\chapter{Conclusion}
In this thesis, we have seen how Dana Scott showed in an elementary way that any $ \lambda $-theory arises as the endomorphism theory of a reflexive object in its category of retracts (Theorem \ref{thm:Scott}). We saw Martin Hyland's proof that any $ \lambda $-theory arises as the endomorphism theory in the presheaf category of its Lawvere theory, using the Yoneda lemma in a very elegant way (Theorem \ref{thm:representation-theorem}).

We also saw how Paul Taylor shows that Scott's category of retracts is relatively cartesian closed (Theorem \ref{thm:Taylor}), and that Hyland gives an interesting new proof of this (Corollary \ref{cor:relatively-cartesian-closed}), using the pre-established fact that the presheaf category is locally cartesian closed. Here it was interesting to note that Taylor's and Hyland's proofs are about the same category in classical mathematics, but that these categories become nonequivalent in univalent foundations (Remark \ref{rem:difference-Taylor-Hyland}), one being the Rezk completion of the other (Corollary \ref{cor:karoubi-candidates}).

As we saw, there are two ways to study the $ \lambda $-calculus using tools from universal algebra: both via $ \lambda $-theories and $ \Lambda $-algebras. We saw that Hyland gives an equivalence between these two in his Fundamental Theorem of the $ \lambda $-calculus (Theorem \ref{thm:final-fundamental-theorem}), where part of his construction again uses a presheaf category, $ \RAct{A_1} $ (Theorem \ref{thm:Hyland-fundamental-theorem}), parallel to his proof of Scott's representation theorem. The equivalence sends a $ \lambda $-theory $ L $ to the $ \Lambda $-algebra $ L_0 $ and sends a $ \Lambda $-algebra $ A $ to its theory of extensions $ \Lambda_A $. We also saw a couple of variations on the proof of this fundamental theorem (Theorem \ref{thm:elementary-fundamental-theorem} and Remark \ref{rem:alternative-algebra-to-theory}), exhibiting multiple equivalent ways to construct a $ \lambda $-theory from a $ \Lambda $-algebra.

Lastly, we saw how part of the material in this thesis was formalized, and we evaluated the choices that were made in the formalization (Chapter \ref{ch:the-formalization}). For some of the very complicated mathematics with a lot of bookkeeping, like the proofs about the Yoneda embedding or the theory of extensions, this formalization constitutes an additional guarantee that it is correct. In some other instances, the process of formalizing contributed to the realization that a lemma should be stated in more generality (see Section \ref{sec:quotients}). Unfortunately, due to the very time-consuming nature of formalization, not all of the material in Hyland's paper could be formalized. In future work, it would be interesting to see which version of the fundamental theorem would lend itself best to formalization. Personally, I would guess it is the most elementary one, exhibited in Section \ref{sec:elementary-fundamental-theorem}.

Also, since we saw that in univalent foundations, there are two nonequivalent definitions $ \bar \C $ and $ \hat \C $ for the category of retracts, it would be interesting to see how well Scott's and Taylor's proofs about $ \bar \C $, can be made to work on its Rezk completion $ \hat \C $. More generally, note that the Karoubi envelope is not specific to the material in this thesis. For example, one way to construct a cartesian closed category is by taking the Karoubi envelope of a `semi cartesian closed category', and the cartesian closed structure on $ \R $ in Scott's representation theorem can be viewed as a special case of this \autocite{hayashi-1985-semifunctors}. In another direction, the category of smooth manifolds can be constructed as the Karoubi envelope of the category $ \C $ of the open subsets of all Euclidean spaces, with smooth maps between them (see \autocite{Lawvere-Karoubi}, page 267). For such classical results about the Karoubi envelope, it would be interesting to study how they hold up in univalent foundations for the different choices $ \overline \C $ and $ \hat \C $ of `the Karoubi envelope'.
