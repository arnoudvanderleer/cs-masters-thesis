\chapter{Previous Work in Categorical Semantics}\label{ch:previous-work}

Hyland, like every scientist, stands on the shoulder of giants. As we saw in the last chapter, he borrows concepts from universal algebra to study the $ \lambda $-calculus. Furthermore, two main theorems in his paper have been proven by others before him, although in a different way. This chapter strives to improve the understanding of Hyland's work by expositing the work of those who came before.

To understand what categorical semantics for the untyped $ \lambda $-calculus is about, we first briefly take a look at categorical semantics for the simply typed $ \lambda $-calculus (Section \ref{sec:lambek-correspondence}). After this, we provide Scott's result about categorical semantics for the untyped $ \lambda $-calculus (Section \ref{sec:scott}), for which we first study the `category of retracts' (Section \ref{sec:retracts-category}). We then discuss Paul Taylor's result about the structure of this category of retracts (Section \ref{sec:taylor}), which says something about its internal logic. Lastly, we briefly discuss a result of Peter Selinger (Section \ref{sec:selinger}), which seems to have inspired Hyland's fundamental theorem of the $ \lambda $-calculus.

\section{The Correspondence Between Categories and Typed \texorpdfstring{$ \lambda $}{lambda}-calculi}\label{sec:lambek-correspondence}
In \autocite{curry}, page 413, Scott and Lambek argue that there is a correspondence between simply typed $ \lambda $-calculi and cartesian closed categories (categories with products and `function objects').

Types in the $ \lambda $-calculus correspond to objects in the category.

Types $ A \to B $ in the $ \lambda $-calculus correspond to exponential objects $ B^A $ in the category.

Terms in the $ \lambda $-calculus of type $ B $, with free variables $ x_1: A_1, \dots, x_n: A_n $, correspond to morphisms $ A_1 \times \dots \times A_n \to B $.

A free variable $ x_i: A_i $ in a context with free variables $ x_1: A_1, \dots, x_n: A_n $ corresponds to the projection morphism $ \pi_i : A_1 \times \dots \times A_n \to A_i $.

Given a term $ s: B_1 \to B_2 $ and a term $ t: B_1 $, both with free variables $ x_1: A_1, \dots, x_n: A_n $, corresponding to morphisms $ \overline s: A_1 \times \dots \times A_n \to B_2 $ and $ \overline t: A_1 \times \dots \times A_n \to B_1 $, the application $ st: B_2 $ corresponds to the composite of the product morphism with the evaluation morphism $ A_1 \times \dots \times A_n \to B_2^{B_1} \times B_1 \to B_2 $.
\begin{center}
  \begin{tikzcd}
    & A_1 \times \dots \times A_n \arrow[ld, "\overline s"'] \arrow[rd, "\overline t"] \arrow[d, "{\langle \overline s, \overline t \rangle}", dashed] &\\
    B_2^{B_1} & B_2^{B_1} \times B_1 \arrow[l, "\pi_1"] \arrow[r, "\pi_2"'] \arrow[d, "ev"] & B_1\\
    & B_2 &
  \end{tikzcd}
\end{center}

Given a term $ t: B $ with free variables $ x_1: A_1, \dots, x_n: A_n $, the abstraction $ (\lambda x_n, t): A_n \to B $ corresponds to using the adjunction corresponding to the exponential object of $ A_n $:
\[ C(A_1 \times \dots \times A_{n-1} \times A_n, B) \simeq C(A_1 \times \dots \times A_{n-1}, B^{A_n}). \]

\section{The Category of Retracts}\label{sec:retracts-category}

The next sections make extensive use of a category called $ \R $, which Hyland calls the `category of retracts'. In this section, we will define the category, and show some properties about it.

Let $ L $ be a $ \lambda $-theory. First of all, for $ a_1, a_2: L_0 $, we define
\begin{align*}
  a_1 \circ a_2 &= \lambda x_1, a_1 (a_2 x_1);\\
  (a_1, a_2) &= \lambda x_1, x_1 a_1 a_2;\\
  \langle a_1, a_2 \rangle &= \lambda x_1, (a_1 x_1, a_2 x_1);\\
  \pi_i &= \lambda x_1, x_1 (\lambda x_2 x_3, x_{i + 1}).
\end{align*}
Although, actually, since every one of these starts with a $ \lambda $-abstraction, we need to lift the constants $ a_i $ to $ \iota_{0, 1}(a_i): L_1 $ to make the definitions above typecheck.

Note that $ \pi_i (a_1, a_2) = a_i $ and $ \pi_i \circ \langle a_1, a_2 \rangle = \lambda x_1, a_i x_1 $, which is exactly what we would expect of a projection.

Also, note that by replacing the $ x_i $ by $ x_{n + i} $ and the $ \iota_{0, 1}(a_i) $ by $ \iota_{n, 1}(a_i) $, we obtain definitions not only for elements of $ L_0 $, but for all $ L_n $.

Later on, we will need not only pairs and their projects, but also $ n $-tuples with projections. Therefore, we define
\begin{align*}
  (a_i)_i = (a_1, \dots, a_n) &= ((\dots((c, a_1), a_2), \dots), a_n);\\
  \langle a_i \rangle_i = \langle a_1, \dots, a_n \rangle &= \langle\langle\dots\langle\langle c, a_1\rangle, a_2\rangle, \dots\rangle, a_n\rangle;\\
  \pi_{n, i} &= \pi_2 \circ \underbrace{\pi_1 \circ \dots \circ \pi_1}_{n - i}
\end{align*}
for some constant $ c $, which usually is something like $ \lambda x_{n + 1}, x_{n + 1} $.

Recall from Section \ref{sec:monoid-category} that we can view any monoid $ M $ as a one-object category $ C_M $, where the morphisms are the elements of $ M $.

\begin{definition}[\coqident{AlgebraicTheories.CategoryOfRetracts}{R}]
  Note that the set of $ \lambda $-terms without free variables $ L_0 $ has a monoid structure under the composition defined above. The category \index{R}$ \R $ is defined as the Karoubi envelope $ \overline C_{L_0} $ of the category $ C_{L_0} $ of this monoid.

  We choose to implement the Karoubi envelope using a slightly different (but equivalent) very syntactical construction using idempotents, because Scott reasons about them in this way:
  \[ \R_0 = \{ A : L_0 \mid A \circ A = A \} \quad \text{and} \quad \R(A, B) = \{ f: L_0 \mid B \circ f \circ A = f \}, \]
  with $ \id A = A $ and composition given by $ \circ $.
\end{definition}

\begin{remark}[\coqident{AlgebraicTheories.CategoryOfRetracts}{R_ob_weq_R'}]
  Hyland instead defines $ \R $ as the Karoubi envelope $ \overline C_{L_1} $ (again, the construction using the idempotents) of the monoid $ (L_1, - \bullet (-)) $ with identity element $ x_{1, 1} $:
  \[ \R_0 = \{ A : L_1 \mid A \bullet A = A \} \quad \text{and} \quad \R(A, B) = \{ f : L_1 \mid B \bullet f \bullet A = f \}, \]
  writing $ s \bullet t $ instead of $ s \bullet (t) $ for $ s, t : L_1 $. Note that we have a monoid morphism:
  \begin{center}
    \begin{tikzcd}
      L_1 \times L_1 \arrow[r, "\bullet"] \arrow[d, "{(s, t) \mapsto ((\lambda x_1, s), (\lambda x_1, t))}"'] & L_1 \arrow[d, "{s \mapsto (\lambda x_1, s)}"]\\
      L_0 \times L_0 \arrow[r, "\circ"] & L_0
    \end{tikzcd}
  \end{center}
  and since for $ A : \R $, $ A = \lambda x_1, \iota_{0, 1}(A) (\iota_{0, 1}(A) x_1) $, we have $ \lambda x_1, \iota_{0, 1}(A) x_1 = A $. This shows that $ s \mapsto \lambda x_1, s $ and $ t \mapsto \iota_{0, 1}(t) x_1 $ (both on objects and morphisms) constitute an equivalence between Hyland's category of retracts and Scott's category of retracts.
\end{remark}

Now, to give a bit more intuition for the objects of $ \R $, we can pretend that an object $ A : \R $ consists of the set of elements that satisfy $ A a = a $. Then a morphism $ f : \R(A, B) $ gives $ B (f a) = (B \circ f) a = f a $. This actually constitutes a functor from $ R $ to $ \Pshf L $:

\begin{definition}\label{def:retracts-embedding}
  We define a functor $ \varphi: \R \to \Pshf L $ by taking
  \[ \varphi(A)_n = \{ a : L_n \mid \iota_{0, n}(A) a = a \} \quad \text{and} \quad \varphi(f)_n(a) = \iota_{0, n}(f) a \]
  for $ A, B : \R $, $ f: \R(A, B) $ and $ a : A $. The presheaf action on $ \varphi(A) $ is given by the substitution of $ L $:
  \[ (a, f) \mapsto a \bullet f \]
  for $ f : L_n^m $ and $ a : L_m $ such that $ \iota_{0, m}(A) a = a $.
\end{definition}

It turns out that this is an embedding:
\begin{lemma}
  This functor $ \varphi $ is fully faithful.
\end{lemma}
\begin{proof}
  Take $ A, B : \R $. We need to show that $ f \mapsto \varphi(f) $ is an equivalence between $ \R(A, B) $ and $ \Pshf L(\varphi(A), \varphi(B)) $. We have a function
  \[ \psi: \Pshf L(\varphi(A), \varphi(B)) \to \R(A, B), \quad g \mapsto (\lambda x_1, g_1(\iota_1(A) x_1)) \]
  which gives us the inverse. To see that this even typechecks, note that we have
  \[ \iota_1(A) x_1 : \varphi(A)_1 \quad \text{and} \quad g_1(\iota_1(A) x_1) : \varphi(B)_1 \subseteq L_1. \]
  Using the fact that $ g $ is a presheaf morphism, we can show that
  \[ B \circ \psi(g) \circ A = \psi(g), \]
  and that $ \varphi $ and $ \psi $ are inverses.
\end{proof}

\begin{remark}
  Note that for all $ A : \R $, we can `reduce' any $ x: L_n $ to
  \[ \iota_{0, n}(A) x : \varphi(A)_n \]
  and in the same way, for all $ A, B: \R $, we can turn any $ f: L_0 $ into a morphism $ B \circ f \circ A : \R(A, B) $. In particular, we have $ B \circ A : \R(A, B) $.
  Of course, all elements of $ \varphi(A)_n $ and all elements of $ \R(A, B) $ arise this way.
\end{remark}

\begin{remark}
  Note that if $ L $ is a nontrivial $ \lambda $-theory, $ \R $ is not a univalent category. To see this, note, for example, that we have an object $ X := \langle \pi_1, \pi_2 \rangle : \R $ (corresponding to the type of `pairs' of $ \lambda $-terms). Since $ L_0 $ is a set, $ X = X $ is a proposition. However, $ X $ has (at least) two automorphisms:
  \[ \langle \pi_1, \pi_2 \rangle \quad \text{and} \quad \langle \pi_2, \pi_1 \rangle. \]
  These are the identity, and the automorphism (of order 2) that swaps the elements of the pair. To see that these are indeed different morphisms, note that applying them (or their lifted versions) to $ (x_{2, 1}, x_{2, 2}) $ gives respectively $ x_{2, 1} $ and $ x_{2, 2} $, which are distinct elements by Lemma \ref{lem:nontrivial-algebraic-theory}.
\end{remark}

In the next chapter, the `universal object' $ U : \R $, given by the identity $ \lambda x_1, x_1 $ (\coqident{AlgebraicTheories.CategoryOfRetracts}{U}), plays a major role. Note that for all $ A: \R $, we have morphisms $ A: \R(U, A) $ and $ A: \R(A, U) $, which exhibit $ A $ as a retract of $ U $ (\coqident{AlgebraicTheories.CategoryOfRetracts}{R_retraction_is_retraction}). Also note that $ \varphi(U) = L $.

Note that $ \R $ has many (isomorphic, but not equal for nontrivial $ L $) terminal objects, given by $ I_c := \lambda x_1, \iota_{0, 1}(c) : \R $ for any $ c: L_0 $ (\coqident{AlgebraicTheories.CategoryOfRetracts}{R_terminal}). These are terminal because for $ f: A \to I_c $, we have $ f = I_c \circ f = I_c $. Note that $ \varphi(I_c)_n = \{ \iota_{0, n}(c) \} $. We will choose $ I = \lambda x_1 x_2, x_2 : \R $ as our main example of the terminal object.

We might wonder whether $ \R $ also has an initial object $ O $. However, for all $ c: L_0 $, we would have a constant morphism to the universal object
\[ \lambda x_1, \iota_{0, 1}(c) : \R(O, U), \]
so if $ L $ is nontrivial, $ \R $ has no initial object.

$ \R $ has binary products with projections and product morphisms (\coqident{AlgebraicTheories.CategoryOfRetracts}{R_binproducts})
\[ A_1 \times A_2 = \langle p_1, p_2 \rangle, \quad p_i = A_i \circ \pi_i \quad \text{and} \quad \langle f, g \rangle. \]
Recall that for any object $ A $, we have $ A = \id A $, which for $ A_1 \times A_2 $ is $ \langle p_1, p_2 \rangle $ by the universal property of the product, which explains why the product is of this form.

$ \R $ also has exponential objects (\coqident{AlgebraicTheories.CategoryOfRetracts}{R_exponentials})
\[ C^B = \lambda x_1, C \circ x_1 \circ B \]
with evaluation morphism $ \epsilon_{BC}: C^B \times B \to C $ given by
\[ \epsilon_{BC} = \lambda x_1, C(\pi_1 x_1 (B (\pi_2 x_1))), \]
which is universal because we can lift a morphism $ f: \R(A \times B, C) $ to a morphism $ \psi(f): \R(A, C^B) $ given by
\[ \psi(f) = \lambda x_1 x_2, f (x_1, x_2). \]
Note that for $ g: \R(A, C^B) $, the inverse $ \psi^{-1}(g) $ is given by
\[ \epsilon \circ \langle g \circ \pi_1, B \circ \pi_2 \rangle = \lambda x_1, g (\pi_1 x_1) (\pi_2 x_1): \R(A \times B, C). \]
Also note that
\[ \psi(\epsilon_{BC}) = \lambda x_1, C \circ x_1 \circ B = C^B = \id{C^B}. \]
Note that $ \varphi(C^B)_0 = \R(B, C) $, as we would expect.

Now, we might wonder whether there exists some $ A: \R $ such that $ \varphi(A)_0 = \emptyset $. However, for any $ c : L_0 $, we have $ A c : \varphi(A)_0 $, because
\[ A c = (A \circ A) c = A (A c). \]
Note that we can lift constants from $ \varphi(A)_0 $ to any $ \varphi(A)_n $, so they are all nonempty.

Combining this with the embedding of $ \R \hookrightarrow \Pshf L $, we would expect $ \R $ to not have all pullbacks. This is because in $ \Pshf L $, for a cospan $ B \xrightarrow f A \xleftarrow g C $ with $ f_n(B_n) \cap g_n(C_n) = \emptyset $ for some $ n $, the pullback $ Q $ would have $ Q_n = \emptyset $, which could never happen with an object coming from $ \R $. Note, however, that this can not be made rigorous, because a fully faithful embedding reflects limits, but does not necessarily preserve them.

However, it is true that $ \R $ does not have all pullbacks (if $ L $ is nontrivial). Consider for example the following cospan:
\[ I \xrightarrow f U \xleftarrow g I \]
for different $ f, g: \R(I, U) $. For example, $ f = (\lambda x_1, \iota_{0, 1}(\pi_1)) $ and $ g = (\lambda x_1, \iota_{0, 1}(\pi_2)) $. Now, take any object $ Q: \R $. Note that we have a unique morphism $ I: \R(Q, I) $. Then we have the following diagram:
\begin{center}
  \begin{tikzcd}
    Q \arrow[r, "I"] \arrow[d, "I"] & I \arrow[d, "g"]\\
    I \arrow[r, "f"] & U
  \end{tikzcd}
\end{center}
with
\[ f \circ I = f \not = g = g \circ I, \]
so the diagram does not commute, and $ Q $ is not a pullback of this cospan.

Taylor notes \autocite[][Section 1.5]{taylor} that the objects in $ \R $ have very strong properties with respect to fixpoints. One of the properties also arises via Lawvere's fixed point theorem \autocite[][page 136]{lawvere-fixpoints}: For all $ B: \R $, since $ B^U $ is a retract of $ U $, every endomorphism $ f: B \to B $ has a fixpoint (\coqident{AlgebraicTheories.CategoryOfRetracts}{fixpoint_is_fixpoint}). That is, there exists $ s: I \to B $ such that $ f \circ s = s $. Working out the proof even yields an explicit term:
\[ s = \lambda i, (\lambda x, f (x x)) (\lambda x, f (x x)). \]
Indeed, $ s = \lambda i, f ((\lambda x, f (x x)) (\lambda x, f (x x))) = f \circ s $, and $ B \circ s = B \circ f \circ s = s = s \circ I $, so $ s : \R(I, B) $.

From this, Taylor deduces \autocite[][\S 1.5.12]{taylor} that $ \R $ does not have all coproducts if $ L $ is nontrivial, because suppose that $ \R $ has all coproducts. Then $ B = I + I $ is a `boolean algebra object': We define $ \bot, \top: I \to B $ to be the injections on the left and right components. Since $ \R $ is cartesian closed, binary products distribute over binary coproducts, so we have $ B \times B \cong ((I \times I) + (I \times I)) + ((I \times I) + (I \times I)) $, and note that $ I \times I \cong I $. Using the universal property of the coproduct, we can define $ \lnot: B \to B $ componentwise as $ \lnot = [\top, \bot] $, the coproduct arrow
\begin{center}
  \begin{tikzcd}[sep=large]
    I \arrow[r, "\bot"] \arrow[rd, "\top"'] & I + I \arrow[d, dashed, "{[\top, \bot]}"description] & I \arrow[l, "\top"'] \arrow[ld, "\bot"]\\
    & I + I &
  \end{tikzcd}
  % \begin{tikzcd}[sep=large]
  %   I \arrow[rrrd, "\bot"'] \arrow[r, "\bot"] & I + I \arrow[rrd, "{[\bot, \bot]}" description, dashed] \arrow[rr, bend left] & I \arrow[rd, "\bot"] \arrow[l, "\top"'] & I + I + I + I \arrow[d, dashed, "{[[\bot, \bot], [\bot, \top]]}" description] & I \arrow[ld, "\bot"'] \arrow[r, "\bot"] & I + I \arrow[lld, "{[\bot, \top]}" description, dashed] \arrow[ll, bend right] & I \arrow[llld, "\top"] \arrow[l, "\top"'] \\
  %   & & & I + I & & &
  % \end{tikzcd}
\end{center}
Note that by the definition of $ \lnot $, $ \lnot \circ \bot = \top $. Similarly, we define $ \land, \lor : B \times B \to B $ as $ \land = [[\bot, \bot], [\bot, \top]] $ and $ \lor = [[\bot, \top], [\top, \top]] $. Using the same universal property, we can also verify some properties of $ \bot, \top, \land $ and $ \lnot $, like the fact that the following diagrams commute:
\begin{center}
  \begin{tikzcd}[sep=large]
    B \arrow[rd, "{\id B}"] \arrow[r, "{\langle \id B, \id B \rangle}"] & B \times B \arrow[d, "\land"]\\
    & B
  \end{tikzcd}
  \begin{tikzcd}[sep=large]
    B \arrow[d, "!"] \arrow[r, "{\langle \lnot, \id B \rangle}"] & B \times B \arrow[d, "\land"]\\
    I \arrow[r, "\bot"] & B
  \end{tikzcd}
\end{center}
for $ !: B \to I $ the terminal projection. We do this by checking, for example, that $ \id B \circ \top = \land \circ \langle \id B, \id B \rangle \circ \top $ and $ \id B \circ \bot = \land \circ \langle \id B, \id B \rangle \circ \bot $. Now, as mentioned above, every endomorphism has a fixed point. In particular, we have some $ \star: I \to B $ such that $ \lnot \circ \star = \star $. Now, note that
\[ \star = \land \circ \langle \id B, \id B \rangle \circ \star = \land \circ \langle \star, \star \rangle = \land \circ \langle \star, \lnot \circ \star \rangle = \land \circ \langle \id B, \lnot \rangle \circ \star = \bot \circ ! \circ \star = \bot \]
and then
\[ \bot = \star = \lnot \circ \star = \lnot \circ \bot = \top. \]
Now, for any object $ A: \R $ and any two global elements $ f, g: I \to A $, we have the following diagram:
\begin{center}
  \begin{tikzcd}
    I \arrow[r, shift left, "\bot"] \arrow[r, shift right, "\top"'] \arrow[rr, bend left, "f"] \arrow[rr, bend right, "g"'] & I + I \arrow[r, "{[f, g]}"] & A
  \end{tikzcd}
\end{center}
and we have
\[ f = [f, g] \circ \bot = [f, g] \circ \top = g. \]
In particular, for any two objects $ A, A^\prime: \R $, since we have $ A \circ A^\prime : \R(A^\prime, A) $, so $ \R(A^\prime, A) $ is nonempty, we have
\[ \R(A^\prime, A) \simeq \R(I \times A^\prime, A) \simeq \R(I, A^{A^\prime}) \simeq \{ \star \}, \]
so $ \R $ is the trivial category and $ L $ is trivial.

\section{Scott's Representation Theorem}\label{sec:scott}
The correspondence in Section \ref{sec:lambek-correspondence} between simply-typed $ \lambda $-calculi and cartesian closed categories raises a question whether such a correspondence also exists for untyped $ \lambda $-calculi. Definition \ref{def:endomorphism-theory} shows that in fairly general circumstances we can take one object $ c $ in a category $ C $ and consider the morphisms $ t: C(c^n, c) $ as terms in an untyped $ \lambda $-calculus. Hyland calls this the `endomorphism theory' of $ c $.

\begin{remark}
  To construct a \textit{simply typed} $ \lambda $-calculus from a category, we just need a cartesian closed category. In a simply typed $ \lambda $-calculus, there is a lot of restriction on which terms we can apply to each other. A term of type $ A \to B $ can only be applied to a term of type $ A $, which gives a term of type $ B $. In particular, a term can be applied only finitely many times to other terms, and every time, the result has a different type.

  On the other hand, for an \textit{untyped} $ \lambda $-calculus, we need a cartesian closed category with a `reflexive object'. This is because in the untyped $ \lambda $-calculus, we can apply arbitrary terms to each other. For example, we can apply the term $ (\lambda x_1, x_1 x_1) $ to itself, which would not be typable in the simply typed $ \lambda $-calculus. Suppose that we have a category $ C $ and an object $ U $ such that the morphisms $ C(U^n, U) $ give the untyped $ \lambda $-terms in $ n $ free variables, for all $ n $. Now, given two terms $ f, g : C(U^n, U) $, for the application $ f g $, we need to consider $ f $ as a morphism in $ C(U^n, U^U) $. We can do this by postcomposing with a morphism $ \varphi: C(U, U^U) $. On the other hand, if $ n > 0 $, then $ (\lambda x_n, f) $ is a morphism in $ C(U^{n - 1}, U^U) $, but it is a term in $ n - 1 $ free variables, so it should be in $ C(U^{n - 1}, U) $. For this, we postcompose with a morphism $ \psi: C(U^U, U) $. Now, for our untyped $ \lambda $-calculus to have $ \beta $-equality, we need $ \psi \cdot \varphi = \id{U^U} $, which means that the exponential $ U^U $ of $ U $ is a retract of $ U $. This is exactly what it means for $ U $ to be a \iindex{reflexive object}: an object $ U $ in a cartesian closed category, that has a retraction onto its `function space' $ U^U $. Note that if we want our $ \lambda $-theory to also have $ \eta $-equality, the retraction must be an isomorphism.

  Note that $ \SET $ is a cartesian closed category, but that for sets $ X $ and $ Y $, the function space $ X^Y $ has cardinality $ \vert X \vert^{\vert Y \vert} $, and therefore $ U^U $ cannot be a retract of $ U $, unless $ U = \{ \star \} $, in which case we have a very trivial $ \lambda $-calculus: $ \SET(U^n, U) = \{ \star \} $.

  During the 1960s, computer scientists sought for nontrivial examples of reflexive objects, there is a quote by Dana Scott that ``Lambda-calculus has no mathematical models!'' \autocite{strachey}. However, in 1969, the same Dana Scott discovered that the category of (continuous) lattices with (Scott-)continuous functions between them has a nontrivial reflexive object, with the retraction onto the function spaces being even an isomorphism. Such a reflexive object $ D_\infty $ is obtained by starting with an arbitrary lattice $ D_0 $, iteratively taking $ D_{n + 1} = D_n^{D_n} $, with a retraction (in the `wrong' direction) $ r: D_n^{D_n} \to D_n $, and then passing to the limit $ D_\infty = \lim_{\leftarrow} D_n $ (for the main result, see Theorem 4.4 in \autocite{scott-continuous}, page 127).
\end{remark}

Since Lambek showed an equivalence between simply typed $ \lambda $-calculi and cartesian closed categories, and since we have a construction for an untyped $ \lambda $-calculus from a cartesian closed category with a reflexive object, we can wonder whether this construction constitutes contitutes an equivalence between untyped $ \lambda $-calculi and some class of categories. This question finds a partial answer in the following theorem, originally proved in a very syntactical way by Dana Scott \autocite[][418]{curry}.
\begin{theorem}[\coqident{AlgebraicTheories.OriginalRepresentationTheorem}{representation_theorem_iso}]\label{thm:Scott}
  We can obtain every untyped $ \lambda $-calculus as the endomorphism theory of some object in some category.
\end{theorem}
\begin{proof}
  Let $ L $ be a $ \lambda $-theory. Scott considers its category of retracts $ \R $, with `universal object' $ U $.

  Note that $ U^U $ is a retract of $ U $, so $ U $ is a reflexive object.

  Therefore, $ E(U) $, the endomorphism theory of $ U $, has a $ \lambda $-theory structure. Note that the finite powers of $ U $ in $ \R $ are given by $ U^0 = I $ and $ U^{n + 1} = U^n \times U $.

  We have $ E(U)_n = \R(U^n, U) = \{ f: L_0 \mid U \circ f \circ U^n = f \} $. The variables of $ E(U) $ are the projections $ \pi_{n, i} $ of $ U^n $. The substitution is given by composition with the product morphism:
  \[ f \bullet g = f \circ \langle \langle \langle I, g_1 \rangle, \dots \rangle, g_n \rangle. \]
  We have $ U^U = \lambda f, U \circ f \circ U = \lambda x_1 x_2, x_1 x_2 $. Using the equivalence $ \R(U^n \times U, U) \simeq \R(U^n, U^U) $ and the retraction $ U^U: U \to U^U $, the abstraction and application $ \lambda $ and $ \rho $ are given by
  \[ \lambda(f) = \lambda x_1 x_2, \iota_{0, 2}(f)(x_1, x_2), \quad \rho(g) = \lambda x_1, \iota_{0, 1}(g) (\pi_1 x_1) (\pi_2 x_1). \]
  for $ f: \R(U^{n + 1}, U) $ and $ g: \R(U^n, U) $.

  Now, we have bijections $ \psi_0: E(U)_0 \xrightarrow{\sim} L_0 $, given by
  \[ \psi_0(f) = f (\lambda x_1, x_1) \quad \text{and} \quad \psi_0^{-1}(g) = \lambda x_1, \iota_{0, 1}(g). \]
  We can extend this to any $ n $, by reducing any term to a constant by repeatedly using $ \lambda $, then applying the bijection, and then lifting it again using $ \rho $. Explicitly, we obtain
  \[ \psi_n(f) = \iota_{0, n}(f)(x_i)_i, \quad \text{and} \quad \psi^{-1}_n(g) = \lambda x_1, g \bullet (\pi_{n, i} x_1)_i. \]
  It is not hard to verify that this is indeed a bijection, using at one point the fact that $ f: \R(U^n, U) $ is defined by $ f \circ U^n = f $, for
  \[ U^n = \langle \pi_{n, i} \rangle_i. \]
  It is also pretty straightforward to check that
  \begin{align*}
    \psi(\pi_{n, i}) &= x_i, & \psi(f) \bullet (\psi(g_i))_i &= \psi(f \bullet g),\\
    \psi(\lambda(h)) &= \lambda(\psi(h)), & \psi(\rho(h^\prime)) &= \rho(\psi(h^\prime))
  \end{align*}
  for $ f: E(U)_m $, $ g: E(U)_n^m $, $ h: E(U)_{n + 1} $ and $ h^\prime : E(U)_n $. Therefore, $ \psi $ is an isomorphism of $ \lambda $-theories.
\end{proof}

\section{The Taylor Fibration}\label{sec:taylor}

In his dissertation, Paul Taylor shows that $ \R $ is not only cartesian closed, but also \iindex{relatively cartesian closed}. Recall we have chosen to interpret $ \R $ as the category $ \overline C_{L_0} $, constructed using idempotents, because both Scott's and Taylor's proof are very syntactical in nature.

In Section \ref{sec:dependent-products}, we studied internal and external representations of families of objects in a category and how they behaved under substitutions (pullbacks). This was to arrive at a definition for dependent products and sums, as the right and left adjoints to the pullback (or substitution) functor $ \alpha^*: (C \downarrow A) \to (C \downarrow B) $ along some morphism $ \alpha : C(B, A) $.

Now, some categories are not locally cartesian closed. That is: not all pullback/substitution functors $ \alpha^* $ exist or have a right adjoint. For example, $ \R $ does not have all pullbacks, so the substitution functors do not always exist. In such a category $ C $, the functor $ C^2 \to C $ is not a fibration. One way to look at this, is that in such a category not every morphism $ X \to A $ represents a family of objects. In such a category, we can carefully choose a subset of the morphisms to represent our indexed families. We will call a morphism that we choose to represent an indexed family a \iindex{display map}. In most cases, we have quite a bit of choice how big we want our subtype of display maps to be. However, to make sure that indexed families are well-behaved, the subtype of display maps needs to have some properties:
\begin{enumerate}
  \item The pullback of a display map along any morphism exists and is a display map.
  \item The composite of two display maps is a display map.
  \item $ C $ has a terminal object and any terminal projection is a display map.
\end{enumerate}
\begin{remark}
  A `maximal' example of a subtype of display maps is, for example, in the category $ \SET $, where we can take our subtype of display maps to equal the full type of all morphisms of $ C $. This works in any category that has all limits.
\end{remark}
\begin{remark}
  A `minimal' example of a subtype of display maps is the subtype of product projections in a univalent category with finite products. In this case, all indexed families are constant, and then dependent sums and products become binary products and exponential objects.
\end{remark}

Now, let $ C $ be a category with a subtype of display maps $ D \subseteq C $. We will denote the subtype of display maps from $ X $ to $ A $ with $ D(X, A) \subseteq C(X, A) $. For any $ A : C $, we define the category $ (C \downarrow_D A) $ as a full subcategory of the slice category $ (C \downarrow A) $, with as objects the display maps $ f: D(X, A) $. The morphisms between two objects of $ (C \downarrow_D A) $ are still all the morphisms of $ (C \downarrow A) $ (i.e. the morphisms of $ C $ that commute with the display maps).

Note that for the terminal object $ I : C $, $ (C \downarrow_D I) $ is still equivalent to $ C $, since every terminal projection is a display map.

Also note that since the pullback of display maps against any map exist (and are display maps again), we get a pullback functor $ \alpha^*: (C \downarrow_D A) \to (C \downarrow_D B) $. This is the restriction of the pullback functor $ \alpha^*: (C \downarrow A) \to (C \downarrow B) $, if it exists.

Note that since composing two display maps gives a display map again, and since the dependent sum is given by postcomposition, $ \alpha^* $ has a left adjoint for all display maps $ \alpha $. That is: the fiber categories $ (C \downarrow_D A) $ have dependent sums over display maps.

The question whether the fiber categories $ (C \downarrow_D A) $ also have dependent products over display maps, brings us to the definition of relative cartesian closedness.
\begin{definition}
  A category $ C $ is \index{cartesian closed!relatively}\textit{cartesian closed relative} to a class of display maps $ D $, if the substitution functors $ \alpha^* $ along display maps have right adjoints.
\end{definition}

Now, analogously to Lemma \ref{lem:locally-cartesian-closed}, Taylor shows:
\begin{lemma}
  If a category $ C $ is cartesian closed relative to a class of display maps $ D $, then the fiber categories $ (C \downarrow_D A) $ are cartesian closed and the substitution functors $ \alpha^* $ preserve this structure.
\end{lemma}
\begin{proof}
  Taylor proves this in a series of lemmas leading up to \S 4.3.7 in \autocite{taylor}. It is also proved as Proposition 6 of \autocite{theory-of-constructions}.
\end{proof}

\subsection{Taylor's proof}

As Hyland remarks, Taylor shows that $ \R $ is relatively cartesian closed using a very syntactical argument, in the spirit of Scott.

He starts out with a kind of `external' representation of indexed families $ (X_a)_a $ in $ \R $. He denotes these as `functions' $ A \to \R_0 $. They are the elements $ X: L_0 $ with
\[ X \circ A = X \quad \text{and} \quad (\lambda x_1, (X x_1) \circ (X x_1)) = X. \]
These $ X $ form a category $ \R^A $, with the morphisms in $ \R^A(X, Y) $ given by $ f: L_0 $ with $ f \circ A = f $ and $ (\lambda x_1, (Y x_1) \circ (f x_1) \circ (X x_1)) = f $. The identity on $ X $ is given by $ X $, and the composition is given by $ f \cdot g = \lambda x_1, g x_1 \circ f x_1 $.

Taylor shows that these categories $ \R^A $ are cartesian closed. He notes that assigning these categories $ \R^A $ to objects $ A: \R $ again constitutes a contravariant (pseudo)functor $ \op \R \to \Cat $, sending morphisms $ A \to B $ to precomposition functors $ \R^B \to \R^A $.

For $ A: \R $, we introduce the combinator
\[ \sum_A = \lambda x_1 x_2, (A (\pi_1 x_2), x_1 (\pi_1 x_2) (\pi_2 x_2)). \]

For $ A: \SET $, we have an equivalence between the elements of $ \SET^A $ and the elements of the fiber $ (\SET \downarrow A) $ of the fibration $ \SET^2 \to \SET $. Now, for $ A: \R $, $ \sum_A $ gives a functor from $ \R^A $ to $ (\R \downarrow A) $. Note that it sends objects $ X: \R^A $ to $ \sum_A X $, together with a projection morphism $ p_X = A \circ \pi_1: \sum_A X \to A $. Note that with the embedding in Definition \ref{def:retracts-embedding} into $ \Pshf L $, we can consider $ \sum_A X $ as consisting of pairs $ (a, x) $ with $ a : A $ and $ x : X a $, which is exactly what we would expect from a dependent sum.

Now, $ \sum_A $ is fully faithful: given a morphism $ g: \R\left(\sum_A X, \sum_A Y\right) $, we take
\[ \psi(g) = \lambda x_1 x_2, \pi_2 (g (x_1, x_2)). \]
For verifying that $ \psi(g) $ is indeed a morphism in $ \R^A(X, Y) $, it helps to recall that $ g \circ \sum_A X = g = \sum_A Y \circ g $ and $ p_Y \circ g = p_X $. Since $ \psi(\sum_A f) = f $ for all $ f: \R^A(X, Y) $ and $ \sum_A \psi(g) = g $ for all $ g: \R(\sum_A X, \sum_A Y) $, $ \sum_A $ is indeed fully faithful.

On the other hand, $ \sum_A $ is generally not surjective. However, we can choose our subtype $ D $ of display maps in such a way that the restriction $ \R^A \to (C \downarrow_D A) $ becomes essentially surjective.

\begin{definition}
  For $ \R $, we will consider a morphism $ f: X \to A $ to be a display map if we have some $ Y : \R^A $ like above, and some isomorphism $ g: (X, f) \xrightarrow{\sim} (\sum_A Y, p_Y) $ in $ (\R \downarrow A) $.
\end{definition}

\begin{remark}
  Hyland actually gives a different characterization of Taylor's display maps. He claims that Taylor takes the display maps $ X \to A $ to be the retracts in $ (\R \downarrow A) $ of $ p_1: A \times U \to A $,
  the projection onto the first coordinate. The two characterizations are equivalent:

  Given $ Y: \R^A $, intuitively $ Y a $ is a retract of $ U $ for all $ a $. Concretely, both morphisms $ r : A \times U \to \sum_A Y $ and $ s : \sum_A Y \to A \times U $ are given by $ \sum_A Y $ and these commute with the projection to $ A $. Therefore, if we have an isomorphism $ g: X \xrightarrow{\sim} \sum_A Y $ in $ (\R \downarrow A) $, then $ \sum_A Y \cdot g^{-1} $ and $ g \cdot \sum_A Y $ make $ X $ into a retract of $ A \times U $.

  Conversely, given a retraction $ r : A \times U \to X $ and section $ s : X \to A \times U $ in $ (\R \downarrow A) $, we have
  \[ Y = \lambda x_1 x_2, \pi_2 (s (r (x_1, x_2))) : \R^A. \]
  Using the properties of $ r $ and $ s $ and the definition of $ \sum_A Y $, one can show that $ r $ gives a morphism $ \sum_A Y \to X $ and $ s $ gives a morphism $ X \to \sum_A Y $, and that $ r \circ s = \id{\sum_A Y} $. Combined with the fact that $ s \circ r = \id X $, this shows that $ X $ is isomorphic to $ \sum_A Y $.
\end{remark}

\begin{remark}
  Recall that if $ L $ is nontrivial, $ \R $ is not univalent, and the existence of $ Y: \R^A $ with the isomorphism $ g: X \xrightarrow{\sim} \sum_A Y $ is not a proposition. Take, for example, $ Y : \R^I $ given by $ (\lambda x_1, U \times U) $. Recall that $ (\R \downarrow I) $ is equivalent to $ \R $. Under this equivalence $ \sum_I Y $ is equivalent to $ U \times U $. As mentioned before, $ \sum_I Y $ has (at least) two distinct isomorphisms to itself: the identity and the isomorphism that swaps both sides of the product.

  In a similar way, being a retract of $ A \times U $ is not a proposition. Therefore, if we want the class of display maps to really be a subclass of the class of morphisms, we need the existence of $ Y: \R^A $ and the isomorphism $ g : X \xrightarrow \sim \sum_A Y $, or the existence of the retraction $ A \times U \to X $, to mean \textit{mere existence} in this case. This means that we cannot use these in constructions, except for mere propositions.
\end{remark}

Taylor shows that these display maps indeed satisfy the three properties mentioned above. However, in univalent foundations, there are some subtleties in the case of pullbacks:
\begin{enumerate}
  \item Given a display map $ (X, f): (\R \downarrow_D A) $ and a morphism $ \alpha: \R(B, A) $. Recall that this means that there merely exists a $ Y: \R^A $ and an isomorphism $ X \xrightarrow \sim \sum_A Y $ in $ (\R \downarrow A) $. Therefore, we have mere existence of a pullback $ \sum_B (Y \circ \alpha) $ of $ \sum_A Y $ along $ \alpha $. For all $ C: \R $ with $ \beta: \R(C, B) $ and $ \gamma: \R(C, \sum_A Y) $ that make the square commute, we have the morphism $ \lambda x_1, (\beta x_1, \pi_2 (\gamma x_1)): \R(C, \sum_B (Y \circ \alpha)) $.
    \begin{center}
      \begin{tikzcd}[sep=large]
        C \arrow[rrrd, bend left, "\gamma"] \arrow[dd, bend right=90, "\beta"'] \arrow[d, dashed, "{\lambda x_1, (\beta x_1, \pi_2 (\gamma x_1))}"] \\
        \sum_B (Y \circ \alpha) \arrow[rrr, "{\lambda x_1, (\alpha (\pi_1 x_1), Y (\alpha (\pi_1 x_1)) (\pi_2 x_1))}"] \arrow[d, "p_{(Y \circ \alpha)}"] &&& \sum_A Y \arrow[d, "p_Y"] & X \arrow[l, "g"] \arrow[ld, "f"]\\
        B \arrow[rrr, "\alpha"] &&& A
      \end{tikzcd}
    \end{center}
    By the isomorphism between $ X $ and $ \sum_A Y $, $ \sum_B (Y \circ \alpha) $ is also a pullback of $ X $: by postcomposing with $ g $ and its inverse, we see that morphisms from $ \sum_B (Y \circ A) $ and $ C $ to $ X $ are equivalent to morphisms to $ \sum_A Y $.
  \item Given display maps $ f: B \to A $ and $ g: C \to B $. We have $ X : \R^A $ and $ Y: \R^B $ with isomorphisms $ s: B \xrightarrow \sim \sum_A X $ and $ t: C \xrightarrow \sim \sum_B Y $. We need to show that $ f \circ g: C \to A $ is also a display map.
    \begin{center}
      \begin{tikzcd}
        C \arrow[d, "g"] \arrow[r, "\sim", "t"'] & \sum_B Y \arrow[ld, "p_Y"] \arrow[r, "u"', "\sim"] & \sum_A Z \arrow[ldld, bend left, "p_Z"]\\
        B \arrow[d, "f"] \arrow[r, "\sim", "s"'] & \sum_A X \arrow[ld, "p_X"]\\
        A
      \end{tikzcd}
    \end{center}
    Intuitively, $ \sum_B Y \cong \sum_A Z $ represents
    \[ \sum_{b: \sum_{a: A} X a} Y b \cong \sum_{a: A} \sum_{x: X a} Y(a, x), \]
    by the isomorphism that relates $ ((a, x), y) $ to $ (a, (x, y)) $. Therefore, consider
    \[ Z = \lambda x_1 x_2, (X x_1 (\pi_1 x_2), Y (s^{-1} (x_1, \pi_1 x_2)) (\pi_2 x_2)): \R^A. \]
    we have an isomorphism $ u: \sum_B Y \xrightarrow \sim \sum_A Z $ given by:
    \[ u = \lambda x_1, (A a, (X a x, Y b y)) \]
    with
    \[ b = \pi_1 x_1;\quad a = \pi_1 (s b); \quad x = \pi_2 (s b); \quad y = \pi_2 x_1, \]
    and
    \[ u^{-1} = \lambda x_1, (s^{-1}(A a, X a x), Y b y) \]
    with
    \[ a = \pi_1 x_1; \quad x = \pi_1 (\pi_2 x_1); \quad b = s^{-1}(a, x); \quad y = \pi_2 (\pi_2 x_1) \]
    and then $ u \circ t $ is an isomorphism in $ (\R \downarrow A) $ between $ (C, f \circ g) $ and $ (\sum_A Z, p_Z) $.

    % \begin{itemize}
    %   \item $ Z: \R^A $
    %     \begin{align*}
    %       &Z x_1 (Z x_1 x_2)\\
    %       &= (X x_1 (X x_1 (\pi_1 x_2)), Y (s^{-1} (x_1, (X x_1 (\pi_1 x_2)))) (Y (s^{-1} (x_1, (\pi_1 x_2))) (\pi_2 x_2)))\\
    %       &= (X x_1 (\pi_1 x_2), Y (s^{-1} (x_1, \pi_1 x_2)) (\pi_2 x_2))\\
    %       &= Z x_1 x_2.
    %     \end{align*}
    %   \item $ u $ is a morphism
    %     \begin{align*}
    %       \sum_A Z (u (\sum_B Y x_1))
    %       &= \sum_A Z (u (B (\pi_1 x_1), Y (\pi_1 x_1) (\pi_2 x_1)))\\
    %       &= \sum_A Z (\pi_1 (s (\pi_1 x_1)), (\pi_2 (s (\pi_1 x_1)), Y (\pi_1 x_1) (\pi_2 x_1)))\\
    %       &= (A (\pi_1 (s (\pi_1 x_1))), (X (\pi_1 (s (\pi_1 x_1))) ((\pi_2 (s (\pi_1 x_1)))), Y (\pi_1 x_1) (\pi_2 x_1)))\\
    %       &= u x_1.
    %     \end{align*}
    %   \item $ u^{-1} $ is a morphism
    %     \begin{align*}
    %       \sum_B Y (u^{-1} (\sum_A Z x_1))
    %       &= \sum_B Y (u^{-1} (A (\pi_1 x_1), (X (\pi_1 x_1) (\pi_1 (\pi_2 x_1)), Y (s^{-1} ((\pi_1 x_1), \pi_1 (\pi_2 x_1))) (\pi_2 (\pi_2 x_1)))))\\
    %       &= \sum_B Y (s^{-1}(\pi_1 x_1, \pi_1 (\pi_2 x_1)), Y (s^{-1} (\pi_1 x_1, \pi_1 (\pi_2 x_1))) (\pi_2 (\pi_2 x_1)))\\
    %       &= (s^{-1}(\pi_1 x_1, \pi_1 (\pi_2 x_1)), Y (s^{-1} (\pi_1 x_1, \pi_1 (\pi_2 x_1))) (\pi_2 (\pi_2 x_1)))\\
    %       &= u^{-1} x_1
    %     \end{align*}
    %   \item $ u \circ u^{-1} = \id{\sum_A Z} $
    %     \begin{align*}
    %       u(u^{-1} x_1)
    %       &= u(s^{-1} (\pi_1 x_1, \pi_1 (\pi_2 x_1)), Y (s^{-1} (\pi_1 x_1, \pi_1 (\pi_2 x_1))) (\pi_2 (\pi_2 x_1)))\\
    %       &= (A (\pi_1 x_1), (X (\pi_1 x_1) (\pi_1 (\pi_2 x_1)), Y (s^{-1} (\pi_1 x_1, \pi_1 (\pi_2 x_1))) (\pi_2 (\pi_2 x_1))))\\
    %       &= \sum_A Z x_1
    %     \end{align*}
    %   \item $ u^{-1} \circ u = \id{\sum_B Y} $.
    %     \begin{align*}
    %       u^{-1} (u x_1)
    %       &= u^{-1}(\pi_1 (s (\pi_1 x_1)), (\pi_2 (s (\pi_1 x_1)), Y (\pi_1 x_1) (\pi_2 x_1)))\\
    %       &= (B (\pi_1 x_1), Y (\pi_1 x_1) (\pi_2 x_1))
    %       &= \sum_B Y x_1.
    %     \end{align*}
    %   \item $ p_Z \circ u = g \circ p_Y $.
    %     \begin{align*}
    %       p_Z (u x_1)
    %       &= p_Z (\pi_1 (s (\pi_1 x_1)), (\pi_2 (s (\pi_1 x_1)), Y (\pi_1 x_1) (\pi_2 x_1)))\\
    %       &= A (\pi_1 (s (\pi_1 x_1)))\\
    %       &= A (\pi_1 (s (B (\pi_1 x_1))))\\
    %       &= p_X (s (p_Y x_1))\\
    %       g (p_Y x_1).
    %     \end{align*}
    % \end{itemize}
  \item Let $ B \to I $ be a terminal projection. Consider $ Y = (\lambda x_1, B) : \R^I $. We have
    \[ \sum_I Y = \lambda x_1, ((\lambda x_2, x_2), B (\pi_2 x_1)) = \langle I \circ \pi_1, B \circ \pi_2 \rangle = I \times B, \]
    which is isomorphic to $ B $.
\end{enumerate}

\begin{remark}
  There is an awful lot of properties to verify here (e.g. that some terms are elements of some $ \R^A $, that others are morphisms in some $ \R(\sum_A X, \sum_B Y) $, that some diagrams commute), but most of it boils down to writing down the equations that should hold, reducing them via $ \beta $-equality, and then using the facts that objects $ A: \R $ are idempotent, that morphisms $ f: \R(A, B) $ equal $ B \circ f \circ A $ and that objects $ X: \R^A $ satisfy $ X \circ A = X $ and $ X a (X a x) = X a x $ for all $ a $ and $ x $. Note that for a morphism $ f: \sum_A X \to B $,
  \[ f(A x_1, x_2) = f(x_1, x_2) = f(x_1, X x_1 x_2), \]
  so $ f $ `absorbs' such instances of $ A $ and $ X $, so to speak.
\end{remark}

\begin{remark}
  To talk about `relatively cartesian closed', we need a pullback functor $ \alpha^* : (\R \downarrow_D A) \to (\R \downarrow_D B) $ for all (display maps) $ \alpha: B \to A $. However, as remarked before, since the definition of display maps involves a propositional truncation, we only have mere existence of pullbacks, and to pass from the statement ``for all $ f $, there exists a pullback $ \alpha^* f $'' to the statement ``there exists a function that sends every morphism $ f : (\R \downarrow_D A) $ to its pullback $ \alpha^* f : (\R \downarrow_D B) $'', we need to assume the axiom of choice.

  However, note that we have a weak equivalence between strict categories (categories in which the type of objects is a homotopy set) $ \sum_A: \R^A \xrightarrow \sim (\R \downarrow_D A) $. If we assume the axiom of choice, $ \sum_A $ becomes an equivalence of categories \autocite[][Chapter 9, introduction, bullet (ii)]{hottbook} and then our pullback functor is given by
  \[ \sum_A^{-1} \bullet (\alpha \cdot -) \bullet \sum_B, \]
  where $ (\alpha \cdot -): \R^A \to \R^B $ denotes the functor given by precomposition with $ \alpha $.
\end{remark}

This brings us to the main theorem of this section \autocite[][\S 5.1.8]{taylor}.
\begin{theorem}\label{thm:Taylor}
  If we assume the axiom of choice, $ \R $ is cartesian closed, relative to the given class of display maps.
\end{theorem}
\begin{proof}
  Let $ \alpha: B \to A $ be a display map. We have mere existence of some $ X: \R^A $ and an isomorphism $ g: (X, \alpha) \xrightarrow \sim (\sum B, p_X) $. We need to show that there is a right adjoint to the pullback functor $ \alpha^* $. Since under assumption of the axiom of choice, $ \alpha^* $ is defined as a composition of a precomposition functor with two equivalences of categories, this is equivalent to showing that there is a right adjoint to the precomposition functor $ (\alpha \cdot -): \R^A \to \R^B $ (note that $ \alpha \cdot - $ sends $ f $ to $ f \circ \alpha $).

  Now, intuitively, any $ Y: \R^B $ denotes an indexed family $ ((Y (a, b))_{b: X a})_{a: A} $. We want the right adjoint $ \alpha_* Y $ to be the indexed family of dependent products $ (\prod_{b: B_a} Y (a, b))_a $. An element $ f $ of the dependent product $ \alpha_* Y a $ is then a function from $ X a $, that satisfies $ f b : Y (a, b) $ for all $ b : X a $. Therefore, for all $ a : A $ and all $ f : L_0 $, we want $ (\alpha_* Y a) f = f $ to mean ``$ f \circ X a = f $ and, $ f b : Y (a, b) $ for all $ b: X a $''. We encode these two parts as
  \[ \lambda x_1, x_1 \circ X a \quad \text{and} \quad \lambda x_1 x_2, Y (g^{-1} (a, x_2)) (x_1 x_2). \]
  It turns out that these are idempotents and they commute, so we can compose them into one object
  \[ \alpha_* Y a = \lambda x_1 x_2, Y (g^{-1} (a, x_2)) (x_1 (X a x_2)) : \R \]
  and we define the combinator
  \[ \alpha_* = \lambda x_1 x_2 x_3 x_4, x_1 (g^{-1} (x_2, x_4)) (x_3 (X x_2 x_4)). \]
  Now, applying $ \alpha_* $ to both objects and morphisms in $ \R^A $ gives a functor $ \alpha_* : \R^B \to \R^A $.

  Note that for all $ Y : \R^B $ and all $ b $, we have a morphism $ (\lambda x_1, x_1 (\pi_2 (g b))) : \R(\alpha_* Y (\alpha b), Y b) $. Making this parametric in $ b $ gives us a counit
  \[ \epsilon_Y = \lambda x_1 x_2, Y x_1 (x_2 (\pi_2 (g x_1))) : \R^B((\alpha_* Y) \circ \alpha, Y). \]

  % First of all, $ \epsilon_Y \circ B = \epsilon_Y $, also
  % \begin{align*}
  %   (Y x_1 \circ \epsilon_Y x_1 \circ (\alpha_* Y \circ \alpha) x_1) x_2
  %   &= Y x_1 (\epsilon_Y x_1 (\alpha_* Y (\alpha x_1) x_2))\\
  %   &= Y x_1 ((\lambda f, Y x_1 (f (X (\alpha x_1) (\pi_2 (g x_1))))) (\lambda x_3, Y (g^{-1} (\alpha x_1, x_3)) (x_2 (X (\alpha x_1) x_3))))\\
  %   &= Y x_1 (((\lambda x_3, Y (g^{-1} (\alpha x_1, x_3)) (x_2 (X (\alpha x_1) x_3)))) (X (\alpha x_1) (\pi_2 (g x_1))))\\
  %   &= Y x_1 (Y (g^{-1} (\alpha x_1, (\pi_2 (g x_1)))) (x_2 (X (\alpha x_1) (X (\alpha x_1) (\pi_2 (g x_1))))))\\
  %   &= Y x_1 (Y x_1 (x_2 (X (\alpha x_1) (X (\alpha x_1) (\pi_2 (g x_1))))))\\
  %   &= Y x_1 (x_2 (X (\alpha x_1) (\pi_2 (g x_1))))\\
  %   &= \epsilon_Y x_1 x_2.
  % \end{align*}

  To show that $ (\alpha \cdot -) \dashv \alpha_* $, we need to show that for all $ Y : \R^B $, $ (\alpha_* Y, \epsilon_Y) $ is a universal arrow from $ (\alpha \cdot -) $ to $ Y $. Recall the following diagram:
  \begin{center}
    \begin{tikzcd}
      Z \circ \alpha \arrow[rd, "f"'] \arrow[r, "{\hat f \circ \alpha}", dashed] & (\alpha_* Y) \circ \alpha \arrow[d, "{\epsilon_Y}"]\\
      & Y
    \end{tikzcd}
  \end{center}
  Now, suppose that we have some $ Z : \R^A $ and some $ f: \R^B(Z \circ \alpha, Y) $. We define
  \[ \hat f = \lambda x_1 x_2 x_3, f(g^{-1}(x_1, x_3)) (Z x_1 x_2): \R^A(Z, \alpha_* Y) \]
  and we have $ ((\hat f) \circ \alpha) \cdot \epsilon_Y = f $.
  % \begin{align*}
  %   (((\alpha \cdot -) \hat f) \cdot \epsilon_Y) b z
  %   &= \epsilon_Y b ((\hat f (\alpha b)) z)\\
  %   &= \epsilon_Y b (\lambda x_3, f (g^{-1} ((\alpha b), x_3)) (Z (\alpha b) z))\\
  %   &= Y b ((\lambda x_3, f (g^{-1} ((\alpha b), x_3)) (Z (\alpha b) z)) (\pi_2 (g b)))\\
  %   &= Y b (f (g^{-1} ((\alpha b), \pi_2 (g b))) (Z (\alpha b) z))\\
  %   &= Y b (f (g^{-1} (g b)) (Z (\alpha b) z))\\
  %   &= f b z.
  % \end{align*}
  Now, suppose that we have some (other) $ \hat f^\prime : \R^A(Z, \alpha_*(Y)) $ such that $ (\hat f^\prime \circ \alpha) \cdot \epsilon_Y = f $. Then substituting $ (\hat f^\prime \circ \alpha) \cdot \epsilon_Y $ for $ f $ in $ \hat f $ yields
  \[ \hat f = \lambda x_1, (\alpha_* Y x_1) \circ (\hat f^\prime x_1) = \hat f^\prime, \]
  so $ \hat f $ is unique and $ (\alpha_* Y, \epsilon_Y) $ is a universal arrow, which concludes the proof that $ (\alpha \cdot -) \dashv \alpha_* $ and we see that $ \R $ is cartesian closed relative to the given class of display maps.
  % \begin{align*}
  %   \hat f
  %   &= \lambda x_1 x_2 x_3, f (g^{-1} (x_1, x_3))(Z x_1 x_2)\\
  %   &= \lambda x_1 x_2 x_3, ((\epsilon_Y (g^{-1} (x_1, x_3))) \circ ((\hat f^\prime \circ \alpha) (g^{-1} (x_1, x_3)))) (Z x_1 x_2)\\
  %   &= \lambda x_1 x_2 x_3, (\epsilon_Y (g^{-1} (x_1, x_3))) (\hat f^\prime (\alpha (g^{-1} (x_1, x_3))) (Z x_1 x_2))\\
  %   &= \lambda x_1 x_2 x_3, (\epsilon_Y (g^{-1} (x_1, x_3))) (\hat f^\prime x_1 x_2)\\
  %   &= \lambda x_1 x_2 x_3, Y (g^{-1} (x_1, x_3)) (\hat f^\prime x_1 x_2 (\pi_2 (g (g^{-1} (x_1, x_3)))))\\
  %   &= \lambda x_1 x_2 x_3, Y (g^{-1} (x_1, x_3)) (\hat f^\prime x_1 x_2 (X x_1 x_3))\\
  %   &= \lambda x_1 x_2, \alpha_* Y x_1 (\hat f^\prime x_1 (Z x_1 x_2))\\
  %   &= \hat f^\prime
  % \end{align*}
\end{proof}

\begin{remark}
  Recall that in the definition of $ \alpha_* $, we composed commuting idempotents
  \[ \lambda x_1 x_2, x_1 (X a x_2) \quad \text{and} \quad \lambda x_1 x_2, Y (a, x_2) (x_1 x_2). \]
  In his PhD thesis, Taylor instead composes the idempotents
  \[ \lambda x_1, x_1 (X a x_2) \quad \text{and} \quad \lambda x_1 x_2, Y x_2 (x_1 x_2). \]
  Note that in the first term, $ x_2 $ plays the role of an element of $ X a $, whereas in the second term, it plays the role of an element of $ B $. Of course, if we worked in $ \SET $, we would indeed have $ X a \subseteq \bigsqcup_{a : A} X_a \cong B $. However, in $ \R $, the `terms' of $ B \cong \sum_A X $ look like pairs $ (a, x) $ with $ a : A $ and $ x : X a $, so we cannot consider terms of $ X a $ to be terms of $ B $.

  Therefore, the idempotents that he uses do not commute, and the resulting term
  \[ \lambda x_1 x_2, Y (B x_2) (x_1 (X a x_2)) \]
  (notice the redundant usage of $ B $) is not idempotent.
\end{remark}

\section{The \texorpdfstring{$ \lambda $}{lambda}-calculus is Algebraic}\label{sec:selinger}
As we mentioned in Remark \ref{rem:lambda-theory-or-algebra}, using tools from universal algebra, there are two approaches to study $ \lambda $-calculus-like structures. One can either study $ \lambda $-theories or algebras for the algebraic theory $ \Lambda $. We will see that Hyland shows that these are equivalent.

In his proof, he refers to the 2002 paper `The lambda calculus is algebraic' by Peter Selinger \autocite{selinger-lambda-calculus-algebraic}. We will not study this paper in detail, but it is interesting to have a quick look at it, because it seems that Hyland took some inspiration from it.

Selinger studies two kinds of objects which he calls $ \lambda $-algebras and $ \lambda $-theories:
\begin{definition}
  A \textit{combinatory algebra} is a set $ A $, together with a binary (application) operation $ (a_1, a_2) \mapsto a_1 a_2 $, and distinguished elements $ k, s : A $ such that for all $ a_1, a_2, a_3 : A $,
  \[ k a_1 a_2 = a_1 \quad \text{and} \quad s a_1 a_2 a_3 = a_1 a_3 (a_2 a_3). \]
  A \textit{$ \lambda $-algebra} is a combinatory algebra which `behaves nicely' in some sense\footnote{We can interpret terms of combinatory algebra (abstract terms involving just application, the constants $ S $ and $ K $, constants from $ A $ and some variables) as functions on $ A $. We can also interpret them as $ \lambda $-terms with constants in $ A $, by sending $ S $ and $ K $ to the $ S $ and $ K $ combinators. Then `behaving nicely' means that if the $ \lambda $-terms of two such terms are equal, then their functions on $ A $ are equal as well.}.
\end{definition}

For a set $ C $ and a countable set $ V $, we define $ \Lambda_C $ to be the inductive type with constructors
\[ \mathtt{Var}(x), \quad \mathtt{App}(f, g), \quad \mathtt{Abs}(x, f) \quad \text{and} \quad \mathtt{Con}(c) \]
for $ x : V $, $ f, g : \Lambda_C $ and $ c : C $.
\begin{definition}
  A \textit{$ \lambda $-theory} is an equivalence relation `$ = $' on $ \Lambda_C $ for some $ C $, containing $ \alpha $- and $ \beta $-equality and closed under application and abstraction.
\end{definition}

Selinger shows in his paper that the categories of $ \lambda $-algebras and $ \lambda $-theories are equivalent. As we will see in the next chapter, Hyland shows that his own notions of $ \Lambda $-algebras and $ \lambda $-theories are equivalent.

Now, it is not hard to show that Hyland's and Selinger's notions of $ \lambda $-theory are equivalent: If we have one of Selinger's $ \lambda $-theories with a countable set of variables, enumerate the variables as $ x_1, x_2, \dots $ and let $ L_n $ be the set of terms with only the free variables $ x_1, \dots, x_n $. Then, if we quotient by the equivalence relation, we get one of Hyland's $ \lambda $-theories. Conversely, if we start with one of Hyland's $ \lambda $-theories $ L $, we can consider $ L_n $ as a subset of $ L_{n + 1} $ by sending $ x_{n, i} $ to $ x_{n + 1, i} $. If we take the union $ \bar L = \bigcup_n L_n $ (officially, this is a directed limit), we have a function $ f: \Lambda_{\bar L} \to \bar L $, because the constants of $ \Lambda_{\bar L} $ are exactly the terms of $ \bar L $ and because $ \lambda $-theories in Hyland's sense support the operations of the $ \lambda $-calculus (see Subsection \ref{subsec:lambda-calculus-operations}). This gives an equivalence relation on $ \Lambda_{\bar L} $ by taking $ a\ `='\ b $ if $ f(a) = f(b) $.

Note, however, that is very hard to show that Hyland's notion of $ \Lambda $-algebra and Selinger's notion of $ \lambda $-algebra are equivalent. If we start with a $ \Lambda $-algebra $ A $, this is already a combinatory algebra, because $ \Lambda $ contains the $ S $ and $ K $ combinators and their image are the designated elements $ s $ and $ k $. However, if we start with a $ \lambda $-algebra and want to turn this into a $ \Lambda $-algebra, we need to give a $ t $-action for every $ t : \Lambda $, not just for $ S $ and $ K $. Therefore, we need to show that any $ \Lambda $-term can be expressed in terms of $ K $, $ S $ and application. However, at that point we are already halfway through a proof that any $ \Lambda $-algebra can be given a $ \lambda $-theory structure.
