\documentclass{amsbook}

\usepackage{hyperref}
\usepackage{tikz-cd}

\newcommand{\Catc}[1]{\mathcal{#1}}
\newcommand{\BB}{\Catc{B}}
\newcommand{\CC}{\Catc{C}}
\newcommand{\DD}{\Catc{D}}
\newcommand{\EE}{\Catc{E}}
\newcommand{\Catb}[1]{\mathbf{#1}}
\newcommand{\SET}{\Catb{Set}}
\newcommand{\TYPE}{\Catb{Type}}
\newcommand{\FIN}{\Catb{Fin}}
\newcommand{\Ob}[1]{{#1}_0}
\newcommand{\Hom}[3]{{#1}(#2,#3)}
\newcommand{\op}[1]{\ensuremath{{#1}^\text{op}}}

\title{Summary of the things that I learned}
\author{Arnoud van der Leer}

\begin{document}
  \maketitle

  \chapter{Lessons about coq and unimath}
  When writing coq code, make sure you understand why a proof should work, instead of blindly unfolding and applying Lemmas. That improves the overall quality of the proofs.

  A proof closed with \texttt{Qed} is opaque, whereas a proof that closes with \texttt{Defined} is transparent (i.e. is remembered can be unfolded). Which one is the right one requires some thought.

  Only use \texttt{destruct} in opaque proofs.

  Path induction (or induction on proofs of equality) helps a lot when proving something about a \texttt{transportf}.

  \section{Default API for any \texttt{object}}
  \begin{itemize}
    \item \texttt{object\_data}: A definition for the data of the object;
    \item \texttt{make\_object\_data}: A function to make the object data out of its parts;
    \item (Coercions from \texttt{object\_data} to some of its parts;)
    \item (Explicit functions to access the constituents;)
    \item \texttt{is\_object}: A definition for the properties of the object;
    \item \texttt{make\_is\_object}: A function to make the properties part of the object;
    \item \texttt{make\_object}: A function to make the object out of its data and property components;
    \item \texttt{object}:
    \item (A coercion from the object to its data;)
    \item (\texttt{object\_has\_property}: Explicit functions to access the properties;)
    \item \texttt{is\_object\_isaprop}: A lemma that the properties form a proposition;
    \item \texttt{object\_eq}: A lemma about sufficient (and necessary) conditions for two terms of type \texttt{object} to be equal (usually some conditions on the constituents of \texttt{object\_data});
  \end{itemize}

  \section{Cleaning up code}
  It is okay to simplify proofs as much as possible. When one wants to step through a proof, they can add \texttt{simpl}s to their own liking.

  \texttt{cbn} is stronger than \texttt{simpl}, so it can be good to try and replace \texttt{cbn}s with \texttt{simpl}s when cleaning up code.

  \section{Things I was not able to find on my own}
  \texttt{isweq\_iso}

  \texttt{weqdnicoprod}

  \section{Things I would have liked to know at the start}
  The fact that if a rewrite does not succeed because a lemma has a slightly different (folded or unfolded) description, we can just force its form like \texttt{rewrite (lemma : form1 = form2)}.

  Setting just one implicit argument like \texttt{action (n := 5)}.

  The fact that instead of finishing a (sub)proof with \texttt{[tactic]. apply idpath.}, one can finish it with \texttt{now [tactic].}.

  \section{Things that I would like to have a standard for}
  \begin{itemize}
    \item Naming (of folders, files, terms): what abbreviations are allowed? In what order do words appear? How are words separated? How are names capitalized?
    \item Documentation headers in files.
    \item In which cases do we want to use unicode?
    \item In which cases do we want definitions to be (fully) typed?
    \item In which cases do we want accessor functions for all data and properties?
  \end{itemize}

  No lines should have trailing whitespace and all files should have a trailing newline.

  \chapter{Week 08}
  \section{Univalent Categories}
  A univalent category is a category in which the univalence axiom holds. I.e., a category $ \mathcal C $ in which, for all $ A, B \in \Ob \CC $, the canonical map $ (A =_{\mathcal C} B) \to (A \cong B) $ is an equivalence.

  \section{Categories}
  An \textbf{$ n $-category} is a category with $ 0 $-cells (objects), $ 1 $-cells (morphisms), $ 2 $-cells (morphisms between morphisms), up to $ n $-cells and various compositions: $ A \to B \to C $. $ A \xrightarrow{f, g, h} B $, $ f \Rightarrow g \Rightarrow h $. $ A \xrightarrow{f, g} B \xrightarrow{f^\prime, g^\prime} C $, $ \alpha: f \Rightarrow g $, and identities $ \alpha^\prime: f^\prime \Rightarrow g^\prime $ gives $ \alpha^\prime * \alpha: f^\prime \circ f \Rightarrow g^\prime \circ g $. These all need to work together `nicely'. An $ \omega $-category is the same, but all the way up.

  A topological space gives a (weak) $ \omega $-category. $ 0 $-cells are points, $ 1 $-cells are paths, $ 2 $-cells are homotopies etc. Composition is by glueing. It is a `groupoid', in the sense that all homotopies of dimension $ \geq 1 $ are invertible. However, glueing is not associative, so it is a `weak' $ \omega $-category.

  A category with only one object $ \star $ is equivalent to a monoid (with elements being the set $ \Hom{\CC}{\star}{\star} $). A $ 2 $-category with only one $ 0 $-cell is the same thing as a monoidal category (objects: the $ 1 $-cells. Morphisms: the $ 2 $-cells). A monoidal category with just one object gives $ 2 $ monoid structures on its set of morphisms. These are the same, and commutative.

  A \textbf{monoid} is a set with a multiplication and a unit. A \textbf{monad} on a category $ \CC $ is a functor $ T: \CC \to \CC $, together with natural transformations $ \mu : T \circ T \to T $ (satisfying associativity) and $ \eta : 1_{\mathcal A} \to T $ (acting as a two-sided unit).

  A \textbf{presheaf} on a category $ \mathcal A $ is a functor $ \mathcal A^{\mathop{opp}} \to \mathbf{Set} $.

  Given a category $ \mathcal E $ and an object $ E \in \Ob \EE $, the \textbf{slice category} $ \mathcal E / E $ with objects being the maps $ D \xrightarrow{p} E $ and morphisms being commutative triangles.

  A \textbf{multicategory}, not necessarily the same as an $ n $-category, is a category in which arrows go from multiple objects to one, instead of from one object to one. I.e. it is a category with a class $ C_0 $ of objects, for all $ n $, and all $ a, a_1, \dots, a_n \in C_0 $, a class $ C(a_1, \dots, a_n; a) $ of `morphisms', and a composition
  \[ C(a_1, \dots, a_n; a) \times C(a_{1, 1}, \dots, a_{1, k_1}; a_1) \times \dots \times C(a_{n, 1}, \dots, a_{n, k_n}; a_n) \to C(a_{1, 1}, \dots, a_{n, k_n}; a), \]
  written $ (\theta, \theta_1, \dots, \theta_n) \mapsto \theta (\theta_1, \dots, \theta_n) $ and for each $ a \in C_0 $ an identity $ 1_a \in C(a; a) $. It must satisfy associativity
  \[ \theta \circ(\theta_1 \circ(\theta_{1, 1}, \dots, \theta_{1, k_1}), \dots, \theta_n \circ (\theta_{n, 1}, \dots, \theta_{n, k_n})) = (\theta \circ (\theta_1, \dots, \theta_n)) \circ (\theta_{1, 1}, \dots, \theta_{n, k_n}) \]
  and identity
  \[ \theta \circ (1_{a_1}, \dots, 1_{a_n}) = \theta = 1_a \circ \theta. \]

  A \textbf{map of multicategories} is a function $ f_0: C_0 \to C_0^\prime $ and maps $ C(a_1, \dots, a_n; a) \to C(f_0(a_1), \dots, f_0(a_n); f_0(a)) $, preserving composition and identities.

  For $ C $ a multicategory, a \textbf{$ C $-algebra} is a map from $ C $ into the multicategory $ \mathbf{Set} $ (with objects $ \SET_0 $ and maps $ \SET(a_1, \dots, a_n; a) = \SET(a_1 \times \dots \times a_n ; a) $). I.e., for each $ a \in C_0 $, a set $ X(a) $, and for each map $ \theta: a_1, \dots, a_n \to a $, a function $ X(\theta): X(a_1) \times X(a_n) \to X(a) $. An example is, for a multicategory $ C $, to take $ X(a) = C(; a) $ (maps from the empty sequence into $ a $).

  Of course, there is a concept of \textbf{free multicategory}: Given a set $ X $, and for all $ n \in N $, and $ x, x_1, \dots, x_n \in X $, a set $ X(x_1, \dots, x_n ; x) $, we get a multicategory $ X^\prime $ with $ X^\prime_0 = X_0 $, and $ X^\prime(x_1, \dots, x_n; x) $ given by formal compositions of elements of the $ X(y_1, \dots, y_m; y) $.

  A \textbf{bicategory} consists of a class $ \mathcal B_0 $ of $ 0 $-cells, or objects; For each $ A, B \in \mathcal B_0 $, a category $ \mathcal B(A, B) $ of $ 1 $-cells (objects) and $ 2 $-cells (morphisms); for each $ A, B, C \in \mathcal B_0 $, a functor $ \mathcal B(B, C) \times \mathcal B(A, B) \to \mathcal B(A, C) $ written $ (g, f) \mapsto g \circ f $ on $ 1 $-cells and $ (\delta, \gamma) \mapsto \delta * \gamma $ on $ 2 $-cells; For each $ A \in \mathcal B_0 $ an object $ 1_A \in \mathcal B(A, A) $; isomorphisms representing associativity and identity axioms (e.g. $ f \circ 1_A \cong f \in \mathcal B(A, B) $), natural in their arguments, satisfying pentagon and triangle axioms.

  The collection of categories $ \mathrm{Cat} $ forms a bicategory. In analogy, we define a monad in a bicategory to be an object $ A $, together with a $ 1 $-cell $ t: A \to A $ and $ 2 $-cells $ \mu: t \circ t \to t $ and $ \eta: 1_A \to t $ satisfying a couple of commutativity axioms (those of 1.1.3 in \cite{higher-operads-higher-categories}).

  \section{Operads}
  \subsection{Definitions}
  An \textbf{operad} is a multicategory with only one object. More explicitly, an operad has a set $ P(k) $ for every $ k \in \mathbb N $, whose elements can be thought of as $ k $-ary operations. It also has, for all $ n, k_1, \dots, k_n \in \mathbb N $, a \textit{composition} function
  \[ P(n) \times P(k_1) \times \dots \times P(k_n) \to P(k_1 + \dots + k_n) \]
  and an element $ 1 = 1_P \in P(1) $ called the \textbf{identity}, satisfying
  \[ \theta \circ (1, 1, \dots, 1) = \theta = \theta \circ 1 \]
  for all $ \theta $, and
  \[ \theta \circ(\theta_1 \circ(\theta_{1, 1}, \dots, \theta_{1, k_1}), \dots, \theta_n \circ (\theta_{n, 1}, \dots, \theta_{n, k_n})) = (\theta \circ (\theta_1, \dots, \theta_n)) \circ (\theta_{1, 1}, \dots, \theta_{n, k_n}) \]
  for all $ \theta \in P(n) $, $ \theta_1 \in P(k_1) $, \dots, $ \theta_n \in P(k_n) $ and all $ \theta_{1, 1} \dots \theta_{n, k_n} $.

  A \textbf{morphism of operads} is a family
  \[ f_n : (P(n) \to Q(n))_{n \in \mathbb N} \]
  of functions, preserving composition and identities.

  A \textbf{$ P $-algebra} for $ P $ an operad, is a set $ X $ and, for each $ n $, and $ \theta \in P(n) $, a function $ \overline{\theta}: X^n \to X $, satisfying the evident axioms (identity is the identity function, the function of a composition is the composition of the functions?).

  \subsection{Examples}
  For any vector space $ V $, there is an operad with $ P(k) = V^{\otimes k} \to V $.

  The terminal operad $ 1 $ has $ P(n) = \{ \star_1 \} $ for all $ n $. An algebra for $ 1 $ is a set $ X $ together with a function $ X^n \to X $, denoted as $ (x_1, \dots, x_n) \mapsto (x_1 \cdot \dots \cdot x_n) $, satisfying
  \[ ((x_{1, 1} \cdot \dots x_{1, k_1}) \cdot \dots \cdot (x_{n, 1} \cdot \dots \cdot x_{n, k_n})) = (x_{1, 1} \cdot \dots \cdot x_{n, k_n}) \]
  and
  \[ x = (x). \]
  The category of $ 1 $-algebras is the category of monoids.

  There exist various sub-operads of $ 1 $. For example, the smallest one has $ P(1) = \{ \star \} $ and $ P(n) = \emptyset $ for $ n \not = 1 $.

  Or the operad with $ P(0) = \emptyset $ and $ P(n) = \{ \star_n \} $ for $ n > 0 $, which has semigroups as its algebras (sets with associative binary operations).

  The suboperad with $ P(n) = \{ \star_n \} $ exactly when $ n \leq 1 $ has as its algebras the pointed sets.

  The \textbf{operad of curves} has $ P(n) = \{ \text{smooth maps}\ \mathbb R \to \mathbb R^n \} $.

  Given a monad on $ \SET $, we get a natural operad structure $ T(n)_{n \in \mathbb N} $, with $ T(n) $ the set of words in $ n $ variables and composition given by `substitution'.

  Given a monoid $ M $ (a category with one object), there is a operad given by $ P(n) = M^n $ and composition
  \[ (\alpha_1, \dots, \alpha_n) \circ ((\alpha_{1, 1}, \dots, \alpha_{1, k_1}), \dots, (\alpha_{n, 1}, \dots, \alpha_{n, k_n})). \]

  The \textbf{Little $ 2 $-disks} operad $ D $ has
  \[ D(n) = \{ \text{set of non-overlapping disks contained within the unit disk} \}, \]
  with composition being geometric "substitution". I.e.: scale and move a unit disk and its contained disks to match one of the smaller disks, and replace the smaller disk with the transformed contents of our original unit disk. See also: \href{https://upload.wikimedia.org/wikipedia/en/thumb/0/0b/Composition_in_the_little_discs_operad.svg/1920px-Composition_in_the_little_discs_operad.svg.png}{this image that explains a lot}

  Given sets $ X(n) $ for all $ n \in \mathbb N $, the \textbf{free operad} $ X^\prime $ on these is defined exactly by $ X(n) \subseteq X^\prime(n) $, $ 1 \in X^\prime(1) $ and for all $ m, n_1, \dots, n_m \in \mathbb N $ and $ f \in X(m) $ and $ f_i \in X^\prime(n_i) $, we have $ f \circ (f_1, \dots, f_m) \in X^\prime(n_1 + \dots + n_m) $.

  \section{\texorpdfstring{$ T $}{T}-operads}
  \subsection{Definitions}
  A category is \textbf{cartesian} if it has all pullbacks. A functor is cartesian if it preserves pullbacks. A natural transformation $ \alpha: S \to T $ is cartesian if for all $ f: A \to B $, the naturality diagram
  \begin{center}
    \begin{tikzcd}
      SA \arrow[r, "S f"] \arrow[d, "\alpha_A"] & SB \arrow[d, "\alpha_B"]\\
      TA \arrow[r, "T f"] & TB
    \end{tikzcd}
  \end{center}
  is a pullback. A monad $ (T, \mu, \eta) $ on a category $ \mathcal E $ is cartesian if the category $ \mathcal E $, the functor $ T $ and the natural transformations $ \mu $ and $ \eta $ are cartesian.

  We can represent (the morphism structure of) an ordinary category using diagrams $ C_0 \xleftarrow{\text{domain}} C_1 \xrightarrow{\text{codomain}} C_0 $, $ C_1 \times_{C_0} C_1 \xrightarrow{\text{composition}} C_1 $ and $ C_0 \xrightarrow{\text{id}} C_1 $ together with some axioms. For a multicategory, we need to slightly modify this, using a functor $ T: \SET \to \SET $, $ A \mapsto \bigsqcup A^n $, to $ T C_0 \xleftarrow{d} C_1 \xrightarrow{c} C_0 $ and $ C_1 \times_{T C_0} T C_1 \xrightarrow{\circ} C_1 $.

  Given a cartesian monad $ (T, \mu, \eta) $ on a category $ \mathcal E $, we can define a bicategory $ \mathcal E_{(T)} $, with the class of $ 0 $-cells being $ \mathcal E_0 $, the $ 1 $-cells $ E \to E^\prime $ being diagrams $ TE \xleftarrow{d} M \xrightarrow{c} E^\prime $, $ 2 $-cells $ (M, d, c) \to (N, q, p) $ are maps $ M \to N $ such that the diagram with $ E, E^\prime, M, N $ commutes. The composite of $ 1 $-cells $ TE \xleftarrow{d} M \xrightarrow{c} E^\prime $ and $ TE \xleftarrow{d^\prime} M^\prime \xrightarrow{c^\prime} E^{\prime\prime} $ is given by
  \[ T E \xleftarrow{\mu_E} T^2 E \xleftarrow{T d} TM \xleftarrow{} T M \times_{T E^\prime} M^\prime \xrightarrow{} M^\prime \xrightarrow{c^\prime} E^{\prime\prime} \]
  in which the coproduct in the middle is formed using $ T c $ and $ d $.
  We can define a $ T $-multicategory to be a monad on $ \mathcal E_{(T)} $. Equivalently, we can define it as an object $ C_0 \in \mathcal E $, together with a diagram $ t: T C_0 \xleftarrow{d} C_1 \xrightarrow{c} C_0 $ and maps $ C_1 \circ C_1 := T C_1 \times_{T C_0} C_1 \xrightarrow{\circ} C_1 $ and $ C_0 \xrightarrow{id} C_1 $ satisfying associativity and identity axioms.

  A $ T $-operad is a $ T $-multicategory such that $ C_0 $ is the terminal object of $ \mathcal E $. Equivalently, it is an object over $ T 1 $, (so we have a morphism $ P \to T 1 $), together with maps $ P \times_{T 1} T P \to P $ and $ 1 \xrightarrow{id} P $, both over $ T 1 $, satisfying associativity and identity axioms.

  \subsection{Examples}
  For $ T $ the identity monad on $ \SET $, a $ T $-operad is exactly a monoid (or an operad with only unary functions) (since there is always a unique map to $ \{ 1 \} $).

  If $ \mathcal E $ is $ \SET $, the terminal object $ 1 $ will always be $ \{ 1 \} $.

  For the free monoid monad $ T A = \bigsqcup A^n $, the $ T $-operads are precisely the operads that we defined before.

  For the monad $ T A = 1 + A $, we can view $ T A $ as a subset of the free monoid on $ A $, and this gives an operad with $ 0 $-ary and $ 1 $-ary functions. The $ 1 $-ary arrows form a monoid, and the $ 0 $-ary arrows are a set, with an action of the monoid.

  \section{Cartesian Operads}
  \subsection{Theory}
  Using \href{https://ncatlab.org/toddtrimble/published/Towards+a+doctrine+of+operads}{Towards a doctrine of operads}.

  NLab uses notation: $ \FIN $ for what we would call a standard skeleton of finite sets (i.e. the category of finite sets $ \{ 0, \dots, n - 1 \} $ and maps between them). $ A^B $ denotes all morphisms/functors $ B \to A $. I.e., the class of functors $ \FIN \to \SET $ is denoted $ \SET^\FIN $.

  Take $ I = \Hom{\FIN}{1}{-}: \SET^\FIN = \FIN \to \SET $.

  Let $ [\SET^\FIN, \SET^\FIN] $ be the category of finite-product-preserving, cocontinuous endofunctors on $ \SET^\FIN $. The map $ \mathop{Ev}_I: [\SET^\FIN, \SET^\FIN] \to \SET^\FIN $, given by $ F \mapsto F(I) $ is an equivalence. $ [\SET^\FIN, \SET^\FIN] $ has a monoidal product $ \odot $ given by endofunctor composition, and we can transfer this to $ \SET^\FIN $.

  Concretely, we have $ F \odot G = \int^{n \in \FIN} F(n) G^n $.

  \subsection{Cartesian operads}
  A \textbf{cartesian operad} is a monoid in this monoidal category $ (\SET^{\FIN}, \odot, I) $. I.e., it is a triple $ (M, \mu, \eta) $, with $ M \in \SET^\FIN $, $ \mu: M \odot M \to M $ and $ \eta: I \to M $.

  More concretely, this is a functor $ M : \FIN \to \SET $, together with maps
  \[ m_{n, k}: M(n) \times M(k)^n \to M(k), \]
  natural in $ k $ and dinatural in $ n $, and an element $ e \in M(1) $.

  Dinaturality in $ n $ means the following. Fix $ k \in \FIN $. We have the functors $ \op \FIN \times \FIN \to \SET $ given by
  \[ F: (n, n^\prime) \mapsto M(n^\prime) \times M(k)^{n} \quad \text{and} \quad G: (n, n^\prime) \mapsto M(k). \]
  For all $ n \in \Ob \FIN $, we have a morphism
  \[ \bullet: F(n, n) = M(n) \times M(k)^n \to M(k) = G(n, n). \]
  Naturality means that for all $ a: n \to n^\prime $,
  \[ G(n, a) \circ \bullet \circ F(a, n) = F(a, n^\prime) \circ \bullet \circ G(n^\prime, a). \]
  i.e., for all $ f \in M(n) $, $ g_1, \dots, g_{n^\prime} \in M(k) $,
  \[ f \bullet (g_{a(1)}, \dots, g_{a(n)}) = M(a)(f) \bullet (g_1, \dots, g_{n^\prime}). \]

  Now, if we have, in $ \FIN $, a decomposition $ k = k_1 + \dots + k_n $, and we have inclusion maps $ i_j: k_j \hookrightarrow k $, then we have
  \[ M(n) \times M(k_1) \times \dots \times M(k_n) \xrightarrow{1 \times M(i_1) \times \dots \times M(i_n)} M(n) \times M(k)^n \xrightarrow{m_{n, k}} M(k), \]
  which gives an operad structure.

  \subsection{Clones}
  In other parts of mathematics, a cartesian operad is called a \textbf{clone}. An abstract clone consists of sets $ M(n) $ for all $ n \in \mathbb N $, for all $ n, k \in \mathbb N $ a function $ \bullet: M(n) \times M(k)^n \to M(k) $ and for each $ 1 \leq i \leq n \in \mathbb N $, an element $ \pi_{i, n} \in M(n) $ such that for $ f \in M(i) $, $ g_1, \dots, g_i \in M(j) $ and $ h_1, \dots, h_j \in M(k) $,
  \[ f \bullet (g_1 \bullet (h_1, \dots, h_j), \dots, g_i \bullet (h_1 \dots, h_j)) = (f \bullet (g_1, \dots, g_i)) \bullet (h_1, \dots, h_j), \]
  for $ f_1, \dots, f_n \in M(k) $,
  \[ \pi_{i, n} \bullet (f_1, \dots, f_n) = f_i, \]
  and for $ f \in M(n) $,
  \[ f \bullet (\pi_{1, n}, \dots, \pi_{n, n}) = f. \]
  (It is claimed that this automatically gives naturality)

  Naturality means for all $ a \in \Hom{\FIN}{n}{n^\prime} $, for all $ f \in M(n) $, $ g_1, \dots, g_{n^\prime} \in M(k) $,
  \[ f \bullet (g_{a(1)}, \dots, g_{a(n)}) = (f \bullet (\pi_{a(1), n^\prime}, \dots, \pi_{a(n), n^\prime})) \bullet (g_1, \dots, g_{n^\prime}). \]
  However, by associativity and since $ \pi_{i, n} \bullet (f_1, \dots, f_n) $, we have
  \begin{align*}
    & (f \bullet (\pi_{a(1), n^\prime}, \dots, \pi_{a(n), n^\prime})) \bullet (g_1, \dots, g_{n^\prime})\\
    &= (f \bullet (\pi_{a(1), n^\prime} \bullet (g_1, \dots, g_{n^\prime}), \dots, \pi_{a(n), n^\prime} \bullet (g_1, \dots, g_{n^\prime})))\\
    &= f \bullet (g_{a(1)}, \dots, g_{a(n)}).
  \end{align*}

  Naturality in $ k $ is as follows. Fix $ n \in \Ob \FIN $. We have functors $ F, G: \FIN \to \SET $, given by
  \[ F: k \mapsto M(n) \times M(k)^n \quad \text{and} \quad G: k \mapsto M(k). \]
  For $ a: k \to k^\prime $, we must have
  \[ G(a) \circ \bullet = \bullet \circ F(a). \]
  That is, for all $ f \in M(n) $, $ g_1, \dots, g_n \in M(k) $,
  \[ M(a) (f \bullet (g_1, \dots, g_n)) = f \bullet (M(a)(g_1), \dots, M(a)(g_n)). \]
  Now, $ M(a) $ is given by
  \[ M(a)(f) = f \bullet (\pi_{a(1), k^\prime}, \dots, \pi_{a(k), k^\prime}). \]
  Therefore, we have
  \begin{align*}
    & M(a) (f \bullet (g_1, \dots, g_n))\\
    &= (f \bullet (g_1, \dots, g_n)) \bullet (\pi_{a(1), k^\prime}, \dots, \pi_{a(k), k^\prime})\\
    &= f \bullet (g_1 \bullet (\pi_{a(1), k^\prime}, \dots, \pi_{a(k), k^\prime}), \dots, g_n \bullet (\pi_{a(1), k^\prime}, \dots, \pi_{a(k), k^\prime}))\\
    &= f \bullet (M(a)(g_1), \dots, M(a)(g_n)).
  \end{align*}
  So the associativity and projection axioms ensure naturality.

  \chapter{Week 09}

  \section{Adjunctions}
  For three sets $ X, Y, Z $, we have a bijection
  \[ \Hom \SET {X \times Y} Z \cong \Hom \SET X {Y \to Z} \]
  that is natural in $ X $ and $ Z $. This is an example of a general notion:

  For functors $ F: \CC \to \DD $ and $ G: \DD \to \CC $, $ F $ is a \textbf{left adjoint} for $ G $ if there is an isomorphism $ \Hom \CC {F X} Y \cong \Hom \DD X {G Y} $ that is dinatural in $ X $ and $ Y $.

  \subsection{Examples}
  A special case is for $ G: \CC \to \SET $ the forgetful functor: if we have a left adjoint $ F: \SET \to \CC $ to $ G $, we call $ F(X) $ the \textbf{free $ \CC $} of $ X $ (for example: free group, free monoid, free category, etc.). That means that $ \Hom \CC {F(X)} Y \cong \Hom \SET X {G(Y)} $, or in other words:

  Adjunctions can be composed: If $ F $ and $ G $ both forget part of the structure of an object, then we can construct the free object `piecewise' using the composition of the left adjoints.

  Let $ G: \Catb{Group} \to \SET $ be the forgetful functor and $ F: \SET \to \Catb{Group} $ be the free group functor. For any set $ S $, we have an inclusion $ i: S \hookrightarrow F S $. Now, given any group $ Y $ and any morphism of sets from $ S $ to $ Y $ (formally: $ f \in \Hom \SET S {G Y} $). Then the fact that $ \Hom \SET {FS} Y \cong \Hom \SET S {G Y} $ is expressed in the fact that $ f $ factors uniquely as $ g \circ i $.
  \begin{center}
    \begin{tikzcd}
      S \arrow[r, "i"] \arrow[rd, "f"] & F(S) \arrow[d, "\exists! g"]\\
      & Y
    \end{tikzcd}
  \end{center}

  In topology, the forgetful functor from topological spaces to sets has a left adjoint that turns a set $ S $ into a space $ F(S) $ with the discrete topology, since it has the smallest topology on $ S $ that still can factor all morphisms $ S \to [0, 1] $ for example.

  In algebra, the inclusion functor from the category of monoids (sets with an associative operation and a unit) to the category of semigroups (sets with an associative operation) has a left adjoint that sends the semigroup $ S $ to the monoid $ S \sqcup \{ \star \} $ and gives it operations such that $ \star $ is the unit.

  In algebra, the inclusion functor from the category of rings to the category of rngs (rings but without an identity) has a left adjoint that sends the rng $ R $ to $ R \times \mathbb Z $ and defines $ (r, x) (s, y) = (rs + xs + ry, xy) $.

  For $ \Catb{Dom_m} $ the category of integral domains with injective morphisms. Then the forgetful functor $ \Catb{Fields} \to \Catb{Dom_m} $ has a left adjoint that sends a domain to its field of fractions.

  For $ \rho: R \to S $ a morphism of rings, we can view every $ S $-module as an $ R $-module: we have a functor $ S\Catb{-Mod} \to R\Catb{-Mod} $. This has a left adjoint, $ M \mapsto M \otimes_R S $.

  \section{Operads and cartesian operads}
  \subsection{Recap operads and cartesian operads}
  In this section, we will represent a cartesian operad $ C $ by a clone:
  \begin{itemize}
    \item For all $ n \in \mathbb N $, a set $ C(n) $;
    \item For all $ 1 \leq i \leq n $ projections $ \pi_{n, i} \in P(n) $;
    \item For all $ m, n $, a function $ \rho_{m, n}: C(m) \times C(n)^m \to C(n) $.
  \end{itemize}
  such that
  \begin{itemize}
    \item For $ f \in C(l) $, $ g_1, \dots, g_l \in C(m) $ and $ h_1, \dots, h_m \in C(n) $,
      \[ \rho(f, (\rho(g_1, (h_1, \dots, h_m)), \dots, \rho(g_l, (h_1, \dots, h_m)))) = \rho(\rho(f, (g_1, \dots, g_l)), (h_1, \dots, h_l)). \]
    \item for $ f_1, \dots, f_n \in M(n) $ and $ 1 \leq i \leq n $, $ \rho(\pi_{i, n}, (f_1, \dots, f_n)) = f_i $;
    \item For $ f \in M(n) $, $ \rho(f, (\pi_{1, n}, \dots, \pi_{n, n})) = f $.
  \end{itemize}
  and a morphism of cartesian operads $ \varphi: C \to C^\prime $ by componentwise functions $ \varphi_n: C(n) \to C^\prime(n) $ such that
  \begin{itemize}
    \item For all $ 1 \leq i \leq n $, $ \varphi_n(\pi_{n, i}) = \pi^\prime_{n, i} $;
    \item For all $ m, n $, $ f \in C(m) $, $ g_1, \dots, g_m \in C(n) $
      \[ \rho(\varphi(f), (\varphi(g_1), \dots, \varphi(g_m))) = \varphi(\rho(f, (g_1, \dots, g_m))). \]
  \end{itemize}

  We will represent an operad $ P $ by
  \begin{itemize}
    \item For all $ n \in \mathbb N $, a set $ P(n) $;
    \item An element $ e \in P(1) $;
    \item For all $ m, n_1, \dots, n_m \in \mathbb N $, a function
      \[ \gamma_{m, n_1, \dots, n_m}: C(m) \times C(n_1) \times \dots \times C(n_m) \to C(n_1 + \dots + n_m) \]
      (we will usually just call this $ \gamma $).
  \end{itemize}
  such that
  \begin{itemize}
    \item For all $ f \in P(n) $,
      \[ \gamma(f, (1, \dots, 1)) = f = \gamma(1, f); \]
    \item For all $ f \in P(l) $, $ g_1 \in P(m_1) $, \dots, $ g_l \in P(m_l) $ and all $ h_{1, 1} \in P(n_{1, 1}) $, \dots, $ h_{l, m_l} \in P(n_{l, m_l}) $,
      \[ f \circ(g_1 \circ(h_{1, 1}, \dots, h_{1, m_1}), \dots, g_l \circ (h_{l, 1}, \dots, h_{l, m_l})) = (f \circ (g_1, \dots, g_l)) \circ (h_{1, 1}, \dots, h_{l, m_l}). \]
  \end{itemize}
  and a morphism of operads $ \psi: P \to P^\prime $ by componentwise functions $ \psi_n: P(n) \to P^\prime(n) $ such that
  \begin{itemize}
    \item $ \psi_1(e) = e^\prime $;
    \item For all $ f \in P(m) $, $ g_1 \in P(n_1) $, \dots, $ g_m \in P(n_m) $,
      \[ \gamma(\psi(f), (\psi(g_1), \dots, \psi(g_m)))  \psi(\gamma(f, (g_1, \dots, g_m))). \]
  \end{itemize}

  \subsection{Free operad?}
  Now, for all $ 1 \leq i \leq m $, we have an injection $ j_i: C(n_i) \to C(n_1 + \dots + n_m) $. Intuitively, it maps terms in a context with $ n_i $ variables to terms in a context with $ n_1 + \dots + n_m $ variables, by mapping variable $ 1 \leq k \leq n_1 $ to variable $ n_1 + \dots + n_{i - 1} + k $. This gives a morphism
  \[ \gamma: C(m) \times C(n_1) \times \dots \times C(n_m) \xrightarrow{\mathop{id} \times j_1 \times \dots \times j_m} C(m) \times C(n_1 + \dots + n_m)^m \xrightarrow{\rho} C(n_1 + \dots + n_m), \]
  which gives an operad structure on the same sets. This gives a functor from cartesian operads to operads.

  Then, the question arises whether this functor has a left adjoint.

  \textbf{Still open}

  \section{Cartesian multicategory}
  Yet another way to think of a cartesian operad is as a one-object cartesian multicategory. A \textbf{cartesian multicategory} is a multicategory with an $ S_n $-action on the hom-sets (in other words: we can permute the arguments of the morphsims) and duplication/diagonal/contraction operations
  \[ \mathrm{Hom}(c_1, \dots, c_k, c_k, \dots, c_n; c) \to \mathrm{Hom}(c_1, \dots, c_k, \dots, c_n; c) \]
  and deletion/projection/weakening operations
  \[ \mathrm{Hom}(c_1, \dots, c_k, \dots, c_n; c) \to \mathrm{Hom}(c_1, \dots, c_{k-1}, c_{k+1}, \dots, c_n; c) \]

  \section{The paper}
  A summary of the results in the paper:
  \begin{enumerate}
    \item A definition of the category of algebraic theories (clones).
    \item A definition of the category of $ T $-algebras (for $ T $ an algebraic theory), the pullback of an algebra, the free algebra, and some properties of these.
    \item A definition of a presheaf (like an algebra, but with a right action instead of a left one) and some properties of this.
    \item A sidenote about the more classical approach to algebra.
    \item A definition for a $ \lambda $-theory and some properties of its constituents.
    \item A definition for an interpretation of the $ \lambda $-calculus in a $ \lambda $-theory, and some properties.
    \item The notion that the algebraic theory $ \Lambda $ of all terms of the $ \lambda $-calculus is the initial $ \lambda $-theory.
    \item The notion that we can add constants to a theory to make sure that it `has enough points'.
    \item A (new) proof that any $ \lambda $-theory is isomorphic to the endomorphism $ \lambda $-theory of some object.
    \item A section that concludes an equivalence between the presheaf categories of a Lawvere theory, a $ \lambda $-theory and the category of retracts.
    \item The definition of $ \Lambda $-algebra and a couple of its properties. In particular, a functor from $ \lambda $-theories to $ \Lambda $-algebras.
    \item An equivalene between $ \Lambda $-algebras and their presheaves.
    \item A characterisation of the function space $ U^U $ for $ U \in PA $ the universal.
    \item An equivalence of categories between the category of $ \Lambda $-algebras and $ \lambda $-theories.
    \item The Fundamental Theorem of the $ \lambda $-calculus: there is an adjoint equivalence between $ \lambda $-theories and $ \Lambda $-algebras.
    \item A remark that this is much harder without the category theoretic framework.
  \end{enumerate}

  \chapter{Week 11}

  \section{A Formalization of Operads in Coq}
  A paper was uploaded to the arXiv (see \cite{arxiv-operads}). Unfortunately, this paper did not contain any code, which makes it harder to compare their technique to mine. The paper is about the formalization of a symmetric multicategory (kind of. They do not require $ (\tau \sigma) f = \tau (\sigma f) $ for $ \sigma, \tau \in S_n $ and $ f \in \Hom{\CC}{a_1}{\dots, a_n;a} $)

  The operad data presented in this paper is a bit different (but equivalent) from the data that I have seen so far. Notably: the composition is on one element at a time:
  \[ \Hom{\CC}{a_1}{\dots, a_n;a} \times \Hom{\CC}{b_1}{\dots, b_m; a_i} \to \Hom{\CC}{a_1}{\dots a_{i-1}, b_1, \dots, b_m, a_{i+1}, \dots, a_n; a}, \]
  in which $ a_1, \dots a_{i-1}, b_1, \dots, b_m, a_{i+1}, \dots, a_n $ is denoted $ \underline{a} \bullet_{i} \underline{b} $.

  I expect the axioms to become a bit more complicated this way. For example, now there are two associativity axioms, and even to state them we need a proof that $ (\underline c \bullet_i \underline a) \bullet_{l - 1 + j} \underline b = (\underline c \bullet_j \underline b) \bullet_i \underline a $.

  Also, their signature of $ \ensuremath{Hom} $ is $ \TYPE \to \mathbf{List}\ \TYPE \to \TYPE $. The advantage of using $ \mathbf{List}\ \TYPE $ instead of $ \sum_{n \in \mathbb N} (\mathtt{stn}\ n \to \TYPE) $ is that it sounds simpler. The disadvantage is that, for example, to say something about $ c_i $, one needs a function to fetch the $ i $th element of $ \underline c $ and a proof that $ i $ is less than the length of $ \underline{c} $.

  To get elements in the right space, they add a lot of typecasting functions. For example, the function $ \mathcal C_{assoc}: \mathcal O((\underline c \bullet_i \underline a) \bullet_{l - 1 + j} \underline b; d) \to \mathcal O((\underline c \bullet_j \underline b) \bullet_i \underline a; d) $ I believe we call such a function a `transport function`. The paper claims that if $ A = B $, a typecast function will be the identity. The way that it is currently stated, it is not compatible with univalence, I believe.

  I feel like the paper lacks a section explaining the choices that were made and evaluating their results.

  \bibliographystyle{alpha}
  \bibliography{citations}

\end{document}
