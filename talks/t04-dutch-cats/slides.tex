\documentclass[aspectratio=169]{fancyslides} % or beamer/slides

\usepackage{tikz-cd}
\usepackage{datetime2}
\usepackage{stackengine}

\setlength{\parskip}{.5\baselineskip}
\setbeamertemplate{footline}[frame number]

\title{Semantics for the \texorpdfstring{$ \lambda $}{lambda}-calculus}
\author{Arnoud van der Leer\\Delft University of Technology\\\texttt{arnoudvanderleer@gmail.com}}


\begin{document}
  \maketitle

  \section{Intro}

  \begin{frame}
    \frametitle{Classical lambda calculus in modern dress}

    \begin{itemize}
      \item Paper by Martin Hyland.
      \item About models for the $ \lambda $-calculus.
      \item Three `big' theorems.
      \item My job: `annotate'.
    \end{itemize}
  \end{frame}

  \begin{frame}[fragile]
    \begin{columns}
      \begin{column}{.5\textwidth}
        \tableofcontents
      \end{column}
      \begin{column}{.5\textwidth}
        \begin{tikzcd}
          \Lambda-algebras \arrow[d, harpoon, shift left]           &                                                   \\
          \lambda-theories \arrow[d] \arrow[u, harpoon, shift left] & \stackanchor{Endomorphism}{theories} \arrow[l, two heads, hook] \\
          \stackanchor{Relatively\ Cartesian}{Closed\ Categories}   &
        \end{tikzcd}
      \end{column}
    \end{columns}
  \end{frame}

  \section{Talking about the \texorpdfstring{$\lambda$}{lambda}-calculus}

  \begin{frame}[fragile]
    \begin{columns}
      \begin{column}{.5\textwidth}
        \tableofcontents[currentsection]
      \end{column}
      \begin{column}{.5\textwidth}
        \begin{tikzcd}
          \Lambda-algebras \arrow[d, harpoon, shift left]           &                                                   \\
          \lambda-theories \arrow[d] \arrow[u, harpoon, shift left] & \stackanchor{Endomorphism}{theories} \arrow[l, two heads, hook] \\
          \stackanchor{Relatively\ Cartesian}{Closed\ Categories}   &
        \end{tikzcd}
      \end{column}
    \end{columns}
  \end{frame}

  \begin{frame}
    \frametitle{The \textbf{untyped} $ \lambda $-calculus}

    Describes a collection consisting of (only) functions.

    \pause

    Has terms, consisting of \textit{variables}, \textit{application} and \textit{abstraction}:
    \[ x_1 \]
    \[ x_1 (x_2 x_1) \]
    \[ \lambda x_1, x_1 \]
    \[ \lambda x_3 x_2 x_1, x_1 (x_2 x_3). \]

    Can have $ \beta $- and $ \eta $-equality:
    \[ (\lambda x_n, f) g = f [x_n := g] \qquad \lambda x_n, (f x_n) = f.\]

    \pause

    The \textit{(pure) $ \lambda $-calculus}: Described exactly by the above.
  \end{frame}

  \subsection{Models}
  \begin{frame}
    \frametitle{Algebraic theories: objects with variables and substitution}

    \begin{example}
      $ \lambda $-calculus: $ \Lambda_n = \{ (\lambda x_1, x_1), x_5, (\lambda x_3, x_7) x_{42} \} $.
    \end{example}

    \pause

    \begin{example}
      Polynomial ring: $ \mathbb Z[x_1, ..., x_n] = \{ 1, x_3, 2048 + 7 x_1^{37} - x_6 x_{13}^{42} x_{17}^{1729}, \dots \} $.
    \end{example}

    \pause

    \begin{definition}
      An \textit{algebraic theory} $ T $ is a sequence of sets $ T_n $ with variables $ x_{i, n} \in T_n $ (for $ 0 \leq i < n $) and a substitution operation $ \bullet : T_m \times T_n^m \to T_n $.
    \end{definition}
  \end{frame}

  \begin{frame}
    \frametitle{$ \lambda $-theory: structure with app and abs}

    \begin{definition}
      A \textit{$ \lambda $-theory} $ L $ is an algebraic theory, together with abstraction functions $ \lambda: L_{n + 1} \to L_n $ and application functions $ \rho: L_n \to L_{n + 1} $ (both compatible with the substitution).
    \end{definition}

    \pause

    The pure $ \lambda $-calculus $ \Lambda $ is the initial $ \lambda $-theory.

    \pause

    $ \beta $- and $ \eta $-equality:
    \[ \rho_n \circ \lambda_n = \mathrm{Id}_{L_{n + 1}} \qquad \lambda_n \circ \rho_n = \mathrm{Id}_{L_n}. \]
  \end{frame}

  \subsection{Semantics}
  \begin{frame}
    \frametitle{Algebras: Interpretations (or denotations)}
    We want to interpret terms with free variables as functions from a context to a set

    \begin{example}
      In $ T(n) = \mathbb Z[x_1, \dots, x_n] $, we can take a set $ A = \mathbb Q $ and get
      \[ 2 x_1 + 3 x_1^2 x_2: A^2 \to A, \quad (a_1, a_2) \mapsto 2 \cdot a_1 + 3 \cdot a_1^2 \cdot a_2. \]
    \end{example}

    \pause

    \begin{definition}
      For an algebraic theory $ T $, a \textit{$ T $-algebra} $ A $ is a set $ A $, together with interpretation functions $ T_n \times A^n \to A $ for all $ n $ (respecting the variables and substitution).
    \end{definition}
  \end{frame}

  \section{The main theorems}

  \subsection{Scott's representation theorem}

  \begin{frame}[fragile]
    \begin{columns}
      \begin{column}{.5\textwidth}
        \tableofcontents[currentsubsection]
      \end{column}
      \begin{column}{.5\textwidth}
        \begin{tikzcd}
          \Lambda-algebras \arrow[d, harpoon, shift left]           &                                                   \\
          \lambda-theories \arrow[d] \arrow[u, harpoon, shift left] & \stackanchor{Endomorphism}{theories} \arrow[l, two heads, hook] \\
          \stackanchor{Relatively\ Cartesian}{Closed\ Categories}   &
        \end{tikzcd}
      \end{column}
    \end{columns}
  \end{frame}

  \begin{frame}
    \frametitle{Scott's representation theorem (1980)}

    \textit{For every $ \lambda $-theory $ L $, we can find a category $ C $ and an object $ X : C_0 $, such that $ L $ is isomorphic to the \textit{endomorphism theory} of $ X $: the $ \lambda $-theory $ E(X) $ given by $ E(X)_n = X^n \to X $.}

    \pause

    The variables of $ E(X)_n $ are the projections $ \pi_i: X^n \to X $. Also, substituting $ g_1, \dots, g_m : X^n \to X $ into $ f: X^m \to X $ composes $ f $ with $ \langle g_1, \dots, g_m \rangle : X^n \to X^m $.

    \pause

    We obtain $ \lambda: E(X)_{n + 1} \to E(X)_n $ as
    \[ \lambda: E(X)_{n + 1} = (X^{n + 1} \to X) \simeq (X^n \to X^X) \xrightarrow{\overline{abs}\ \circ\ -} (X^n \to X) = E(X)_n. \]
    for some morphism $ \overline{abs}: X^X \to X $. In the same way, we get $ \rho: E(X)_n \to E(X)_{n + 1} $ from a morphism $ \overline{app}: X \to X^X $.
  \end{frame}

  \begin{frame}
    \frametitle{Scott's representation theorem (1980)}

    \textit{For every $ \lambda $-theory $ L $, we can find a category $ C $ and an object $ X : C_0 $, such that $ L $ is isomorphic to the \textit{endomorphism theory} of $ X $: the $ \lambda $-theory $ E(X) $ given by $ E(X)_n = X^n \to X $.}

    $ C $ is the category of sequences of sets $ (P_i)_i $ with a composition $ P_m \times L_n^m \to P_n $ and $ X $ is the sequence $ (L_i)_i $.

    \pause

    With Hyland's definitions and some lemmas, the representation theorem arises before you know it \textit{(on paper)}.
  \end{frame}

  \subsection{The fundamental theorem of the \texorpdfstring{$\lambda$}{lambda}-calculus}

  \begin{frame}[fragile]
    \begin{columns}
      \begin{column}{.5\textwidth}
        \tableofcontents[currentsubsection]
      \end{column}
      \begin{column}{.5\textwidth}
        \begin{tikzcd}
          \Lambda-algebras \arrow[d, harpoon, shift left]           &                                                   \\
          \lambda-theories \arrow[d] \arrow[u, harpoon, shift left] & \stackanchor{Endomorphism}{theories} \arrow[l, two heads, hook] \\
          \stackanchor{Relatively\ Cartesian}{Closed\ Categories}   &
        \end{tikzcd}
      \end{column}
    \end{columns}
  \end{frame}
  \begin{frame}
    \frametitle{``The fundamental theorem of the $ \lambda $-Calculus''}

    There is a functor from $ \lambda $-theories to $ \Lambda $-algebras, sending $ L $ to $ L_0 $: its set of constants.

    \pause

    There is also a functor from $ \Lambda $-algebras to $ \lambda $-theories.
    This functor again uses the endomorphism theory $ E(X) $ for some object $ X $ to construct the $ \lambda $-theory.

    \pause

    Hyland shows that these functors constitute an adjoint equivalence.
  \end{frame}

  \subsection{The category of retracts}

  \begin{frame}[fragile]
    \begin{columns}
      \begin{column}{.5\textwidth}
        \tableofcontents[currentsubsection]
      \end{column}
      \begin{column}{.5\textwidth}
        \begin{tikzcd}
          \Lambda-algebras \arrow[d, harpoon, shift left]           &                                                   \\
          \lambda-theories \arrow[d] \arrow[u, harpoon, shift left] & \stackanchor{Endomorphism}{theories} \arrow[l, two heads, hook] \\
          \stackanchor{Relatively\ Cartesian}{Closed\ Categories}   &
        \end{tikzcd}
      \end{column}
    \end{columns}
  \end{frame}
  \begin{frame}
    \frametitle{The category of retracts}

    Given a $ \lambda $-theory $ L $, we can view elements $ f: L_1 $ as one-argument functions, and we can compose them like $ f \circ g := f \bullet g $.

    Now we construct a category $ R $
    \[ R_0 = \{ a : L_1 \mid a \circ a = a \}, \qquad a \to b = \{ f: L_1 \mid b \circ f \circ a = f \}. \]

    \pause

    This category is cartesian closed: it has products, and `exponentials'. So its morphisms constitute a simply typed $ \lambda $-calculus: we can do \textit{type theory} with the morphisms.
  \end{frame}

  \begin{frame}[fragile]
    \frametitle{Relatively cartesian closed}
    This category is cartesian closed: it has products, and `exponentials'. So its morphisms constitute a simply typed $ \lambda $-calculus: we can do \textit{type theory} with the morphisms.

    \pause

    If we want to do \textit{dependent type theory}, we need dependent products and sums.

    \begin{center}
      \begin{tikzcd}
        R / A \arrow[d, dashed] \arrow[r, "f^*" description] & R / B \arrow[d, dashed] \arrow[l, "\sum_f"', bend right] \arrow[l, "\prod_f", bend left] \\
        A & B \arrow[l, "f"]
      \end{tikzcd}
    \end{center}

    \textit{Locally cartesian closed}: all pullback functors have both adjoints.

    \pause

    In $ R $, not all pullback functors have both adjoints, but some do: \textit{relatively cartesian closed}.

    \pause

    I am still working on understanding the proof.
  \end{frame}

  \section{My contribution}
  \begin{frame}[fragile]
    \begin{columns}
      \begin{column}{.5\textwidth}
        \tableofcontents[currentsection]
      \end{column}
      \begin{column}{.5\textwidth}
        \begin{tikzcd}
          \Lambda-algebras \arrow[d, harpoon, shift left]           &                                                   \\
          \lambda-theories \arrow[d] \arrow[u, harpoon, shift left] & \stackanchor{Endomorphism}{theories} \arrow[l, two heads, hook] \\
          \stackanchor{Relatively\ Cartesian}{Closed\ Categories}   &
        \end{tikzcd}
      \end{column}
    \end{columns}
  \end{frame}

  \subsection{Annotating the paper}
  \begin{frame}
    \frametitle{Annotating the paper}

    \textit{An algebraic theory $ T $ is first a functor $ T : \mathbf{F} \to Sets $: so we have sets $ T(n) $ of $ n $-ary multimaps with variable renamings. In addition, $ T $ is equipped with projections $ pr_1, \dots, pr_n : T(n) $ including as special case the identity $ id \in T(1) $. Finally there are compositions $ T(n) \times T(m)^n \to T(m) $ which are} \textbf{associative, unital, compatible with projections and natural in $ n $ and $ m $}. \textit{A map $ F : S \to T $ of algebraic theories is a natural transformation with components $ F_n : S(n) \to T(n) $ preserving projections and composition.}

    \pause

    \begin{itemize}
      \item Learn the background.
      \item Decode the definitions and theorems.
      \item Find examples.
      \item Formalize (on paper).
      \item Mechanize.
    \end{itemize}

  \end{frame}

  \subsection{Mechanization}
  \begin{frame}
    \frametitle{Mechanization}

    \begin{itemize}
      \item Displayed categories:
      \begin{itemize}
        \item Univalence;
        \item Limits (twice);
      \end{itemize}
      \item Higher inductive types;
      \item $ X^{n + 1} = X \times X^n $;
      \item $ X_{n + 1} = X_{1 + n} $;
    \end{itemize}

  \end{frame}

  \section{Conclusion}

  \begin{frame}[fragile]
    \begin{columns}
      \begin{column}{.5\textwidth}
        \tableofcontents[currentsection]
      \end{column}
      \begin{column}{.5\textwidth}
        \begin{tikzcd}
          \Lambda-algebras \arrow[d, harpoon, shift left]           &                                                   \\
          \lambda-theories \arrow[d] \arrow[u, harpoon, shift left] & \stackanchor{Endomorphism}{theories} \arrow[l, two heads, hook] \\
          \stackanchor{Relatively\ Cartesian}{Closed\ Categories}   &
        \end{tikzcd}
      \end{column}
    \end{columns}
  \end{frame}

  \begin{frame}
    \frametitle{Conclusion}

    Algebraic theories, $ \lambda $-theories and their algebras (and `presheaves') seem to be a promising way to work with models for the $ \lambda $-calculus.

    \pause

    3 `big' theorems:
    \begin{itemize}
      \item Every model of the $ \lambda $-calculus arises as the endomorphism theory of some category.
      \item There is an equivalence between models of the $ \lambda $-calculus, and interpretations of the $ \lambda $-calculus as functions on a set.
      \item From a model for the untyped $ \lambda $-calculus, we can create a category in which we can do some form of dependent type theory.
    \end{itemize}

    \pause

    I am slowly processing the paper.

    \pause

    Mechanization is hard.

  \end{frame}

  \begin{frame}
    \frametitle{Do you have questions?}

    \pause

    Because I have one: I am still a bit unsure about the exact `meaning' of relative cartesian closedness. Can someone explain that better to me?

  \end{frame}

\end{document}
