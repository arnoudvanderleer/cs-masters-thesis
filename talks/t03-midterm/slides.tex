\documentclass[aspectratio=169]{fancyslides} % or beamer/slides

\usepackage{stmaryrd}
\usepackage{tikz-cd}

\title{Midterm review}

\begin{document}
  \maketitle

  \begin{frame}
    \frametitle{Classical lambda calculus in modern dress}

    As far as I understand it:
    \begin{itemize}
      \item Paper by Martin Hyland.
      \item Talks about models for the $ \lambda $-calculus.
      \item Scott's representation theorem (1980)
        \begin{quote}
          Any model is isomorphic to the ``endomorphism theory of a reflexive object in a cartesian closed category''.
        \end{quote}
      \item Fundamental theorem of the $ \lambda $-calculus.
        \begin{quote}
          There is an adjoint equivalence between the category of models and the category of semantics for the $ \lambda $-calculus with $ \beta $-equality.
        \end{quote}
    \end{itemize}
  \end{frame}

  \begin{frame}
    \frametitle{Algebraic theories: objects with variables and substitutions}

    \begin{example}
      Polynomial ring: $ \mathbb Z[x_1, ..., x_n] = \{ 1, x_3, x_5 + 7 x_2 - 17 x_3^2 x_5 x_{75}^{371}, \dots \} $.
    \end{example}

    \pause

    \begin{example}
      $ \lambda $-calculus: $ \Lambda(n) = \{ (\lambda x_1, x_1), x_5, (\lambda x_3, x_7) x_{42} \} $.
    \end{example}

    \pause

    \begin{definition}
      An \textit{algebraic theory} $ T $ is a sequence of sets $ T(n) $ with variables $ x_{i, n} \in T(n) $ (for $ 0 \leq i < n $) and a substitution operation $ \bullet : T(m) \times T(n)^m \to T(n) $.
    \end{definition}

    \pause

    \begin{example}
      In $ T(n) = \mathbb Z[x_1, \dots, x_n] $ for $ (m, n) = (2, 1) $:
      \[ (x_1 x_2) \bullet ((\lambda x_2, x_1 x_2), x_1 x_1) = (\lambda x_2, x_1 x_2) (x_1 x_1). \]
    \end{example}
  \end{frame}

  \begin{frame}
    \frametitle{Algebras: semantics}
    We want to interpret terms with free variables as functions from a context to a set

    \begin{example}
      In $ T(n) = \mathbb Z[x_1, \dots, x_n] $, we can take a set $ A = \mathbb Q $ and get
      \[ 2 x_1 + 3 x_1^2 x_2: A^2 \to A, \quad (a_1, a_2) \mapsto 2 \cdot a_1 + 3 \cdot a_1^2 \cdot a_2. \]
    \end{example}

    \pause

    \begin{definition}
      For an algebraic theory $ T $, a \textit{$ T $-algebra} $ A $ is a set $ A $, together with an interpretation function $ T(n) \times A^n \to A $ for all $ n $ (respecting the variables and substitution).
    \end{definition}
  \end{frame}

  \begin{frame}
    \frametitle{$ \lambda $-theory: structure with app and abs}

    To obtain a model for the $ \lambda $-calculus, we want to talk about structures that also have $ \lambda $-abstraction and application.

    \begin{definition}
      A \textit{$ \lambda $-theory} $ L $ is an algebraic theory, together with abstraction functions $ \lambda: L(n + 1) \to L(n) $ and application functions $ \rho: L(n) \to L(n + 1) $ (respecting the substitution).
    \end{definition}

    \pause

    The $ \lambda $-calculus $ \Lambda $ is the initial $ \lambda $-theory.

    \pause

    $ \beta $- and $ \eta $-equality boil down to the requirement that respectively $ \rho \circ \lambda $ and $ \lambda \circ \rho $ are the identity.
  \end{frame}

  \begin{frame}
    \frametitle{Scott's representation theorem (1980)}

    Given an object $ U $ in a cartesian closed category $ C $ such that we have a retraction $ U \xrightarrow{r} U^U $, we get a $ \lambda $-theory with $ \beta $-equality $ \mathcal U $ (the endomorphism theory of $ U $) by setting $ \mathcal U(n) = C(U^n, U) $.

    \pause

    The paper shows that $ \lambda $-theory is the endomorphism theory for some object $ U $.

    \pause

    It does this by explicitly constructing $ U $.
  \end{frame}

  \begin{frame}
    \frametitle{The Fundamental Theorem of the $ \lambda $-Calculus}

    The paper shows that there is an adjoint equivalence between $ \lambda $-theories with $ \beta $-equality and algebras for $ \Lambda $.

    \pause

    The functor one way sends $ L $ to $ L(0) $.

    \pause

    The functor the other way is a bit more complicated. Given a $ \Lambda $-algebra, it
    \begin{itemize}
      \item constructs a monoid $ M_A $;
      \item takes the category of presheaves $ P M_A $;
      \item takes the object $ U = M_A \in P(M_A) $;
      \item takes its endomorphism theory $ \mathcal U(n) $;
      \item notes that $ \Lambda_A $, the algebraic theory of extensions of $ A $, is isomorphic to $ \mathcal U(n) $.
    \end{itemize}
    \pause
    The fundamental theorem then states that $ A \mapsto \Lambda_A $ and $ L \mapsto L(0) $ constitute an adjoint equivalence of these categories.
  \end{frame}

  \begin{frame}
    \frametitle{Progress}
    \begin{itemize}
      \item Definitions and examples for algebraic theories, algebras and $ \lambda $-theories; \pause
      \item Showed equivalence between two definitions for algebraic theories; \pause
      \item Defined their categories as displayed categories; \pause
      \item Showed univalence for all of these categories; \pause
      \item Showed univalence of two standard constructions; \pause
      \item Constructed the $ \Lambda $-calculus with beta and eta equality as a (hypothetical) higher inductive type;
    \end{itemize}
  \end{frame}

  \begin{frame}
    \frametitle{Displayed categories}

    Many categories are based on other categories. They just have `more structure'. We formalized our categories as (towers of) displayed categories, because this matches their mathematical structure more closely, and helps when proving things like univalence. We have the following towers of displayed categories:

    \[ algebraic\_theory \to algebraic\_theory\_data \to pointed\_functor \to [F, HSET]. \]
    \[ algebra \to algebra\_data \to algebraic\_theory \times HSET \to algebraic\_theory;\]
    \[ lambda\_theory \to lambda\_theory\_data \to algebraic\_theory; \]
  \end{frame}

  \begin{frame}
    \frametitle{Univalence of two standard constructions}
    We can view any category $ C $ as a displayed category $ C^\prime $ over the unit category. I showed that $ C^\prime $ is displayed univalent if $ C $ is univalent.

    \pause

    Given displayed categories $ D $ over $ C $ and $ E $ over $ \int_C D $, we have the category $ \sum_D E $, which is displayed over $ C $. If $ D $ and $ E $ are displayed univalent, then $ \sum_D E $ is displayed univalent as well.
  \end{frame}

  \begin{frame}
    \frametitle{The $ \lambda $-calculus as a (hypothetical) higher inductive type}

    Need:
    \begin{itemize}
      \item Sets $ \Lambda(n) $ for all $ n $.
      \item Constructors.
      \item Substitution.
      \item Beta equality.
    \end{itemize}

    \pause

    Solution: Higher inductive type.

    \pause

    Objection: No inductive types in UniMath.

    \pause

    Solution: Hypothetical higher inductive type.
  \end{frame}

  \begin{frame}
    \frametitle{Future work}

    \begin{itemize}
      \item Remainder of the paper;
      \item Verify Hyland's claims about the inefficiency of alternative approaches;
      \item Compare Hyland's approach with the approaches of his predecessors;
      \item Explorations:
        \begin{itemize}
          \item How about the $ \lambda $-calculus with $ \beta $-conversion (as a special relation) instead of $ \beta $-equality?
          \item What if we also have $ \eta $-equality?
          \item What if we don't have $ \beta $-reduction?
        \end{itemize}
    \end{itemize}
  \end{frame}

\end{document}
