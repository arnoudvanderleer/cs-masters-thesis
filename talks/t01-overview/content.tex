\documentclass{amsart}

\usepackage{stmaryrd}

\newtheorem{theorem}{Theorem}
\newtheorem{lemma}{Lemma}

\theoremstyle{definition}
\newtheorem{definition}{Definition}
\newtheorem{example}{Example}

\newcommand{\SET}{\ensuremath{SET}}

\begin{document}
  \begin{definition}
    An algebraic theory is a functor $ \mathcal T: F \to \SET $ together with a composition morphism $ \bullet: \mathcal T(m) \times \mathcal T(n)^m \to \mathcal T(n) $ for all $ m, n $ and elements (projections) $ \pi_{n, i} \in \mathcal T(n) $ for all $ 1 \leq i \leq n $. The composition must be associative, unital, compatible with projections and dinatural in $ m $.
  \end{definition}

  This is equivalent to:

  \begin{definition}
    An abstract clone is a function $ C : \mathbb N \to \SET $, together with a composition morphism $ \bullet: C(m) \times C(n)^m \to C(n) $ for all $ m, n $ and elements (projections) $ \pi_{n, i} \in C(n) $ for all $ 1 \leq i \leq n $, such that
    \[ \pi_{n, i} \bullet (f_1, \dots, f_n) = f_i; \]
    \[ f \bullet (\pi_{1, n}, \dots, \pi_{n, n}) = f; \]
    \[ (f \bullet (g_1, \dots, g_m)) \bullet (h_1, \dots, h_n) = f \bullet (g_1 \bullet (h_1, \dots, h_n), \dots, g_m \bullet (h_1, \dots, h_n)). \]
  \end{definition}

  \begin{definition}
    An algebra for an algebraic theory $ \mathcal T $ is a set $ A $ with an associative unital action $ \mathcal T(n) \times A^n \to A $, natural in $ n $.
  \end{definition}

  \begin{definition}
    A $ \lambda $-theory is an algebraic theory $ \mathcal L $ together with retractions $ \mathcal L(n + 1) \triangleleft \mathcal L(n) $ with retraction $ \rho: \mathcal L(n) \to \mathcal L(n + 1) $ and section $ \lambda: \mathcal L(n + 1) \to \mathcal L(n) $ natural in $ n $ and compatible (?) with the actions $ \mathcal L(m) \times \mathcal L(n)^m \to \mathcal L(n) $ and $ \mathcal L(m+1) \times \mathcal L(n)^m \to \mathcal L(n+1) $ (which ``ignores the last variable'').
  \end{definition}

  \begin{definition}
    An algebra for a $ \lambda $-theory $ \mathcal L $ is an algebra $ A $ for the underlying algebraic theory.
  \end{definition}
  For each term $ t(\mathbf x) \in \mathcal L(n) $ and each tuple $ \mathbf a \in A^n $, we get an interpretation $ t(\mathbf a) \in A $.

  Given a $ \lambda $-theory $ \mathcal L $, we can interpret a term $ t $ of the lambda calculus (that has a context $ \Gamma $ of length $ n $) as an element $ \llbracket t \rrbracket \in \mathcal L(n) $.
  
  \begin{example}
    Take $ T(n) = \{ \star \} $, $ \pi_{n, i} = \star $ and $ \star \bullet \{ \star, \dots, \star \} = \star $.

    This theory is the terminal $ \lambda $-theory.
  \end{example}

  \begin{example}
    The $ \lambda $-calculus $ \Lambda $, in which $ \Lambda(n) $ consists of the terms with $ n $ free variables, $ \pi_{n, i} = \texttt{Var}(i) $ (with De Bruijn indices) and $ \bullet $ is substitution.

    For every theory $ \mathcal T $ the $ \mathcal T(n) $ are algebras, so the $ \Lambda(n) $ are algebras.

    This is a $ \lambda $-theory. According to the paper, it is the initial $ \lambda $-theory.
  \end{example}

  \begin{example}
    We can create abstract clones using algebraic signatures. If we have, for all $ n $, a set of constructors $ \Sigma_n $, and we have a sequence of variables $ x_1, x_2, \dots $. Then we can build the elements iteratively with the rules $ x_i \in T(n) $ if $ i \leq n $, and for $ x_1, \dots, x_m \in T(n) $ and $ \sigma \in \Sigma_m $, $ \sigma(x_1, \dots, x_m) \in T(n) $. Then the $ \pi_{n, i} $ are the $ x_i $, and $ f \bullet (g_1, \dots, g_m) $ substitutes the $ g_i $ for the $ x_i $ in $ f $.

    For example, if we talk about rings, we have $ \Sigma_0 = \{ 0, 1 \} $, $ \Sigma_1 = \{ - \} $ (negation) and $ \Sigma_2 = \{ +, \cdot \} $. Then $ T(n) $ is almost the polynomial ring over $ \mathbb Z $ in $ n $ variables (but not quite, because we distinguish, for example, between $ 0 $, $ 0 + 0 $ and $ x_1 - x_1 $).

    If we have a type $ A $ with for all $ \sigma \in \Sigma_n $, a map $ \llbracket \sigma \rrbracket : A^n \to A $. Then $ A $ is an algebra for this theory.
  \end{example}

  \begin{example}
    Let $ R $ be a ring. Take $ T(n) = R[x_1, \dots, x_n] $ the polynomial ring in $ n $ variables. Take $ \pi_{n, i} = x_i $ and let $ f \bullet (g_1, \dots, g_n) $ substitute the $ g_i $ for the $ x_i $ in $ f $.

    If we have a homomorphism of rings $ R \to S $ and we take $ \mathcal T(n) = R[X_1, \dots, X_n] $, then $ S $ is a $ \mathcal T $-algebra.
  \end{example}

  \begin{example}
    Take $ T(n) = \{ 1, 2, \dots, n \} $, $ \pi_{n, i} = i $ and $ i \bullet (f_1, \dots, f_n) = f_i $.

    Any type $ A $ can be an algebra, with $ i (a_1, \dots, a_n) = a_i $.
  \end{example}

  \begin{example}
    Take a semiring $ R $ with identities $ 0, 1 $ and operations $ +, \cdot $. Take $ T(n) = R^n $, $ \pi_{n, i} = (0, \dots, 0, 1, 0, \dots, 0) $ and take 
    \[ f \bullet (g_1, \dots, g_n) = \begin{pmatrix}
      g_{1, 1} & \dots & g_{n, 1}\\
      \vdots & \ddots & \vdots\\
      g_{1, m} & \dots & g_{n, m}\\
    \end{pmatrix} \begin{pmatrix}
      f_1 \\ \vdots \\ f_n
    \end{pmatrix} := \left( \sum_i f_i \cdot g_{i, 1}, \dots, \sum_i f_i \cdot g_{i, m} \right) \]
    in a matrix multiplication like fashion.

    For example, take $ S $ a set, $ \mathcal T $ a topology on $ S $. Then we can take $ T(n) = \mathcal T^n $, with operations $ \cup $, $ \cap $, and units $ \emptyset $, $ S $. Then we have $ \pi_{n, i} = (\emptyset, \dots, \emptyset, S, \emptyset, \dots, \emptyset) $. For $ U = (U_1, \dots, U_n) $, $ V_i = (V_{i, 1}, \dots, V_{i, m}) $ we have
    \[ U \bullet (V_1, \dots, V_n) = (U_1 \cap V_{1, 1} \cup \dots \cup U_n \cap V_{n, 1}, \dots, U_1 \cap V_{1, m} \cup \dots \cup U_n \cap V_{n, m}). \]

    Or take $ R = \mathbb N $, with operations $ +, \cdot $ and units $ 0, 1 $.
  \end{example}

  Example $ 5 $, and example $ 6 $ with $ R $ a field are special cases of the following:
  \begin{example}
    In any category with finite products, take an object $ X $. Then we can set $ T(n) = (X^n \to X) $. Then $ \pi_{n, i} $ is the $ i $th projection morphism. Also, by the universal property of the product, if we have $ m $ terms of $ (X^n \to X) $, we get a term of $ (X^n \to X^m) $ which we can compose with a term of $ (X^m \to X) $ to get a term of $ (X^n \to X) $.

    If we have a retraction $ r: X \to A $ with section $ s $, then $ A $ is an algebra with $ \varphi \cdot (a_1, \dots, a_n) = r(\varphi(s(a_1), \dots, s(a_n))) $. (?)

    If we have a retraction $ X \to X^X $, the theory that we get is a $ \lambda $-theory.
  \end{example}

  \section{Presheaves}
  Let $ A $ be a $ \Lambda $-algebra.

  \begin{definition}
    We define the monoid $ M_A $ with underlying set $ \{ a \in A \mid \mathbf 1 a = a \} $ and composition $ (a, b) \mapsto a \circ b = \lambda x, a(b x) $ (?).
  \end{definition}

  \begin{definition}
    We define $ PA $ to be the category of presheaves on the category $ M_A $. (I.e. the set of contravariant functors into set). This has `universal object' $ U_A = M_A $ with the obvious right action of $ M_A $.
  \end{definition}

  \begin{lemma}
    For $ U_A $, we have a retraction $ U_A \to U_A^{U_A} $.
  \end{lemma}
  \begin{proof}
    We have $ M_A = \{ a \in A \mid \mathbf 1 a = a \} $. We can identify $ U^U $ with $ \{ a \in A \mid \mathbf 1_2 a = a \} $.

    Composition on the left with $ \mathbf 1 $ gives the retraction.
  \end{proof}

  \begin{definition}
    We take $ \mathcal U_A $ to be the theory of the reflexive universal $ U_A \in P(A) $.
  \end{definition}

  \section{The main theorem}

  We have a map $ \Lambda \to \mathcal L $, which makes $ \mathcal L(0) $ into a $ \Lambda $-algebra.

  Given a $ \Lambda $-algebra $ A $. Take $ \Lambda_A(n) = A + \Lambda(n) $ (which is a coproduct of $ \mathcal T $-algebras, defined using a coend).
  
  \begin{lemma}
    $ \Lambda_A $ is a $ \lambda $-theory.
  \end{lemma}
  \begin{proof}
    We can identify $ \Lambda_A $ with $ \mathcal U_A $.

    We have a retraction $ U_A \to U_A^{U_A} $. Composition with this gives a retraction $ \mathcal U_A(n) \to \mathcal U_A(n + 1) $. 
  \end{proof}

  \begin{theorem}
    There is an adjoint equivalence $ \mathcal L \mapsto \mathcal L(0) $ and $ A \mapsto \Lambda_A $ between $ \lambda $-theories and  $ \Lambda $-algebras.

    In particular, each $ \lambda $-theory $ \mathcal L $ is isomorphic to the theory of extensions of its initial algebra $ \mathcal L(0) $.
  \end{theorem}

\end{document}