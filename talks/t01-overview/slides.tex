\documentclass[aspectratio=169]{fancyslides} % or beamer/slides

\usepackage{stmaryrd}

% \usetheme{Madrid}

\title{Decorating ``Classical Lambda Calculus in Modern Dress''}

\begin{document}
  \maketitle

  \begin{frame}
    \frametitle{Definitions - Algebraic Theory}

    \begin{definition}
      An algebraic theory consists of
      \begin{itemize}
        \item a functor $ \mathcal T: F \to \mathrm{SET} $,
        \item elements (projections) $ \pi_{n, i} \in \mathcal T(n) $ for all $ 1 \leq i \leq n $,
        \item a composition morphism $ \bullet: \mathcal T(m) \times \mathcal T(n)^m \to \mathcal T(n) $ for all $ m, n $.
      \end{itemize}
      The composition must be
      \begin{itemize}
        \item associative,
        \item unital,
        \item compatible with projections,
        \item dinatural in $ m $.
      \end{itemize}
    \end{definition}
  \end{frame}

  \begin{frame}
    \frametitle{Definitions - Abstract Clone}

    \begin{definition}
      An abstract clone is
      \begin{itemize}
        \item a function $ C : \mathbb N \to \mathrm{SET} $,
        \item a composition morphism $ \bullet: C(m) \times C(n)^m \to C(n) $ for all $ m, n $,
        \item elements (projections) $ \pi_{n, i} \in C(n) $ for all $ 1 \leq i \leq n $.
      \end{itemize}
      The composition must satisfy
      \begin{itemize}
        \item $ \pi_{n, i} \bullet (f_1, \dots, f_n) = f_i $,
        \item $ f \bullet (\pi_{1, n}, \dots, \pi_{n, n}) = f $,
        \item $ (f \bullet (g_1, \dots, g_m)) \bullet (h_1, \dots, h_n) = f \bullet (g_1 \bullet (h_1, \dots, h_n), \dots, g_m \bullet (h_1, \dots, h_n)) $.
      \end{itemize}
    \end{definition}
  \end{frame}

  \begin{frame}
    \frametitle{Definitions - Algebra for an Algebraic Theory}

    \begin{definition}
      An algebra for an algebraic theory $ \mathcal T $ consists of
      \begin{enumerate}
        \item a set $ A $,
        \item an action $ \alpha_n: \mathcal T(n) \times A^n \to A $.
      \end{enumerate}
      Such that $ \alpha_n $ satisfies:
      \begin{enumerate}
        \item naturality in $ n $,
        \item associativity,
        \item unitality.
      \end{enumerate}
    \end{definition}
  \end{frame}

  \begin{frame}
    \frametitle{Definitions - $ \lambda $-theory}

    \begin{definition}
      A $ \lambda $-theory consists of
      \begin{itemize}
        \item an algebraic theory $ \mathcal L $,
        \item a retraction $ \rho: \mathcal L(n) \to \mathcal L(n + 1) $,
        \item a section $ \lambda: \mathcal L(n + 1) \to \mathcal L(n) $.
      \end{itemize}
      Such that
      \begin{itemize}
        \item $ \rho $ and $ \lambda $ are natural in $ n $;
        \item $ \rho $ and $ \lambda $ are compatible (?) with the actions $ \mathcal L(m) \times \mathcal L(n)^m \to \mathcal L(n) $ and $ \mathcal L(m+1) \times \mathcal L(n)^m \to \mathcal L(n+1) $ (which ``ignores the last variable'').
      \end{itemize}
    \end{definition}

    \pause

    \begin{definition}
      An algebra for a $ \lambda $-theory $ \mathcal L $ is an algebra $ A $ for the underlying algebraic theory.
    \end{definition}

    \pause
    \vfill

    For each term $ t(\mathbf x) \in \mathcal L(n) $ and each tuple $ \mathbf a \in A^n $, we get an interpretation $ t(\mathbf a) \in A $.

    \vfill

    Given a $ \lambda $-theory $ \mathcal L $, we can interpret a term $ t $ of the lambda calculus (that has a context $ \Gamma $ of length $ n $) as an element $ \llbracket t \rrbracket \in \mathcal L(n) $.
  \end{frame}

  \begin{frame}
    \frametitle{Examples}

    \begin{definition}
      An abstract clone is
      \begin{itemize}
        \item a function $ C : \mathbb N \to \mathrm{SET} $,
        \item a composition morphism $ \bullet: C(m) \times C(n)^m \to C(n) $ for all $ m, n $,
        \item elements (projections) $ \pi_{n, i} \in C(n) $ for all $ 1 \leq i \leq n $.
      \end{itemize}
      The composition must satisfy
      \begin{itemize}
        \item $ \pi_{n, i} \bullet (f_1, \dots, f_n) = f_i $,
        \item $ f \bullet (\pi_{1, n}, \dots, \pi_{n, n}) = f $,
        \item $ (f \bullet (g_1, \dots, g_m)) \bullet (h_1, \dots, h_n) = f \bullet (g_1 \bullet (h_1, \dots, h_n), \dots, g_m \bullet (h_1, \dots, h_n)) $.
      \end{itemize}
    \end{definition}

    \vfill

    \begin{example}
      Take $ C(n) = \{ \star \} $, $ \pi_{n, i} = \star $ and $ \star \bullet \{ \star, \dots, \star \} = \star $.

      This theory is the terminal $ \lambda $-theory.
    \end{example}
  \end{frame}

  \begin{frame}
    \frametitle{Examples}

    The $ \lambda $-calculus $ \Lambda $, in which $ \Lambda(n) $ consists of the terms with $ n $ free variables, $ \pi_{n, i} = \texttt{Var}(i) $ (with De Bruijn indices) and $ \bullet $ is substitution.

    \pause
    \vfill

    For every theory $ \mathcal T $ the $ \mathcal T(n) $ are algebras, so the $ \Lambda(n) $ are algebras.

    \pause
    \vfill

    This is a $ \lambda $-theory. According to the paper, it is the initial $ \lambda $-theory.
  \end{frame}

  \begin{frame}
    \frametitle{Examples}

    More generally, we can create abstract clones using algebraic signatures: Given sets of $ n $-ary constructors $ \Sigma_n $ and a sequence of variables $ x_1, \dots, x_n $, we can build the $ C(n) $ inductively:
    \begin{itemize}
      \item $ x_i : C(n) $ for $ 1 \leq i \leq n $;
      \item $ \sigma(c_1, \dots, c_n) : C(n) $ for $ c_1, \dots, c_n : C(n) $ and $ \sigma: \Sigma_n $.
    \end{itemize}
    Then the $ \pi_{n, i} $ are the $ x_i $, and $ f \bullet (g_1, \dots, g_m) $ substitutes the $ g_i $ for the $ x_i $ in $ f $.

    \pause
    \vfill

    For example, if we talk about rings, we have $ \Sigma_0 = \{ 0, 1 \} $, $ \Sigma_1 = \{ - \} $ (negation) and $ \Sigma_2 = \{ +, \cdot \} $. Then $ C(n) $ is almost the polynomial ring over $ \mathbb Z $ in $ n $ variables (but not quite, because we distinguish, for example, between $ 0 $, $ 0 + 0 $ and $ x_1 - x_1 $).

    \pause
    \vfill

    If we have a type $ A $ and we have for all $ \sigma \in \Sigma_n $, a map $ \llbracket \sigma \rrbracket : A^n \to A $. Then we can make $ A $ into an algebra for this theory.
  \end{frame}

  \begin{frame}
    \frametitle{Examples}
    Let $ R $ be a ring. Take $ T(n) = R[x_1, \dots, x_n] $ the polynomial ring in $ n $ variables. Take $ \pi_{n, i} = x_i $ and let $ f \bullet (g_1, \dots, g_n) $ substitute the $ g_i $ for the $ x_i $ in $ f $.

    \pause
    \vfill

    If we have a homomorphism of rings $ R \to S $ and we take $ \mathcal T(n) = R[X_1, \dots, X_n] $, then $ S $ is a $ \mathcal T $-algebra.
  \end{frame}

  \begin{frame}
    \frametitle{Examples}

    Let $ R $ be a ring. Take $ T(n) = R[x_1, \dots, x_n] $ the polynomial ring in $ n $ variables. Take $ \pi_{n, i} = x_i $ and let $ f \bullet (g_1, \dots, g_n) $ substitute the $ g_i $ for the $ x_i $ in $ f $.

    \pause
    \vfill

    If we have a homomorphism of rings $ R \to S $ and we take $ \mathcal T(n) = R[X_1, \dots, X_n] $, then $ S $ is a $ \mathcal T $-algebra.
  \end{frame}

  \begin{frame}
    \frametitle{Examples}

    Take $ T(n) = \{ 1, 2, \dots, n \} $, $ \pi_{n, i} = i $ and $ i \bullet (f_1, \dots, f_n) = f_i $.
    
    This is the initial algebraic theory.

    \pause
    \vfill
      
    Any type $ A $ can be an algebra, with $ i (a_1, \dots, a_n) = a_i $.

    \pause
    \vfill

    I am still pondering whether $ T(n) = \mathbb Z $ could be a $ \lambda $-algebra.
  \end{frame}

  \begin{frame}
    \frametitle{Examples}

    Take a semiring $ R $ with identities $ 0, 1 $ and operations $ +, \cdot $. Take $ T(n) = R^n $, $ \pi_{n, i} = (0, \dots, 0, 1, 0, \dots, 0) $ and take 
    \[ f \bullet (g_1, \dots, g_n) = \begin{pmatrix}
      g_{1, 1} & \dots & g_{n, 1}\\
      \vdots & \ddots & \vdots\\
      g_{1, m} & \dots & g_{n, m}\\
    \end{pmatrix} \begin{pmatrix}
      f_1 \\ \vdots \\ f_n
    \end{pmatrix} := \left( \sum_i f_i \cdot g_{i, 1}, \dots, \sum_i f_i \cdot g_{i, m} \right) \]
    in a matrix multiplication like fashion. ($ T(n) $ is the free $ R $-module on $ n $ generators)

    \pause
    \vfill

    For example, take $ S $ a set, $ \mathcal T $ a topology on $ S $. Then we can take $ T(n) = \mathcal T^n $, with operations $ \cup $, $ \cap $, and units $ \emptyset $, $ S $. Then we have $ \pi_{n, i} = (\emptyset, \dots, \emptyset, S, \emptyset, \dots, \emptyset) $. For $ U = (U_1, \dots, U_n) $, $ V_i = (V_{i, 1}, \dots, V_{i, m}) $ we have
    \[ U \bullet (V_1, \dots, V_n) = (U_1 \cap V_{1, 1} \cup \dots \cup U_n \cap V_{n, 1}, \dots, U_1 \cap V_{1, m} \cup \dots \cup U_n \cap V_{n, m}). \]

    \pause
    \vfill

    Or take $ R = \mathbb N $, with operations $ +, \cdot $ and units $ 0, 1 $.
  \end{frame}

  \begin{frame}
    \frametitle{Examples}

    In any category with finite products, take an object $ X $. Then we can set $ T(n) = (X^n \to X) $. Then $ \pi_{n, i} $ is the $ i $th projection morphism. Also, by the universal property of the product, if we have $ m $ terms of $ (X^n \to X) $, we get a term of $ (X^n \to X^m) $ which we can compose with a term of $ (X^m \to X) $ to get a term of $ (X^n \to X) $.

    \pause
    \vfill

    I believe that if we have a retraction $ r: X \to A $ with section $ s $, then $ A $ is an algebra with $ \varphi \cdot (a_1, \dots, a_n) = r(\varphi(s(a_1), \dots, s(a_n))) $.

    \pause
    \vfill

    If we have a retraction $ X \to X^X $, the theory that we get is a $ \lambda $-theory.
  \end{frame}

  \begin{frame}
    \frametitle{About $ \lambda $-theories}

    Any algebra $ (A, (\alpha_i)_i) $ for a $ \lambda $-theory can be interpreted as a $ \Lambda $-algebra.

    \pause
    \vfill

    Any $ \lambda $-theory $ \mathcal T $ has an element $ e \in \mathcal T(1) $, retractions $ \rho: \mathcal T(n) \to \mathcal T(n + 1) $ and sections $ \sigma: \mathcal T(n + 1) \to \mathcal T(n) $. Define $ app = \rho(e) $. For $ \mathcal T = \Lambda $, we have $ app = xy $.

    \pause
    
    Then $ \alpha_2(app, (a, -)): A \to A $ gives an interpretation of $ a : A $ as a function.

    \vfill
    \pause

    Take $ \mathbf 1_n = \sigma^{n+1}(\rho^n(e)) $. Note that we can inject this as $ \mathbf 1_n \bullet () $ into any $ \mathcal T(n) $.
    
    For $ \mathcal T = \Lambda $, this is
    \[ \lambda x, \lambda y_1, \dots, \lambda y_n, x y_1 \dots y_n. \]
    \pause
    For $ n = 1 $, this is
    \[ \lambda x, \lambda y, xy. \]
  \end{frame}

  \begin{frame}
    \frametitle{Preliminaries to the main theorem}

    For a $ \lambda $-theory $ \mathcal L $, $ \mathcal L(0) $ is a $ \mathcal L $-algebra and therefore a $ \Lambda $-algebra.

    \vfill
    \pause

    Given a $ \Lambda $-algebra $ A $. Take $ \Lambda_A(n) = A + \Lambda(n) $, as a coproduct of $ \mathcal T $-algebras, defined using a coend.

    \vfill
    \pause
    
    \begin{lemma}
      $ \Lambda_A $ is a $ \lambda $-theory.
    \end{lemma}
    \begin{proof}
      We can identify $ \Lambda_A $ with $ \mathcal U_A $.

      We have a retraction $ U_A \to U_A^{U_A} $. Composition with this gives a retraction $ \mathcal U_A(n) \to \mathcal U_A(n + 1) $. 
    \end{proof}
  \end{frame}

  \begin{frame}
    \frametitle{$ \mathcal U_A $}
    Let $ A $ be a $ \Lambda $-algebra.

    \begin{definition}
      We define a monoid $ M_A = (\{ a \in A \mid \mathbf 1 a = a \}, \circ) $ with $ \circ $ being function composition.
    \end{definition}
    
    \pause
    \vfill

    \begin{definition}
      We define $ P(A) $ to be the category of presheaves on the category $ M_A $. This has `universal object' $ U_A = M_A $ with the obvious right action of $ M_A $.
    \end{definition}
    
    \pause
    \vfill

    \begin{lemma}
      We have a retraction $ U_A \to U_A^{U_A} $.
    \end{lemma}
    \begin{proof}
      We have $ M_A = \{ a \in A \mid \mathbf 1 a = a \} $. We can identify $ U_A^{U_A} $ with $ \{ a \in A \mid \mathbf 1_2 a = a \} $.

      Composition on the left with $ \mathbf 1 $ gives the retraction.
    \end{proof}
    
    \pause
    \vfill

    \begin{definition}
      We take $ \mathcal U_A $ to be the endomorphism theory of the reflexive universal $ U_A \in P(A) $.
    \end{definition}
  \end{frame}

  \begin{frame}
    \frametitle{The Main Theorem of the Lambda Calculus}
    \begin{theorem}
      There is an adjoint equivalence $ \mathcal L \mapsto \mathcal L(0) $ and $ A \mapsto \Lambda_A $ between $ \lambda $-theories and  $ \Lambda $-algebras.

      In particular, each $ \lambda $-theory $ \mathcal L $ is isomorphic to the theory of extensions of its initial algebra $ \mathcal L(0) $.
    \end{theorem}
  \end{frame}

\end{document}