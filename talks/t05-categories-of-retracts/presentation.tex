\documentclass{amsart}

\usepackage{tikz-cd}
\usepackage{tabularx}
\usepackage{pifont}% http://ctan.org/pkg/pifont
\newcommand{\cmark}{\ding{51}}%
\newcommand{\xmark}{\ding{55}}%

\renewcommand{\tabularxcolumn}[1]{m{#1}}

\DeclareFontFamily{U}{dmjhira}{}
\DeclareFontShape{U}{dmjhira}{m}{n}{ <-> dmjhira }{}

\DeclareRobustCommand{\yo}{\text{\usefont{U}{dmjhira}{m}{n}\symbol{"48}}}

\DeclareMathOperator\Psh{Psh}
\DeclareMathOperator\id{id}

\title{The categories of retracts}

\begin{document}
  \maketitle

  \section*{Locally cartesian closed}

  Let $ C $ be any category, and take $ f: C(B, A) $.
  \begin{center}
    \begin{tikzcd}
      C \downarrow A \arrow[r, "f^*", description] & C \downarrow B \arrow[l, "\sum_f"', bend right] \arrow[l, "\prod_f", bend left]
    \end{tikzcd}
  \end{center}

  \section*{Relatively cartesian closed}
  Some categories do not have $ f^* $ or $ \prod_f $ for all $ f $. But sometimes we can choose full subcategories of `display maps' (or `fibrations') $ C \downarrow_H B \subseteq C \downarrow B $ such that
  \begin{itemize}
    \item Composition of display maps is a display map;
    \item Terminal projections are display maps (so $ C \downarrow_H T = C \downarrow T \simeq C $);
    \item Pullbacks of display maps along any map exist.
  \end{itemize}
  \begin{center}
    \begin{tikzcd}
      C \downarrow_H A \arrow[r, "f^*" description] & C \downarrow_H B \arrow[l, "\sum_f"', bend right] \arrow[l, "\prod_f", bend left]
    \end{tikzcd}
  \end{center}
  Cartesian closed relative to $ H $ if $ f^* $ has a right adjoint.

  \newpage

  Let $ A $ be a category. Let $ \yo: A \hookrightarrow \Psh A $ be the Yoneda embedding. Let $ B, C, D $ be the categories given by:

  \section*{Scott's version}

  \[ B_0 = \sum_{X : A}\ \sum_{f : A(X, X)} f \cdot f = f \]
  \[ B((X_1, f_1), (X_2, f_2)) = \{ g: A(X_1, X_2) \mid f_1 \cdot g \cdot f_2 = g \}: \]
  \begin{center}
    \begin{tikzcd}
      X_1
        \arrow["f_1"', loop, distance=2em, in=-150, out=150]
        \arrow[r, "g"] &
      X_2
        \arrow["f_2"', loop, distance=2em, in=30, out=-30]
    \end{tikzcd}
  \end{center}

  \section*{First interpretation of the equivalent notion}

  \[ C_0 = \sum_{P : \Psh A}\ \sum_{X : A}\ \sum_{s: A(P, \yo(X))}\ \sum_{r: A(\yo(X), P)} s \cdot r = \id \]
  \[ C((P_1, X_1, r_1, s_1), (P_2, X_2, r_2, s_2)) = \Psh A(P_1, P_2): \]
  \begin{center}
    \begin{tikzcd}
      \yo(X_1) \arrow[r, "r_1", bend left] &
      P_1 \arrow[r, "g"] \arrow[l, "s_1", bend left] &
      P_2 \arrow[r, "s_2", bend left] &
      \yo(X_2) \arrow[l, "r_2", bend left]
    \end{tikzcd}
  \end{center}

  \section*{Second interpretation of the equivalent notion}

  \[ D_0 = \sum_{P : \Psh A} \left\Vert \sum_{X : A}\ \sum_{s: A(P, \yo(X))}\ \sum_{r: A(\yo(X), P)} s \cdot r = \id \right\Vert \]
  \[ D((P_1, X_1, r_1, s_1), (P_2, X_2, r_2, s_2)) = \Psh A(P_1, P_2): \]
  \begin{center}
    \begin{tikzcd}
      \mathit{\yo(X_1)} \arrow[r, "r_1", bend left, dashed] &
      P_1 \arrow[l, "s_1", bend left, dashed] \arrow[r, "g"] &
      P_2 \arrow[r, "s_2", bend left, dashed] &
      \mathit{\yo(X_2)} \arrow[l, "r_2", bend left, dashed]
    \end{tikzcd}
  \end{center}

  \newpage

  \begin{center}
    \begin{tabularx}{.7\textwidth}{c >{\centering\arraybackslash}X}
      B &
      \begin{tikzcd}[ampersand replacement = \&]
        X_1
          \arrow["f_1"', loop, distance=2em, in=-150, out=150]
          \arrow[r, "g"] \&
        X_2
          \arrow["f_2"', loop, distance=2em, in=30, out=-30]
      \end{tikzcd}\\
      C &
      \begin{tikzcd}[ampersand replacement = \&]
        \yo(X_1) \arrow[r, "r_1", bend left] \&
        P_1 \arrow[r, "g"] \arrow[l, "s_1", bend left] \&
        P_2 \arrow[r, "s_2", bend left] \&
        \yo(X_2) \arrow[l, "r_2", bend left]
      \end{tikzcd}\\
      D &
      \begin{tikzcd}[ampersand replacement = \&]
        \mathit{\yo(X_1)} \arrow[r, "r_1", bend left, dashed] \&
        P_1 \arrow[l, "s_1", bend left, dashed] \arrow[r, "g"] \&
        P_2 \arrow[r, "s_2", bend left, dashed] \&
        \mathit{\yo(X_2)} \arrow[l, "r_2", bend left, dashed]
      \end{tikzcd}
    \end{tabularx}
  \end{center}

  \section*{Properties}

  \begin{center}
    \begin{tabularx}{.8\textwidth}{l| *3{>{\centering\arraybackslash}X}}
      & B & C & D\\\hline
      Embeds fully faithfully into $ \Psh A $ & \cmark & \cmark & \cmark\\
      Subcategory of $ \Psh A $ & \xmark & \xmark & \cmark\\
      Univalent & \xmark & \xmark & \cmark\\
      Scott's construction & \cmark & \xmark & \xmark\\
      Taylor's construction & \cmark & \xmark & \xmark\\
    \end{tabularx}
  \end{center}

  \begin{center}
    \begin{tikzcd}
      B \arrow[r, "\sim"] & C \arrow[r, hook, two heads] & D \arrow[r, phantom, "\subseteq"] & \Psh A
    \end{tikzcd}
  \end{center}

  \newpage

  \section*{Taylor}
  Taylor works in Scott's category of retracts $ B $ of a category $ A $ that is based on a monoid:

  \begin{center}
    \begin{tikzcd}
      *
        \arrow["f_1"', loop, distance=2em, in=-150, out=150]
        \arrow[r, "g"] &
      *
        \arrow["f_2"', loop, distance=2em, in=30, out=-30]
    \end{tikzcd}
  \end{center}

  For $ X : B $, Taylor constructs a category of `indexed types' $ B^X $ (which behaves like functions $ X \to B $) with a fully faithful embedding
  \[ \sum_X : B^X \hookrightarrow B \downarrow X. \]
  He chooses $ D $ such that $ i $ is essentially surjective onto $ B \downarrow_H X $ and shows that we have an adjunction
  \begin{center}
    \begin{tikzcd}[row sep=large]
      B^X \arrow[d, hook, two heads, "\sum_X"] \arrow[r, "f^*" description] & B^Y \arrow[d, hook, two heads, "\sum_Y"] \arrow[l, bend left, "\prod_f"]\\
      B \downarrow_H X & B \downarrow_H Y
    \end{tikzcd}
  \end{center}
  In this case, $ B $ is a setcategory, so the functors $ \sum $ become equivalences under the axiom of choice and we can lift $ f^* \dashv \prod_f $ to $ B \downarrow_H X $ and $ B \downarrow_H Y $.
  In fact we have mere existence of pullbacks and to turn this into the mere existence of a pullback functor already requires the axiom of choice.

  Note that if we try to do a similar thing in $ C $, we would expect to have easy candidates for $ \prod_f $ and $ \sum_f $, but again we only have mere existence of pullbacks and now that is an even bigger problem because $ C $ is not a setcategory.
  \begin{center}
    \begin{tikzcd}
      \yo(*) \arrow[r, "r_1", bend left] &
      P_1 \arrow[r, "g"] \arrow[l, "s_1", bend left] &
      P_2 \arrow[r, "s_2", bend left] &
      \yo(*) \arrow[l, "r_2", bend left]
    \end{tikzcd}
  \end{center}

  \newpage

  \section*{Hyland}
  For $ X : \Psh A $, Hyland considers `the category of retracts' $ \mathbb R(X) $ of
  \[ \pi_1: X \times \yo(*) \to X \]
  in $ \Psh A \downarrow X $. He proceeds to construct retractions of $ (Y \times \yo(*), \pi_1) $ onto $ \prod_f(X, g) $ and $ \sum_f(X, g) $ for $ f: Y \to X $ in $ \mathbb R(X) $ and for a retraction of $ (X \times \yo(*), \pi_1) $ onto $ (X, g) $. That is,
  \[ \prod_f, \sum_f : \Psh A \downarrow X \to \Psh A \downarrow Y \]
  send objects in $ \mathbb R(X) $ to $ \mathbb R(Y) $.

  Furthermore, he claims that
  \begin{itemize}
    \item Scott's category of retracts $ B $ ($ = \mathbb R(\yo(*)) $) is a subcategory of $ \Psh A $.
    \item For $ X \in B $ and $ Y \in \mathbb R(X) $, $ Y \in B $.
    \item From this, Taylor's theorem follows.
  \end{itemize}

  This suggests that Hyland attempts to lift $ \prod_f $ and $ \sum_f $ along the weak equivalences
  \begin{center}
    \begin{tikzcd}[sep=huge]
      (\mathbb R \downarrow_H X)
        \arrow[d, hook, two heads, "\psi_X"]
        \arrow[r, dashed, bend left, "\sum_f"']
        \arrow[r, dashed, bend right, "\prod_f"]
      & (\mathbb R \downarrow_H Y)
        \arrow[d, hook, two heads, "\psi_Y"]
        \arrow[l, dashed, "f^*" description]\\
      \mathbb R(\varphi(X))
        \arrow[r, bend left, "\sum_{\varphi(f)}"']
        \arrow[r, bend right, "\prod_{\varphi(f)}"]
      & \mathbb R(\varphi(Y))
        \arrow[l, "\varphi(f)^*" description]\\
    \end{tikzcd}
  \end{center}

  But since we are working with presheaf categories here, a weak equivalence does not give an adjoint equivalence.

  \section*{The solution}
  If we abandon the need to replicate Taylor's proof, we can actually do a lot better:
  We will work in $ D $, the Rezk completion of $ B $ and $ C $:
  \begin{center}
    \begin{tikzcd}
      \mathit{\yo(X_1)} \arrow[r, "r_1", bend left, dashed] &
      P_1 \arrow[l, "s_1", bend left, dashed] \arrow[r, "g"] &
      P_2 \arrow[r, "s_2", bend left, dashed] &
      \mathit{\yo(X_2)} \arrow[l, "r_2", bend left, dashed]
    \end{tikzcd}
  \end{center}

  We take $ (Y, f) : D \downarrow_H X $ if there merely exists a retraction from $ \pi_1 : X \times \yo(*) \to X $ onto $ (Y, f) $. We can show that this forms a class of display maps, and using Hyland's proof, we can show that $ \prod_f $ restricts to the relative slices, so $ D $ is cartesian closed relative to $ H $.
\end{document}
