\documentclass[aspectratio=169]{fancyslides} % or beamer/slides

\usepackage{stmaryrd}
\usepackage{tikz-cd}

% \usetheme{Madrid}

\newcommand\abstractCloneDefinition{
  \begin{definition}
    An abstract clone is
    \begin{itemize}
      \item a function $ C : \mathbb N \to \mathrm{SET} $,
      \item a composition morphism $ \bullet: C(m) \times C(n)^m \to C(n) $ for all $ m, n $,
      \item elements (projections) $ x_{n, i} \in C(n) $ for all $ 1 \leq i \leq n $.
    \end{itemize}
    The composition must satisfy $ x_{n, i} \bullet (f_1, \dots, f_n) = f_i $ and $ f \bullet (\pi_{1, n}, \dots, \pi_{n, n}) = f $ and
    \[ (f \bullet (g_1, \dots, g_m)) \bullet (h_1, \dots, h_n) = f \bullet (g_1 \bullet (h_1, \dots, h_n), \dots, g_m \bullet (h_1, \dots, h_n)). \]
  \end{definition}
}

\newcommand\algebraDefinition{
  \begin{definition}
    An algebra for an abstract clone $ C $ consists of
    \begin{enumerate}
      \item a set $ A $,
      \item an action $ \alpha_n: C(n) \times A^n \to A $.
    \end{enumerate}
    Such that $ \alpha_n $ satisfies $ \alpha_n(x_{n, i}, a) = a_i $ and $ \alpha_n(f \bullet g, a) = \alpha_m(f, (\alpha_n(g_1, a), \dots, \alpha_n(g_m, a))) $.
  \end{definition}
}

\newcommand\lambdaCloneDefinition{
  \begin{definition}
    A $ \lambda $-clone consists of
    \begin{itemize}
      \item an abstract clone $ \mathcal L $,
      \item functions $ \rho_n: \mathcal L(n) \to \mathcal L(n + 1) $ and $ \lambda_n: \mathcal L(n + 1) \to \mathcal L(n) $.
    \end{itemize}
    Such that
    \[ \rho(f \bullet g) = \rho(f) \bullet (g_1, \dots, g_m, x_{n+1, n+1}) \]
    and
    \[ \lambda(f) \bullet g = \lambda(f \bullet (g_1, \dots, g_m, x_{n+1, n+1})). \]
    Hyland also requires $ (\lambda, \rho) $ to be a section-retraction pair.
  \end{definition}
}

\title{Classical Lambda Calculus in Modern Dress}

\begin{document}
  \maketitle

  \begin{frame}
    \frametitle{Denotational Semantics}

    \texttt{expr = 1 | -<expr> | <expr> + <expr>}\\
    \pause
    For $ t \in \mathtt{expr} $, $ \llbracket t \rrbracket \in \mathbb Z $.
  \end{frame}

  \begin{frame}
    \frametitle{Denotational Semantics}

    \texttt{expr = 1 | x | -<expr> | <expr> + <expr>}\\
    \pause
    For $ t \in \mathtt{expr} $, $ \llbracket t \rrbracket \in \mathbb Z \oplus x \cdot \mathbb Z $.
  \end{frame}

  \begin{frame}
    \frametitle{Denotational Semantics}

    \texttt{expr = 1 | x * <expr> | -<expr> | <expr> + <expr>}\\
    \pause
    For $ t \in \mathtt{expr} $, $ \llbracket t \rrbracket \in \mathbb Z[x] $: polynomials with coefficients in $ \mathbb Z $.
    \pause
    Or $ \llbracket t \rrbracket \in \mathbb Z \to \mathbb Z $. Or maybe $ \llbracket t \rrbracket \in \mathbb R \to \mathbb R $.
  \end{frame}

  \begin{frame}
    \frametitle{Denotational Semantics}

    \texttt{expr = Var(<nat>) | App(<expr>, <expr>) | Abs(Var(<nat>), <expr>)}\\
    \pause
    For $ t \in \mathtt{expr} $, $ \llbracket t \rrbracket \in ? $.\\
  \end{frame}

  \begin{frame}
    \frametitle{Algebraic Theory}

    \begin{definition}
      An algebraic theory consists of
      \begin{itemize}
        \item a functor $ \mathcal T: F \to \mathrm{SET} $,
        \item elements (projections) $ x_{n, i} \in \mathcal T(n) $ for all $ 1 \leq i \leq n $,
        \item a composition morphism $ \bullet: \mathcal T(m) \times \mathcal T(n)^m \to \mathcal T(n) $ for all $ m, n $.
      \end{itemize}
      The composition must be associative, unital, compatible with projections and dinatural in $ m $.
    \end{definition}
  \end{frame}

  \begin{frame}
    \frametitle{Abstract Clone}

    \abstractCloneDefinition
  \end{frame}

  \begin{frame}
    \frametitle{Abstract Clone - Example}

    \abstractCloneDefinition

    \begin{example}
      Take $ C(n) = \{ 1, 2, \dots, n \} $, $ x_{n, i} = i $ and $ i \bullet (f_1, \dots, f_n) = f_i $.
    \end{example}
  \end{frame}

  \begin{frame}
    \frametitle{Abstract Clone - Example}


    \abstractCloneDefinition

    \begin{example}
      Take $ C(n) $ the free monoid on $ n $ generators. The $ x_{n, i} $ are the generators and $ f \bullet (g_1, \dots, g_m) $ applies the mapping $ C(m) \to C(n) $ on $ f $, given by sending $ x_{m, i} $ to $ g_i $.
    \end{example}
  \end{frame}

  \begin{frame}
    \frametitle{Abstract Clone - Example}

    \abstractCloneDefinition

    \begin{example}
      Let $ X $ be an object in a category with finite products. The endomorphism clone has $ C(n) = (X^n \to X) $. Then $ x_{n, i} $ is the $ i $th projection morphism. The universal property of the product gives $ \bullet $.
    \end{example}
  \end{frame}

  \begin{frame}
    \frametitle{Abstract Clone - Example}

    \abstractCloneDefinition

    \begin{example}
      The $ \lambda $-calculus $ \Lambda $, in which $ \Lambda(n) $ consists of the terms with $ n $ free variables, $ x_{n, i} = \texttt{Var}(i) $ (with De Bruijn indices) and $ \bullet $ is substitution.
    \end{example}
  \end{frame}

  \begin{frame}
    \frametitle{Abstract Clone Algebra}

    \abstractCloneDefinition
  \end{frame}

  \begin{frame}
    \frametitle{Abstract Clone Algebra - Example}

    \abstractCloneDefinition

    \begin{example}
      For the clone $ C(n) = \{ 1, \dots, n \} $, any set $ A $ can be an algebra, setting $ \alpha_n(i, a) = a_i $.
    \end{example}
  \end{frame}

  \begin{frame}
    \frametitle{Abstract Clone Algebra - Example}

    \abstractCloneDefinition

    \begin{example}
      For the clone with $ C(n) $ the free monoid on $ n $ generators, the algebras are exactly the monoids. Setting $ a \star b := \alpha_2(x_1 \star x_2, (a, b)) $.
    \end{example}
  \end{frame}

  \begin{frame}
    \frametitle{Abstract Clone Algebra - Properties}

    \begin{example}
      An algebra $ A $ for the lambda calculus clone $ \Lambda $ gets a lot of structure. For each term $ t \in \Lambda(n) $ and $ a \in A^n $, we have an interpretation $ t(a) \in A $.

      \pause

      $ A $ has function application $ a b := \alpha_2(x_1 x_2, (a, b)) $.

      \pause

      And lambda abstraction $ \lambda x, a x := \alpha_1((\lambda x, x_1 x), (a)) $.

      \pause

      It can inherit beta and eta reduction:
      \begin{align*}
          (\lambda x, a x)(b) &= \alpha_2(x_1 x_2, (\alpha_1((\lambda x, x_1 x), (a)), b))\\
          &= \alpha_2(x_1 x_2, (\alpha_2((\lambda x, x_1 x), (a, b)), \alpha_2(x_2, (a, b))))\\
          &= \alpha_2((x_1 x_2) \bullet ((\lambda x, x_1 x), x_2), (a, b))\\
          &= \alpha_2((\lambda x, x_1 x)(x_2), (a, b))\\
          &= \alpha_2(x_1 x_2, (a, b))\\
          &= a b.\\
      \end{align*}
    \end{example}
  \end{frame}

  \begin{frame}
    \frametitle{$ \lambda $-clone}

    \lambdaCloneDefinition

    Given a $ \lambda $-clone $ \mathcal L $, we can interpret a term $ t $ of the lambda calculus (that has a context $ \Gamma $ of length $ n $) as an element $ \llbracket t \rrbracket \in \mathcal L(n) $.
  \end{frame}

  \begin{frame}
    \frametitle{$ \lambda $-clone - Example}

    \lambdaCloneDefinition

    \begin{example}
      If we have an object $ X $ in a category with products, and a retraction $ X \to (X \to X) $, the endomorphism clone of $ X $ is a $ \lambda $-clone.
    \end{example}
  \end{frame}

  \begin{frame}
    \frametitle{$ \lambda $-clone - Example}

    \lambdaCloneDefinition

    \begin{example}
      The lambda calculus clone $ \Lambda $ is a $ \lambda $-clone, with $ \lambda_n f = \lambda x_{n + 1}, f $ and $ \rho_n(f) = f x_{n + 1} $.

      It is the initial $ \lambda $-clone, so any algebra for a $ \lambda $-clone is a "$ \Lambda $-algebra".
    \end{example}
  \end{frame}

  \begin{frame}
    \frametitle{$ \lambda $-clone - Properties}

    \lambdaCloneDefinition

    Any algebra $ (A, (\alpha_i)_i) $ for a $ \lambda $-theory can be interpreted as a $ \Lambda $-algebra.
  \end{frame}

  \begin{frame}
    work in progress
  \end{frame}

  \begin{frame}
    \frametitle{Preliminaries to the main theorem}

    For a $ \lambda $-theory $ \mathcal L $, $ \mathcal L(0) $ is a $ \mathcal L $-algebra and therefore a $ \Lambda $-algebra.

    \vfill
    \pause

    Given a $ \Lambda $-algebra $ A $. Take $ \Lambda_A(n) = A + \Lambda(n) $, as a coproduct of $ \Lambda $-algebras, defined as a coend of sets
    \[ A + B = \int^{m, n} \mathop{Alg_\Lambda}(\Lambda(m), A) \times \mathop{Alg_\Lambda}(\Lambda(n), B) \times \Lambda (m + n) \]

    \vfill
    \pause

    \begin{lemma}
      $ \Lambda_A $ is a $ \lambda $-theory.
    \end{lemma}
    \begin{proof}
      We can identify $ \Lambda_A $ with $ \mathcal U_A $.

      We have a retraction $ U_A \to U_A^{U_A} $. Composition with this gives a retraction $ \mathcal U_A(n) \to \mathcal U_A(n + 1) $.
    \end{proof}
  \end{frame}

  \begin{frame}
    \frametitle{$ \mathcal U_A $}
    Let $ A $ be a $ \Lambda $-algebra.

    \begin{definition}
      We define a monoid $ M_A = (\{ a \in A \mid \mathbf 1 a = a \}, \circ) $ with $ \mathbf 1 := \lambda x y, x y = \alpha_0((\lambda x y, x y), ()) $ and $ a \circ b := \alpha_2((\lambda x, x_1 (x_2 x)), (a, b)) $.
    \end{definition}

    \pause
    \vfill

    \begin{definition}
      We define $ P(A) $ to be the category of presheaves on the category $ M_A $. This has `universal objec t' $ U_A = M_A $ with the obvious right action of $ M_A $.
    \end{definition}

    \pause
    \vfill

    \begin{lemma}
      We have a retraction $ U_A \to (U_A \to U_A) $.
    \end{lemma}
    \begin{proof}
      We can identify $ U_A \to U_A $ with $ \{ a \in A \mid \mathbf 1_2 a = a \} $ with $ \mathbf 1_2 := \lambda x y_1 y_2, (x y_1) y_2 $.

      Composition on the left with $ \mathbf 1 $ gives the retraction.
    \end{proof}

    \pause
    \vfill

    \begin{definition}
      We take $ \mathcal U_A $ to be the endomorphism theory of the reflexive universal $ U_A \in P(A) $.
    \end{definition}
  \end{frame}

  \begin{frame}
    \frametitle{The Main Theorem of the Lambda Calculus}
    \begin{theorem}
      There is an adjoint equivalence $ \mathcal L \mapsto \mathcal L(0) $ and $ A \mapsto \Lambda_A $ between $ \lambda $-theories and  $ \Lambda $-algebras.

      In particular, each $ \lambda $-theory $ \mathcal L $ is isomorphic to the theory of extensions of its initial algebra $ \mathcal L(0) $.
    \end{theorem}
  \end{frame}

  \begin{frame}
    \frametitle{Conclusion}

    In this framework, we can study the denotations/interpretations/models for the lambda calculus by studying the $ \lambda $-theories.
  \end{frame}

\end{document}
