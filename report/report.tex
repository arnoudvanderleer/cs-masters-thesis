\documentclass[a4paper]{amsbook}

\usepackage{hyperref}
\usepackage{tikz-cd}

\title{Semantics for the $ \lambda $-calculus}

\newtheorem{lemma}{Lemma}
\newtheorem{theorem}{Theorem}
\newtheorem{corollary}{Corollary}

\theoremstyle{definition}
\newtheorem{definition}{Definition}

\theoremstyle{remark}
\newtheorem{remark}{Remark}

\newcommand{\id}[1]{\ensuremath{\text{id}_{#1}}}
\newcommand{\op}[1]{\ensuremath{#1^{\text{op}}}}
\newcommand\SET{\mathbf{SET}}
\newcommand\TODO{\textbf{(TODO)} }

\begin{document}
  \maketitle

  \tableofcontents

  \chapter{Definitions}

  \section{Algebraic Theories}

  \begin{definition}[algebraic theory]
    We define an algebraic theory $ T $ to be a sequence of sets $ T_n $ indexed over $ \mathbb N $ with for all $ 1 \leq i \leq n $ elements ("variables" or "projections") $ x_{n, i}: T_n $ (we usually leave $ n $ implicit), together with a substitution operation
    \[ \_ \bullet \_: T_m \times T_n^m \to T_n \]
    for all $ m, n $, such that
    \begin{align*}
      x_j \bullet g &= g_j\\
      f \bullet (x_{l, i})_i &= f\\
      (f \bullet g) \bullet h &= f \bullet (g_i \bullet h)_i
    \end{align*}
    for all $ 1 \leq j \leq l $, $ f: T_l $, $ g: T_m^l $ and $ h: T_n^m $.
  \end{definition}

  \begin{definition}[algebraic theory morphism]
    A morphism $ F $ between algebraic theories $ T $ and $ T^\prime $ is a sequence of functions $ F_n: T_n \to T^\prime_n $ (we usually leave the $ n $ implicit) such that
    \begin{align*}
      F_n(x_j) &= x_j\\
      F_n(f \bullet g) &= F_m(f) \bullet (F_n(g_i))_i
    \end{align*}
    for all $ 1 \leq j \leq n $, $ f: T_m $ and $ g: T_n^m $.
  \end{definition}

  \begin{remark}
    We can construct binary products of algebraic theories, with sets $ (T \times T^\prime)_n = T_n \times T^\prime_n $, variables $ (x_i, x_i) $ and substitution
    \[ (f, f^\prime) \bullet (g, g^\prime) = (f \bullet g, f^\prime \bullet g^\prime). \]
    In the same way, the category of algebraic theories has all limits.
  \end{remark}

  \section{Algebras}

  \begin{definition}[algebra]
    An algebra $ A $ for an algebraic theory $ T $ is a set $ A $, together with an action
    \[ \bullet: T_n \times A^n \to A \]
    for all $ n $, such that
    \begin{align*}
      x_j \bullet a &= a_j\\
      (f \bullet g) \bullet a &= f \bullet (g_i \bullet a)_i
    \end{align*}
    for all $ j $, $ f: T_m $, $ g: T_n^m $ and $ a: A^n $.
  \end{definition}

  \begin{definition}[algebra morphism]
    For an algebraic theory $ T $, a morphism $ F $ between $ T $-algebras $ A $ and $ A^\prime $ is a function $ F: A \to A $ such that
    \[ F(f \bullet a) = f \bullet (F(a_i))_i \]
    for all $ f: T_n $ and $ a: A^n $.
  \end{definition}

  \begin{remark}
    The category of algebras has all limits. The set of a limit of algebras is the limit of the underlying algebras.
  \end{remark}

  \begin{remark}
    Note that for an algebraic theory $ T $, the $ T_n $ are all algebras for $ T $, with the action given by $ \bullet $.
  \end{remark}

  \section{Presheaves}

  \begin{definition}[presheaf]
    A presheaf $ P $ for an algebraic theory $ T $ is a sequence of sets $ P_n $ indexed over $ \mathbb N $, together with an action
    \[ \bullet: P_m \times T_n^m \to P_n \]
    for all $ m, n $, such that
    \begin{align*}
      t \bullet (x_{l, i})_i &= t\\
      (t \bullet f) \bullet g &= t \bullet (f_i \bullet g)_i
    \end{align*}
    for all $ t: P_l $, $ f: T_m^l $ and $ g: T_n^m $.
  \end{definition}

  \begin{definition}[presheaf morphism]
    For an algebraic theory $ T $, a morphism $ F $ between $ T $-presheaves $ P $ and $ P^\prime $ is a sequence of functions $ F_n: P_n \to P^\prime_n $ such that
    \[ F_n(t \bullet f) = F_m(t) \bullet f \]
    for all $ t: P_m $ and $ f: T_n^m $.
  \end{definition}

  We will write $ PT $ for the category of $ T $-presheaves and their morphisms.

  \begin{remark}
    The category of presheaves has all limits. The $ n $th set $ \overline{P}_n $ of a limit $ \overline{P} $ of presheaves $ P_i $ is the limit of the $ n $th sets $ P_{i, n} $ of the presheaves in the limit diagram.
  \end{remark}

  \section{\texorpdfstring{$ \lambda $-}{lambda-}theories}

  \begin{definition}[$ \lambda $-theory]
    A $ \lambda $-theory is an algebraic theory $ L $, together with sequences of functions $ \lambda_n: L_{n + 1} \to L_n $ and $ \rho_n: L_n \to L_{n + 1} $, such that
    \begin{align*}
      \lambda(f) \bullet g &= \lambda(f \bullet (g_1, \dots, g_m, x_{n + 1}))\\
      \rho(f \bullet g) &= \rho(f) \bullet (g_1, \dots, g_m, x_{n + 1})\\
    \end{align*}
    for all $ f: L_m $ and $ g: L_n^m $. (\TODO: Fix)
  \end{definition}

  \begin{definition}[$ \beta $- and $ \eta $-equality]
    We say that a $ \lambda $-theory $ L $ satisfies $ \beta $-equality (or that it is a $ \lambda $-theory with $ \beta $) if $ \rho_n \circ \lambda_n = \id{L_n} $ for all $ n $. We say that is satisfies $ \eta $-equality if $ \lambda_n \circ \rho_n = \id{L_{n + 1}} $ for all $ n $.
  \end{definition}

  \begin{definition}[$ \lambda $-theory morphism]
    A morphism $ F $ between $ \lambda $-theories $ L $ and $ L^\prime $ is an algebraic theory morphism $ F $ such that
    \begin{align*}
      F_n(\lambda_n(f)) &= \lambda_n(F_{n + 1}(f))\\
      \rho_n(F_n(g)) &= F_{n + 1}(\rho_n(g))
    \end{align*}
    for all $ f: L_{n + 1} $ and $ g: L_n $.
  \end{definition}

  \begin{remark}
    The category of lambda theories has all limits, with the underlying algebraic theory of a limit being the limit of the underlying algebraic theories.
  \end{remark}

  A $ \lambda $-theory algebra or presheaf is a presheaf for the underlying algebraic theory.

  \chapter{Lemmas}

  \section{The endomorphism theory}

  \begin{definition}[Endomorphism theory]
    Suppose that we have a category $ C $ and an object $ X: C $, such that all powers $ X^n $ of $ X $ are also in $ C $.
    The endomorphism theory $ E(X) $ of $ X $ is the algebraic theory given by $ E(X)_n = C(X^n, X) $ with projections as variables $ x_{n, i}: X^n \to X $ and a substitution that sends $ f: X^m \to X $ and $ g_1, \dots, g_m: X^n \to X $ to $ f \circ \langle g_i \rangle_i: X^n \to X^m \to X $.
  \end{definition}

  \begin{lemma}
    $ E(X) $ is indeed an algebraic theory.
  \end{lemma}
  \begin{proof}
    For $ 1 \leq j \leq l $, $ f: E(X)_l $, $ g: {E(X)_m}^l $ and $ h: {E(X)_n}^m $, we have
    \[ x_j \bullet g = x_j \circ \langle g_i \rangle_i = g_j, \]
    \[ f \bullet (x_{l,i})_i = f \circ \langle x_{l, i} \rangle_i = f \circ \id{X^l} = f \]
    and
    \[ (f \bullet g) \bullet h = f \circ \langle g_i \rangle_i \circ \langle h_i \rangle_i = f \circ \langle g_i \circ \langle h_{i^\prime} \rangle_{i^\prime} \rangle_i = f \bullet (g_i \bullet h)_i. \]
  \end{proof}

  \begin{definition}[Endomorphism $ \lambda $-theory]
    Now, suppose that the exponential object $ X^X $ exists, and that we have morphisms back and forth $ abs: X^X \to X $ and $ app: X \to X^X $. Let, for $ Y: C $, $ \varphi_Y $ be the isomorphism $ C(X \times Y, X) \xrightarrow{\sim} C(Y, X^X) $.
    We can give $ E(X) $ a $ \lambda $-theory structure by setting, for $ f: E(X)_{n + 1} $ and $ g: E(X)_n $,
    \[ \lambda(f) = abs \circ \varphi_{X^n}(f) \qquad \rho(g) = \varphi_{X^n}^{-1}(app \circ g). \]
  \end{definition}

  \begin{lemma}
    $ E(X) $ is indeed a $ \lambda $-theory.
  \end{lemma}
  \begin{proof}
    Note that $ \varphi: C(- \times X, X) \xrightarrow{\sim} C(-, X^X) $ is a natural isomorphism, so for $ g: {E(X)_n}^m $, the following diagram commutes
    \begin{center}
      \begin{tikzcd}[column sep = 1in]
        C(X^m \times X, X) \arrow[r, "- \circ (\langle g_i \rangle_i \times \id X)"]\arrow[d, "\varphi_{X^m}", bend left] & C(X^n \times X, X^X) \arrow[d, "\varphi_{X^n}", bend left]\\
        C(X^m, X^X) \arrow[r, "- \circ \langle g_i \rangle_i"] \arrow[u, "\varphi_{X^m}^{-1}", bend left] & C(X^n, X^X) \arrow[u, "\varphi_{X^n}^{-1}", bend left]
      \end{tikzcd}
    \end{center}
    and note that $ \langle g_i \rangle_i \times \id X = \langle g_1, \dots, g_m, x_{n + 1} \rangle $. Then we have, for all $ f: E(X)_m $
    \begin{align*}
      \lambda_m(f) \bullet g &= abs \circ \varphi_{X^m}(f) \circ \langle g_i \rangle_i\\
      &= abs \circ \varphi_{X^n}(f \circ \langle g_1, \dots, g_m, x_{n + 1} \rangle)\\
      &= \lambda_n(f \bullet (g_1, \dots, g_m, x_{n + 1}))
    \end{align*}
    and
    \begin{align*}
      \rho_n(f \bullet g) &= \varphi_{X^n}^{-1}(app \circ f \circ \langle g_i \rangle_i)\\
      &= \varphi_{X^m}^{-1}(app \circ f) \circ \langle g_1, \dots, g_m, x_{n + 1} \rangle\\
      &= \rho_m(f) \bullet (g_1, \dots, g_m, x_{n + 1}).
    \end{align*}
  \end{proof}

  \section{The theory presheaf}

  \begin{definition}[The theory presheaf]
    Let $ T $ be an algebraic theory. We can turn $ T $ into an $ T $-presheaf $ \tilde T $ by setting $ \tilde T_n = T_n $ and using the substitution from $ T $:
    \[ \bullet: \tilde T_m \times T_n^m \to \tilde T_n. \]
  \end{definition}

  \begin{lemma}
    $ \tilde T $ is indeed a presheaf.
  \end{lemma}
  \begin{proof}
    For all $ t: \tilde T_l $, $ f: T_m^l $ and $ g: T_n^m $,
    \[ t \bullet (x_{l, i})_i = t \]
    and
    \[ (t \bullet f) \bullet g = t \bullet (f_i \bullet g)_i \]
    because $ T $ is an algebraic theory.
  \end{proof}

  \begin{lemma}
    Given an algebraic theory $ T $ and a $ T $-presheaf $ Q $, we have for all $ n $ a bijection of sets
    \[ \varphi: PT(\tilde T^n, Q) \cong Q_n. \]
  \end{lemma}
  \begin{proof}
    Take $ \varphi(f) = f_n(x_1, \dots, x_n) $.

    Conversely, take $ \varphi^{-1}(q) $ to be the presheaf morphism that sends $ t: T^n_m $ to $ q \bullet t : Q_m $. This is indeed a presheaf morphism, since for all $ t: T^n_l $ and $ f: T^l_m $,
    \[ \varphi^{-1}(q)(t \bullet f) = q \bullet t \bullet f = \varphi^{-1}(q)(t) \bullet f. \]

    Now, for a presheaf morphism $ f: T^n \to Q $ and $ t: T^n_m $, we have
    \[ \varphi^{-1}(\varphi(f))(t) = f_n(x_1, \dots, x_n) \bullet t = f_n((x_1, \dots, x_n) \bullet t) = f_n(t_1, \dots, t_n) = f_n(t). \]

    Conversely, given $ q: Q_n $, we have
    \[ \varphi(\varphi^{-1}(q)) = q \bullet (x_1, \dots, x_n) = q. \]
    which concludes the proof.
  \end{proof}

  \section{The `+l' presheaf}

  Let $ \iota_{m, n} : T_m \to T_{m + n} $ denote the function that sends $ f $ to $ f \bullet (x_{m + n, 1}, \dots, x_{m + n, m}) $. Note that
  \[ \iota_{m, n}(f) \bullet g = f \bullet (g_i)_{i \leq m} \]
  and
  \[ \iota_{m, n}(f \bullet g) = f \bullet g \bullet (x_i)_i = f \bullet (g_i \bullet (x_j)_j)_i = f \bullet (\iota_{m, n}(g_i))_i. \]

  For tuples $ x : X^m $ and $ y: X^n $, let $ x + y $ denote the tuple $ (x_1, \dots, x_m, y_1, \dots, y_n) : X^{m + n} $.

  \begin{definition}[The `+l' presheaf]
    Given a $ T $-presheaf $ Q $, we can construct a presheaf $ A(Q, l) $, given by $ A(Q, l)_n = Q_{n + l} $. Then, for $ q: A(Q, l)_m $ and $ f: T_n^m $, the substitution is given by
    \[ q \bullet_{A(Q, l)} f = q \bullet_Q ((\iota_{n, l} (f_i))_i + (x_{n + i})_i) \]
  \end{definition}
  \begin{lemma}
    The +l presheaf is a presheaf
  \end{lemma}
  \begin{proof}
    We have, for $ q: A(Q, l)_n $,
    \begin{align*}
      q \bullet_{A(Q, l)} (x_i)_i &= q \bullet_Q ((\iota_{n, l}(x_i))_i + (x_{n + i})_i)\\
      &= q \bullet_Q ((x_i)_i + (x_{n + i})_i)\\
      &= q \bullet_Q (x_i)_i\\
      &= q.
    \end{align*}
    We have, for $ q : A(Q, k)_l $, $ f: T_m^l $ and $ g: T_n^m $,
    \begin{align*}
      q \bullet_{A(Q, k)} f \bullet_{A(Q, k)} g &= q \bullet_Q ((\iota_{m, l}(f_i))_i + (x_{m + i})_i) \bullet_Q ((\iota_{n, l}(g_i))_i + (x_{n + i})_i)\\
      &= q \bullet_Q (((\iota_{m, l}(f_i) \bullet_T ((\iota_{n, l}(g_j))_j + (x_{n + j})_j))_i + (x_{m + i} \bullet_T ((\iota_{n, l}(g_j))_j + (x_{n + j})_j))_i))\\
      &= q \bullet_Q ((f_i \bullet_T (\iota_{n, l}(g_j))_j)_i + (x_{n + i})_i)\\
      &= q \bullet_Q ((\iota_{n, l}(f_i \bullet_T g))_i + (x_{n + i})_i)\\
      &= q \bullet_{A(Q, k)} (f_i \bullet_T g).
    \end{align*}
  \end{proof}

  \section{Exponentiability of the theory presheaf}

  \begin{lemma}
    For all $ l $, the presheaf $ \tilde T^l $ is exponentiable.
  \end{lemma}
  \begin{proof}
    We will show that $ A(-, l) $ constitutes a right adjoint to the functor $ - \times \tilde T^l $. We will do this using universal arrows (\cite{MacLane}, Chapter IV.1, Theorem 2 (iv)). To that end, we will need for all $ Q: PT $ a universal arrow $ \varphi: A(Q, l) \times \tilde T^l \to Q $.

    For $ q: A(Q, l)_n = Q_{n + l} $ and $ t: \tilde T^l_n $, we take $ \varphi(q, t) = q \bullet_Q ((x_{n, i})_i + t) $.

    This is a presheaf morphism, since for all $ q: A(Q, l)^l_m $, $ t: \tilde T^l_m $ and $ f: T_n^m $,
    \begin{align*}
      \varphi((q, t) \bullet_{A(Q, l) \times \tilde \tilde T^l} f) &= \varphi(q \bullet_{A(Q, l)} f, t \bullet_{\tilde T^l} f)\\
      &= q \bullet_{A(Q, l)} f \bullet_Q ((x_i)_i + (t \bullet_{\tilde T^l} f))\\
      &= q \bullet_Q ((\iota_{n, l}(f_i))_i + (x_{n + i})_i) \bullet_Q ((x_i)_i + (t \bullet_{\tilde T^l} f))\\
      &= q \bullet_Q ((\iota_{n, l}(f_i) \bullet_T ((x_j)_j + (t \bullet_{\tilde T^l} f)))_i + (x_{n + i} \bullet_T ((x_j)_j + (t \bullet_{\tilde T^l} f)))_i)\\
      &= q \bullet_Q ((f_i \bullet_T (x_j)_j)_i + ((t \bullet_{\tilde T^l} f)_i)_i)\\
      &= q \bullet_Q ((f_i)_i + (t_i \bullet_{\tilde T} f)_i)\\
      &= q \bullet_Q ((x_i \bullet_T f)_i + (t_i \bullet_T f)_i)\\
      &= q \bullet_Q ((x_i)_i + t) \bullet_Q f\\
      &= \varphi(q, t) \bullet_Q f.
    \end{align*}

    Now, given any presheaf $ Q^\prime : P T $ we need to show that any morphism $ \psi: Q^\prime \times \tilde T^l \to Q $ factors uniquely as $ \varphi \circ (\tilde \psi \times \id{\tilde T^l}) $ for some $ \tilde \psi: Q^\prime \to A(Q, l) $.

    So, given such a $ \psi $, and given $ q: Q^\prime_n $, we take $ \tilde \psi(q) = \psi(\iota_{n, l}(q), (x_{n + i})_i) $

    This is a presheaf morphism, since for all $ q: Q^\prime_m $ and $ f : T_n^m $,
    \begin{align*}
      \tilde \psi(q \bullet f) &= \psi(\iota_{n, l}(q \bullet f), (x_{n + i})_i)\\
      &= \psi(q \bullet (\iota_{n, l}(f_i))_i, (x_{n + i})_i)\\
      &= \psi((\iota_{m, l}(q), (x_{m + i})_i) \bullet_{Q^\prime \times \tilde T^l} ((\iota_{n, l}(f_i))_i + (x_{n + i})_i))\\
      &= \psi(\iota_{m, l}(q), (x_{m + i})_i) \bullet_Q ((\iota_{n, l}(f_i))_i + (x_{n + i})_i)\\
      &= \tilde \psi(q) \bullet_{A(Q, l)} f.
    \end{align*}
    Note that indeed $ \varphi \circ (\tilde \psi \times \id{\tilde T^l}) = \psi $:
    \begin{align*}
      \varphi(\tilde \psi(q), t) &= \varphi(\psi(\iota_{n, l}(q), (x_{n + i})_i), t)\\
      &= \psi(\iota_{n, l}(q), (x_{n + i})_i) \bullet ((x_i)_i + t)\\
      &= \psi(\iota_{n, l}(q) \bullet ((x_i)_i + t), (x_{n + i})_i \bullet ((x_i)_i + t))\\
      &= \psi(q \bullet (x_i)_i, (t_i)_i)\\
      &= \psi(q, t).\\
    \end{align*}
    Now, suppose that we have another $ \tilde \psi^\prime : Q^\prime \to A(Q, l) $ such that $ \varphi \circ (\tilde \psi^\prime \times \id{\tilde T^l}) = \psi $. Then we have
    \begin{align*}
      \tilde \psi(q) &= \psi(\iota_{n, l}(q), (x_{n + i})_i)\\
      &= (\varphi \circ (\tilde \psi^\prime \times \id{\tilde T^l}))(\iota_{n, l}(q), (x_{n + i})_i)\\
      &= \varphi(\tilde \psi^\prime(\iota_{n, l}(q)), (x_{n + i})_i)\\
      &= \tilde \psi^\prime(\iota_{n, l}(q)) \bullet ((x_i)_i + (x_{n + i})_i)\\
      &= \iota_{n, l}(\tilde \psi^\prime(q)) \bullet ((x_i)_i + (x_{n + i})_i)\\
      &= \tilde \psi^\prime(q) \bullet (x_i)_i\\
      &= \tilde \psi^\prime(q),
    \end{align*}
    so $ \tilde \psi $ is unique, which completes the proof.
  \end{proof}

  Now, this adjunction $ - \times \tilde T^l \dashv A(-, l) $ induces a natural isomorphism
  \[ \varphi: PT(- \times \tilde T^l, \tilde T) \xrightarrow{\sim} PT(-, A(\tilde T, l)) \]
  \begin{lemma}
    For all $ f: PT(\tilde T^n \times \tilde T^l, \tilde T) $,
    \[ \varphi_{\tilde T^n}(f)(q) = f(\iota_{m, l}(q), (x_{m + i})_i) \]
  \end{lemma}
  \begin{proof}
    \TODO
  \end{proof}

  \begin{lemma}
    For all $ f: PT(\tilde T^n, A(\tilde T, l)) $,
    \[ \varphi_{\tilde T^n}^{-1}(f)(q, t) = f(q) \bullet ((x_i)_i + t). \]
  \end{lemma}
  \begin{proof}
    \TODO
  \end{proof}

  \chapter{Theorems}

  \section{Scott's Representation Theorem}
  \begin{theorem}
    Any $ \lambda $-theory $ L $ is isomorphic to the endomorphism $ \lambda $-theory $ E(\tilde L) $ of $ \tilde L $ in the presheaf category of $ L $.
  \end{theorem}
  \begin{proof}
    First of all, remember that $ \tilde L $ is indeed exponentiable and that $ \tilde L^{\tilde L} = A(\tilde L, 1) $.
    Now, since $ L $ is a $ \lambda $-theory, we have functions back and forth $ \lambda: A(\tilde L, 1) \to \tilde L $ and $ \rho: \tilde L \to A(\tilde L, 1) $. These are presheaf morphisms because for all $ f: A(\tilde L, 1)_m $ and $ g: \tilde L_m $ and $ t: T_n^m $,
    \[ \lambda(f \bullet_{A(\tilde L, 1)} t) = \lambda(f \bullet_{\tilde L} ((\iota_{m, 1}(t_i))_i + (x_{n + 1}))) = \lambda(f) \bullet_{\tilde L} t \]
    and
    \[ \rho(g \bullet_{\tilde L} t) = \rho(g) \bullet_{\tilde L} ((\iota_{m, 1}(t_i))_i + (x_{n + 1})) = \rho(g) \bullet_{A(\tilde L, 1)} t. \]
    Therefore, $ E(\tilde L) $ is indeed a $ \lambda $-theory.

    For any presheaf $ Q $ and for any $ n $, we have a bijection $ PL(L^n, Q) \cong Q_n $.
    Then we have $ \varphi: E(\tilde L)_n \cong L_n $.
    This bijection is an isomorphism of $ \lambda $-theories, since it preserves the $ x_i $, $ \bullet $, $ \rho $ and $ \lambda $: for all $ 1 \leq j \leq n $, $ f: E(\tilde L)_m $, $ g: E(\tilde L)_{m + 1} $ and $ h: E(\tilde L)_n^m $.
    \begin{align*}
      \varphi(x_j) &= x_j(x_1, \dots, x_n)\\
      &= x_j;\\
      \varphi(f \bullet h) &= f \circ \langle h_i \rangle_i((x_i)_i)\\
      &= f((h_i((x_j)_j))_i)\\
      &= f((x_i)_i \bullet (h_i((x_j)_j))_i)\\
      &= f((x_i)_i) \bullet (h_i((x_j)_j))_i\\
      &= \varphi(f) \bullet (\varphi(h_i))_i;\\
      \varphi(\rho(f)) &= \rho(f)((x_i)_i)\\
      &= \rho(f((x_i)_i)) \bullet (x_i)_i\\
      &= \rho(f((x_i)_i))\\
      &= \rho(\varphi(f));\\
      \varphi(\lambda(g)) &= \lambda(g)((x_i)_i)\\
      &= \lambda(\varphi_{X^n}(g)((x_i)_i))\\
      &= \lambda(g(\iota_{m, l}((x_i)_i) + (x_{m + 1})))\\
      &= \lambda(g((x_i)_i))\\
      &= \lambda(\varphi(g)).
    \end{align*}
  \end{proof}

  \section{Locally cartesian closedness of the category of retracts}
  \begin{definition}[Category of retracts]
    The category of retracts for a $ \lambda $-theory $ L $ is the category with objects $ f: L_n $ such that $ f \bullet f = f $ and it has as morphisms $ g: f \to f^\prime $ the terms $ g: L_n $ such that $ f^\prime \bullet g \bullet f = g $. The object $ f: L_n $ has identity element $ f $, and we have composition $ g \circ g^\prime = g \bullet g^\prime $. These are morphisms \TODO
  \end{definition}

  \begin{lemma}
    The category of retracts is indeed a category.
  \end{lemma}
  \begin{proof}
    \TODO
  \end{proof}

  \begin{theorem}
    The category of retracts is locally cartesian closed \TODO.
  \end{theorem}

  \section{The Fundamental Theorem of the \texorpdfstring{$ \lambda $-}{lambda }calculus}

  \begin{definition}[$ \Lambda $]
    There is a special $ \lambda $-theory, given by the $ \lambda $-calculus itself. $ \Lambda_n $ is the set of $ \lambda $-terms with $ n $ free variables, the $ x_i $ are the free variables, and $ \bullet $ is given by substitution. $ \lambda $ sends $ f: \Lambda_{n + 1} $ to $ \lambda x_{n + 1}, f $ and $ \rho $ sends $ f: \Lambda_n $ to $ \iota_{n, 1}(f) x_{n + 1} $ in $ \Lambda_n $.
  \end{definition}

  \begin{lemma}
    $ \Lambda $ is indeed a $ \lambda $-theory.
  \end{lemma}
  \begin{proof}
    \TODO
  \end{proof}

  \begin{lemma}
    $ \Lambda $ is the initial $ \lambda $-theory.
  \end{lemma}
  \begin{proof}
    Given a $ \lambda $-theory $ L $, we construct a morphism $ f: \Lambda \to L $ by induction on the $ \lambda $-terms. We set $ f(x_i) = x_i $, $ f(\lambda(t)) = \lambda(f(t)) $ and $ f(st) = \rho(f(s)) \bullet ((x_i)_i + (f(t))) $.

    This is a $ \lambda $-theory morphism because \TODO

    It is unique, since \TODO
  \end{proof}

  \begin{definition}[Pullback of algebras]
    If we have a morphism of algebraic theories $ f: T^\prime \to T $, we have a functor $ AT \to AT^\prime $.

    On objects, it sends a $ T $-algebra $ A $ to a $ T^\prime $-algebra with set $ A $ and action $ g \bullet_{T^\prime} a = f(g) \bullet_T a $. This is a $ T^\prime $-algebra because \TODO.

    On morphisms, it sends $ \varphi: A \to A $ to $ \varphi: A \to A $. This is a $ T^\prime $-algebra morphism because for all $ g: T^\prime_n $ and $ a: A^n $, we have
    \[ \varphi(g \bullet_{T^\prime} a) = \varphi(f(g) \bullet_T a) = f(g) \bullet_T \varphi(a) = g \bullet_{T^\prime} \varphi(a). \]
  \end{definition}
  \begin{lemma}
    This is indeed a functor.
  \end{lemma}
  \begin{proof}
    \TODO
  \end{proof}

  \begin{definition}[Term algebra]
    Given an algebraic theory $ T $, for every $ n $, $ T_n $ together with the action operator $ \bullet: T_m \times T_n^m \to T_n $ gives a $ T $-algebra.
  \end{definition}

  \begin{lemma}
    $ T_n $ is indeed a $ T $-algebra.
  \end{lemma}
  \begin{proof}
    \TODO
  \end{proof}

  \begin{definition}
    For all $ n $, we have a functor from lambda theories to $ \Lambda $-algebras. It sends the $ \lambda $-theory $ L $ to the $ L $-algebra $ L_n $ and then turns this into a $ \Lambda $-algebra via the morpism $ \Lambda \to L $.

    It sends morphisms $ f: L \to L^\prime $ to $ f_n : L_n \to L^\prime_n $. This is a $ \Lambda $-algebra morphism because \TODO
  \end{definition}

  \begin{lemma}
    This indeed constitutes a functor.
  \end{lemma}
  \begin{proof}
    \TODO
  \end{proof}

  \begin{remark}
    Note that for a monoid $ M $, if we view $ M $ as a category, the category $ [\op{M}, \SET] $ consists of sets with a right $ M $-action.
  \end{remark}

  \begin{definition}[The exponential object in the presheaf category]
    Given a monoid $ M $, if we have two presheaves (sets with right $ M $-actions) $ P $ and $ P^\prime $, we have a set of $ M $-equivariant maps
    \[ F_{P, P^\prime} = \left\{ f: M \times P \to P^\prime \mid \prod_{p : P, m, m^\prime: M} f(m, p)m^\prime = f(m m^\prime, p m^\prime) \right\} \]
    with a right $ M $-action, given by $ f m^\prime(m, p) = f(m^\prime m, p) $. This is again $ M $-equivariant because
    \[ fm^\prime(m, p)m^{\prime \prime} = f(m^\prime m, p)m^{\prime \prime} = f(m^\prime m m^{\prime \prime}, p m^{\prime \prime}) = f m^\prime(m m^{\prime \prime}, p m^{\prime \prime}), \]
    so $ F_{P, P^\prime} $ is a presheaf.

    Now, to show that $ F_{P, P^\prime} $ is the exponential object $ {P^\prime}^P $, we show that for any $ P $, $ F_{P, -} $ is the left adjoint of $ - \times P $. So we need for all $ P^\prime: PT $, a universal arrow $ \varphi: F_{P, P^\prime} \times P \to P^\prime $.

    First of all, we have an evaluation map $ \varphi: F_{P, P^\prime} \times P \to P^\prime $ given by $ (f, p) \mapsto f(I, p) $ for $ I $ the unit of the monoid. This map is equivariant because for all $ m $,
    \[ (f, p) m = (f m, p m) \mapsto f m(I, p m) = f(m, p m) = f(I, p) m. \]
    Now, given any presheaf $ Q $ and any morphism $ \psi: Q \times P \to P^\prime $, take $ \tilde \psi: Q \to F_{P, P^\prime} $ given by $ \psi(q)(m, p) = \psi(q m, p) $. This is equivariant because
    \[ \tilde \psi(q)m(m^\prime, p) = \tilde \psi(q)(m m^\prime, p) = \psi(q m m^\prime, p) = \tilde \psi(q m)(m^\prime, p) \]
    and we have
    \[ \varphi(\tilde \psi(q), p) = \tilde \psi(q)(I, p) = \psi(q, p). \]
    Now, suppose that we have $ \tilde \psi^\prime: Q \to F_{P, P^\prime} $ such that $ \varphi \circ (\tilde \psi^\prime \times \id{P}) = \psi $. Then for all $ q : Q $, $ m: M $ and $ p: P $,
    \[ \tilde \psi(q)(m, p) = \psi(q m, p) = \varphi(\tilde \psi^\prime(q m), p) = \tilde \psi^\prime(q m)(I, p) = \psi^\prime(q) m(I, p) = \psi^\prime(q)(m, p), \]
    so $ \tilde \psi $ is unique and $ F_{P, P^\prime} $ is an exponential object.
  \end{definition}

  \begin{definition}[n-functional terms]
    Let $ A $ be a $ \Lambda $-algebra. We define
    \[ A(n) = \{ a : A \mid (\lambda x_2 x_3 \dots x_{n + 1}, x_1 x_2 x_3 \dots x_{n + 1}) \bullet a = a \}. \]
  \end{definition}

  \begin{definition}
    Take $ \mathbf 1_n = (\lambda x_1 \dots x_n, x_1 \dots x_n) \bullet () : A $.
  \end{definition}

  \begin{definition}
    We define composition as $ a \circ b = (\lambda x_3, x_1 (x_2 x_3)) \circ (a, b) $ for $ a, b : A $.
  \end{definition}

  \begin{lemma}
    This composition is associative.
  \end{lemma}
  \begin{proof}
    \TODO
  \end{proof}

  \begin{definition}[The monoid of a $ \Lambda $-algebra]
    Now we make $ A(1) $ into a monoid with unit $ \lambda x_1, x_1 $.
  \end{definition}

  \begin{lemma}
    This is indeed a monoid.
  \end{lemma}
  \begin{proof}
    \TODO
  \end{proof}

  From here on, we will assume that $ \Lambda $ (and therefore, any $ \lambda $-theory) satisfies $ \beta $-equality.

  \begin{lemma}
    For $ a: A $, $ a $ is in $ A(n) $ iff $ \mathbf 1_n \circ a = a $.
  \end{lemma}
  \begin{proof}
    \begin{align*}
      \mathbf 1_n \circ a
      &= (\lambda x_3, x_1 (x_2 x_3)) \bullet (((\lambda x_1 \dots x_n, x_1 \dots x_n) \bullet ()), a)\\
      &= (\lambda x_3, x_1 (x_2 x_3)) \bullet (((\lambda x_2 \dots x_{n + 1}, x_2 \dots x_{n + 1}) \bullet a), x_1 \bullet a)\\
      &= ((\lambda x_3, x_1 (x_2 x_3)) \bullet ((\lambda x_2 \dots x_{n + 1}, x_2 \dots x_{n + 1}), x_1)) \bullet a\\
      &= (\lambda x_2, (\lambda x_3 \dots x_{n + 2}, x_3 \dots x_{n + 2}) (x_1 x_2)) \bullet a\\
      &= (\lambda x_2 x_3 \dots x_{n + 1}, x_1 x_2 \dots x_{n + 1}) \bullet a.
    \end{align*}
  \end{proof}

  \begin{definition}[The presheaf category of a $ \Lambda $-algebra]
    Let $ A $ be a $ \Lambda $-algebra. If we view the monoid $ A(1) $ as a one-object category, we define the category $ PA $ to be the category of presheaves $ [\op{A(1)}, \SET] $.
  \end{definition}

  \begin{definition}[The objects $ A(n) $ in $ PA $]
    Given $ a: A(n) $ and $ b: A(1) $, we have
    \[ \mathbf 1_n \circ (a \circ b) = (\mathbf 1_n \circ a) \circ b = a \circ b, \]
    so $ a \circ b: A(n) $ and we have a right $ A(1) $-action on $ A(n) $, which makes $ A(n) $ into an object in $ PA $.
  \end{definition}

  \begin{lemma}
    We have $ A(1)^{A(1)} \cong A(2) $.
  \end{lemma}
  \begin{proof}
    \TODO
  \end{proof}

  \begin{definition}[Endomorphism $ \lambda $-theory of a $ \Lambda $-algebra]
    $ PA $ borrows products from $ \SET $. Therefore, the algebraic theory $ E(A(1)) $ exists. Now note that $ A(1) $ is exponentiable and $ A(1)^{A(1)} \cong A(2) $.
    Note that $ A(2) \subseteq A(1) $ and that $ (\lambda x_2 x_3, x_1 x_2 x_3) \bullet - $ gives a function from $ A(1) $ to $ A(2) $. This gives $ E(A(1)) $ a $ \lambda $-theory structure.
  \end{definition}

  \begin{theorem}
    There exists an adjoint equivalence between the category of $ \lambda $-theories, and the category of algebras of $ \Lambda $.
  \end{theorem}
  \begin{proof}
    We will show that the functor $ L \mapsto L_0 $ is an equivalence of categories.

    It is essentially surjective, because $ L $ is isomorphic \TODO to $ E(A(1)) $.

    Now, given morphisms $ f, f^\prime: L \to L^\prime $. Suppose that $ f_0 = f^\prime_0 $. Suppose that $ L $ and $ L^\prime $ have $ \beta $-equality. Then, given $ l: L_n $, we have
    \[ f_n(l) = \rho^n(\lambda^n(f_n(l))) = \rho^n(f_0(\lambda^n(l))) = \rho^n(f^\prime_0(\lambda^n(l))) = \rho^n(\lambda^n(f^\prime_n(l))) = f^\prime_n(l), \]
    so the functor is faithful.

    The functor is full because a $ \Lambda $-algebra morphism $ f: A \to A^\prime $ induces a morphism $ f^*: PA^\prime \to PA $ \TODO, and via left Kan extension we get a left adjoint $ f_*: PA \to PA^\prime $ with $ f_*(A(1)) \cong A^\prime(1) $ \TODO. Now, $ f_* $ preserves (finite) products \TODO, so we have maps $ PA(A(1)^n, A(1)) \to PA^\prime(A^\prime(1)^n, A^\prime(1)) $ and so a map $ E(A(1)) \to E(A^\prime(1)) $ \TODO. This map, when restricted to a map $ PA(1, A(1)) \to PA^\prime(1, A(1)) $, and transported along the isomorphism $ a \mapsto a I $ \TODO, is equal to $ f $ \TODO.
  \end{proof}

  \begin{lemma}
    The category of $ T $-algebras has coproducts.
  \end{lemma}
  \begin{proof}
    \TODO
  \end{proof}

  \begin{definition}[Theory of extensions]
    Let $ T $ be an algebraic theory and $ A $ a $ T $-algebra. We can define an algebraic theory $ T_A $ called `the theory of extensions of $ A $' with $ (T_A)_n = T_n + A $. The left injection of the variables $ x_i : T_n $ gives the variables.
    Now, take $ h: (T_n + A)^m $. Sending $ g: T_m $ to $ \varphi(g) := g \bullet h $ gives a $ T $-algebra morphism $ T_m \to T_n + A $ since
    \[ \varphi(f \bullet g) = f \bullet g \bullet h = f \bullet (g_i \bullet h) = f \bullet (\varphi(g_i))_i. \]
    This, together with the injection morphism of $ A $ into $ T_n + A $, gives us a $ T $-algebra morphism from the coproduct: $ T_m + A \to T_n + A $. We especially have a function on sets $ (T_m + A) \times (T_n + A)^m \to T_n + A $, which we will define our substitution to be.
  \end{definition}

  \begin{lemma}
    $ T_A $ is indeed an algebraic theory.
  \end{lemma}
  \begin{proof}
    \TODO
  \end{proof}

  \bibliographystyle{alpha}
  \bibliography{citations}

\end{document}
