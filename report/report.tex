\documentclass[a4paper]{amsbook}

\usepackage{hyperref}
\usepackage{tikz-cd}

\title{Semantics for the $ \lambda $-calculus}

\newtheorem{lemma}{Lemma}
\newtheorem{theorem}{Theorem}
\newtheorem{corollary}{Corollary}

\theoremstyle{definition}
\newtheorem{definition}{Definition}

\theoremstyle{remark}
\newtheorem{remark}{Remark}

\newcommand{\id}[1]{\ensuremath{\text{id}_{#1}}}
\newcommand{\op}[1]{\ensuremath{#1^{\text{op}}}}
\newcommand{\Lan}[2]{\ensuremath{\text{Lan}_{#1}#2}}
\newcommand\SET{\mathbf{SET}}
\newcommand\TODO{\textbf{(TODO)} }

\begin{document}
  \maketitle

  \tableofcontents

  \chapter{Definitions}

  \section{Algebraic Theories}

  \begin{definition}[algebraic theory]
    We define an algebraic theory $ T $ to be a sequence of sets $ T_n $ indexed over $ \mathbb N $ with for all $ 1 \leq i \leq n $ elements ("variables" or "projections") $ x_{n, i}: T_n $ (we usually leave $ n $ implicit), together with a substitution operation
    \[ \_ \bullet \_: T_m \times T_n^m \to T_n \]
    for all $ m, n $, such that
    \begin{align*}
      x_j \bullet g &= g_j\\
      f \bullet (x_{l, i})_i &= f\\
      (f \bullet g) \bullet h &= f \bullet (g_i \bullet h)_i
    \end{align*}
    for all $ 1 \leq j \leq l $, $ f: T_l $, $ g: T_m^l $ and $ h: T_n^m $.
  \end{definition}

  \begin{definition}[algebraic theory morphism]
    A morphism $ F $ between algebraic theories $ T $ and $ T^\prime $ is a sequence of functions $ F_n: T_n \to T^\prime_n $ (we usually leave the $ n $ implicit) such that
    \begin{align*}
      F_n(x_j) &= x_j\\
      F_n(f \bullet g) &= F_m(f) \bullet (F_n(g_i))_i
    \end{align*}
    for all $ 1 \leq j \leq n $, $ f: T_m $ and $ g: T_n^m $.
  \end{definition}

  \begin{remark}
    We can construct binary products of algebraic theories, with sets $ (T \times T^\prime)_n = T_n \times T^\prime_n $, variables $ (x_i, x_i) $ and substitution
    \[ (f, f^\prime) \bullet (g, g^\prime) = (f \bullet g, f^\prime \bullet g^\prime). \]
    In the same way, the category of algebraic theories has all limits.
  \end{remark}

  \section{Algebras}

  \begin{definition}[algebra]
    An algebra $ A $ for an algebraic theory $ T $ is a set $ A $, together with an action
    \[ \bullet: T_n \times A^n \to A \]
    for all $ n $, such that
    \begin{align*}
      x_j \bullet a &= a_j\\
      (f \bullet g) \bullet a &= f \bullet (g_i \bullet a)_i
    \end{align*}
    for all $ j $, $ f: T_m $, $ g: T_n^m $ and $ a: A^n $.
  \end{definition}

  \begin{definition}[algebra morphism]
    For an algebraic theory $ T $, a morphism $ F $ between $ T $-algebras $ A $ and $ A^\prime $ is a function $ F: A \to A $ such that
    \[ F(f \bullet a) = f \bullet (F(a_i))_i \]
    for all $ f: T_n $ and $ a: A^n $.
  \end{definition}

  \begin{remark}
    The category of algebras has all limits. The set of a limit of algebras is the limit of the underlying sets.
  \end{remark}

  \begin{remark}
    Note that for an algebraic theory $ T $, the $ T_n $ are all algebras for $ T $, with the action given by $ \bullet $.
  \end{remark}

  \section{Presheaves}

  \begin{definition}[presheaf]
    A presheaf $ P $ for an algebraic theory $ T $ is a sequence of sets $ P_n $ indexed over $ \mathbb N $, together with an action
    \[ \bullet: P_m \times T_n^m \to P_n \]
    for all $ m, n $, such that
    \begin{align*}
      t \bullet (x_{l, i})_i &= t\\
      (t \bullet f) \bullet g &= t \bullet (f_i \bullet g)_i
    \end{align*}
    for all $ t: P_l $, $ f: T_m^l $ and $ g: T_n^m $.
  \end{definition}

  \begin{definition}[presheaf morphism]
    For an algebraic theory $ T $, a morphism $ F $ between $ T $-presheaves $ P $ and $ P^\prime $ is a sequence of functions $ F_n: P_n \to P^\prime_n $ such that
    \[ F_n(t \bullet f) = F_m(t) \bullet f \]
    for all $ t: P_m $ and $ f: T_n^m $.
  \end{definition}

  We will write $ PT $ for the category of $ T $-presheaves and their morphisms.

  \begin{remark}
    The category of presheaves has all limits. The $ n $th set $ \overline{P}_n $ of a limit $ \overline{P} $ of presheaves $ P_i $ is the limit of the $ n $th sets $ P_{i, n} $ of the presheaves in the limit diagram.
  \end{remark}

  \section{\texorpdfstring{$ \lambda $-}{lambda-}theories}

  \begin{definition}[$ \lambda $-theory]
    A $ \lambda $-theory is an algebraic theory $ L $, together with sequences of functions $ \lambda_n: L_{n + 1} \to L_n $ and $ \rho_n: L_n \to L_{n + 1} $, such that
    \begin{align*}
      \lambda_m(f) \bullet h &= \lambda_n(f \bullet (h_1, \dots, h_m, x_{n + 1}))\\
      \rho_n(g \bullet h) &= \rho_m(g) \bullet (h_1, \dots, h_m, x_{n + 1})\\
    \end{align*}
    for all $ f: L_{m + 1} $, $ g: L_m $ and $ h: L_n^m $.
  \end{definition}

  \begin{definition}[$ \beta $- and $ \eta $-equality]
    We say that a $ \lambda $-theory $ L $ satisfies $ \beta $-equality (or that it is a $ \lambda $-theory with $ \beta $) if $ \rho_n \circ \lambda_n = \id{L_n} $ for all $ n $. We say that is satisfies $ \eta $-equality if $ \lambda_n \circ \rho_n = \id{L_{n + 1}} $ for all $ n $.
  \end{definition}

  \begin{definition}[$ \lambda $-theory morphism]
    A morphism $ F $ between $ \lambda $-theories $ L $ and $ L^\prime $ is an algebraic theory morphism $ F $ such that
    \begin{align*}
      F_n(\lambda_n(f)) &= \lambda_n(F_{n + 1}(f))\\
      \rho_n(F_n(g)) &= F_{n + 1}(\rho_n(g))
    \end{align*}
    for all $ f: L_{n + 1} $ and $ g: L_n $.
  \end{definition}

  \begin{remark}
    The category of lambda theories has all limits, with the underlying algebraic theory of a limit being the limit of the underlying algebraic theories.
  \end{remark}

  A $ \lambda $-theory algebra or presheaf is a presheaf for the underlying algebraic theory.

  \section{Alternate definitions}
  \begin{definition}
    Lawvere theory: \TODO
  \end{definition}
  \begin{definition}
    Relative monad: \TODO
  \end{definition}
  \begin{definition}
    Abstract clone: \TODO
  \end{definition}
  \begin{definition}
    Cartesian Operad: \TODO
  \end{definition}
  (https://ncatlab.org/nlab/show/lambda+theory)

  \chapter{Category Theoretic Preliminaries}

I will assume a familiarity with the category-theoretical concepts presented in \autocite{CT4P}. These include categories, functors, isomorphisms, natural transformations, adjunctions, equivalences and limits.

\section{Notation}
For an object $ c $ in a category $ C $, I will write $ c: C $.

For a morphism $ f $ between objects $ c $ and $ c^\prime $ in a category $ C $, I will write $ f: C(c, c^\prime) $ or $ f: c \to c^\prime $.

For composition of morphisms $ f: C(c, d) $ and $ g: C(d, e) $, I will write $ f \cdot g $.

For composition of functors $ F: A \to B $ and $ G: B \to C $, I will write $ F \bullet G $.

\section{Universal Arrows}

One concept in category theory that can be used to describe a lot of limits and adjunctions is that of a universal arrow (see for example \autocite{MacLane}, Part III)
\begin{definition}
  A \iindex{universal arrow} from an object $ c: C $ to a functor $ F: D \to C $ consists of an object $ d: D $ and a morphism $ f: D(c, F(d)) $ such that for every similar pair $ (d^\prime, f^\prime) $, $ f^\prime $ factors uniquely as $ f \cdot F(g) $ for some $ g: C(d, d^\prime) $:
  \begin{center}
    \begin{tikzcd}
      c \arrow[d, "f"'] \arrow[rd, "f^\prime"] &\\
      F(d) \arrow[r, "F(g)"', dashed] & F(d^\prime)
    \end{tikzcd}
  \end{center}
\end{definition}

Alternatively, we can characterize universal arrows by their action on hom-sets:
\begin{lemma}
  Let $ F: C \to D $ be a functor and $ d: D $ an object. An object $ c: C $ and an arrow $ f: D(d, F(c)) $ form a universal arrow from $ d $ to $ F $ if and only if the function
  \[ (g \mapsto f \cdot F(g)) : C(c, x) \cong D(d, F(x)) \]
  is a bijection.

  Conversely, for all $ c: C $ and $ d: D $, every bijection
  \[ C(c, x) \cong D(d, F(x)) \]
  that is natural in $ x $ arises in this way from some universal arrow $ f: D(d, F(c)) $.
\end{lemma}
\begin{proof}
  See \autocite{MacLane}, Chapter III.2, Proposition 1.
\end{proof}

There is also the dual concept of a universal arrow $ (d, f) $ from a functor $ F $ to an object $ c: C $. Its universal property can be summarized in the following diagram:
\begin{center}
  \begin{tikzcd}
    F(d^\prime)\arrow[rd, "f^\prime"']  \arrow[r, "F(g)", dashed] & F(d) \arrow[d, "f"]\\
    & c
  \end{tikzcd}
\end{center}

\section{Adjunctions and equivalences}

Recall that an \iindex{adjunction} $ L \dashv R $ is a pair of functors
\begin{center}
  \begin{tikzcd}
    D \arrow[r, bend right, "R"'{name=R}] & C \arrow[l, bend right, "L"'{name=L}]
  \end{tikzcd}
\end{center}
with natural transformations (the unit and counit)
\[ \eta: \id C \Rightarrow L \bullet R \quad \text{and} \quad \epsilon: R \bullet L \Rightarrow \id D \]
such that the diagrams
\begin{center}
  \begin{tikzcd}
    L \arrow[rd, "\eta \bullet L"'] \arrow[rr, "\id L"] & & L\\
    & L \bullet R \bullet L \arrow[ur, "L \bullet \epsilon"']
  \end{tikzcd}
  \qquad
  \begin{tikzcd}
    R \arrow[rd, "R \bullet \eta"'] \arrow[rr, "\id R"] & & R\\
    & R \bullet L \bullet R \arrow[ur, "\epsilon \bullet R"']
  \end{tikzcd}
\end{center}
commute (these are called the \iindex{triangle identities} or \iindex{zigzag identities}). Here the natural transformation $ \eta \bullet L: L \bullet R \bullet L $ is the natural transformation $ \eta $ whiskered on the right by $ L $, and the other whiskered transformations are similar.

An alternative characterization (see \autocite{MacLane}, chapter IV.1, Theorem 2) of an adjunction $ L \dashv R $ is as a natural bijection
\[ \varphi: D(L(c), d) \xrightarrow{\sim} C(c, R(d)). \]
Naturality means that for all $ f: C(c^\prime, c) $, $ g: D(d, d^\prime) $ and $ h: D(L(c), d) $,
\[ \varphi(L(f) \cdot h \cdot g) = f \cdot \varphi(h) \cdot R(g). \]

Lastly, one can construct an adjunction using universal arrows. This lends itself particularly well for a formalization, where it is often preferable to have as little `demonstranda' as possible:
\begin{lemma}
  One can construct an adjunction $ (L, R, \eta, \epsilon) $ as above from only the functor $ L: C \to D $ and, for each $ c: C $, a universal arrow $ (R(c), \epsilon_c) $ from $ L $ to $ c $.
\end{lemma}
\begin{proof}
  See \autocite{MacLane}, Chapter IV.1, Theorem 2 (iv).
\end{proof}

\subsection{Adjoint equivalences}
An (adjoint) equivalence of categories has multiple definitions. The one we will use here is the following:

\begin{definition}\label{def:equivalence-of-categories}
  An \iindex{adjoint equivalence} between categories $ C $ and $ D $ is a pair of adjoint functors $ L \dashv R $ like above such that the unit $ \eta: \id{C} \Rightarrow L \bullet R $ and counit $ \epsilon: R \bullet L \Rightarrow \id{D} $ are isomorphisms of functors.
\end{definition}

\subsection{Weak equivalences}
There is also the notion of `weak equivalence'. In some cases, this is equivalent to an adjoint equivalence (for example, when its domain is univalent).
\begin{definition}
  A functor $ F: C \to D $ is called a \iindex{weak equivalence} if it is essentially surjective and fully faithful.
\end{definition}

\subsection{Exponential objects}
Note that in the category of sets, for all $ X, Y: \SET $, we have a set $ (X \to Y) $. Also, for all $ X, Y, Z $, there is a (natural) bijection
\[ (X \times Y \to Z) \cong (X \to (Y \to Z)) \]
which we can also write as
\[ \SET(X \times Y, Z) \cong \SET(X, (Y \to Z)). \]
In other words, we have functors $ X \mapsto X \times Y $ and $ Z \mapsto (Y \to Z) $, and these two form an adjunction. The following generalizes this
\begin{definition}
  A category $ C $ has \textit{exponential objects}\index{exponential objects} (or \textit{exponentials}) if for all $ c: C $, the functor $ c^\prime \mapsto c^\prime \times c $ has a right adjoint, which we denote $ d \mapsto d^c $.
\end{definition}

\begin{remark}
  It is actually very well possible that a category does not have all exponentials, but it has some objects $ c, d, d^c: C $ with a natural bijection
  \[ C(d^\prime \times c, d) \cong C(d^\prime, d^c). \]
  Then $ d^c $ is still called an exponential object.
\end{remark}

\subsection{Forgetful functors and free objects}

In mathematics, we often deal with objects that are `based on' other objects. For example, a ring is a set with some additional structure. Often, this is a relation between the respective categories (for example, in the case of a displayed category, see Section \ref{sec:displayed-categories}), and such a relation gives rise to a \iindex{forgetful functor}, that `forgets' about the additional structure. In the examples of rings and sets, the forgetful functor sends a ring to its underlying set, and a ring morphism to the function between the sets. However, note that there is no formal definition of forgetful functors. The name is more of a way to talk about the perceived relation between the categories.

\begin{definition}
  Given a forgetful functor $ F: C \to D $, we define the \textit{free functor}\index{free!functor} associated to $ F $ to be the left adjoint to $ F $, if it exists.
\end{definition}

\begin{example}
  Consider the forgetful functor from the category of commutative rings to the category of sets, sending a ring to its underlying set. This has a left adjoint, sending the set $ \{ 1, 2, \dots, n \} $ to the polynomial ring $ \mathbb Z[X_1, \dots, X_n] $, and more generally, sending $ S $ to the polynomial ring $ \mathbb Z[X_s]_{s : S} $. This ring is then called `the free commutative ring on $ S $'. If $ S $ is finite of size $ n $, the ring is also called `the free commutative ring on $ n $ generators'.

  The free functor sends a function $ f: S \to T $ to the ring morphism $ \mathbb Z[X_s]_{s: S} \to \mathbb Z[X_t]_{t: T} $ that sends $ X_s $ to $ X_{f(s)} $.

  The natural bijection
  \[ \mathbf{Rng}(\mathbb Z[X_s]_{s : S}, R) \cong \SET(S, R) \]
  then sends $ f: \mathbf{Rng}(\mathbb Z[X_s]_{s : S}, R) $ to $ s \mapsto f(X_s) $ and $ g: \SET(S, R) $ to the morphism that sends $ X_s $ to $ g(s) $.
\end{example}

However, sometimes, we have a forgetful functor $ F: C \to D $, but we cannot give a free functor on the entire category $ D $. In such a case, we might still talk about free `objects':
\begin{definition}
  Let $ F: C \to D $ be a forgetful functor. Given $ d: D $, the \textit{free object}\index{free!object} on $ d $ is a universal arrow $ (c, f) $ from $ d $ to $ F $.
\end{definition}

\begin{remark}
  By \autocite{MacLane}, Chapter IV.1, Theorem 2 (ii), if we have a free object on every $ d: D $, we can piece these together to get a free functor associated to $ F $.
\end{remark}

\section{Yoneda}
We can embed a category $ C $ fully faithfully into the functor category $ P C = [\op C, \SET] $ as follows (see \autocite{Kashiwara}, Section 1.4):
\begin{definition}\label{def:Yoneda-embedding}
  The \iindex{Yoneda embedding} $ Y : C \hookrightarrow P C $ is given on objects by $ Y(c) = C(-, c) $:
  \[ Y(c)(d) = C(d, c) \quad \text{and} \quad Y(c)(f)(g) = f \cdot g \]
  for $ d: C $, $ f: C(d, d^\prime) $ and $ g: C(d^\prime, c) $. It sends a morphism $ f: C(c, c^\prime) $ to the natural transformation $ Y(f): C(-, c) \Rightarrow C(-, c^\prime) $ given by
  \[ Y(f)(d)(g) = g \cdot f \]
  for $ d: C $ and $ g: C(d, c) $.
\end{definition}

Now, this embedding has a couple of properties:
\begin{lemma}
  For any $ c: C $ and $ F : PC $, we have an equivalence $ PC(Y(c), F) \simeq F(c) $, and this equivalence is natural in $ c $ and $ F $.
\end{lemma}
\begin{proof}
  See \autocite{Kashiwara}, Proposition 1.4.3.
\end{proof}

\begin{lemma}
  The Yoneda embedding functor preserves limits.
\end{lemma}
\begin{proof}
  See \autocite{borceux}, Volume 1, Proposition 2.15.5.
\end{proof}

\begin{lemma}
  The Yoneda embedding functor is \iindex{cartesian}. That is, not only does it preserve binary products, but it also preserves exponential objects (\autocite{stackexchange:yoneda-exponentials}).
\end{lemma}
\begin{proof}
  \TODO
\end{proof}

For a functor between categories $ f: C \to D $, given the Yoneda embeddings $ Y_C : C \to P C $ and $ Y_D : D \to P D $ (we will often omit the subscript $ C $ and $ D $), we can create a diagram
\begin{center}
  \begin{tikzcd}
    C \arrow[r, "f"] \arrow[d, hookrightarrow, "Y"] & D \arrow[d, hookrightarrow, "Y"]\\
    P C & P D \arrow[l, "\op f_*"]
  \end{tikzcd}
\end{center}
Note that the arrows in this diagram are functors, so objects in a category, instead of elements of a set. Therefore, it does not make sense to talk about `equality' of the morphisms along the different paths, but we rather talk about isomorphism in the functor category $ [C, P C] $. If we have such an isomorphism, we say the diagram `2-commutes':
\begin{lemma}\label{lem:Yoneda-restriction-commutes}
  If $ f: C \to D $ is a fully faithful functor, the diagram above 2-commutes.
\end{lemma}
\begin{proof}
  For $ c, d : C $, since $ f $ is fully faithful, we have isomorphisms of sets, given by
  \[ Y(c)(d) = C(d, c) \xrightarrow[f_{d, c}]{\sim} D(f(d), f(c)) = \op f_*(Y(f(c)))(d). \]
  Also, for $ g: C(d^\prime, d) $, the following diagram commutes
  \begin{center}
    \begin{tikzcd}
      Y(c)(d) \arrow[d, "g \cdot -"] \arrow[r, "f_{d, c}"] & \op f_*(Y(f(c)))(d) \arrow[d, "f_{d^\prime, d}(g) \cdot -"]\\
      Y(c)(d^\prime) \arrow[r, "f_{d^\prime, c}"] & \op f_*(Y(f(c)))(d^\prime)
    \end{tikzcd}
  \end{center}
  so the isomorphism is natural in $ d $ and we have $ Y(c) \cong \op f_*(Y(f(c))) $ in $ P C $. Lastly, for $ g: C(c, c^\prime) $ and $ d: C $, the following diagram commutes
  \begin{center}
    \begin{tikzcd}
      Y(c)(d) \arrow[d, "- \cdot g"] \arrow[r, "f_{d, c}"] & \op f_*(Y(f(c)))(d) \arrow[d, "- \cdot f_{c, c^\prime}(g)"]\\
      Y(c^\prime)(d) \arrow[r, "f_{d, c^\prime}"] & \op f_*(Y(f(c^\prime)))(d)
    \end{tikzcd}
  \end{center}
  so the isomorphism is natural in $ c $ and we have $ Y \cong f \bullet Y \bullet \op f_* $ in $ [C, P C] $.
\end{proof}

\section{Fibrations}
Let $ P : E \to B $ be a functor. In this case, we will view this as the category $ E $ `lying over' the category $ B $, with for every point $ b: B $, a slice $ E_b = P^{-1}(B) $ lying `above' $ b $.

\begin{definition}
  A morphism $ f: E(y, z) $ is called \iindex{cartesian} if for all $ g: E(x, z) $ and $ h: B(P(x), P(y)) $ with $ h \cdot P(f) = P(g) $, there exists $ \bar h: E(x, y) $ such that $ P(\bar h) = h $ and $ \bar h \cdot f = g $, like illustrated in the following diagram from \autocite{nlab:grothendieck_fibration}
  \begin{center}
    \begin{tikzcd}[sep=large]
      E \arrow[d, "P"] &x \arrow[rr, "\forall g", bend left] \arrow[r, "\exists! \bar h"', dashed] & y \arrow[r, "f"'] & z\\
      B & P(x) \arrow[r, "\forall h"'] \arrow[rr, "P(g)", bend left] & P(y) \arrow[r, "P(f)"'] & P(z)
    \end{tikzcd}
  \end{center}
\end{definition}

\begin{definition}
  $ P $ is a \iindex{fibration} if for all $ y: E $ and morphisms $ f: B(x, P(y)) $, there exist an object $ \bar x: E $ and a cartesian morphism $ \bar f: E(\bar x, y) $ such that $ P(\bar x) = x $ and $ P(\bar f) = f $:
  \begin{center}
    \begin{tikzcd}
      E \arrow[d, "P"] & \bar x \arrow[r, "\exists \bar f", dashed] & y\\
      B & x \arrow[r, "\forall f"] & P(y)
    \end{tikzcd}
  \end{center}
\end{definition}

\section{(Co)slice categories}
Given an object in a category $ c: C $, the morphisms to and from $ c $ constitute the slice and coslice categories
\begin{definition}
  The \iindex{slice category} $ C \downarrow c $ is the category with as objects the morphisms to $ c $:
  \[ (C \downarrow c)_0 = \sum_{c^\prime: C} C(c^\prime, c). \]
  The morphisms from $ (c^\prime, f) $ to $ (c^{\prime\prime}, f^\prime) $ are the morphisms $ g: c^\prime \to c^{\prime\prime} $ making the following diagram commute.
  \begin{center}
    \begin{tikzcd}
      c^\prime \arrow[rr, "g"] \arrow[rd, "f"'] & & c^{\prime\prime} \arrow[ld, "f^\prime"]\\
      & c
    \end{tikzcd}
  \end{center}
\end{definition}
The \textit{coslice category}\index{slice category!co-} $ c \downarrow C $ is similar, but with the morphisms \textit{from} $ c $ instead of \textit{to} $ c $:
\[ (c \downarrow C)_0 = \sum_{c^\prime: C} C(c, c^\prime). \]

Now, if we have an object in a slice category, we can again look at the slice of the slice category over that object. However, this gives us nothing new:
\begin{lemma}
  Let $ (d, f) $ be an object in the slice category $ (C \downarrow c) $. The slice category $ ((C \downarrow c) \downarrow (d, f)) $ is equivalent to $ (C \downarrow d) $.
\end{lemma}
\begin{proof}
  An object of $ ((e, g), \alpha) : ((C \downarrow c) \downarrow (d, f)) $ is an object $ (e, g) : (C \downarrow c) $, together with a morphism $ \alpha: (C \downarrow c)((e, g), (d, f)) $. That is, a morphism $ \alpha: C(e, d) $ such that the obvious triangle commutes (shown in the diagram below on the far left).

  Then a morphism between $ ((e, g), \alpha) $ and $ ((e^\prime, g^\prime), \alpha^\prime) $ is a morphism $ \beta $ between $ (e, g) $ and $ (e^\prime, g^\prime) $ that commutes with $ \alpha $ and $ \alpha^\prime $. Note that a morphism between $ (e, g) $ and $ (e^\prime, g^\prime) $ is a morphism between $ e $ and $ e^\prime $ that commutes with $ g $ and $ g^\prime $.
  \begin{center}
    \begin{tikzcd}
      e \arrow[rdd, "g"'] \arrow[rd, "\alpha"] \arrow[rr, "\beta"] && e^\prime \arrow[ldd, "g^\prime"] \arrow[ld, "\alpha^\prime"']\\
      & d \arrow[d, "f" description]\\
      & c
    \end{tikzcd}
    $ \Leftrightarrow $
    \begin{tikzcd}
      e \arrow[rd, "\alpha"] \arrow[rr, "\beta"] && e^\prime \arrow[ld, "\alpha^\prime"']\\
      & d
    \end{tikzcd}
  \end{center}

  Now, note that $ g $ and $ g^\prime $ are completely determined by $ g = \alpha \cdot f $ and $ g^\prime = \alpha^\prime \cdot f $, so we can leave them out. Also, if $ \beta $ commutes with $ \alpha $ and $ \alpha^\prime $, it automatically also commutes with $ g $ and $ g^\prime $. Therefore, as shown above, we have a correspondence between objects and morphisms
  \[ \beta: ((e, g), \alpha) \to ((e^\prime, g^\prime), \alpha^\prime) \Leftrightarrow \beta: (e, \alpha) \to (e^\prime, \alpha^\prime). \]
\end{proof}

Also, it turns out that we can derive some structure of the slice categories from the original category:
\begin{lemma}
  For $ (d, f), (d^\prime, f^\prime) : (C \downarrow c) $, their product in the slice category is given by their pullback / fibered product $ d \times_c d^\prime $, together with the induced morphism $ g: d \times_c d^\prime \dasharrow c $.
\end{lemma}
\begin{proof}
  Consider the diagram below. The fibered product gives `projections' $ h $ and $ h^\prime $. Also, if we have some $ (e, \alpha) : (C \downarrow c) $, together with morphisms $ \beta : (e, \alpha) \to (d, f) $ and $ \beta^\prime : (e, \alpha) \to (d^\prime, f^\prime) $. Then $ \beta $ and $ \beta^\prime $ commute with $ f $ and $ f^\prime $, so by the universal property of the fibred product, there exists a unique morphism $ \gamma : e \to d \times_c d^\prime $ that makes the triangles with $ \beta $ and $ h $, and with $ \beta^\prime $ and $ h^\prime $ commute. Then $ \gamma $ commutes with $ \alpha $ and $ g $ as well, so it is a morphism in $ (C \downarrow c) $. This shows that $ (d \times_c d^\prime, g) $ has the universal property of the product in $ (C \downarrow c) $.
  \begin{center}
    \begin{tikzcd}
      e \arrow[rd, "\beta"] \arrow[rr, "\beta^\prime", bend left] \arrow[rrd, "\alpha"', bend right=49, shift right] \arrow[r, dashed, "\gamma"] & d \times_c d^\prime \arrow[r, "h^\prime"'] \arrow[d, "h"] \arrow[rd, dashed, "g"] & d^\prime \arrow[d, "f^\prime"] \\
      & d \arrow[r, "f"] & c
    \end{tikzcd}
  \end{center}
\end{proof}

For a category $ C $ with products, Hyland introduces the notation $ \Delta_X Y $ for the element $ (X \times Y, p_1) : (C \downarrow X) $, and we will follow his example in this.

In fact, $ \Delta_X : C \to C / X $ is a functor, with $ \Delta_X(f) = \id X \times f $ for $ f: C(Y, Y^\prime) $. This functor preserves the terminal object, products and pullbacks:
\begin{lemma}\label{lem:delta-limits}
  $ \Delta_X $ preserves all limits.
\end{lemma}
\begin{proof}
  Take a diagram $ ((A_i)_i, (f_j)_j) $ in $ C $. Suppose that this has a limit $ (L, (g_i)_i) : C $. Now, consider an object $ (B, q) : (C \downarrow X) $, together with morphisms $ h_i : (C \downarrow X)((B, q), \Delta_X A_i) $, that commute with the $ \Delta_X f_j $:
  \begin{center}
    \begin{tikzcd}
      & B \arrow[ldd, "h_{m_j}"'] \arrow[rdd, "h_{n_j}"]\\
      & X \times L \arrow[ld, "\Delta_X g_{m_j}"] \arrow[rd, "\Delta_X g_{n_j}"']\\
      X \times A_{m_j} \arrow[rr, "\Delta_X f_j"'] & & X \times A_{n_j}
    \end{tikzcd}
  \end{center}
  Then the morphisms in $ (C \downarrow X)((B, q), \Delta_X L) $, commuting with the $ \Delta_X g_j $ and $ h_j $ are the morphisms in $ C(B, X \times L) \cong C(B, X) \times C(B, L) $ that commute with the projections to $ X $ and the $ \id X \times g_j $ and $ h_j $. Since the morphisms in $ C(B, X) $ commuting with $ q $ and $ \id X $ are exactly $ q $, we can forget about this component, and the morphisms we are looking for correspond to the morphisms in $ C(B, L) $ that commute with the $ g_j $ and $ h_j \cdot p_1 $. Since $ (L, (g_j)_j) $ is a limit, this is a unique morphism.
\end{proof}

\begin{lemma}\label{lem:delta-exponentials}
  $ \Delta_X $ preserves exponential objects:
\end{lemma}
\begin{proof}
  See \autocite{borceux}, Volume 3, Lemma 5.8.2. This lemma shows this via the following equivalences:
  \begin{align*}
    (C \downarrow X)((A, f), \Delta_X Y^Z) &\cong C(A, Y^Z)\\
    &\cong C(A \times Z, Y)\\
    &\cong (C \downarrow X)((A \times X \times Z, \langle f, p_1 \rangle), \Delta_X Y)\\
    &\cong (C \downarrow X)((A, f) \times \Delta_X Z, \Delta_X Y)\\
  \end{align*}
\end{proof}

\section{Dependent products and sums}\label{sec:dependent-products}
The following is based loosely on Section 4.1 of \autocite{taylor}.

Take a category $ C $. To talk about dependent sums $ \sum_{a: A} X_a $ and products $ \prod_{a: A} X_a $ in $ C $, we first need some way to construct the family of objects $ (X_a)_a $. Of course, we can do this \textit{externally} using a set $ A $, and picking an object $ X_a : C $ for every element $ a : A $. We then have a category of such families $ C^A $, with objects $ (X_a)_a $ and morphisms $ (f_a)_a: C^A((X_a)_a, (Y_a)_a) $, with $ f_a: X_a \to Y_a $. We write $ C^A $ because we can view this as just the $ A $-fold power of $ C $. Now, this assignment of categories $ A \mapsto C^A $ can be turned into a contravariant (pseudo)functor of $ 2 $-categories $ \op \SET \to \Cat $. It sends a morphism $ f: A \to B $ to the `relabeling' or `substitution' functor $ C^B \to C^A $, $ (X_b)_b \mapsto (X_{f(a)})_a $.

However, there is also an \textit{internal} representation, as a morphism $ X \to A $. We can turn the collection of these morphisms over all the $ A : C $ simultaneously into a category $ C^2 $ (abusing notation a bit, writing $ 2 $ for the two-point category $ \bullet \to \bullet $). Then taking codomains $ (X \to A) \mapsto A $ gives a functor $ C^2 \to C $. The fiber of this functor above $ A $ is the slice category $ (C \downarrow A) $.

In $ \SET $, the external and internal ways of indexing are actually equivalent, because given a family $ (X_a)_a $ we can construct a morphism $ f: \sum_{a : A} X_a \to A $ and conversely, we can recover the family $ (X_a)_a $ as $ (f^{-1}(a))_a $.

Also note that for $ \SET $, if we consider an indexed family $ (X_a)_a $ as some function $ X: A \to \SET $, then substitution over $ \alpha: B \to A $ is just given by postcomposition $ X \circ \alpha: B \to \SET $. It turns out that in the internal representation this arises as the pullback
\begin{center}
  \begin{tikzcd}
    \sum_{b : B} X_{\alpha(b)} \arrow[r] \arrow[d, "\alpha^* f"] \arrow[dr, phantom, "\lrcorner", very near start] & \sum_{a : A} X_a \arrow[d, "f"]\\
    B \arrow[r, "\alpha"] & A
  \end{tikzcd}
\end{center}
This can be turned into a pullback or `substitution' functor $ \alpha^* $, which Taylor calls $ \mathtt{P}\alpha $. This turns the functor $ \SET^2 \to \SET $ into a fibration. We can construct $ \alpha^* : (C \downarrow A) \to (C \downarrow B) $ for any category $ C $ with pullbacks and any morphism $ \alpha: C(B, A) $. The existence of this pullback makes the functor $ C^2 \to C $ into a fibration.

Now, in $ \SET $, for a family $ (X_a)_a $, consider the dependent product $ \prod_{a : A} X_a $. Its elements $ (x_a)_a $ can be identified with morphisms from the terminal set: $ \{ \star \} \to \prod_{a : A} X_a $, sending $ \star $ to $ (x_a)_a $. However, they can also be identified with the morphisms $ \varphi: A \to \sum_{a : A} X_a $ that make the following diagram commute, sending $ a $ to $ x_a $:
\begin{center}
  \begin{tikzcd}
    A \arrow[rd, "\id A"] \arrow[rr, " \varphi "] & & \sum_{a : A} X_a \arrow[ld, "f"]\\
    & A
  \end{tikzcd}
\end{center}
These are morphisms in $ (\SET \downarrow A) $ from $ (A, \id A) $ to $ (\sum_{a : A} X_a, f) $. Note that for the terminal morphism $ \alpha: A \to \{ \star \} $, we have $ \id A = \alpha^*(\id{\{\star\}}) $. To summarize, we have an equivalence
\[ (\SET \downarrow A)(\alpha^*(\id{\{\star\}}), f) \simeq (\SET \downarrow {\{\star\}})( \id{\{ \star \}}, \prod_{a : A} X_a). \]
Now, given a family of families $ ((X_b)_{b : B_a})_{a : A} $, we can wonder whether we can construct the family of dependent products $ (\prod_{b : B_a} X_b)_a $. In $ \SET $, this is definitely possible, and from this, we get an equivalence again
\[ (\SET \downarrow (\sum_{a : A} B_a))(\alpha^*(\id A), f) \simeq (\SET \downarrow A)(\id A, (\prod_{b : B_a} X_b)_a). \]
These equivalences suggest an adjunction $ \alpha^* \dashv \prod_{\dots} $. We can use this to define in general
\begin{definition}
  For a category $ C $ and a morphism $ \alpha: B \to A $, the \iindex{dependent product} along $ \alpha $ is, if it exists, the right adjoint to the pullback functor:
  \begin{center}
    \begin{tikzcd}
      (C \downarrow A) \arrow[r, shift left=2, "\alpha^*"{name=A}] &
      (C \downarrow B) \arrow[l, shift left=2, "\prod_\alpha"{name=B}]
      \ar[from=A, to=B, symbol=\dashv]
    \end{tikzcd}
  \end{center}
\end{definition}
\begin{remark}
  As argued above, we can recover the familiar dependent product $ \prod_{a : A} X_a $ of a family $ (X_a)_a $ as the dependent product $ \prod_\alpha f $ along the terminal morphism $ \alpha: A \to I $, with $ f: \sum_a X_a \to A $ the internal representation of the family. Here we use the equivalence between $ (C \downarrow I) $ and $ C $.
\end{remark}

Now we turn our attention to dependent sums. In $ \SET $, let $ (X_a)_a $ and $ (B_a)_a $ be two families over $ A $ and let $ ((Y_b)_{b : B_a})_{a : A} $ be a family of families. Let $ \alpha: \sum_{a : A} B_a \to A $ be the internal representation of $ (B_a)_a $. A family of maps $ f_a : (\sum_{b : B_a} Y_b)_a \to X_a $ consists of maps $ Y_b \to X_a $ for all $ b : B_a $, so these are maps $ f_b : Y_b \to X_{\alpha(b)} $. This gives an equivalence
\[ (\SET \downarrow A)((\sum_{b : B_a} Y_b)_a, (X_a)_a) \simeq (\SET \downarrow (\sum_{a : A} B_a))((Y_b)_b, \alpha^*((X_a)_a)). \]
This, again, suggests an adjunction which we will use as a definition.
\begin{definition}
  For a category $ C $ and a morphism $ \alpha: B \to A $, the \iindex{dependent sum} along $ \alpha $ is, if it exists, the left adjoint to the pullback functor:
  \begin{center}
    \begin{tikzcd}
      (C \downarrow A) \arrow[r, shift right=2, "\alpha^*"'{name=B}] &
      (C \downarrow B) \arrow[l, shift right=2, "\sum_\alpha"'{name=A}]
      \ar[from=A, to=B, symbol=\dashv]
    \end{tikzcd}
  \end{center}
\end{definition}

However, note that the conversion from an external to an internal representation in $ \SET $ already contained a dependent sum, which is no coincidence. It turns out that in practice, we will never have a hard time obtaining dependent sums:
\begin{lemma}\label{lem:sum-postcomposition}
  Let $ \alpha : C(B, A) $ be a morphism in a category. If the pullback functor $ \alpha^*: (C \downarrow A) \to (C \downarrow B) $ exists, it has a left adjoint given by postcomposition with $ \alpha $.
\end{lemma}
\begin{proof}
  For morphisms $ f: X \to A $, $ g: Y \to B $, the universal property of the pullback, with the following diagram
  \begin{center}
    \begin{tikzcd}
      Y \arrow[r, dashed, "\varphi"'] \arrow[rr, "\psi", bend left] \arrow[dr, "g"'] & \alpha^* X \arrow[d, "\alpha^* f"'] \arrow[r] \arrow[rd, phantom, "\lrcorner", very near start] & X \arrow[d, "f"]\\
      & B \arrow[r, "\alpha"'] & A
    \end{tikzcd}
  \end{center}
  gives an equivalence between morphisms $ \varphi: Y \to \alpha^* X $ that commute with $ g $ and $ \alpha^* f $, and morphisms $ \psi: Y \to X $ that commute with $ g $, $ f $ and $ \alpha $. In other words:
  \[ (C \downarrow A)(g \cdot \alpha, f) \simeq (C \downarrow B)(g, \alpha^*(f)), \]
  which shows the adjunction.
\end{proof}

Now, let $ f : X \to A $ be the internal representation of an indexed family $ (X_a)_a $ and let $ \alpha : A \to I $ be the terminal projection. We have $ \sum_{a : A} X_a = f \circ \alpha : X \to I $. By the equivalence between $ (C \downarrow I) $ and $ C $, we see that the dependent sum of the family $ (X_a)_a $ is exactly $ X $. Therefore, our attention is mainly focused on the dependent product.

We will close this section with a name for a category that has all dependent products:

\begin{definition}
  A \index{cartesian closed!locally}\textit{locally cartesian closed} category is a category $ C $ with pullbacks such that each pullback functor $ \alpha^* $ has a right adjoint.
\end{definition}

Apart from having dependent sums and products, there also is the following theorem that shows the significance of locally cartesian closedness:
\begin{lemma}\label{lem:locally-cartesian-closed}
  A category $ C $ is locally cartesian closed iff $ (C \downarrow A) $ is cartesian closed for each $ A : C $.
\end{lemma}
\begin{proof}
  See the end of Section 1.3 of \autocite{freyd}.
\end{proof}

\begin{remark}\label{rem:pullback-of-projection}
  Note that for $ X, Y, Z : C $ and $ f: C(Y, X) $, the following diagram shows that $ f^* \Delta_X Z \cong \Delta_Y Z $:
  \begin{center}
    \begin{tikzcd}
      Y \times Z \arrow[d, "p_1"] \arrow[r, "f \times \id Z"] & X \times Z \arrow[d, "p_1"] \arrow[r, "p_2"] & Z \arrow[d, "!"]\\
      Y \arrow[r, "f"] & X \arrow [r, "!"] & T
    \end{tikzcd}
  \end{center}
\end{remark}

\begin{lemma}\label{lem:constant-dependent-product}
  For $ W, X, Z : C $ and $ p_1 : X \times Z \to X $,
  \[ \prod_{p_1} \Delta_{X \times Z} W \cong \Delta_X W^Z \]
\end{lemma}
\begin{proof}
  First of all, note that $ (C \downarrow X \times Z) \cong ((C \downarrow X) \downarrow \Delta_X Z) $. Also note that the composite morphism $ (X \times Z) \times W \xrightarrow{p_1} X \times Z \xrightarrow{p_1} X $ is the element $ \Delta_X (W \times Z) : (C \downarrow X) $.

  By Proposition 1.34 in \autocite{freyd}, $ \prod_{p_1} \Delta_{X \times Z} W $ is given as the following pullback:
  \begin{center}
    \begin{tikzcd}
      \prod_{p_1} \Delta_{X \times Z} W \arrow[r] \arrow[d] & (\Delta_X Z \times W)^{\Delta_X Z} \arrow[d, "(\Delta_X p_1)^{\Delta_X Z}"]\\
      X \arrow[r] & (\Delta_X Z)^{\Delta_X Z}
    \end{tikzcd}
  \end{center}

  By Lemma \ref{lem:delta-exponentials}, $ \Delta_X $ preserves exponential objects, so the morphism on the right is $ \Delta_X p_1^Z : (C \downarrow X)(\Delta_X (Z \times W)^Z, \Delta_X Z^Z) $. However, we have an isomorphism $ (Z \times W)^Z \cong Z^Z \times Z^W $, and then the morphism on the right becomes
  \[ \Delta_X p_1 : (C \downarrow X)(\Delta_X(Z^Z \times Z^W), \Delta_X Z^Z) \]
  We also have an isomorphism $ X \cong \Delta_X T $. Then by Lemma \ref{lem:delta-limits} and Remark \ref{rem:pullback-of-projection}, the pullback of this diagram is $ \Delta_X W^Z $:
  \begin{center}
    \begin{tikzcd}
      \Delta_X W^Z \arrow[r] \arrow[d] & \Delta_X (Z^Z \times W^Z) \arrow[d, "\Delta_X p_1"]\\
      \Delta_X T \arrow[r] & \Delta_X Z^Z
    \end{tikzcd}
  \end{center}
\end{proof}


\section{(Weakly) terminal objects}
\begin{definition}
  If a category has an object $ t $, such that there is a (not necessarily unique) morphism to it from every other object in the category, $ t $ is said to be a \iindex{weakly terminal object}.
\end{definition}

\begin{definition}
  Let $ C $ be a category with terminal object $ t $. For an object $ c: C $, a \iindex{global element} of $ c $ is a morphism $ f: C(t, c) $.
\end{definition}


\section{Kan Extensions}
One of the most general and abstract concepts in category theory is the concept of \textit{Kan extensions}. In \autocite{MacLane}, Section X.7, MacLane notes that

\enquote{The notion of Kan extensions subsumes all the other fundamental concepts of category theory.}

In this thesis, we will use left Kan extension a handful of times. It comes in handy when we want to extend a functor along another functor in the following way:

Let $ A $, $ B $ and $ C $ be categories and let $ F : A \to B $ be a functor.
\begin{definition}
  Precomposition gives a functor between functor categories $ F_* : [B, C] \to [A, C] $. If $ F_* $ has a left adjoint, we will denote call this adjoint functor the \textit{left Kan extension}\index{Kan extension!left} along $ F $ and denote it $ \mathrm{Lan}_F : [A, C] \to [B, C] $.

  \begin{center}
    \begin{tikzcd}
      A \arrow[rr, "F"] \arrow[rd, dashed, "F_* G"'] & & B \arrow[ld, "G"]\\
      & C
    \end{tikzcd}
    \qquad
    \begin{tikzcd}
      A \arrow[rr, "F"] \arrow[rd, "G"'] & & B \arrow[ld, dashed, "\Lan F G"]\\
      & C
    \end{tikzcd}
  \end{center}

  Analogously, when $ F_* $ has a right adjoint, one calls this the \textit{right Kan extension}\index{Kan extension!right} along $ F $ and denote it $ \mathrm{Ran}_F: [A, C] \to [B, C] $.
\end{definition}

If a category has limits (resp. colimits), we can construct the right (resp. left) Kan extension in a `pointwise' fashion (see Theorem X.3.1 in \autocite{MacLane} or Theorem 2.3.3 in \autocite{Kashiwara}). Below, I will outline the parts of the construction that we will need explicitly in this thesis.
\begin{lemma}
  If $ C $ has colimits, $ \Lan F {} $ exists.
\end{lemma}
\begin{proof}
  First of all, for objects $ b: B $, we take
  \[ \Lan F G(b) := \text{colim} \left( (F \downarrow b) \to A \xrightarrow G C \right). \]

  Here, $ (F \downarrow b) $ denotes the comma category with as objects the morphisms $ B(F(a), b) $ for all $ a: A $, and as morphisms from $ f: B(F(a), b) $ to $ f^\prime: B(F(a^\prime), b) $ the morphisms $ g: A(a, a^\prime) $ that make the diagram commute:
  \begin{center}
    \begin{tikzcd}
      F(a) \arrow[rr, "F(g)"] \arrow[rd, "f"'] & & F(a^\prime) \arrow[ld, "f^\prime"]\\
      & b
    \end{tikzcd}
  \end{center}
  and $ (F \downarrow b) \to A $ denotes the projection functor that sends $ f: B(F(a_1), b) $ to $ a_1 $.

  Now, a morphism $ h: B(b, b^\prime) $ gives a morphism of diagrams, sending the $ F(a) $ corresponding to $ f: B(G(a), b) $ to the $ F(a) $ corresponding to $ f \cdot h: B(G(a), b^\prime) $. From this, we get a morphism $ \Lan F G(h): C(\Lan F G(b), \Lan F G(b^\prime)) $.

  The unit of the adjunction is a natural transformation $ \eta: \id{[A, C]} \Rightarrow \Lan F {} \bullet F_* $. We will define this pointwise, for $ G: [A, C] $ and $ a: A $. Our diagram contains the $ G(a) $ corresponding to $ \id{F(a)}: (F \downarrow F(a)) $ and the colimit cocone gives a morphism
  \[ \eta_G(a) : C(G(a), \Lan F G (F(a))), \]
  the latter being equal to $ (\Lan F {} \bullet F_*)(G)(a) $.

  The counit of the adjunction is a natural transformation $ \epsilon: F_* \bullet \Lan F {} \Rightarrow \id{[B, C]} $. We will also define this pointwise, for $ G: [B, C] $ and $ b: B $. The diagram for $ \Lan F (F_* G)(b) $ consists of $ G(F(a)) $ for all $ f: B(F(a), b) $. Then, by the universal property of the colimit, the morphisms $ G(f): C(G(F(a)), G(b)) $ induce a morphism
  \[ \epsilon_G(b) : C(\Lan F (F_* G)(b), G(b)). \]
\end{proof}

\begin{lemma}\label{lem:lan-precomp-iso}
  If $ F : A \to B $ is a fully faithful functor, and $ C $ is a category with colimits, $ \eta: \id{[A, C]} \Rightarrow \Lan F {} \bullet F_* $ is a natural isomorphism.
\end{lemma}
\begin{proof}
  To show that $ \eta $ is a natural isomorphism, we have to show that $ \eta_G(a^\prime): G(a^\prime) \Rightarrow \Lan F G(F(a^\prime)) $ is an isomorphism for all $ G: [A, C] $ and $ a^\prime: A $. Since a left adjoint is unique up to natural isomorphism (\autocite{CT4P}, Exercise 153), we can assume that $ \Lan F G(F(a^\prime)) $ is given by
  \[ \text{colim} ((F \downarrow F(a^\prime)) \to A \xrightarrow G C). \]
  Now, the diagram for this colimit consists of $ G(a) $ for each arrow $ f: B(F(a), F(a^\prime)) $. Since $ F $ is fully faithful, we have $ f = F(\overline f) $ for some $ \overline f: A(a, a^\prime) $. If we now take the arrows $ G(\overline f): C(G(a), G(a^\prime)) $, the universal property of the colimit gives an arrow
  \[ \varphi: C(\Lan F G(F(a^\prime)), G(a^\prime)) \]
  which constitutes an inverse to $ \eta_G(a^\prime) $. The proof of this revolves around properties of the colimit and its (induced) morphisms.
\end{proof}

\begin{remark}
  In the same way, if $ C $ has limits, $ \epsilon $ is a natural isomorphism.
\end{remark}

\begin{corollary}\label{cor:surjective-precomposition}
  If $ C $ has limits or colimits, precomposition of functors $ [B, C] $ along a fully faithful functor is (split) essentially surjective.
\end{corollary}
\begin{proof}
  For each $ G: [A, C] $ we take $ \Lan F G: [B, C] $, and we have $ F_*(\Lan F G) \cong G $.
\end{proof}

\begin{corollary}
  If $ C $ has colimits (resp. limits), left (resp. right) Kan extension of functors $ [A, C] $ along a fully faithful functor is fully faithful.
\end{corollary}
\begin{proof}
  Since left Kan extension along $ F $ is the left adjoint to precomposition, we have
  \[ [A, C](\Lan F G, \Lan F G^\prime) \cong [B, C](G, F_*(\Lan F G^\prime)) \cong [B, C](G, G^\prime). \]
\end{proof}

\section{Coends}\label{sec:coends}
This section is based on Section 1.2 of \autocite{riehl}.

In this thesis, we will encounter co-ends a couple of times, so we will introduce them here.
\begin{definition}
  Let $ C, D $ be categories and $ F : \op C \times C \to D $ a functor. We define the \iindex{coend} $ \int^{c : C} F(c, c) $ to be the colimit
  \begin{center}
    \begin{tikzcd}
      \displaystyle\coprod_{\substack{f : C(a, b)}} F(b, a)
        \arrow[r, shift left, "{F(f, \id b)}"]
        \arrow[r, shift right, "{F(\id a, f)}"'] &
      \displaystyle\coprod_{c : C} F(c, c)
        \arrow[r, dashed] &
      \int^C F
    \end{tikzcd}
  \end{center}
\end{definition}

\begin{remark}
  An alternative way to phrase this, is that $ \int^C F : D $ is an object, equipped with arrows $ F(c, c) \to \int^C F $ such that for all $ f: C(a, b) $, the following diagram must commute
  \begin{center}
    \begin{tikzcd}
      F(b, a) \arrow[r, "{F(f, \id a)}"] \arrow[d, "{F(\id b, f)}"] & F(a, a) \arrow[d]\\
      F(b, b) \arrow[r] & \int^C F
    \end{tikzcd}
  \end{center}
  and such that for any other $ G : D $ with the same properties, we have a unique morphism $ \int_C F \to G $, making the triangles commute
  \begin{center}
    \begin{tikzcd}
      F(c, c) \arrow[r] \arrow[rd] & \int_C F \arrow[d]\\
      & G
    \end{tikzcd}
  \end{center}
\end{remark}

\begin{remark}
  Of course, a co-end is actually the dual notion of an end, which can be defined as the equalizer of the diagram above, but with the arrows reversed.
\end{remark}

\begin{remark}
  Left Kan extension can be expressed as a coend:
  \[ \Lan F G(b) = \int^{a : A} D(F a, b) \cdot G a \]
  where $ S \cdot X $ for $ S $ a set and $ X : C $ denotes the `copower'. In most cases, we will have
  \[ S \cdot X = \coprod_{s : S} X. \]
\end{remark}

\section{The Karoubi envelope}
Let $ C $ be a category. If we have a retraction-section pair
\begin{tikzcd}
  c \arrow[r, shift left, "r"] & d \arrow[l, shift left, "s"]
\end{tikzcd}
we have (by definition) $ s \cdot r = \id d $. On the other hand, $ r \cdot s: c \to c $ is an idempotent morphism, since $ r \cdot s \cdot r \cdot s = r \cdot s $. Conversely, we can wonder whether for some idempotent morphism $ a: c \to c $, we can find a retraction-section pair $ (r, s) $ such that $ a = r \cdot s $. If this is the case, we say that the idempotent $ a $ \textit{splits}\index{split idempotent}. If $ a $ does not split, we can wonder whether we can find an embedding $ \iota_C : C \hookrightarrow \overline C $ such that the idempotent $ \iota_C(a): \iota_C(c) \to \iota_C(c) $ does split. This is one way to arrive at the \textit{Karoubi envelope}.

\begin{definition}
  We define the category $ \overline C $. The objects of $ \overline C $ are tuples $ (c, a) $ with $ c: C $, $ a: C(c, c) $ such that $ a \cdot a = a $. The morphisms between $ (c, a) $ and $ (d, b) $ are morphisms $ f: C(a, b) $ such that $ a \cdot f \cdot b = f $. The identity morphism on $ (c, a) $ is given by $ a $ and $ \overline C $ inherits morphism composition from $ C $.
\end{definition}
This category is called the \iindex{Karoubi envelope}, the \textit{idempotent completion}\index{idempotent completion|see{Karoubi envelope}}, the \textit{category of retracts}\index{category of retracts|see{Karoubi envelope}}, or the \textit{Cauchy completion}\index{Cauchy completion|see{Karoubi envelope}} of $ C $.

\begin{remark}
  Note that for a morphism $ f: \overline C((c, a), (d, b)) $,
  \[ a \cdot f = a \cdot a \cdot f \cdot b = a \cdot f \cdot b = f \]
  and in the same way, $ f \cdot b = f $.
\end{remark}

\begin{definition}
  We have an embedding $ \iota_C: C \to \overline C $, sending $ c: C $ to $ (c, \id{c}) $ and $ f: C(c, d) $ to $ f $.
\end{definition}

\begin{lemma}\label{lem:karoubi-is-retract}
  Every object $ c: \overline C $ is a retract of $ \iota_C(c_0) $ for some $ c_0: C $.
\end{lemma}
\begin{proof}
  Note that $ c = (c_0, a) $ for some $ c_0: C $ and an idempotent $ a: c \to c $. We have morphisms
  \begin{tikzcd}
    \iota_C(c) \arrow[r, shift left, "a_\rightarrow"] & (c, a) \arrow[l, shift left, "a_\leftarrow"]
  \end{tikzcd}, both given by $ a $. We have $ a_\leftarrow \cdot a_\rightarrow = a = \id{(c, a)} $, so $ (c, a) $ is a retract of $ \iota_C(c) $.
\end{proof}

\begin{lemma}
  In $ \overline C $, every idempotent splits.
\end{lemma}
\begin{proof}
  Take an idempotent $ e: \overline C(c, c) $. Note that $ c $ is given by an object $ c_0: C $ and an idempotent $ a: C(c_0, c_0) $. Also, $ e $ is given by some idempotent $ e: C(c_0, c_0) $ with $ a \cdot e \cdot a = e $.

  Now, we have $ (c_0, e): \overline C $ and morphisms
  \begin{tikzcd}
    (c_0, a) \arrow[r, shift left, "e_\rightarrow"] & (c_0, e) \arrow[l, shift left, "e_\leftarrow"]
  \end{tikzcd}, both given by $ e $. We have $ e_\leftarrow \cdot e_\rightarrow = e = \id{(c_0, e)} $, so $ (c_0, e) $ is a retract of $ (c_0, a) $. Also, $ e = e_\rightarrow \cdot e_\leftarrow $, so $ e $ is split.
\end{proof}

\begin{remark}
  Note that the embedding is fully faithful, since
  \[ \overline C((c, \id c), (d, \id d)) = \{ f: C(c, d) \mid \id c \cdot f \cdot \id d = f \} = C(c, d). \]
\end{remark}

\begin{remark}\label{rem:retract-coequalizer}
  Let $ D $ be a category. Suppose that we have a retraction-section pair in $ D $, given by
  \begin{tikzcd}
    d \arrow[r, shift left, "r"] & d^\prime \arrow[l, shift left, "s"]
  \end{tikzcd}.
  Now, suppose that we have an object $ c: D $ and a morphism $ f $ with $ (r \cdot s) \cdot f = f $. Then we get a morphism $ s \cdot f: d^\prime \to c $ such that $ f $ factors as $ r \cdot (s \cdot f) $. Also, for any $ g $ with $ r \cdot g = f $, we have
  \[ g = s \cdot r \cdot g = s \cdot f. \]
  \begin{center}
    \begin{tikzcd}
      d \arrow[r, left, "r"] \arrow[rd, "f"'] & d^\prime \arrow[r, "s"] \arrow[d, "s \cdot f" description] & d \arrow[ld, "f"]\\
      & c
    \end{tikzcd}.
  \end{center}
  Therefore, $ d^\prime $ is the equalizer of \begin{tikzcd}
    d \arrow[r, shift left, "\id d"] \arrow[r, shift right, "r \cdot s"'] & d
  \end{tikzcd}. In the same way, it is also the coequalizer of this diagram.

  Now, note that if we have a coequalizer $ c^\prime $ of $ \id c $ and $ a $, and an equalizer $ d^\prime $ of $ \id d $ and $ b $, the universal properties of these give an equivalence
  \[ D(c^\prime, d^\prime) \cong \{ f: D(c, d^\prime) \mid a \cdot f = f \} \cong \{ f: D(c, d) \mid a \cdot f = f = f \cdot b \}. \]
  \begin{center}
    \begin{tikzcd}
      c \arrow[r, shift left, "\id c"] \arrow[r, shift right, "a"'] & c \arrow[r] \arrow[d] \arrow[rd] & c^\prime \arrow[d]\\
      d & d \arrow[l, shift right, "\id d"'] \arrow[l, shift left, "b"] & d^\prime \arrow[l]
    \end{tikzcd}
  \end{center}
\end{remark}

Since a functor preserves retracts, and since every object of $ \overline C $ is a retract of an object in $ C $, one can lift a functor from $ C $ (to a category with (co)equalizers) to a functor on $ \overline C $.

For convenience, the lemma below works with pointwise left Kan extension using colimits, but one could also prove this using just (co)equalizers (or right Kan extension using limits).
\begin{lemma}
  Let $ D $ be a category with colimits. We have an adjoint equivalence between $ [C, D] $ and $ [\overline C, D] $.
\end{lemma}
\begin{proof}
  We already have an adjunction $ \Lan {\iota_C} {} \dashv \iota_{C*} $. Also, since $ \iota_C $ is fully faithful, we know that $ \eta $ is a natural isomorphism. Therefore, we only have to show that $ \epsilon $ is a natural isomorphism. That is, we need to show that $ \epsilon_G(c, a): D(\Lan {\iota_C} (\iota_{C*} G) (c, a), G(c, a)) $ is an isomorphism for all $ G: [\overline C, D] $ and $ (c, a): \overline C $.

  One of the components in the diagram of $ \Lan {\iota_C} (\iota_{C*} G) (c, a) $ is the $ \iota_{C*} G(c) = G(c, \id c) $ corresponding to $ a: \iota_C(c) \to (c, a) $. This component has a morphism into our colimit
  \[ \varphi: C(G(\iota_C(c)), \Lan {\iota_C} (\iota_{C*} G) (c, a)). \]
  Note that we can view $ a $ as a morphism $ a: \overline C((c, a), \iota_C(c)) $. This gives us our inverse morphism
  \[ G(a) \cdot \varphi: C(G(c, a), \Lan {\iota_C} (\iota_{C*} G) (c, a)). \]
\end{proof}

\begin{lemma}
  The formation of the opposite category commutes with the formation of the Karoubi envelope.
\end{lemma}
\begin{proof}
  An object in $ \overline{\op C} $ is an object $ c: \op C $ (which is just an object $ c: C $), together with an idempotent morphism $ a: \op C(c, c) = C(c, c) $. This is the same as an object in $ \op{\overline C} $.

  A morphism in $ \overline{\op C}((c, a), (d, b)) $ is a morphism $ f: \op C(c, d) = C(d, c) $ such that
  \[ b \cdot_C f \cdot_C a = a \cdot_{\op C} f \cdot_{\op C} b = f. \]
  A morphism in $ \op{\overline C}((c, a), (d, b)) = \overline C((d, b), (c, a)) $ is a morphism $ f: C(d, c) $ such that $ b \cdot f \cdot a = f $.

  Now, in both categories, the identity morphism on $ (c, a) $ is given by $ a $.

  Lastly, $ \overline {\op C} $ inherits morphism composition from $ \op C $, which is the opposite of composition in $ C $. On the other hand, composition in $ \op{\overline C} $ is the opposite of composition in $ \overline C $, which inherits composition from $ C $.
\end{proof}

\begin{corollary}\label{cor:karoubi-presheaf}
  As the category $ \SET $ is cocomplete, we have an equivalence between the category of presheaves on $ C $ and the category of presheaves on $ \overline C $:
  \[ [\op C, \SET] \cong [\overline{\op C}, \SET] \cong [\op{\overline C}, \SET]. \]
\end{corollary}

Now, there is also another (classically equivalent) definition of the Karoubi envelope:
\begin{definition}\label{def:karoubi'}
  Let $ Y: C \to P C $ be the Yoneda embedding of $ C $ into its category of presheaves. We define the category $ \hat C $ as the full subcategory of $ P C $ consisting of objects $ F : P C $ such that there exists an object $ c: C $ and a retraction of $ Y(c) $ onto $ F $. Note that the existence of $ c $ and the retraction is \textit{mere existence} (see the chapter on Univalent Foundations) because we need this to be a subcategory of $ P C $.
\end{definition}

\begin{lemma}\label{lem:retracts-rezk}
  $ \hat C $ is the Rezk completion of $ \overline C $.
\end{lemma}
\begin{proof}
  First of all, note that $ \hat C $ is univalent, since it is a presheaf category (see Theorem 4.5 in \autocite{univalent-categories}). So we must show that we have a weak equivalence from $ \overline C $ to $ \hat C $.

  Secondly, by Corollary \ref{cor:karoubi-presheaf} we have an adjoint equivalence $ {\op{\iota_C}}_*: P \overline C \xrightarrow \sim P C $ and by Lemma \ref{lem:Yoneda-restriction-commutes}, the following diagram 2-commutes:
  \begin{center}
    \begin{tikzcd}
      C \arrow[r, "\iota_C"] \arrow[d, "Y"] & \overline C \arrow[d, "Y"]\\
      P C & P \overline C \arrow[l, "\sim"', "{\op{\iota_C}}_*"]
    \end{tikzcd}
  \end{center}

  By Lemma \ref{lem:karoubi-is-retract}, every object of $ \overline C $ is a retract of some object in $ c $. Therefore, every object in the image of $ \overline C $ inside $ P C $ is a retract of some object in the image of $ C $, so we can turn $ Y \bullet {\op{\iota_C}}_* : \overline C \to P C $ into a fully faithful embedding into $ \hat C $.

  Now, to show that it is essentially surjective, take an object $ F : \hat C $. That is, take a retract \begin{tikzcd} F \arrow[r, shift left, "s"] & Y(c) \arrow[l, shift left, "r"] \end{tikzcd}. Note that $ r \cdot s $ is an idempotent on $ Y(c) $. Since $ Y $ is a fully faithful embedding, it gives an equivalence on the hom-sets, so we have an element $ e : C(c, c) $ that maps to $ r \cdot s : P C (Y(c), Y(c)) $. We will show that $ (c, e) $ is our preimage in $ \overline C $ of $ F $.

  Also, $ (c, e) $ is a retract of $ \iota_C(c) $ with section and retraction $ e $. Therefore, $ \op{\iota_C} \bullet Y(c, e) $ is a retract of $ \op{\iota_C} \bullet Y (\iota_C (c)) $ with section and retraction $ \op{\iota_C} \bullet Y(e) $. Composing these gives an idempotent
  \[ \op{\iota_C} \bullet Y(e): d \mapsto f \mapsto f \cdot e \]
  on $ \op{\iota_C} \bullet Y (\iota_C (c)) $. Transporting this along the isomorphism with $ Y(c) $, we get an idempotent
  \[ d \mapsto f \mapsto \iota_C^{-1}(\iota_C(f) \cdot e) \]
  on $ Y(c) $. Note that we can view $ e $ both as an idempotent on $ c $ and on $ \iota_C(e) $, and that $ \iota_C(e) = e $, so $ \iota_C^{-1}(\iota_C(f) \cdot e) = f \cdot e $.

  Therefore, the idempotent on $ Y(c) $ is exactly $ Y(e) = r \cdot s $. By Remark \ref{rem:retract-coequalizer}, $ \op{\iota_C} \bullet Y(c, e) $ is the equalizer of $ \id{Y(c)} $ and $ r \cdot s $. Note that by the same remark, $ F $ is the equalizer of $ \id{Y(c)} $ and $ r \cdot s $ as well. Since the equalizer is unique up to unique isomorphism, we have an isomorphism $ F \cong \op{\iota_C} \bullet Y(c, e) $ and we see that $ Y \bullet {\op{\iota_C}}_* $ is essentially surjective.
\end{proof}

\begin{remark}
  As shown above, $ \overline C $ and $ \hat C $ have a very strong relation. In classical mathematics, they are even equivalent. However, since $ \overline C $ is not univalent, we do not have a full equivalence in univalent foundations and our choice of definition will have consequences. One one hand, $ \hat C $ is univalent. On the other hand, every object of $ \overline C $ is uniquely associated with one idempotent morphism in $ C $.

  In this thesis, I will choose to interpret the `category of retracts' or `Karoubi envelope' to mean $ \overline C $. This is because Dana Scott and Paul Taylor perform most of their constructions explicitly using the (idempotent) morphisms of $ C $, which is closest to working in $ \overline C $.

  It is however very well possible that most of the proofs can be made to work in $ \hat C $ as well. For future research, it would be interesting to see what the differences are between working in $ \overline C $ and $ \hat C $ for these proofs.
\end{remark}

\section{Monoids as categories}\label{sec:monoid-category}
Take a monoid $ M $.
\begin{definition}
  We can construct a category $ C_M $ with $ C_{M0} = \{ \star \} $, $ C_M(\star, \star) = M $. The identity morphism on $ \star $ is the identity $ 1: M $. The composition is given by multiplication $ g \cdot_{C_M} f = f \cdot_M g $.
\end{definition}

\begin{remark}
  Actually, we have a functor from the category of monoids to the category of setcategories (categories whose object type is a set).

  A monoid morphism $ f: M \to M^\prime $ is equivalent to a functor $ F_f: C_M \to C_{M^\prime} $. Any functor between $ C_M $ and $ C_{M^\prime} $ sends $ \star_M $ to $ \star_{M^\prime} $. The monoid morphism manifests as $ F_f(m) = f(m) $ for $ m: C_M(\star, \star) = M $.
\end{remark}

\begin{lemma}
  An isomorphism of monoids gives an (adjoint) equivalence of categories.
\end{lemma}
\begin{proof}
  Given an isomorphism $ f: M \to M^\prime $. Then we have functors $ F_f: C_M \to C_{M^\prime} $ and $ F_{f^{-1}}: C_{M^\prime} \to C_M $. Take the identity natural transformations $ \eta: \id{C_M} \Rightarrow F_f \bullet F_{f^{-1}} $ and $ \epsilon: F_{f^{-1}} \bullet F_f \Rightarrow \id{C_{M^\prime}} $. Of course these are natural isomorphisms.
\end{proof}

\begin{definition}
  A \textit{right monoid action}\index{monoid action} of $ M $ on a set $ X $ is a function $ X \times M \to X $ such that for all $ x: X $, $ m, m^\prime: M $,
  \[ x 1 = x \qquad \text{and} \qquad (x m) m^\prime = x (m \cdot m^\prime). \]
\end{definition}

\begin{definition}
  A \textit{morphism}\index{monoid action!morphism} between sets $ X $ and $ Y $ with a right $ M $-action is an $ M $-equivariant function $ f: X \to Y $: a function such that $ f(xm) = f(x)m $ for all $ x: X $ and $ m: M $.
\end{definition}

These, together with the identity and composition from $ \SET $, constitute a category \iindex{$ \RAct M $} of right $ M $-actions.

\begin{lemma}
  Presheaves on $ C_M $ are equivalent to sets with a right $ M $-action.
\end{lemma}
\begin{proof}
  This correspondence sends a presheaf $ F $ to the set $ F(\star) $, and conversely, the set $ X $ to the presheaf $ F $ given by $ F(\star) := X $. The $ M $-action corresponds to the presheaf acting on morphisms as $ xm = F(m)(x) $. A morphism (natural transformation) between presheaves $ F \Rightarrow G $ corresponds to a function $ F(\star) \to G(\star) $ that is $ M $-equivariant, which is exactly a monoid action morphism.
\end{proof}

\begin{remark}
  Since the category of sets with an $ M $-action is equivalent to a presheaf category, it has all limits. However, we can make this concrete. The set of the product $ \prod_i X_i $ is the product of the underlying sets. The action is given pointwise by $ (x_i)_i m = (x_i m)_i $.
\end{remark}

Note that the initial set with $ M $-action is $ \{ \star \} $, with action $ \star m = \star $.

\begin{lemma}\label{lem:global-action-elements}
  The global elements of a set with right $ M $-action correspond to the elements that are invariant under the $ M $-action.
\end{lemma}
\begin{proof}
  A global element of $ X $ is a morphism $ \varphi: \{ \star \} \to X $ such that for all $ m: M $, $ \varphi(\star)m = \varphi(\star m) = \varphi(\star) $. Therefore, it is given precisely by the element $ \varphi(\star): X $, which must be invariant under the $ M $-action.
\end{proof}

\begin{lemma}
  The category $ C $ of sets with an $ M $-action has exponentials.
\end{lemma}
\begin{proof}
  Given sets with $ M $-action $ X $ and $ Y $. Consider the set $ C(M \times X, Y) $ with an $ M $-action given by $ \phi m^\prime(m, x) = \phi(m^\prime m, x) $. This is the exponential object $ X^Y $, with the (universal) evaluation morphism $ X \times X^Y \to Y $ given by $ (x, \phi) \mapsto \phi(1, x) $. Explicitly, we get a natural isomorphism $ \psi: \RAct M(Z \times Y, X) \xrightarrow \sim \RAct M(Z, X^Y) $ given by
  \[ \psi(f)(z)(m, y) = f(z m, y) \quad \text{and} \quad \psi^{-1}(g)(z, y) = g(z)(1, y). \]
\end{proof}

\begin{definition}
  We can view $ M $ as a set $ U_M $ with right $ M $-action $ m n = m \cdot_M n $ for $ m: U_M $ and $ n: M $.
\end{definition}

\subsection{Extension and restriction of scalars}

Let $ \varphi: M \to M^\prime $ be a morphism of monoids.

Remember that sets with a right monoid action are equivalent to presheaves on the monoid category. Also, $ \varphi $ is equivalent to a functor between the monoid categories. The following is a specific case of the concepts in the section about Kan extension:

\begin{lemma}
  We get a \iindex{restriction of scalars} functor $ \varphi^* $ from sets with a right $ M^\prime $-action to sets with a right $ M $-action.
\end{lemma}
\begin{proof}
  Given a set $ X $ with right $ M^\prime $-action, take the set $ X $ again, and give it a right $ M $-action, sending $ (x, m) $ to $ x \varphi(m) $.

  On morphisms, send an $ M^\prime $-equivariant morphism $ f: X \to X^\prime $ to the $ M $-equivariant morphism $ f: X \to X^\prime $.
\end{proof}

Since $ \SET $ has colimits, and restriction of scalars corresponds to precomposition of presheaves (on $ C_{M^\prime} $), we can give it a left adjoint. This is the (pointwise) left Kan extension, which boils down to a very concrete definition, reminiscent of a tensor product:

\begin{lemma}\label{lem:scalar-extension}
  We get an \iindex{extension of scalars} functor $ \varphi_* $ from sets with a right $ M $-action to sets with a right $ M^\prime $-action.
\end{lemma}
\begin{proof}
  Given a set $ X $ with right $ M $-action. Take $ Y = X \times M^\prime / \sim $ with the relation $ (x m, m^\prime) \sim (x, f(m) \cdot m^\prime) $ for $ m: M $. This has a right $ M^\prime $-action given by $ (x, m^\prime)n^\prime = (x, m^\prime n^\prime) $.

  On morphisms, it sends the $ m $-equivariant $ f: X \to X^\prime $ to the morphism $ (x, m^\prime) \mapsto (f(x), m^\prime) $.
\end{proof}

\begin{lemma}\label{lem:scalar-extension-monoid-monoid-action}
  For $ U_M $ the set $ M $ with right $ M $-action, we have $ \varphi_*(U_M) \cong U_{M^\prime} $.
\end{lemma}
\begin{proof}
  The proof relies on the fact that for all $ m: U_M $ and $ m^\prime : M^\prime $, we have
  \[ (m, m^\prime) \sim (1, \varphi(m) m^\prime). \]
\end{proof}

Consider the category $ D $ with $ D_0 = M^\prime $ and
\[ D(m^\prime, \overline m^\prime) = \{ m: M \mid \varphi(m) \cdot m^\prime = \overline m^\prime \}. \]

\begin{lemma}\label{lem:scalar-extension-terminal}
  Suppose that $ D $ has a weakly terminal element. Then for $ I_M $ the terminal object in the category of sets with a right $ M $-action, we have $ \varphi_*(I_M) \cong I_{M^\prime} $.
\end{lemma}
\begin{proof}
  If $ D $ has a weakly terminal object, there exists $ m_0 : M^\prime $ such that for all $ m^\prime: M^\prime $, there exists $ m: M $ such that $ \varphi(m) \cdot m^\prime = m_0 $.

  The proof relies on the fact that every element of $ \varphi_*(I_M) $ is given by some $ (\star, m^\prime) $, but then
  \[ (\star, m^\prime) = (\star \cdot m, m^\prime) \sim (\star, \varphi(m) \cdot m^\prime) = (\star, m_0), \]
  so $ \varphi_*(I_M) $ has exactly $ 1 $ element.
\end{proof}

\begin{remark}
  For $ \varphi_* $ to preserve terminal objects, we actually only need $ D $ to be connected. The fact that $ \varphi_*(I_M) $ is a quotient by a symmetric and transitive relation then allows us to `walk' from any $ (\star, m^\prime_1) $ to any other $ (\star, m^\prime_2) $ in small steps.
\end{remark}

For any $ m^\prime_1, m^\prime_2: M^\prime $, consider the category $ D_{m^\prime_1, m^\prime_2} $, given by
\[ D_{m^\prime_1, m^\prime_2, 0} = \{ (m^\prime, m_1, m_2): M^\prime \times M \times M \mid m_i^\prime = \varphi(m_i) \cdot m^\prime \} \]
and
\[ D_{m^\prime_1, m^\prime_2}((m^\prime, m_1, m_2), (\overline m^\prime, \overline m_1, \overline m_2)) = \{ m: M \mid \varphi(m) \cdot m^\prime = \overline m^\prime, m_i = \overline m_i \cdot m \}. \]

\begin{lemma}\label{lem:scalar-extension-product}
  Suppose that $ D_{m^\prime_1, m^\prime_2} $ has a weakly terminal object for all $ m^\prime_1, m^\prime_2: M^\prime $. Then for sets $ A $ and $ B $ with right $ M $-action, we have $ \varphi_*(A \times B) \cong \varphi_*(A) \times \varphi_*(B) $.
\end{lemma}
\begin{proof}
  Now, any element in $ \varphi_*(A) \times \varphi_*(B) = (A \times M^\prime / \sim) \times (B \times M^\prime / \sim) $ is given by some $ (a, m^\prime_1, b, m^\prime_2) $.

  The fact that $ D_{m^\prime_1, m^\prime_2} $ has a weakly terminal object means that we have some $ \overline m^\prime: M^\prime $ and $ \overline m_1, \overline m_2: M $ with $ m_i^\prime = \varphi(\overline m_i) \cdot \overline m^\prime $. Therefore,
  \[ (a, m^\prime_1, b, m^\prime_2) = (a, \varphi(\overline m_1) \cdot \overline m^\prime, b, \varphi(\overline m_2) \cdot \overline m^\prime) \sim (a \overline m_1, \overline m^\prime, b \overline m_2, \overline m^\prime), \]
  so this is equivalent to some element in $ \varphi_*(A \times B) = (A \times B \times M^\prime / \sim) $. Note that this trivially respects the right $ M^\prime $-action.

  The fact that $ (\overline m^\prime, \overline m_1, \overline m_2) $ is weakly terminal also means that for all $ m^\prime: M^\prime $ and $ m_1, m_2: M $ with $ m_i^\prime = \varphi(m_i) \cdot m^\prime $, there exists $ m: M $ such that $ \varphi(m) \cdot m^\prime = \overline m^\prime $ and $ m_i = \overline m_i \cdot m $. This means that the equivalence that we established is actually well-defined: equivalent elements in $ \varphi_*(A) \times \varphi_*(B) $ are sent to equivalent elements in $ \varphi_*(A \times B) $.

  Therefore, we have an isomorphism $ \psi: \varphi^*(A) \times \varphi^*(B) \xrightarrow{\sim} \varphi^*(A \times B) $. Now we only need to show that the projections are preserved by this isomorphism. To that end, take $ x = (a, m^\prime_1, b, m^\prime_2) \sim (a \overline m_1, \overline m^\prime, b \overline m_2, \overline m^\prime) : \varphi^*(A) \times \varphi^*(B) $. We have
  \[ \varphi^*(\pi_1)(\psi(x)) = (a \overline m_1, \overline m^\prime) = \pi^\prime_1(x). \]
  In the same way, $ \varphi^*(\pi_2) \circ \psi = \pi^\prime_2 $ and this concludes the proof.
\end{proof}


  \chapter{Lemmas}

  \section{The endomorphism theory}

  \begin{definition}[Endomorphism theory]
    Suppose that we have a category $ C $ and an object $ X: C $, such that all powers $ X^n $ of $ X $ are also in $ C $.
    The endomorphism theory $ E(X) $ of $ X $ is the algebraic theory given by $ E(X)_n = C(X^n, X) $ with projections as variables $ x_{n, i}: X^n \to X $ and a substitution that sends $ f: X^m \to X $ and $ g_1, \dots, g_m: X^n \to X $ to $ f \circ \langle g_i \rangle_i: X^n \to X^m \to X $.
  \end{definition}

  \begin{lemma}
    $ E(X) $ is indeed an algebraic theory.
  \end{lemma}
  \begin{proof}
    For $ 1 \leq j \leq l $, $ f: E(X)_l $, $ g: {E(X)_m}^l $ and $ h: {E(X)_n}^m $, we have
    \[ x_j \bullet g = x_j \circ \langle g_i \rangle_i = g_j, \]
    \[ f \bullet (x_{l,i})_i = f \circ \langle x_{l, i} \rangle_i = f \circ \id{X^l} = f \]
    and
    \[ (f \bullet g) \bullet h = f \circ \langle g_i \rangle_i \circ \langle h_i \rangle_i = f \circ \langle g_i \circ \langle h_{i^\prime} \rangle_{i^\prime} \rangle_i = f \bullet (g_i \bullet h)_i. \]
  \end{proof}

  \begin{definition}[Endomorphism $ \lambda $-theory]
    Now, suppose that the exponential object $ X^X $ exists, and that we have morphisms back and forth $ abs: X^X \to X $ and $ app: X \to X^X $. Let, for $ Y: C $, $ \varphi_Y $ be the isomorphism $ C(X \times Y, X) \xrightarrow{\sim} C(Y, X^X) $.
    We can give $ E(X) $ a $ \lambda $-theory structure by setting, for $ f: E(X)_{n + 1} $ and $ g: E(X)_n $,
    \[ \lambda(f) = abs \circ \varphi_{X^n}(f) \qquad \rho(g) = \varphi_{X^n}^{-1}(app \circ g). \]
  \end{definition}

  \begin{lemma}
    $ E(X) $ is indeed a $ \lambda $-theory.
  \end{lemma}
  \begin{proof}
    Note that $ \varphi: C(- \times X, X) \xrightarrow{\sim} C(-, X^X) $ is a natural isomorphism, so for $ g: {E(X)_n}^m $, the following diagram commutes
    \begin{center}
      \begin{tikzcd}[column sep = 1in]
        C(X^m \times X, X) \arrow[r, "- \circ (\langle g_i \rangle_i \times \id X)"]\arrow[d, "\varphi_{X^m}", bend left] & C(X^n \times X, X^X) \arrow[d, "\varphi_{X^n}", bend left]\\
        C(X^m, X^X) \arrow[r, "- \circ \langle g_i \rangle_i"] \arrow[u, "\varphi_{X^m}^{-1}", bend left] & C(X^n, X^X) \arrow[u, "\varphi_{X^n}^{-1}", bend left]
      \end{tikzcd}
    \end{center}
    and note that $ \langle g_i \rangle_i \times \id X = \langle g_1, \dots, g_m, x_{n + 1} \rangle $. Then we have, for all $ f: E(X)_m $
    \begin{align*}
      \lambda_m(f) \bullet g &= abs \circ \varphi_{X^m}(f) \circ \langle g_i \rangle_i\\
      &= abs \circ \varphi_{X^n}(f \circ \langle g_1, \dots, g_m, x_{n + 1} \rangle)\\
      &= \lambda_n(f \bullet (g_1, \dots, g_m, x_{n + 1}))
    \end{align*}
    and
    \begin{align*}
      \rho_n(f \bullet g) &= \varphi_{X^n}^{-1}(app \circ f \circ \langle g_i \rangle_i)\\
      &= \varphi_{X^m}^{-1}(app \circ f) \circ \langle g_1, \dots, g_m, x_{n + 1} \rangle\\
      &= \rho_m(f) \bullet (g_1, \dots, g_m, x_{n + 1}).
    \end{align*}
  \end{proof}

  \section{The theory presheaf}

  \begin{definition}[The theory presheaf]
    Let $ T $ be an algebraic theory. We can turn $ T $ into an $ T $-presheaf $ \tilde T $ by setting $ \tilde T_n = T_n $ and using the substitution from $ T $:
    \[ \bullet: \tilde T_m \times T_n^m \to \tilde T_n. \]
  \end{definition}

  \begin{lemma}
    $ \tilde T $ is indeed a presheaf.
  \end{lemma}
  \begin{proof}
    For all $ t: \tilde T_l $, $ f: T_m^l $ and $ g: T_n^m $,
    \[ t \bullet (x_{l, i})_i = t \]
    and
    \[ (t \bullet f) \bullet g = t \bullet (f_i \bullet g)_i \]
    because $ T $ is an algebraic theory.
  \end{proof}

  \begin{lemma}
    Given an algebraic theory $ T $ and a $ T $-presheaf $ Q $, we have for all $ n $ a bijection of sets
    \[ \varphi: PT(\tilde T^n, Q) \cong Q_n. \]
  \end{lemma}
  \begin{proof}
    Take $ \varphi(f) = f_n(x_1, \dots, x_n) $.

    Conversely, take $ \varphi^{-1}(q) $ to be the presheaf morphism that sends $ t: T^n_m $ to $ q \bullet t : Q_m $. This is indeed a presheaf morphism, since for all $ t: T^n_l $ and $ f: T^l_m $,
    \[ \varphi^{-1}(q)(t \bullet f) = q \bullet t \bullet f = \varphi^{-1}(q)(t) \bullet f. \]

    Now, for a presheaf morphism $ f: T^n \to Q $ and $ t: T^n_m $, we have
    \[ \varphi^{-1}(\varphi(f))(t) = f_n(x_1, \dots, x_n) \bullet t = f_n((x_1, \dots, x_n) \bullet t) = f_n(t_1, \dots, t_n) = f_n(t). \]

    Conversely, given $ q: Q_n $, we have
    \[ \varphi(\varphi^{-1}(q)) = q \bullet (x_1, \dots, x_n) = q. \]
    which concludes the proof.
  \end{proof}

  \section{The `+l' presheaf}

  Let $ \iota_{m, n} : T_m \to T_{m + n} $ denote the function that sends $ f $ to $ f \bullet (x_{m + n, 1}, \dots, x_{m + n, m}) $. Note that
  \[ \iota_{m, n}(f) \bullet g = f \bullet (g_i)_{i \leq m} \]
  and
  \[ \iota_{m, n}(f \bullet g) = f \bullet g \bullet (x_i)_i = f \bullet (g_i \bullet (x_j)_j)_i = f \bullet (\iota_{m, n}(g_i))_i. \]

  For tuples $ x : X^m $ and $ y: X^n $, let $ x + y $ denote the tuple $ (x_1, \dots, x_m, y_1, \dots, y_n) : X^{m + n} $.

  \begin{definition}[The `+l' presheaf]
    Given a $ T $-presheaf $ Q $, we can construct a presheaf $ A(Q, l) $, given by $ A(Q, l)_n = Q_{n + l} $. Then, for $ q: A(Q, l)_m $ and $ f: T_n^m $, the substitution is given by
    \[ q \bullet_{A(Q, l)} f = q \bullet_Q ((\iota_{n, l} (f_i))_i + (x_{n + i})_i) \]
  \end{definition}
  \begin{lemma}
    The +l presheaf is a presheaf
  \end{lemma}
  \begin{proof}
    We have, for $ q: A(Q, l)_n $,
    \begin{align*}
      q \bullet_{A(Q, l)} (x_i)_i &= q \bullet_Q ((\iota_{n, l}(x_i))_i + (x_{n + i})_i)\\
      &= q \bullet_Q ((x_i)_i + (x_{n + i})_i)\\
      &= q \bullet_Q (x_i)_i\\
      &= q.
    \end{align*}
    We have, for $ q : A(Q, k)_l $, $ f: T_m^l $ and $ g: T_n^m $,
    \begin{align*}
      q \bullet_{A(Q, k)} f \bullet_{A(Q, k)} g &= q \bullet_Q ((\iota_{m, l}(f_i))_i + (x_{m + i})_i) \bullet_Q ((\iota_{n, l}(g_i))_i + (x_{n + i})_i)\\
      &= q \bullet_Q (((\iota_{m, l}(f_i) \bullet_T ((\iota_{n, l}(g_j))_j + (x_{n + j})_j))_i + (x_{m + i} \bullet_T ((\iota_{n, l}(g_j))_j + (x_{n + j})_j))_i))\\
      &= q \bullet_Q ((f_i \bullet_T (\iota_{n, l}(g_j))_j)_i + (x_{n + i})_i)\\
      &= q \bullet_Q ((\iota_{n, l}(f_i \bullet_T g))_i + (x_{n + i})_i)\\
      &= q \bullet_{A(Q, k)} (f_i \bullet_T g).
    \end{align*}
  \end{proof}

  \section{Exponentiability of the theory presheaf}

  \begin{lemma}
    For all $ l $, the presheaf $ \tilde T^l $ is exponentiable.
  \end{lemma}
  \begin{proof}
    We will show that $ A(-, l) $ constitutes a right adjoint to the functor $ - \times \tilde T^l $. We will do this using universal arrows (\cite{MacLane}, Chapter IV.1, Theorem 2 (iv)). To that end, we will need for all $ Q: PT $ a universal arrow $ \varphi: A(Q, l) \times \tilde T^l \to Q $.

    For $ q: A(Q, l)_n = Q_{n + l} $ and $ t: \tilde T^l_n $, we take $ \varphi(q, t) = q \bullet_Q ((x_{n, i})_i + t) $.

    This is a presheaf morphism, since for all $ q: A(Q, l)^l_m $, $ t: \tilde T^l_m $ and $ f: T_n^m $,
    \begin{align*}
      \varphi((q, t) \bullet_{A(Q, l) \times \tilde \tilde T^l} f) &= \varphi(q \bullet_{A(Q, l)} f, t \bullet_{\tilde T^l} f)\\
      &= q \bullet_{A(Q, l)} f \bullet_Q ((x_i)_i + (t \bullet_{\tilde T^l} f))\\
      &= q \bullet_Q ((\iota_{n, l}(f_i))_i + (x_{n + i})_i) \bullet_Q ((x_i)_i + (t \bullet_{\tilde T^l} f))\\
      &= q \bullet_Q ((\iota_{n, l}(f_i) \bullet_T ((x_j)_j + (t \bullet_{\tilde T^l} f)))_i + (x_{n + i} \bullet_T ((x_j)_j + (t \bullet_{\tilde T^l} f)))_i)\\
      &= q \bullet_Q ((f_i \bullet_T (x_j)_j)_i + ((t \bullet_{\tilde T^l} f)_i)_i)\\
      &= q \bullet_Q ((f_i)_i + (t_i \bullet_{\tilde T} f)_i)\\
      &= q \bullet_Q ((x_i \bullet_T f)_i + (t_i \bullet_T f)_i)\\
      &= q \bullet_Q ((x_i)_i + t) \bullet_Q f\\
      &= \varphi(q, t) \bullet_Q f.
    \end{align*}

    Now, given any presheaf $ Q^\prime : P T $ we need to show that any morphism $ \psi: Q^\prime \times \tilde T^l \to Q $ factors uniquely as $ \varphi \circ (\tilde \psi \times \id{\tilde T^l}) $ for some $ \tilde \psi: Q^\prime \to A(Q, l) $.

    So, given such a $ \psi $, and given $ q: Q^\prime_n $, we take $ \tilde \psi(q) = \psi(\iota_{n, l}(q), (x_{n + i})_i) $

    This is a presheaf morphism, since for all $ q: Q^\prime_m $ and $ f : T_n^m $,
    \begin{align*}
      \tilde \psi(q \bullet f) &= \psi(\iota_{n, l}(q \bullet f), (x_{n + i})_i)\\
      &= \psi(q \bullet (\iota_{n, l}(f_i))_i, (x_{n + i})_i)\\
      &= \psi((\iota_{m, l}(q), (x_{m + i})_i) \bullet_{Q^\prime \times \tilde T^l} ((\iota_{n, l}(f_i))_i + (x_{n + i})_i))\\
      &= \psi(\iota_{m, l}(q), (x_{m + i})_i) \bullet_Q ((\iota_{n, l}(f_i))_i + (x_{n + i})_i)\\
      &= \tilde \psi(q) \bullet_{A(Q, l)} f.
    \end{align*}
    Note that indeed $ \varphi \circ (\tilde \psi \times \id{\tilde T^l}) = \psi $:
    \begin{align*}
      \varphi(\tilde \psi(q), t) &= \varphi(\psi(\iota_{n, l}(q), (x_{n + i})_i), t)\\
      &= \psi(\iota_{n, l}(q), (x_{n + i})_i) \bullet ((x_i)_i + t)\\
      &= \psi(\iota_{n, l}(q) \bullet ((x_i)_i + t), (x_{n + i})_i \bullet ((x_i)_i + t))\\
      &= \psi(q \bullet (x_i)_i, (t_i)_i)\\
      &= \psi(q, t).\\
    \end{align*}
    Now, suppose that we have another $ \tilde \psi^\prime : Q^\prime \to A(Q, l) $ such that $ \varphi \circ (\tilde \psi^\prime \times \id{\tilde T^l}) = \psi $. Then we have
    \begin{align*}
      \tilde \psi(q) &= \psi(\iota_{n, l}(q), (x_{n + i})_i)\\
      &= (\varphi \circ (\tilde \psi^\prime \times \id{\tilde T^l}))(\iota_{n, l}(q), (x_{n + i})_i)\\
      &= \varphi(\tilde \psi^\prime(\iota_{n, l}(q)), (x_{n + i})_i)\\
      &= \tilde \psi^\prime(\iota_{n, l}(q)) \bullet ((x_i)_i + (x_{n + i})_i)\\
      &= \iota_{n, l}(\tilde \psi^\prime(q)) \bullet ((x_i)_i + (x_{n + i})_i)\\
      &= \tilde \psi^\prime(q) \bullet (x_i)_i\\
      &= \tilde \psi^\prime(q),
    \end{align*}
    so $ \tilde \psi $ is unique, which completes the proof.
  \end{proof}

  Now, this adjunction $ - \times \tilde T^l \dashv A(-, l) $ induces a natural isomorphism
  \[ \varphi: PT(- \times \tilde T^l, \tilde T) \xrightarrow{\sim} PT(-, A(\tilde T, l)) \]
  \begin{lemma}
    For all $ f: PT(\tilde T^n \times \tilde T^l, \tilde T) $,
    \[ \varphi_{\tilde T^n}(f)(q) = f(\iota_{m, l}(q), (x_{m + i})_i) \]
  \end{lemma}
  \begin{proof}
    \TODO
  \end{proof}

  \begin{lemma}
    For all $ f: PT(\tilde T^n, A(\tilde T, l)) $,
    \[ \varphi_{\tilde T^n}^{-1}(f)(q, t) = f(q) \bullet ((x_i)_i + t). \]
  \end{lemma}
  \begin{proof}
    \TODO
  \end{proof}

  \chapter{Theorems}

  \section{Scott's Representation Theorem}
  \begin{theorem}
    Any $ \lambda $-theory $ L $ is isomorphic to the endomorphism $ \lambda $-theory $ E(\tilde L) $ of $ \tilde L $ in the presheaf category of $ L $.
  \end{theorem}
  \begin{proof}
    First of all, remember that $ \tilde L $ is indeed exponentiable and that $ \tilde L^{\tilde L} = A(\tilde L, 1) $.
    Now, since $ L $ is a $ \lambda $-theory, we have functions back and forth $ \lambda: A(\tilde L, 1) \to \tilde L $ and $ \rho: \tilde L \to A(\tilde L, 1) $. These are presheaf morphisms because for all $ f: A(\tilde L, 1)_m $ and $ g: \tilde L_m $ and $ t: T_n^m $,
    \[ \lambda(f \bullet_{A(\tilde L, 1)} t) = \lambda(f \bullet_{\tilde L} ((\iota_{m, 1}(t_i))_i + (x_{n + 1}))) = \lambda(f) \bullet_{\tilde L} t \]
    and
    \[ \rho(g \bullet_{\tilde L} t) = \rho(g) \bullet_{\tilde L} ((\iota_{m, 1}(t_i))_i + (x_{n + 1})) = \rho(g) \bullet_{A(\tilde L, 1)} t. \]
    Therefore, $ E(\tilde L) $ is indeed a $ \lambda $-theory.

    For any presheaf $ Q $ and for any $ n $, we have a bijection $ PL(L^n, Q) \cong Q_n $.
    Then we have $ \varphi: E(\tilde L)_n \cong L_n $.
    This bijection is an isomorphism of $ \lambda $-theories, since it preserves the $ x_i $, $ \bullet $, $ \rho $ and $ \lambda $: for all $ 1 \leq j \leq n $, $ f: E(\tilde L)_m $, $ g: E(\tilde L)_{m + 1} $ and $ h: E(\tilde L)_n^m $.
    \begin{align*}
      \varphi(x_j) &= x_j(x_1, \dots, x_n)\\
      &= x_j;\\
      \varphi(f \bullet h) &= f \circ \langle h_i \rangle_i((x_i)_i)\\
      &= f((h_i((x_j)_j))_i)\\
      &= f((x_i)_i \bullet (h_i((x_j)_j))_i)\\
      &= f((x_i)_i) \bullet (h_i((x_j)_j))_i\\
      &= \varphi(f) \bullet (\varphi(h_i))_i;\\
      \varphi(\rho(f)) &= \rho(f)((x_i)_i)\\
      &= \rho(f((x_i)_i)) \bullet (x_i)_i\\
      &= \rho(f((x_i)_i))\\
      &= \rho(\varphi(f));\\
      \varphi(\lambda(g)) &= \lambda(g)((x_i)_i)\\
      &= \lambda(\varphi_{X^n}(g)((x_i)_i))\\
      &= \lambda(g(\iota_{m, l}((x_i)_i) + (x_{m + 1})))\\
      &= \lambda(g((x_i)_i))\\
      &= \lambda(\varphi(g)).
    \end{align*}
  \end{proof}

  \section{Locally cartesian closedness of the category of retracts}
  \begin{definition}[Category of retracts]
    The category of retracts for a $ \lambda $-theory $ L $ is the category with objects $ f: L_n $ such that $ f \bullet f = f $ and it has as morphisms $ g: f \to f^\prime $ the terms $ g: L_n $ such that $ f^\prime \bullet g \bullet f = g $. The object $ f: L_n $ has identity element $ f $, and we have composition $ g \circ g^\prime = g \bullet g^\prime $. These are morphisms \TODO
  \end{definition}

  \begin{lemma}
    The category of retracts is indeed a category.
  \end{lemma}
  \begin{proof}
    \TODO
  \end{proof}

  \begin{theorem}
    The category of retracts is locally cartesian closed \TODO.
  \end{theorem}

  \section{The Fundamental Theorem of the \texorpdfstring{$ \lambda $-}{lambda }calculus}

  \begin{definition}[$ \Lambda $]
    There is a special $ \lambda $-theory, given by the $ \lambda $-calculus itself. $ \Lambda_n $ is the set of $ \lambda $-terms with $ n $ free variables, the $ x_i $ are the free variables, and $ \bullet $ is given by substitution. $ \lambda $ sends $ f: \Lambda_{n + 1} $ to $ \lambda x_{n + 1}, f $ and $ \rho $ sends $ f: \Lambda_n $ to $ \iota_{n, 1}(f) x_{n + 1} $ in $ \Lambda_n $.
  \end{definition}

  \begin{lemma}
    $ \Lambda $ is indeed a $ \lambda $-theory.
  \end{lemma}
  \begin{proof}
    \TODO
  \end{proof}

  \begin{lemma}
    $ \Lambda $ is the initial $ \lambda $-theory.
  \end{lemma}
  \begin{proof}
    Given a $ \lambda $-theory $ L $, we construct a morphism $ f: \Lambda \to L $ by induction on the $ \lambda $-terms. We set $ f(x_i) = x_i $, $ f(\lambda(t)) = \lambda(f(t)) $ and $ f(st) = \rho(f(s)) \bullet ((x_i)_i + (f(t))) $.

    This is a $ \lambda $-theory morphism because \TODO

    It is unique, since \TODO
  \end{proof}

  \begin{definition}[Pullback of algebras]
    If we have a morphism of algebraic theories $ f: T^\prime \to T $, we have a functor $ AT \to AT^\prime $.

    On objects, it sends a $ T $-algebra $ A $ to a $ T^\prime $-algebra with set $ A $ and action $ g \bullet_{T^\prime} a = f(g) \bullet_T a $. This is a $ T^\prime $-algebra because \TODO.

    On morphisms, it sends $ \varphi: A \to A $ to $ \varphi: A \to A $. This is a $ T^\prime $-algebra morphism because for all $ g: T^\prime_n $ and $ a: A^n $, we have
    \[ \varphi(g \bullet_{T^\prime} a) = \varphi(f(g) \bullet_T a) = f(g) \bullet_T \varphi(a) = g \bullet_{T^\prime} \varphi(a). \]
  \end{definition}
  \begin{lemma}
    This is indeed a functor.
  \end{lemma}
  \begin{proof}
    \TODO
  \end{proof}

  \begin{definition}[Term algebra]
    Given an algebraic theory $ T $, for every $ n $, $ T_n $ together with the action operator $ \bullet: T_m \times T_n^m \to T_n $ gives a $ T $-algebra.
  \end{definition}

  \begin{lemma}
    $ T_n $ is indeed a $ T $-algebra.
  \end{lemma}
  \begin{proof}
    \TODO
  \end{proof}

  \begin{definition}
    For all $ n $, we have a functor from lambda theories to $ \Lambda $-algebras. It sends the $ \lambda $-theory $ L $ to the $ L $-algebra $ L_n $ and then turns this into a $ \Lambda $-algebra via the morpism $ \Lambda \to L $.

    It sends morphisms $ f: L \to L^\prime $ to $ f_n : L_n \to L^\prime_n $. This is a $ \Lambda $-algebra morphism because \TODO
  \end{definition}

  \begin{lemma}
    This indeed constitutes a functor.
  \end{lemma}
  \begin{proof}
    \TODO
  \end{proof}

  \begin{remark}
    Note that for a monoid $ M $, if we view $ M $ as a category, the category $ [\op{M}, \SET] $ consists of sets with a right $ M $-action.
  \end{remark}

  \begin{definition}[The exponential object in the presheaf category]
    Given a monoid $ M $, if we have two presheaves (sets with right $ M $-actions) $ P $ and $ P^\prime $, we have a set of $ M $-equivariant maps
    \[ F_{P, P^\prime} = \left\{ f: M \times P \to P^\prime \mid \prod_{p : P, m, m^\prime: M} f(m, p)m^\prime = f(m m^\prime, p m^\prime) \right\} \]
    with a right $ M $-action, given by $ f m^\prime(m, p) = f(m^\prime m, p) $. This is again $ M $-equivariant because
    \[ fm^\prime(m, p)m^{\prime \prime} = f(m^\prime m, p)m^{\prime \prime} = f(m^\prime m m^{\prime \prime}, p m^{\prime \prime}) = f m^\prime(m m^{\prime \prime}, p m^{\prime \prime}), \]
    so $ F_{P, P^\prime} $ is a presheaf.

    Now, to show that $ F_{P, P^\prime} $ is the exponential object $ {P^\prime}^P $, we show that for any $ P $, $ F_{P, -} $ is the left adjoint of $ - \times P $. So we need for all $ P^\prime: PT $, a universal arrow $ \varphi: F_{P, P^\prime} \times P \to P^\prime $.

    First of all, we have an evaluation map $ \varphi: F_{P, P^\prime} \times P \to P^\prime $ given by $ (f, p) \mapsto f(I, p) $ for $ I $ the unit of the monoid. This map is equivariant because for all $ m $,
    \[ (f, p) m = (f m, p m) \mapsto f m(I, p m) = f(m, p m) = f(I, p) m. \]
    Now, given any presheaf $ Q $ and any morphism $ \psi: Q \times P \to P^\prime $, take $ \tilde \psi: Q \to F_{P, P^\prime} $ given by $ \psi(q)(m, p) = \psi(q m, p) $. This is equivariant because
    \[ \tilde \psi(q)m(m^\prime, p) = \tilde \psi(q)(m m^\prime, p) = \psi(q m m^\prime, p) = \tilde \psi(q m)(m^\prime, p) \]
    and we have
    \[ \varphi(\tilde \psi(q), p) = \tilde \psi(q)(I, p) = \psi(q, p). \]
    Now, suppose that we have $ \tilde \psi^\prime: Q \to F_{P, P^\prime} $ such that $ \varphi \circ (\tilde \psi^\prime \times \id{P}) = \psi $. Then for all $ q : Q $, $ m: M $ and $ p: P $,
    \[ \tilde \psi(q)(m, p) = \psi(q m, p) = \varphi(\tilde \psi^\prime(q m), p) = \tilde \psi^\prime(q m)(I, p) = \psi^\prime(q) m(I, p) = \psi^\prime(q)(m, p), \]
    so $ \tilde \psi $ is unique and $ F_{P, P^\prime} $ is an exponential object.
  \end{definition}

  \begin{definition}[n-functional terms]
    Let $ A $ be a $ \Lambda $-algebra. We define
    \[ A(n) = \{ a : A \mid (\lambda x_2 x_3 \dots x_{n + 1}, x_1 x_2 x_3 \dots x_{n + 1}) \bullet a = a \}. \]
  \end{definition}

  \begin{definition}
    Take $ \mathbf 1_n = (\lambda x_1 \dots x_n, x_1 \dots x_n) \bullet () : A $.
  \end{definition}

  \begin{definition}
    We define composition as $ a \circ b = (\lambda x_3, x_1 (x_2 x_3)) \circ (a, b) $ for $ a, b : A $.
  \end{definition}

  \begin{lemma}
    This composition is associative.
  \end{lemma}
  \begin{proof}
    \TODO
  \end{proof}

  \begin{definition}[The monoid of a $ \Lambda $-algebra]
    Now we make $ A(1) $ into a monoid with unit $ \lambda x_1, x_1 $.
  \end{definition}

  \begin{lemma}
    This is indeed a monoid.
  \end{lemma}
  \begin{proof}
    \TODO
  \end{proof}

  From here on, we will assume that $ \Lambda $ (and therefore, any $ \lambda $-theory) satisfies $ \beta $-equality.

  \begin{lemma}
    For $ a: A $, $ a $ is in $ A(n) $ iff $ \mathbf 1_n \circ a = a $.
  \end{lemma}
  \begin{proof}
    \begin{align*}
      \mathbf 1_n \circ a
      &= (\lambda x_3, x_1 (x_2 x_3)) \bullet (((\lambda x_1 \dots x_n, x_1 \dots x_n) \bullet ()), a)\\
      &= (\lambda x_3, x_1 (x_2 x_3)) \bullet (((\lambda x_2 \dots x_{n + 1}, x_2 \dots x_{n + 1}) \bullet a), x_1 \bullet a)\\
      &= ((\lambda x_3, x_1 (x_2 x_3)) \bullet ((\lambda x_2 \dots x_{n + 1}, x_2 \dots x_{n + 1}), x_1)) \bullet a\\
      &= (\lambda x_2, (\lambda x_3 \dots x_{n + 2}, x_3 \dots x_{n + 2}) (x_1 x_2)) \bullet a\\
      &= (\lambda x_2 x_3 \dots x_{n + 1}, x_1 x_2 \dots x_{n + 1}) \bullet a.
    \end{align*}
  \end{proof}

  \begin{definition}[The presheaf category of a $ \Lambda $-algebra]
    Let $ A $ be a $ \Lambda $-algebra. If we view the monoid $ A(1) $ as a one-object category, we define the category $ PA $ to be the category of presheaves $ [\op{A(1)}, \SET] $.
  \end{definition}

  \begin{definition}[The objects $ A(n) $ in $ PA $]
    Given $ a: A(n) $ and $ b: A(1) $, we have
    \[ \mathbf 1_n \circ (a \circ b) = (\mathbf 1_n \circ a) \circ b = a \circ b, \]
    so $ a \circ b: A(n) $ and we have a right $ A(1) $-action on $ A(n) $, which makes $ A(n) $ into an object in $ PA $.
  \end{definition}

  \begin{lemma}
    We have $ A(1)^{A(1)} \cong A(2) $.
  \end{lemma}
  \begin{proof}
    We have a bijection $ \varphi: A(2) \cong F_{A(1), A(1)} $, given by
    \[ \varphi(a)(b, b^\prime) = (\lambda x_4, x_1 (x_2 x_4) (x_3 x_4)) \bullet (a, b, b^\prime). \]
    Note that $ \varphi(d) $ is equivariant because \TODO
    Now, $ \varphi $ is a presheaf morphism because \TODO

    Take $ p = \lambda x_1, x_1 (\lambda x_2 x_3, x_2) $ and $ q = \lambda x_1, x_1 (\lambda x_2 x_3, x_3) $. These are elements of $ A(1) $. Note that for terms $ c_1, c_2 $
    \begin{align*}
      p (\lambda x_1, x_1 c_1 c_2)
      &= (\lambda x_1, x_1 c_1 c_2) (\lambda x_2 x_3, x_2)\\
      &= (\lambda x_1 x_3, x_2) c_1 c_2\\
      &= c_1.
    \end{align*}
    In the same way, $ q \circ (\lambda x_1 x_2, x_2 c_1 c_2) = c_2 $.

    An inverse is given by
    \[ \psi: f \mapsto \lambda x_1 x_2, f(p, q)(\lambda x_3, x_3 x_1 x_2). \]
    This is a presheaf morphism because \TODO

    This is an inverse, because given $ f: F_{A(1), A(1)} $ and$ (a_1, a_2): A(1) \times A(1) $, we have
    \begin{align*}
      \varphi(\psi(f))(a_1, a_2) &= u(\lambda x_1 x_2, f(p, q)(\lambda x_3, x_3 x_1 x_2))(a_1, a_2)\\
      &= \lambda x_1, (\lambda x_2 x_3, f(p, q)(\lambda x_4, x_4 x_2 x_3)) (a_1 x_1) (a_2 x_1)\\
      &= \lambda x_1, f(p, q)(\lambda x_2, x_2 (a_1 x_1) (a_2 x_1))\\
      &= f(p, q) \circ (\lambda x_1, (\lambda x_2, x_2 (a_1 x_1) (a_2 x_1)))\\
      &= f(p \circ (\lambda x_1, (\lambda x_2, x_2 (a_1 x_1) (a_2 x_1))), q \circ (\lambda x_1, (\lambda x_2, x_2 (a_1 x_1) (a_2 x_1))))\\
      &= f(\lambda x_1, p (\lambda x_2, x_2 (a_1 x_1) (a_2 x_1)), \lambda x_1, q (\lambda x_2, x_2 (a_1 x_1) (a_2 x_1)))\\
      &= f(\lambda x_1, a_1 x_1, \lambda x_1, a_2 x_1)\\
      &= f(a_1, a_2).
    \end{align*}
    The last line is because $ a_i : A(1) $ and therefore $ \lambda x_1, a_i x_1 = a_i $.

    On the other hand, if we have $ a_1: A(2) $, we have
    \begin{align*}
      \psi(\varphi(a_1)) &= \psi((a_2, a_3) \mapsto \lambda x_1, a_1 (a_2 x_1) (a_3 x_1))\\
      &= \lambda x_1 x_2, (\lambda x_3, a_1 (p x_3) (q x_3)) (\lambda x_3, x_3 x_1 x_2)\\
      &= \lambda x_1 x_2, a_1 (p (\lambda x_3, x_3 x_1 x_2)) (q (\lambda x_3, x_3 x_1 x_2))\\
      &= \lambda x_1 x_2, a_1 x_1 x_2\\
      &= a_1.
    \end{align*}
    The last line is because $ a_1 : A(2) $ and therefore $ \lambda x_1 x_2, a_1 x_1 x_2 = a_1 $.

    Therefore, this map is a bijection and an isomorphism.
  \end{proof}

  \begin{definition}[Endomorphism $ \lambda $-theory of a $ \Lambda $-algebra]
    $ PA $ borrows products from $ \SET $. Therefore, the algebraic theory $ E(A(1)) $ exists. Now note that $ A(1) $ is exponentiable and $ A(1)^{A(1)} \cong A(2) $.
    Note that $ A(2) \subseteq A(1) $ and that $ (\lambda x_2 x_3, x_1 x_2 x_3) \bullet - $ gives a function from $ A(1) $ to $ A(2) $. This gives $ E(A(1)) $ a $ \lambda $-theory structure.
  \end{definition}

  \begin{definition}[Pullback functor on presheaves for a $ \Lambda $-algebra]
    Let $ f: A \to A^\prime $ be a $ \Lambda $-algebra morphism. Then for all $ a: A(n) $,
    \[ \mathbf 1_n \circ f(a) = f(\mathbf 1_n) \circ f(a) = f(\mathbf 1_n \circ a), \]
    so we have an induced morphism $ f: A(n) \to A^\prime(n) $.

    Now, given a presheaf $ P: PA^\prime $. We can create a presheaf $ f^* P : PA $ by taking the set of $ P $, and, for $ p: P $ and $ a: A $, setting $ pa = p \circ f(a) $. This is indeed a presheaf because \TODO

    Now, given a morphism $ g: P \to P^\prime $, we get a morphism by taking the function on the sets of $ P $ and $ P^\prime $. This is a morphism because \TODO
  \end{definition}

  \begin{lemma}
    The above indeed constitutes a functor.
  \end{lemma}
  \begin{proof}
    \TODO
  \end{proof}

  Left Kan extension then gives a left adjoint $ f_*: PA \to PA^\prime $.

  \begin{lemma}
    We have $ f_*(A(1)) \cong A^\prime(1) $.
  \end{lemma}
  \begin{proof}
    \TODO
  \end{proof}

  \begin{lemma}
    $ f_* $ preserves finite products.
  \end{lemma}
  \begin{proof}
    \TODO
  \end{proof}

  \begin{definition}
    Since $ f_* $ preserves finite products, given an element of $ g: E(A(1))(n) = PA(A(1)^n, A(1)) $, we get
    \[ \# f_*(g): PA^\prime(f(A(1)^n), f(A(1))) \cong PA^\prime(A^\prime(1)^n, A^\prime(1)) = E(A^\prime(1))(n). \]
  \end{definition}

  \begin{lemma}
    $ \# f_*: E(A(1)) \to E(A^\prime(1)) $ is a map of $ \lambda $-theories.
  \end{lemma}
  \begin{proof}
    \TODO
  \end{proof}

  \begin{definition}
    We have an isomorphism $ E(A(1))(0) \cong A $ given by $ a \mapsto a I $.
  \end{definition}

  \begin{lemma}
    This is indeed an isomorphism of $ \Lambda $-algebras.
  \end{lemma}
  \begin{proof}
    \TODO
  \end{proof}

  \begin{lemma}
    Given $ g: A \to A^\prime $,
  \end{lemma}

  \begin{theorem}
    There exists an adjoint equivalence between the category of $ \lambda $-theories, and the category of algebras of $ \Lambda $.
  \end{theorem}
  \begin{proof}
    We will show that the functor $ L \mapsto L_0 $ is an equivalence of categories.

    It is essentially surjective, because $ L $ is isomorphic \TODO to $ E(A(1)) $.

    Now, given morphisms $ f, f^\prime: L \to L^\prime $. Suppose that $ f_0 = f^\prime_0 $. Suppose that $ L $ and $ L^\prime $ have $ \beta $-equality. Then, given $ l: L_n $, we have
    \[ f_n(l) = \rho^n(\lambda^n(f_n(l))) = \rho^n(f_0(\lambda^n(l))) = \rho^n(f^\prime_0(\lambda^n(l))) = \rho^n(\lambda^n(f^\prime_n(l))) = f^\prime_n(l), \]
    so the functor is faithful.

    The functor is full because a $ \Lambda $-algebra morphism $ f: A \to A^\prime $ induces a functor $ f^*: PA^\prime \to PA $, and via left Kan extension we get a left adjoint $ f_*: PA \to PA^\prime $ with $ f_*(A(1)) \cong A^\prime(1) $. Now, $ f_* $ preserves (finite) products, so we have maps $ PA(A(1)^n, A(1)) \to PA^\prime(A^\prime(1)^n, A^\prime(1)) $ and so a map $ E(A(1)) \to E(A^\prime(1)) $. This map, when restricted to a map $ PA(1, A(1)) \to PA^\prime(1, A(1)) $, and transported along the isomorphism $ a \mapsto a I $ \TODO, is equal to $ f $ \TODO.
  \end{proof}

  \begin{lemma}
    The category of $ T $-algebras has coproducts.
  \end{lemma}
  \begin{proof}
    \TODO
  \end{proof}

  \begin{definition}[Theory of extensions]
    Let $ T $ be an algebraic theory and $ A $ a $ T $-algebra. We can define an algebraic theory $ T_A $ called `the theory of extensions of $ A $' with $ (T_A)_n = T_n + A $. The left injection of the variables $ x_i : T_n $ gives the variables.
    Now, take $ h: (T_n + A)^m $. Sending $ g: T_m $ to $ \varphi(g) := g \bullet h $ gives a $ T $-algebra morphism $ T_m \to T_n + A $ since
    \[ \varphi(f \bullet g) = f \bullet g \bullet h = f \bullet (g_i \bullet h) = f \bullet (\varphi(g_i))_i. \]
    This, together with the injection morphism of $ A $ into $ T_n + A $, gives us a $ T $-algebra morphism from the coproduct: $ T_m + A \to T_n + A $. We especially have a function on sets $ (T_m + A) \times (T_n + A)^m \to T_n + A $, which we will define our substitution to be.
  \end{definition}

  \begin{lemma}
    $ T_A $ is indeed an algebraic theory.
  \end{lemma}
  \begin{proof}
    \TODO
  \end{proof}

  \bibliographystyle{alpha}
  \bibliography{citations}

\end{document}
