\chapter{The paper}

\section{Scott's Representation Theorem}
\begin{theorem}
  Any $ \lambda $-theory $ L $ is isomorphic to the endomorphism $ \lambda $-theory $ E(L) $ of $ L: \Pshf L $, which is $ L $, viewed as a presheaf in its own presheaf category.
\end{theorem}
\begin{proof}
  First of all, remember that $ L $ is indeed exponentiable and that $ L^L = A(L, 1) $.
  Now, since $ L $ is a $ \lambda $-theory, we have sequences of functions back and forth $ \lambda_n: A(L, 1)_n \to L_n $ and $ \rho_n: L_n \to A(L, 1)_n $. These commute with the $ L $-actions, so they constitute presheaf morphisms and $ E(L) $ is indeed a $ \lambda $-theory.

  Lemma \ref{lem:presheaf-Yoneda} gives a sequence of bijections $ \varphi_n : \Pshf L(L^n, L) \cong L_n $ for all $ n $, sending $ F: \Pshf L(L^n, L) $ to $ F(x_1, \dots, x_n) $, and conversely sending $ s: L_n $ to $ ((t_1, \dots, t_n) \mapsto s \bullet (t_1, \dots, t_n)) $. It considers $ \lambda $-terms in $ n $ variables as $ n $-ary functions on the $ \lambda $-calculus. Therefore, it should come as no surprise that $ \varphi $ preserves the $ x_i $, $ \bullet $, $ \rho $ and $ \lambda $, which makes it into an isomorphism of $ \lambda $-theories and this concludes the proof.
\end{proof}

\section{Locally cartesian closedness of the category of retracts}
\begin{definition}[Category of retracts]
  The category of retracts for a $ \lambda $-theory $ L $ is the category with objects $ f: L_n $ such that $ f \bullet f = f $ and it has as morphisms $ g: f \to f^\prime $ the terms $ g: L_n $ such that $ f^\prime \bullet g \bullet f = g $. The object $ f: L_n $ has identity element $ f $, and we have composition $ g \circ g^\prime = g \bullet g^\prime $. These are morphisms \TODO
\end{definition}

\begin{lemma}
  The category of retracts is indeed a category.
\end{lemma}
\begin{proof}
  \TODO
\end{proof}

\begin{theorem}
  The category of retracts is locally cartesian closed \TODO.
\end{theorem}

\section{Equivalences}

\section{The Fundamental Theorem of the \texorpdfstring{$ \lambda $-}{lambda }calculus}

\begin{definition}[Pullback of algebras]
  If we have a morphism of algebraic theories $ f: T^\prime \to T $, we have a functor $ AT \to AT^\prime $.

  On objects, it sends a $ T $-algebra $ A $ to a $ T^\prime $-algebra with set $ A $ and action $ g \bullet_{T^\prime} a = f(g) \bullet_T a $. This is a $ T^\prime $-algebra because \TODO.

  On morphisms, it sends $ \varphi: A \to A $ to $ \varphi: A \to A $. This is a $ T^\prime $-algebra morphism because for all $ g: T^\prime_n $ and $ a: A^n $, we have
  \[ \varphi(g \bullet_{T^\prime} a) = \varphi(f(g) \bullet_T a) = f(g) \bullet_T \varphi(a) = g \bullet_{T^\prime} \varphi(a). \]
\end{definition}
\begin{lemma}
  This is indeed a functor.
\end{lemma}
\begin{proof}
  \TODO
\end{proof}

\begin{definition}[Term algebra]
  Given an algebraic theory $ T $, for every $ n $, $ T_n $ together with the action operator $ \bullet: T_m \times T_n^m \to T_n $ gives a $ T $-algebra.
\end{definition}

\begin{lemma}
  $ T_n $ is indeed a $ T $-algebra.
\end{lemma}
\begin{proof}
  \TODO
\end{proof}

\begin{definition}
  For all $ n $, we have a functor from lambda theories to $ \Lambda $-algebras. It sends the $ \lambda $-theory $ L $ to the $ L $-algebra $ L_n $ and then turns this into a $ \Lambda $-algebra via the morpism $ \Lambda \to L $.

  It sends morphisms $ f: L \to L^\prime $ to $ f_n : L_n \to L^\prime_n $. This is a $ \Lambda $-algebra morphism because \TODO
\end{definition}

\begin{lemma}
  This indeed constitutes a functor.
\end{lemma}
\begin{proof}
  \TODO
\end{proof}

\begin{remark}
  Note that for a monoid $ M $, if we view $ M $ as a category, the category $ [\op{M}, \SET] $ consists of sets with a right $ M $-action.
\end{remark}

\begin{definition}[The exponential object in the presheaf category]
  Given a monoid $ M $, if we have two presheaves (sets with right $ M $-actions) $ P $ and $ P^\prime $, we have a set of $ M $-equivariant maps
  \[ F_{P, P^\prime} = \left\{ f: M \times P \to P^\prime \mid \prod_{p : P, m, m^\prime: M} f(m, p)m^\prime = f(m m^\prime, p m^\prime) \right\} \]
  with a right $ M $-action, given by $ f m^\prime(m, p) = f(m^\prime m, p) $. This is again $ M $-equivariant because
  \[ fm^\prime(m, p)m^{\prime \prime} = f(m^\prime m, p)m^{\prime \prime} = f(m^\prime m m^{\prime \prime}, p m^{\prime \prime}) = f m^\prime(m m^{\prime \prime}, p m^{\prime \prime}), \]
  so $ F_{P, P^\prime} $ is a presheaf.

  Now, to show that $ F_{P, P^\prime} $ is the exponential object $ {P^\prime}^P $, we show that for any $ P $, $ F_{P, -} $ is the left adjoint of $ - \times P $. So we need for all $ P^\prime: PT $, a universal arrow $ \varphi: F_{P, P^\prime} \times P \to P^\prime $.

  First of all, we have an evaluation map $ \varphi: F_{P, P^\prime} \times P \to P^\prime $ given by $ (f, p) \mapsto f(I, p) $ for $ I $ the unit of the monoid. This map is equivariant because for all $ m $,
  \[ (f, p) m = (f m, p m) \mapsto f m(I, p m) = f(m, p m) = f(I, p) m. \]
  Now, given any presheaf $ Q $ and any morphism $ \psi: Q \times P \to P^\prime $, take $ \tilde \psi: Q \to F_{P, P^\prime} $ given by $ \psi(q)(m, p) = \psi(q m, p) $. This is equivariant because
  \[ \tilde \psi(q)m(m^\prime, p) = \tilde \psi(q)(m m^\prime, p) = \psi(q m m^\prime, p) = \tilde \psi(q m)(m^\prime, p) \]
  and we have
  \[ \varphi(\tilde \psi(q), p) = \tilde \psi(q)(I, p) = \psi(q, p). \]
  Now, suppose that we have $ \tilde \psi^\prime: Q \to F_{P, P^\prime} $ such that $ \varphi \circ (\tilde \psi^\prime \times \id{P}) = \psi $. Then for all $ q : Q $, $ m: M $ and $ p: P $,
  \[ \tilde \psi(q)(m, p) = \psi(q m, p) = \varphi(\tilde \psi^\prime(q m), p) = \tilde \psi^\prime(q m)(I, p) = \psi^\prime(q) m(I, p) = \psi^\prime(q)(m, p), \]
  so $ \tilde \psi $ is unique and $ F_{P, P^\prime} $ is an exponential object.
\end{definition}

\begin{definition}[n-functional terms]
  Let $ A $ be a $ \Lambda $-algebra. We define
  \[ A(n) = \{ a : A \mid (\lambda x_2 x_3 \dots x_{n + 1}, x_1 x_2 x_3 \dots x_{n + 1}) \bullet a = a \}. \]
\end{definition}

\begin{definition}
  Take $ \mathbf 1_n = (\lambda x_1 \dots x_n, x_1 \dots x_n) \bullet () : A $.
\end{definition}

\begin{definition}
  We define composition as $ a \circ b = (\lambda x_3, x_1 (x_2 x_3)) \circ (a, b) $ for $ a, b : A $.
\end{definition}

\begin{lemma}
  This composition is associative.
\end{lemma}
\begin{proof}
  \TODO
\end{proof}

\begin{definition}[The monoid of a $ \Lambda $-algebra]
  Now we make $ A(1) $ into a monoid with unit $ \lambda x_1, x_1 $.
\end{definition}

\begin{lemma}
  This is indeed a monoid.
\end{lemma}
\begin{proof}
  \TODO
\end{proof}

From here on, we will assume that $ \Lambda $ (and therefore, any $ \lambda $-theory) satisfies $ \beta $-equality.

\begin{lemma}
  For $ a: A $, $ a $ is in $ A(n) $ iff $ \mathbf 1_n \circ a = a $.
\end{lemma}
\begin{proof}
  \begin{align*}
    \mathbf 1_n \circ a
    &= (\lambda x_3, x_1 (x_2 x_3)) \bullet (((\lambda x_1 \dots x_n, x_1 \dots x_n) \bullet ()), a)\\
    &= (\lambda x_3, x_1 (x_2 x_3)) \bullet (((\lambda x_2 \dots x_{n + 1}, x_2 \dots x_{n + 1}) \bullet a), x_1 \bullet a)\\
    &= ((\lambda x_3, x_1 (x_2 x_3)) \bullet ((\lambda x_2 \dots x_{n + 1}, x_2 \dots x_{n + 1}), x_1)) \bullet a\\
    &= (\lambda x_2, (\lambda x_3 \dots x_{n + 2}, x_3 \dots x_{n + 2}) (x_1 x_2)) \bullet a\\
    &= (\lambda x_2 x_3 \dots x_{n + 1}, x_1 x_2 \dots x_{n + 1}) \bullet a.
  \end{align*}
\end{proof}

\begin{definition}[The presheaf category of a $ \Lambda $-algebra]
  Let $ A $ be a $ \Lambda $-algebra. If we view the monoid $ A(1) $ as a one-object category, we define the category $ PA $ to be the category of presheaves $ [\op{A(1)}, \SET] $.
\end{definition}

\begin{definition}[The objects $ A(n) $ in $ PA $]
  Given $ a: A(n) $ and $ b: A(1) $, we have
  \[ \mathbf 1_n \circ (a \circ b) = (\mathbf 1_n \circ a) \circ b = a \circ b, \]
  so $ a \circ b: A(n) $ and we have a right $ A(1) $-action on $ A(n) $, which makes $ A(n) $ into an object in $ PA $.
\end{definition}

\begin{lemma}
  We have $ A(1)^{A(1)} \cong A(2) $.
\end{lemma}
\begin{proof}
  We have a bijection $ \varphi: A(2) \cong F_{A(1), A(1)} $, given by
  \[ \varphi(a)(b, b^\prime) = (\lambda x_4, x_1 (x_2 x_4) (x_3 x_4)) \bullet (a, b, b^\prime). \]
  Note that $ \varphi(d) $ is equivariant because \TODO
  Now, $ \varphi $ is a presheaf morphism because \TODO

  Take $ p = \lambda x_1, x_1 (\lambda x_2 x_3, x_2) $ and $ q = \lambda x_1, x_1 (\lambda x_2 x_3, x_3) $. These are elements of $ A(1) $. Note that for terms $ c_1, c_2 $
  \begin{align*}
    p (\lambda x_1, x_1 c_1 c_2)
    &= (\lambda x_1, x_1 c_1 c_2) (\lambda x_2 x_3, x_2)\\
    &= (\lambda x_1 x_3, x_2) c_1 c_2\\
    &= c_1.
  \end{align*}
  In the same way, $ q \circ (\lambda x_1 x_2, x_2 c_1 c_2) = c_2 $.

  An inverse is given by
  \[ \psi: f \mapsto \lambda x_1 x_2, f(p, q)(\lambda x_3, x_3 x_1 x_2). \]
  This is a presheaf morphism because \TODO

  This is an inverse, because given $ f: F_{A(1), A(1)} $ and$ (a_1, a_2): A(1) \times A(1) $, we have
  \begin{align*}
    \varphi(\psi(f))(a_1, a_2) &= u(\lambda x_1 x_2, f(p, q)(\lambda x_3, x_3 x_1 x_2))(a_1, a_2)\\
    &= \lambda x_1, (\lambda x_2 x_3, f(p, q)(\lambda x_4, x_4 x_2 x_3)) (a_1 x_1) (a_2 x_1)\\
    &= \lambda x_1, f(p, q)(\lambda x_2, x_2 (a_1 x_1) (a_2 x_1))\\
    &= f(p, q) \circ (\lambda x_1, (\lambda x_2, x_2 (a_1 x_1) (a_2 x_1)))\\
    &= f(p \circ (\lambda x_1, (\lambda x_2, x_2 (a_1 x_1) (a_2 x_1))), q \circ (\lambda x_1, (\lambda x_2, x_2 (a_1 x_1) (a_2 x_1))))\\
    &= f(\lambda x_1, p (\lambda x_2, x_2 (a_1 x_1) (a_2 x_1)), \lambda x_1, q (\lambda x_2, x_2 (a_1 x_1) (a_2 x_1)))\\
    &= f(\lambda x_1, a_1 x_1, \lambda x_1, a_2 x_1)\\
    &= f(a_1, a_2).
  \end{align*}
  The last line is because $ a_i : A(1) $ and therefore $ \lambda x_1, a_i x_1 = a_i $.

  On the other hand, if we have $ a_1: A(2) $, we have
  \begin{align*}
    \psi(\varphi(a_1)) &= \psi((a_2, a_3) \mapsto \lambda x_1, a_1 (a_2 x_1) (a_3 x_1))\\
    &= \lambda x_1 x_2, (\lambda x_3, a_1 (p x_3) (q x_3)) (\lambda x_3, x_3 x_1 x_2)\\
    &= \lambda x_1 x_2, a_1 (p (\lambda x_3, x_3 x_1 x_2)) (q (\lambda x_3, x_3 x_1 x_2))\\
    &= \lambda x_1 x_2, a_1 x_1 x_2\\
    &= a_1.
  \end{align*}
  The last line is because $ a_1 : A(2) $ and therefore $ \lambda x_1 x_2, a_1 x_1 x_2 = a_1 $.

  Therefore, this map is a bijection and an isomorphism.
\end{proof}

\begin{definition}[Endomorphism $ \lambda $-theory of a $ \Lambda $-algebra]
  $ PA $ borrows products from $ \SET $. Therefore, the algebraic theory $ E(A(1)) $ exists. Now note that $ A(1) $ is exponentiable and $ A(1)^{A(1)} \cong A(2) $.
  Note that $ A(2) \subseteq A(1) $ and that $ (\lambda x_2 x_3, x_1 x_2 x_3) \bullet - $ gives a function from $ A(1) $ to $ A(2) $. This gives $ E(A(1)) $ a $ \lambda $-theory structure.
\end{definition}

\begin{definition}[Pullback functor on presheaves for a $ \Lambda $-algebra]
  Let $ f: A \to A^\prime $ be a $ \Lambda $-algebra morphism. Then for all $ a: A(n) $,
  \[ \mathbf 1_n \circ f(a) = f(\mathbf 1_n) \circ f(a) = f(\mathbf 1_n \circ a), \]
  so we have an induced morphism $ f: A(n) \to A^\prime(n) $.

  Now, given a presheaf $ P: PA^\prime $. We can create a presheaf $ f^* P : PA $ by taking the set of $ P $, and, for $ p: P $ and $ a: A $, setting $ pa = p \circ f(a) $. This is indeed a presheaf because \TODO

  Now, given a morphism $ g: P \to P^\prime $, we get a morphism by taking the function on the sets of $ P $ and $ P^\prime $. This is a morphism because \TODO
\end{definition}

\begin{lemma}
  The above indeed constitutes a functor.
\end{lemma}
\begin{proof}
  \TODO
\end{proof}

Left Kan extension then gives a left adjoint $ f_*: PA \to PA^\prime $.

\begin{lemma}
  We have $ f_*(A(1)) \cong A^\prime(1) $.
\end{lemma}
\begin{proof}
  \TODO
\end{proof}

\begin{lemma}
  $ f_* $ preserves finite products.
\end{lemma}
\begin{proof}
  \TODO
\end{proof}

\begin{definition}
  Since $ f_* $ preserves finite products, given an element of $ g: E(A(1))(n) = PA(A(1)^n, A(1)) $, we get
  \[ \# f_*(g): PA^\prime(f(A(1)^n), f(A(1))) \cong PA^\prime(A^\prime(1)^n, A^\prime(1)) = E(A^\prime(1))(n). \]
\end{definition}

\begin{lemma}
  $ \# f_*: E(A(1)) \to E(A^\prime(1)) $ is a map of $ \lambda $-theories.
\end{lemma}
\begin{proof}
  \TODO
\end{proof}

\begin{definition}
  We have an isomorphism $ E(A(1))(0) \cong A $ given by $ a \mapsto a I $.
\end{definition}

\begin{lemma}
  This is indeed an isomorphism of $ \Lambda $-algebras.
\end{lemma}
\begin{proof}
  \TODO
\end{proof}

\begin{lemma}
  Given $ g: A \to A^\prime $,
\end{lemma}

\begin{theorem}
  There exists an adjoint equivalence between the category of $ \lambda $-theories, and the category of algebras of $ \Lambda $.
\end{theorem}
\begin{proof}
  We will show that the functor $ L \mapsto L_0 $ is an equivalence of categories.

  It is essentially surjective, because $ L $ is isomorphic \TODO to $ E(A(1)) $.

  Now, given morphisms $ f, f^\prime: L \to L^\prime $. Suppose that $ f_0 = f^\prime_0 $. Suppose that $ L $ and $ L^\prime $ have $ \beta $-equality. Then, given $ l: L_n $, we have
  \[ f_n(l) = \rho^n(\lambda^n(f_n(l))) = \rho^n(f_0(\lambda^n(l))) = \rho^n(f^\prime_0(\lambda^n(l))) = \rho^n(\lambda^n(f^\prime_n(l))) = f^\prime_n(l), \]
  so the functor is faithful.

  The functor is full because a $ \Lambda $-algebra morphism $ f: A \to A^\prime $ induces a functor $ f^*: PA^\prime \to PA $, and via left Kan extension we get a left adjoint $ f_*: PA \to PA^\prime $ with $ f_*(A(1)) \cong A^\prime(1) $. Now, $ f_* $ preserves (finite) products, so we have maps $ PA(A(1)^n, A(1)) \to PA^\prime(A^\prime(1)^n, A^\prime(1)) $ and so a map $ E(A(1)) \to E(A^\prime(1)) $. This map, when restricted to a map $ PA(1, A(1)) \to PA^\prime(1, A(1)) $, and transported along the isomorphism $ a \mapsto a I $ \TODO, is equal to $ f $ \TODO.
\end{proof}

\section{An alternative proof for the fundamental theorem}

\section{Theory of extensions}

\begin{lemma}
  The category of $ T $-algebras has coproducts.
\end{lemma}
\begin{proof}
  \TODO
\end{proof}

\begin{definition}[Theory of extensions]
  Let $ T $ be an algebraic theory and $ A $ a $ T $-algebra. We can define an algebraic theory $ T_A $ called `the theory of extensions of $ A $' with $ (T_A)_n = T_n + A $. The left injection of the variables $ x_i : T_n $ gives the variables.
  Now, take $ h: (T_n + A)^m $. Sending $ g: T_m $ to $ \varphi(g) := g \bullet h $ gives a $ T $-algebra morphism $ T_m \to T_n + A $ since
  \[ \varphi(f \bullet g) = f \bullet g \bullet h = f \bullet (g_i \bullet h) = f \bullet (\varphi(g_i))_i. \]
  This, together with the injection morphism of $ A $ into $ T_n + A $, gives us a $ T $-algebra morphism from the coproduct: $ T_m + A \to T_n + A $. We especially have a function on sets $ (T_m + A) \times (T_n + A)^m \to T_n + A $, which we will define our substitution to be.
\end{definition}

\begin{lemma}
  $ T_A $ is indeed an algebraic theory.
\end{lemma}
\begin{proof}
  \TODO
\end{proof}
