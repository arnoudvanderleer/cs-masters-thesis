\chapter{The paper}


\section{Scott's Representation Theorem}
\begin{theorem}
  Any $ \lambda $-theory $ L $ is isomorphic to the endomorphism $ \lambda $-theory $ E(L) $ of $ L: \Pshf L $, which is $ L $, viewed as a presheaf in its own presheaf category.
\end{theorem}
\begin{proof}
  First of all, remember that $ L $ is indeed exponentiable and that $ L^L = A(L, 1) $.
  Now, since $ L $ is a $ \lambda $-theory, we have sequences of functions back and forth $ \lambda_n: A(L, 1)_n \to L_n $ and $ \rho_n: L_n \to A(L, 1)_n $. These commute with the $ L $-actions, so they constitute presheaf morphisms and $ E(L) $ is indeed a $ \lambda $-theory.

  Lemma \ref{lem:presheaf-Yoneda} gives a sequence of bijections $ \varphi_n : \Pshf L(L^n, L) \cong L_n $ for all $ n $, sending $ F: \Pshf L(L^n, L) $ to $ F(x_1, \dots, x_n) $, and conversely sending $ s: L_n $ to $ ((t_1, \dots, t_n) \mapsto s \bullet (t_1, \dots, t_n)) $. It considers $ \lambda $-terms in $ n $ variables as $ n $-ary functions on the $ \lambda $-calculus. Therefore, it should come as no surprise that $ \varphi $ preserves the $ x_i $, $ \bullet $, $ \rho $ and $ \lambda $, which makes it into an isomorphism of $ \lambda $-theories and this concludes the proof.
\end{proof}


\section{Locally cartesian closedness of the category of retracts}
\begin{definition}[Category of retracts]
  The category of retracts for a $ \lambda $-theory $ L $ is the category with objects $ f: L_n $ such that $ f \bullet f = f $ and it has as morphisms $ g: f \to f^\prime $ the terms $ g: L_n $ such that $ f^\prime \bullet g \bullet f = g $. The object $ f: L_n $ has identity element $ f $, and we have composition $ g \circ g^\prime = g \bullet g^\prime $. These are morphisms \TODO
\end{definition}

\begin{lemma}
  The category of retracts is indeed a category.
\end{lemma}
\begin{proof}
  \TODO
\end{proof}

\begin{theorem}
  The category of retracts is locally cartesian closed \TODO.
\end{theorem}


\section{Equivalences}


\section{Terms of a \texorpdfstring{$ \Lambda $}{Lambda}-algebra}
Let $ A $ be an algebra for the initial $ \lambda $-theory $ \Lambda $. We will assume that $ \Lambda $ (and therefore, any $ \lambda $-theory) satisfies $ \beta $-equality.

The $ \Lambda $-algebra structure gives the terms of $ A $ quite a lot of behaviour. For example, we can define `function application' as
\[ a b = (x_1 x_2) \bullet (a, b) \]
and composition as
\[ a \circ b = (x_1 \circ x_2) \bullet (a, b) \]
for $ a, b : A $, with $ x_1 \circ x_2 = \lambda x_3, x_1 (x_2 x_3) : \Lambda_2 $.

\begin{remark}
  Recall that in Example \ref{ex:free-monoid-theory}, we constructed an algebraic theory $ T $ with a monoid structure. This allowed us to define a monoid operation on $ T $-algebras as well. We then were able to transfer associativity of the operation on the $ T_n $ to associativity of the operation on the algebras. In exactly the same way, the function composition on $ A $ is associative because composition on $ \Lambda_n $ is associative.
\end{remark}

\begin{definition}
  We can consider the sets of elements of $ A $ that behave like functions in $ n $ variables:
  \[ A_n = \{ a : A \mid (\lambda x_2 x_3 \dots x_{n + 1}, x_1 x_2 x_3 \dots x_{n + 1}) \bullet a = a \}. \]
\end{definition}

\begin{definition}
  Take $ \mathbf 1_n = (\lambda x_1 \dots x_n, x_1 \dots x_n) \bullet () : A $.
\end{definition}

\begin{remark}
  Some straightforward rewriting, shows that for all $ a : A $,
  \[ \mathbf 1_n \circ a = (\lambda x_2 x_3 \dots x_{n + 1}, x_1 x_2 \dots x_{n + 1}) \bullet a. \]
  In other words, $ A_n = \{ a : A \mid \mathbf 1_n \circ a = a \} $.
\end{remark}

\begin{remark}
  Also note that $ \mathbf 1_n \circ a \circ \mathbf 1_n = \mathbf 1_n \circ a $, so for $ a : A_n $, $ a \circ \mathbf 1_n = a $.
\end{remark}

\begin{lemma}
  For $ t: \Lambda_{m + n} $ and $ a_1, \dots, a_m: A $, we have $ (\lambda^n t) \bullet (a_1, \dots, a_m) : A $ and we have
  \[ \mathbf 1_n \circ ((\lambda^n t) \bullet (a_1, \dots, a_m)) = (\lambda^n t) \bullet (a_1, \dots, a_m), \]
  so $ (\lambda^n t) \bullet (a_1, \dots, a_m) : A_n $.
\end{lemma}
\begin{proof}
  This follows by straightforward rewriting.
\end{proof}

\begin{corollary}
  By the previous remark,
  \[ ((\lambda^n t) \bullet (a_1, \dots, a_m)) \circ \mathbf 1_n = (\lambda^n t) \bullet (a_1, \dots, a_m). \]
\end{corollary}
\begin{corollary}
  In particular, $ \mathbf 1_m \circ \mathbf 1_n = \mathbf 1_{\max(m, n)} $. From this, it follows that $ A_m \subseteq A_n $ for $ m \leq n $. It also follows that $ a \mapsto \mathbf 1_n \circ a $ gives a function from $ A $ to $ A_n $ (and also from $ A_m \subseteq A $ to $ A_n $).
\end{corollary}


\section{The Fundamental Theorem of the \texorpdfstring{$ \lambda $-}{lambda }calculus}

The fundamental theorem states that there is an equivalence of categories
\[ \LamTh \cong \Alg \Lambda. \]
We will prove this by showing that there is a fully faithful and essentially surjective functor.


\subsection{The functor}

\begin{definition}
  For all $ n $, we have a functor from lambda theories to $ \Lambda $-algebras. It sends the $ \lambda $-theory $ L $ to the $ L $-algebra $ L_n $ and then turns this into a $ \Lambda $-algebra via the morpism $ \Lambda \to L $. It sends morphisms $ f: L \to L^\prime $ to the algebra morphism $ f_n : L_n \to L^\prime_n $.
\end{definition}


\subsection{Lifting \texorpdfstring{$ \Lambda $}{Lambda}-algebras}

\begin{definition}[The monoid of a $ \Lambda $-algebra]
  Now we make $ A_1 $ into a monoid under composition $ \circ $ with unit $ \mathbf 1_1 $. The fact that this is a monoid follows from the remarks in the last section.
\end{definition}

Recall that we have an equivalence $ [\op{C_{A_1}}, \SET] \cong \RAct{A_1} $.

\begin{definition}
  Now, composition $ \circ $ gives a right $ A_1 $-action on the $ A_n $, so we have $ A_n : \RAct{A_1} $.
\end{definition}

\begin{lemma}
  We have $ U_{A_1}^{U_{A_1}} \cong A_2 $ in $ \RAct{A_1} $.
\end{lemma}
\begin{proof}
  Recall that $ U_{A_1}^{U_{A_1}} $ consists of the set of $ A_1 $-equivariant morphisms $ U_{A_1} \times U_{A_1} \to U_{A_1} $.

  We have a bijection $ \varphi: A_2 \xrightarrow{\sim} U_{A_1}^{U_{A_1}} $, given by
  \[ \varphi(a)(b, b^\prime) = (\lambda x_4, x_1 (x_2 x_4) (x_3 x_4)) \bullet (a, b, b^\prime), \]
  \TODO: relate to product in category of retracts.

  with an inverse given by
  \[ \varphi^{-1}(f) = \lambda x_1 x_2, f(p_1, p_2)(\lambda x_3, x_3 x_1 x_2) \]
  \TODO: Make more explicitly that we use $ \bullet $ here.
  for $ p_i = \lambda x_1, x_1 (\lambda x_2 x_3, x_{i + 1}) $. Note that for terms $ c_1, c_2 $, we have $ p_i (\lambda x_1, x_1 c_1 c_2) = c_i $.

  This is an inverse, because given $ f: U_{A_1}^{U_{A_1}} $ and $ (a_1, a_2): U_{A_1} \times U_{A_1} $, we have
  \[ \varphi(\psi(f))(a_1, a_2) = f(p_1, p_2) \circ q = f(p_1 \circ q, p_2 \circ q) = f(a_1, a_2) \]
  for $ q = \lambda x_1, (\lambda x_2, x_2 (a_1 x_1) (a_2 x_1)) $. In the last step of this proof, we use, among other things, the fact that the $ a_i : A_1 $ and therefore $ \lambda x_1, a_i x_1 = a_i $.

  Some straightforward rewriting shows that for $ a: A_2 $, we have $ \psi(\varphi(a)) = a $. In the last step of this proof, we use the fact that $ a : A_2 $ and therefore $ \lambda x_1 x_2, a x_1 x_2 = a $.

  Therefore, $ \varphi $ is a bijection and, as it turns out, an isomorphism.
\end{proof}

\begin{definition}[Construction of the $ \lambda $-theory]
  Since $ \RAct{A_1} $ has products, the algebraic theory $ E(U_{A_1}) $ exists.

  Recall that $ A_2 \subseteq A_1 $ and that $ a \mapsto \mathbf 1_2 \circ a $ gives a function from $ A_1 $ to $ A_2 $, which is, by definition, the identity on $ A_2 $. This gives $ E(U_{A_1}) $ a $ \lambda $-theory structure with $ \beta $-equality.
\end{definition}


\subsection{Lifting algebra morphisms}

\begin{definition}[Pullback functor on presheaves for a $ \Lambda $-algebra]
  A $ \Lambda $-algebra morphism preserves $ \mathbf 1_n $ and $ \circ $, so it sends elements of $ A_n $ to $ A^\prime_n $. In particular, it gives a monoid morphism $ f: A_1 \to A^\prime_1 $.

  Therefore, as described in Section \ref{sec:monoid-category}, we get pullback and pushforward functors $ f_*: \RAct{A^\prime_1} \to \RAct{A_1} $ and $ f^* : \RAct{A_1} \to \RAct{A^\prime_1} $.
\end{definition}

\begin{remark}
  By Lemma \ref{lem:scalar-extension-monoid-monoid-action}, we have $ f^*(U_{A_1}) \cong U_{A^\prime_1} $.
\end{remark}

\begin{lemma}
  $ f^* $ preserves finite products.
\end{lemma}
\begin{proof}
  We will show that $ f^* $ preserves binary products and the terminal object.

  We use Lemma \ref{lem:scalar-extension-terminal} to show that $ f^* $ preserves the terminal object. We take
  \[ a_0 = (\lambda x_1 x_2, x_2) \bullet () : A_1 \quad \text{and} \quad a_0^\prime = (\lambda x_1 x_2, x_2) \bullet () : A^\prime_1 \]
  and $ a_0^\prime $ is weakly terminal because for all $ a : A_1 $, we have $ f(a_0) \circ a = a_0^\prime $.

  We use Lemma \ref{lem:scalar-extension-product} to show that $ f^* $ also preserves the product. Therefore, given $ a_1, a_2 : A_1 $.

  (\TODO: relate to product)

  Take $ a = \lambda x_1 x_2, x_2 (a_1 x_1) (a_2 x_1) $ and $ \pi_i = \lambda x_1, x_1 (\lambda x_2 x_3, x_{i + 1}) $. We have $ a_i = f(\pi_i) \circ a $. Now, for some $ a^\prime: A^\prime_1 $, $ \pi_1^\prime, \pi_2^\prime: A_1 $ such that $ a_i = f(\pi_i^\prime) \circ a^\prime $, take $ m = \lambda x_1 x_2, x_2 (\pi_1^\prime x_1) (\pi_2^\prime x_1) $. Then $ \pi_i \circ m = \pi_i^\prime $ and $ f(m) \circ a^\prime = a $, so $ (a, \pi_1, \pi_2) $ is weakly terminal and $ f^* $ preserves binary products.

  Since any finite product is (isomorphic to) a construction with a repeated binary product and the terminal object, the fact that $ f^* $ preserves binary products and the terminal object shows that $ f^* $ preserves all finite products.
\end{proof}

\begin{definition}
  Since $ f^* $ preserves finite products, given an element of $ g: E(U_{A_1})_n = \RAct{A}(U_{A_1}^n, U_{A_1}) $, we get
  \[ f^*(g): \RAct{A^\prime}(f(U_{A_1}^n), f(U_{A_1})) \cong \RAct{A^\prime}((U_{A^\prime_1})^n, U_{A^\prime_1}) = E(U_{A^\prime_1})_n \]
  so we have a morphism $ f^*: \LamTh(E(U_{A_1}), E(U_{A^\prime_1})) $.
\end{definition}

\begin{remark}
  The fact that $ f^* $ preserves the variables and substitution is not very hard, since these are just defined in terms of finite products $ U_{A_1} $ and $ f^* $ preserves finite products and $ U_{A_1} $.

  However, showing that it is a $ \lambda $-theory morphism is really hard. Hyland claims
  \begin{quote}
    $ f $ preserves $ \mathbf 1_n $, which determines the function space as a retract of the universal. So $ f^* $ preserves the retract and the result follows.
  \end{quote}
  This indeed covers the core of the argument, but actually verifying that it works is much more complicated: We a natural equivalence and isomorphisms
  \begin{align*}
    \alpha_A: \RAct{A_1}(X \times Y, Z) &\xrightarrow{\sim} \RAct{A}(X, Z^Y);\\
    \beta: f^*(U_{A_1}) &\xrightarrow{\sim} U_{A_1^\prime};\\
    \bar \gamma: f^*(A \times B) &\xrightarrow{\sim} f^*(A) \times f^*(B);\\
    \gamma_n: f^*(X^n) &\xrightarrow{\sim} f^*(X)^n;\\
    % \delta: f^*(X^Y) &\xrightarrow{\sim} f^*(X)^{f^*(Y)};\\
    \delta_A: U_{A_1}^{U_{A_1}} &\xrightarrow{\sim} A_2.
  \end{align*}
  with $ \gamma_{n + 1} = \bar{\gamma} \cdot (\gamma_n \times \id X) $.
  Then, for $ s: E(U_{A_1})_{n + 1} $, we have $ \lambda_A(s) = \alpha_A(s) \cdot \delta_A \cdot \pi_A : \RAct{A_1}(U_{A_1}^n, U_{A_1}) $ for the projection $ \pi_A: A_2 \to U_{A_1} $. However, note that $ f^*(\lambda_A(s)): \RAct{A^\prime_1}(f^*(U_{A^\prime_1}^n), f^*(U_{A^\prime_1})) $. To compare this to $ \lambda_{A^\prime}(\dots) $, we actually need to take
  \[ (\beta^{-1})^n \cdot \gamma_n^{-1} \cdot f^*(\lambda_A(s)) \cdot \beta : \RAct{A^\prime_1}(U_{A_1}^n, U_{A_1}). \]
  On the other hand, we also have $ f^*(s): \RAct{A^\prime_1}(f^*(U_{A_1}^{n + 1}), f^*(U_{A_1})) $ and we cannot apply $ \lambda_{A^\prime} $ directly. Instead, the term that we end up with is
  \[ \lambda_{A^\prime}((\beta^{-1})^{n + 1} \cdot \gamma_{n + 1}^{-1} \cdot f^*(s) \cdot \beta): \RAct{A^\prime_1}(U_{A_1}^n, U_{A_1}). \]
  So the equality that we need to prove is (using functoriality of $ f^* $ and naturality of $ \alpha_{A^\prime} $):
  \begin{align*}
    &(\beta^{-1})^n \cdot \gamma_n^{-1} \cdot f^*(\alpha_A(s)) \cdot f^*(\delta_A \cdot \pi_A) \cdot \beta\\
    &= (\beta^{-1})^n \cdot \gamma_n^{-1} \cdot \alpha_{A^\prime}((\id X^n \times \beta^{-1}) \cdot \bar{\gamma}^{-1} \cdot f^*(s)) \cdot \beta^{U_{A_1}} \cdot \delta_{A^\prime} \cdot \pi_{A^\prime}.
  \end{align*}
  Because of the complicated definition of each of these terms, I was unable to verify the correctness of this statement (and the similar statement about $ \rho: E(U_{A_1})_n \to E(U_{A_1})_{n + 1} $) within a day, even though I am willing to believe Hyland's claim.
  \TODO

  % Given a morphism $ t: X \times Y \to Z $, $ \alpha_{A_1}(t): X \to (U_{A_1} \times Y \to Z) $ is given by
  % \[ \alpha(t)(x)(a, y) = t(x a, y) \]

  % Also, $ \beta: f^*(U_{A_1}) \xrightarrow{\sim} U_{A^\prime_1} $ is given, for $ a: A_1 $ and $ a^\prime : A^\prime_1 $, by
  % \[ \beta(a, a^\prime) = f(a) a^\prime \quad \text{and} \quad \beta^{-1}(a^\prime) = (1, a^\prime). \]

  % Given a morphism $ s: X \to Y $, we have $ f^*(s): f^*(X) \to f^*(Y) $ given by
  % \[ f^*(s)(x, a^\prime) = (s(x), a^\prime). \]

  % Finally, we have
  % \[ \delta(s) = \lambda x_1 x_2, s(p_1, p_2)(\lambda x_3, x_3 x_1 x_2) \]
  % with $ p_i = \lambda x_1, x_1 (\lambda x_2 x_3, x_{i + 1}) $.


  % For some $ s: U_{A_1}^{n + 1} \to U_{A_1} $, the left hand side sends $ (a, (a^\prime_1, \dots, a^\prime_n)): f^*(U_{A_1}^n) $ to
  % \begin{align*}
  %   &\lambda x_1 x_2, f(s((a_i p_1)_i, p_2)) (\lambda x_3, x_3 x_1 x_2) \circ a^\prime\\
  %   &= \lambda x_1 x_2, f(s((a_i p_1)_i, p_2)) (\lambda x_3, x_3 a^\prime (x_1) x_2)
  % \end{align*}
  % and the right hand side sends it to
  % \begin{align*}
  %   \lambda x_1 x_2, \lambda x_3, (f(s((a_i p_1)_i, p_2))((\lambda x_4, x_4 (a^\prime p_1 (\lambda x_3, x_3 x_1 x_2)) (p_2 (\lambda x_3, x_3 x_1 x_2)))(x_3)))
  % \end{align*}
\end{remark}

\begin{lemma}
  We have a bijection $ \varphi: E(U_{A_1})_0 \cong A $, sending $ s $ to $ x_1 (\lambda x_2, x_2) \bullet (s(\star)) $.
\end{lemma}
\begin{proof}
  Using \ref{lem:global-action-elements}, we showed that $ E(U_{A_1})_0 $ corresponds to the set of $ A_1 $-equivariant elements of $ U_{A_1} $, sending $ s $ to $ s(\star) $. Now, for an $ A_1 $-equivariant element $ a $,
  \[ (\lambda x_2, x_1 (\lambda x_3, x_3)) \bullet a = a \circ ((\lambda x_1 x_2, x_2) \bullet ()) = a \]
  so we can send $ a $ to $ A $ as $ (x_1 (\lambda x_2, x_2)) \bullet a $ and it has (two-sided) inverse $ (\lambda x_2, x_1) \bullet a $.
\end{proof}

We want to show that $ \varphi $ is an isomorphism of $ \Lambda $-algebras. We use Lemma \ref{lem:make-is-lambda-algebra-morphism} for this, so we need to show that it preserves the application and the $ \Lambda $-definable constants.

\begin{lemma}
  For $ \varphi $ as above, we have for all $ a, b: E(U_{A_1})_0 $, we have
  \[ \varphi((x_1 x_2) \bullet (a, b)) = (x_1 x_2) \bullet (\varphi(a), \varphi(b)). \]
\end{lemma}
\begin{proof}
  For $ a, b: E(U_{A_1})_0 $, we have, for the isomorphism $ \delta: \RAct{A_1}(U_{A_1}^{U_{A_1}}, A_2) $,
  \begin{align*}
    \varphi((x_{2, 1} x_{2, 2}) \bullet (a, b)) &= \varphi(\rho(x_{1, 1}) \bullet (a, b))\\
    &= (x_1 (\lambda x_2, x_2)) \bullet (\langle a, b \rangle \cdot ((s, t) \mapsto (\delta^{-1}(\mathbf 1_2 \circ s)) (1, t)))(\star)\\
    &= (x_1 (\lambda x_2, x_2)) \bullet (\lambda x_3, x_1 x_3 (x_2 x_3)) \bullet (a(\star), b(\star))\\
    &= (x_1 (\lambda x_3, x_3) (x_2 (\lambda x_3, x_3))) \bullet (a(\star), b(\star))\\
    &= (x_1 x_2) \bullet (\varphi(a), \varphi(b))
  \end{align*}
  and this concludes the proof.
\end{proof}

\begin{lemma}
  For $ \varphi $ as above, we have for all $ s: \Lambda_0 $,
  \[ \varphi(s \bullet ()) = s \bullet (). \]
\end{lemma}
\begin{proof}
  \TODO
\end{proof}

\begin{lemma}
  We have an isomorphism $ U_{L_0} \cong L $.
\end{lemma}
\begin{proof}
  \TODO
\end{proof}

\begin{lemma}
  This is indeed an isomorphism of $ \lambda $-theories.
\end{lemma}
\begin{proof}
  \TODO
\end{proof}

\begin{theorem}
  There exists an adjoint equivalence between the category of $ \lambda $-theories, and the category of algebras of $ \Lambda $.
\end{theorem}
\begin{proof}
  We will show that the functor $ L \mapsto L_0 $ is an equivalence of categories.

  It is essentially surjective, because $ L $ is isomorphic \TODO to $ E(U_{A_1}) $.

  Now, given morphisms $ f, f^\prime: L \to L^\prime $. Suppose that $ f_0 = f^\prime_0 $. Suppose that $ L $ and $ L^\prime $ have $ \beta $-equality. Then, given $ l: L_n $, we have
  \[ f_n(l) = \rho^n(\lambda^n(f_n(l))) = \rho^n(f_0(\lambda^n(l))) = \rho^n(f^\prime_0(\lambda^n(l))) = \rho^n(\lambda^n(f^\prime_n(l))) = f^\prime_n(l), \]
  so the functor is faithful.

  The functor is full because a $ \Lambda $-algebra morphism $ f: A \to A^\prime $ induces a functor $ f^*: \RAct{A^\prime} \to \RAct{A} $, and via left Kan extension we get a left adjoint $ f^*: \RAct{A} \to \RAct{A^\prime} $ with $ f^*(A_1) \cong A^\prime_1 $. Now, $ f^* $ preserves (finite) products, so we have maps $ \RAct{A}(A_1^n, A_1) \to \RAct{A^\prime}((A^\prime_1)^n, A^\prime_1) $ and so a map $ E(U_{A_1}) \to E(U_{A^\prime_1}) $. This map, when restricted to a map $ \RAct{A}(1, A_1) \to \RAct{A^\prime}(1, A_1) $, and transported along the isomorphism $ a \mapsto a I $ \TODO, is equal to $ f $ \TODO.
\end{proof}


\section{An alternative proof for the fundamental theorem}


\section{Theory of extensions}

\begin{lemma}
  The category of $ T $-algebras has coproducts.
\end{lemma}
\begin{proof}
  \TODO
\end{proof}

\begin{definition}[Theory of extensions]
  Let $ T $ be an algebraic theory and $ A $ a $ T $-algebra. We can define an algebraic theory $ T_A $ called `the theory of extensions of $ A $' with $ (T_A)_n = T_n + A $. The left injection of the variables $ x_i : T_n $ gives the variables.
  Now, take $ h: (T_n + A)^m $. Sending $ g: T_m $ to $ \varphi(g) := g \bullet h $ gives a $ T $-algebra morphism $ T_m \to T_n + A $ since
  \[ \varphi(f \bullet g) = f \bullet g \bullet h = f \bullet (g_i \bullet h) = f \bullet (\varphi(g_i))_i. \]
  This, together with the injection morphism of $ A $ into $ T_n + A $, gives us a $ T $-algebra morphism from the coproduct: $ T_m + A \to T_n + A $. We especially have a function on sets $ (T_m + A) \times (T_n + A)^m \to T_n + A $, which we will define our substitution to be.
\end{definition}

\begin{lemma}
  $ T_A $ is indeed an algebraic theory.
\end{lemma}
\begin{proof}
  \TODO
\end{proof}
