\chapter{Alternative definitions}
The literature, there are many different but equivalent definitions, carrying many different names, for the objects that called `algebraic theories' in Section \ref{sec:algebraic-theories}. Also, many of these different names are attached to differently defined objects in different sources. This section will showcase some of the various definitions.

In this section, we will denote the finite set $ \llbracket n \rrbracket = \{ 1, \dots, n \} $.

\section{Abstract Clone}
\begin{definition}
  An algebraic theory as presented in Section \ref{sec:algebraic-theories}, is usually called an \iindex{abstract clone}. In this thesis, outside of this specific section, we will call it algebraic theory to be consistent with the names that Hyland attaches to objects.
\end{definition}

\begin{remark}
  The definition of algebraic theory that Hyland gives is closest to that of an abstract clone. However, instead of a sequence of sets $ (T_n)_n $, he requires a functor $ T: F \to \SET $ (with $ F \subseteq \FinSET $ the skeleton category of finite sets $ F_0 = \{ \llbracket 0 \rrbracket, \llbracket 1 \rrbracket, \dots \} $), and he requires $ \bullet: T_m \times T_n^m \to T_n $ to be dinatural in $ n $ and natural in $ m $.

  Using naturality, one can show that with such a functor we have $ x_{n, i} = f(a)(x_{1, 1}) $ for the function $ a(1) = i : \llbracket n \rrbracket $.

  Alternatively, using the same naturality, one can show that this functor sends a morphism $ a: \llbracket m \rrbracket \to \llbracket n \rrbracket $ to the function $ T_m \to T_n $ given by
  \[ f \mapsto f \bullet (x_{n, a(i)})_i. \]
  If we take this to be the definition of our functor on morphisms, the (di)naturality in $ m $ and $ n $ can be shown using the associativity and the laws about the interaction between $ \bullet $ and the $ x_i $.

  Since any additional properties mean extra complexity when formalizing, and since the proofs rarely use the functor structure, we decided to reduce the functor $ T: \FinSET \to \SET $ to a sequence of sets $ (T_n)_n $.
\end{remark}

\section{Lawvere theory}
\begin{definition}
  An \iindex{algebraic theory} as presented in \cite{algebaric-theories-2010} is a small category with finite products.
\end{definition}
\begin{definition}
  An \iindex{algebra} for an algebraic theory $ T $ is a finite-product-preserving functor $ T \to \SET $.
\end{definition}

This definition is more general than the definition of algebraic theory in Section \ref{sec:algebraic-theories}. To make it equivalent, we have to be more specific about the objects of the category:
\begin{definition}
  A \iindex{Lawvere theory}, or \textit{one-sorted algebraic theory} is a category $ L $, with $ L_0 = \{ 0, 1, \dots \} $, such that $ n = 1^n $, the $ n $-fold product.
\end{definition}

\begin{lemma}
  There is an equivalence between abstract clones and Lawvere theories.
\end{lemma}
\begin{proof}
  Let $ C $ be an abstract clone. We construct a Lawvere theory $ L $ as follows: We have objects $ L_0 = \{ 0, 1, \dots \} $ and morphisms $ L(m, n) = C_m^n $. The identity morphism is $ \id n = (x_i)_i : L(n, n) $ and for $ f: L(l, m) $, $ g: L(m, n) $, we have composition
  \[ f \cdot g = (g_i \bullet f)_i: L(l, n). \]
  Lastly, we have product projections $ \pi_{n, i} = x_{n, i}: L(n, 1) $ for all $ 1 \leq i \leq n $.

  Conversely, if $ L $ is a Lawvere theory, we construct an abstract clone $ C $ as follows:
  We take $ C_n = L(n, 1) $. We take the $ x_{n, i} $ to be the product projections $ \pi_i : L(n, 1) $ and we define the substitution $ f \bullet g = \langle g_i \rangle_i \cdot f $ the composite of the product morphism $ \langle g_i \rangle_i $ with $ f $.
\end{proof}

\subsection{Algebras for Lawvere Theories}

\begin{lemma}
  Let $ C $ be an abstract clone, and $ L $ be its associated Lawvere theory by the equivalence given above. A $ C $-algebra is equivalent to an algebra for $ L $.
\end{lemma}
\begin{proof}
  Let $ A $ be a $ C $-algebra. We will construct a functor $ F: L \to \SET $ as follows: We take $ F(n) = A^n $. We define the action on morphisms as
  \[ F(f)(a) = (f_i \bullet a)_i \]
  for $ f: L(m, n) = L(m, 1)^n $ and $ a: A^m $.

  Conversely, let $ F: L \to \SET $ be a functor. We take the $ C $-algebra $ A $, with $ A = F(1) $ and for $ f: C_n = L(n, 1) $ and $ a: A^n $, $ f \bullet a = F(f)(a) $.
\end{proof}

\section{Cartesian Operad}

\begin{definition}
  A \iindex{cartesian operad} T is a functor $ T: F \to \SET $, together with an `identity' element $ \id{}: T(1) $ and for all $ m, n_1, \dots, n_m: \mathbb N $, a composition map
  \[ T(m) \times \prod_i T(n_i) \to T\left(\sum_i n_i\right), \]
  written $ (f, g_1, \dots, g_m) \mapsto f[g_1, \dots, g_m] $, satisfying some identity, associativity and naturality conditions (see \cite{lambda-monoid}, Definition 2.1, for more details).
\end{definition}

\begin{remark}
  One can arrive at this concept as a standalone definition, or one can view a cartesian operad as a \iindex{cartesian multicategory} with one element. A multicategory is a category in which morphisms have type $ C((X_1, X_2, \dots, X_n), Y) $ instead of $ C(X, Y) $. A cartesian multicategory is a multicategory in which one can permute the $ X_i $ of a morphism and has `contraction' and `weakening' operations:
  \begin{align*}
    C((X_1, \dots, X_i, X_i, \dots, X_n), Y) &\to C((X_1, \dots, X_i, \dots, X_n), Y),\\
    C((X_1, \dots, X_{i - 1}, X_{i + 1}, \dots, X_n), Y) &\to C((X_1, \dots, X_{i - 1}, X_i, X_{i + 1}, \dots, X_n), Y).
  \end{align*}
\end{remark}

\begin{lemma}
  There is an equivalence between abstract clones and cartesian operads.
\end{lemma}
\begin{proof}
  Let $ C $ be an abstract clone. We define a cartesian operad $ T $ with $ T(n) = C_n $ and $ T(f)(t) = t \bullet (x_{f(1)}, \dots, x_{f(m)}) $ for $ f: \llbracket m \rrbracket \to \llbracket n \rrbracket $ and $ t: C_m $. The identity element is $ x_{1, 1} $ and the composition $ f[g_1, \dots, g_n] $, for $ f: C_m $ and $ g_i: C_{n_i} $, is given by lifting all terms to $ C_{\sum_i n_i} $ and then substituting:
  \[ f[g_1, \dots, g_n] = f \bullet (T(\iota_i)(g_i)) \]
  for $ \iota_i : \llbracket n_i \rrbracket \hookrightarrow \llbracket \sum_i n_i \rrbracket $ the pairwise disjoint injections.

  Conversely, given a cartesian operad $ T $, we construct an abstract clone $ C $ with $ C_n = T(\llbracket n \rrbracket) $. The variables are $ x_{n, i} = \iota_{n, i}(\id{}) $, for $ \iota_{n, i}: \llbracket 1 \rrbracket \hookrightarrow \llbracket n \rrbracket $ the morphism that sends $ 1 $ to $ i $. The substitution $ f \bullet (g_i)_i $ for $ g_1, \dots, g_m: T(n) $ is given by composing and then identifying some variables:
  \[ f \bullet (g_i)_i = T(\pi)(f[g_1, \dots, g_m]) \]
  for $ \pi: \llbracket m n \rrbracket \to \llbracket n \rrbracket $ the function that sends $ i + 1 $ to $ (i \mod n) + 1 $.
\end{proof}

\section{Relative Monad}
\begin{definition}
  Let $ S: C \to D $ be a functor. A \iindex{relative monad} on $ S $ is a functor $ T: C \to D $, together with a natural transformation $ \eta: S \Rightarrow T $ and a `kleisli extension' $ (-)^*: D(S(X), T(Y)) \to D(T(X), T(Y)) $, natural in both $ S $ and $ T $, such that for all $ f: D(S(X), T(Y)) $ and $ g: D(S(Y), T(Z)) $,
  \[ \eta_X^* = \id{T X}, \qquad f = \eta_X \cdot f^* \qquad \text{and} \qquad (f \cdot g^*)^* = f^* \cdot g^*. \]
\end{definition}

\begin{remark}
  Note that for an adjunction $ F \dashv G $, $ F \bullet G $ gives a monad. In the same way, there exists a notion of \iindex{relative adjunction}, from which we can obtain a relative monad (see \cite{monads-endofunctors}, Theorem 2.10).
\end{remark}

\begin{remark}
  Now, there is a result that states: There is an equivalence between abstract clones and relative monads on the embedding $ \iota: \FinSET \hookrightarrow \SET $.

  Note that the objects of $ \FinSET $ are defined to be sets $ X $, together with a proof that there exists some $ n: \mathbb N $ and some bijection $ f: X \xrightarrow{\sim} \llbracket n \rrbracket $. This existence of $ n $ and $ f $ is given by the propositional truncation $ \left\Vert \sum_{n: \mathbb N}, X \xrightarrow{\sim} \llbracket n \rrbracket \right\Vert $.

  Classically, this construction starts with ``fix, for all $ X : \FinSET $, a bijection $ f: X \xrightarrow{\sim} \llbracket n \rrbracket $''. However, since $ X $ only provides \textit{mere existence} of such a bijection without choosing one (by the propositional truncation), there is no way to obtain $ f $ without using the axiom of choice.

  Now, we can partially circumvent this problem by noting that we have a fully faithful and essentially surjective embedding $ F \to \FinSET $ (for $ F $ a skeleton of finite sets), which induces an adjoint equivalence $ [F, \SET] \xrightarrow{\sim} [\FinSET, \SET] $, so we can lift the functor part of a relative monad on $ F \to \SET $ to $ \FinSET \to \SET $. We can also lift the natural transformation using this equivalence. However, the kleisli extension cannot be lifted using this and we are stuck.

  Therefore, we will prove a modified statement:
\end{remark}

\begin{lemma}
  There is an equivalence between abstract clones and relative monads on the embedding $ \iota: F \hookrightarrow \SET $.
\end{lemma}
\begin{proof}
  Let $ C $ be an abstract clone. We define a relative monad $ T $ as follows: We take $ T(\llbracket n \rrbracket) = C_n $. For a morphism $ a: F(\llbracket m \rrbracket, \llbracket n \rrbracket) $. We take $ T(a)(f) = f \bullet (x_{a(i)})_i $. We define $ \eta_{\llbracket n \rrbracket}(i) = x_{n, i} $. Finally, for $ g: \SET(\llbracket m \rrbracket, T(\llbracket n \rrbracket)) $, we define $ g^*(f) = f \bullet g $.

  Conversely, let $ (T, \eta, (\cdot)^*) $ be a relative monad on the embedding $ \iota: F \hookrightarrow \SET $. We define an abstract clone $ C $ with $ C_n = T(\llbracket n \rrbracket) $. Substitution is defined as $ f \bullet g = g^*(f) $ for $ f: C_m $ and $ g: C_n^m $ and we have variables $ x_{n, i} = \eta_{\llbracket n \rrbracket}(i) $.
\end{proof}

\section{Monoid in a skew-monoidal category}
For details and a general treatment, see \cite{monads-endofunctors}, Section 3 and specifically Theorem 3.4.

\begin{definition}
  Consider the category $ [F, \SET] $. Note that we have a functor $ \iota: F \hookrightarrow \SET $ and that $ \SET $ has colimits. Therefore, we can define a `tensor product' on functors $ [F, \SET] $ as
  \[ F \otimes G = G \bullet \Lan{\iota}{F}. \]
  Together with the `unit' $ \iota $, this gives $ [F, \SET] $ a `skew-monoidal category' structure.
\end{definition}

\begin{definition}
  Given a (skew-)monoidal category $ (C, \otimes, I) $, a \iindex{monoid} in this category is an object $ T : C $, together with a `multiplication' $ \mu: C(T \otimes T, T) $ and a `unit' morphism $ \eta: C(I, T) $ satisfying a couple of laws (see \cite{MacLane}, Section III.6).
\end{definition}

\begin{lemma}
  There exists an equivalence between relative monads on $ \iota: F \hookrightarrow \SET $ and monoids in the skew-monoidal category $ [F, \SET] $.
\end{lemma}
\begin{proof}
  A monoid in $ [F, \SET] $ consists of an object $ T: [F, \SET] $, together with natural transformations $ \mu: T \otimes T \Rightarrow T $ and $ \eta: \iota \Longrightarrow T $. We can immediately see the functor $ T $ and natural transformation $ \eta $ of the relative monad pop up here. The kleisli extension corresponds to $ \mu $; they are related to each other via the properties of the left Kan extension of $ T $.
\end{proof}
