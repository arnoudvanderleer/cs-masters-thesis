\chapter{Previous work in categorical semantics}

\section{The correspondence between categories and typed \texorpdfstring{$ \lambda $}{lambda}-calculi}
In \cite{curry}, page 413, Scott and Lambek argue that there is a correspondence between simply typed $ \lambda $-calculi and cartesian closed categories (categories with products and `function objects').

Types in the $ \lambda $-calculus correspond to objects in the category.

Types $ A \to B $ in the $ \lambda $-calculus correspond to exponential objects $ B^A $ in the category.

Terms in the $ \lambda $-calculus of type $ B $, with free variables $ x_1: A_1, \dots, x_n: A_n $, correspond to morphisms $ A_1 \times \dots \times A_n \to B $.

A free variable $ x_i: A_i $ in a context with free variables $ x_1: A_1, \dots, x_n: A_n $ corresponds to the projection morphism $ \pi_i : A_1 \times \dots \times A_n \to A_i $.

Given a term $ s: B_1 \to B_2 $ and a term $ t: B_1 $, both with free variables $ x_1: A_1, \dots, x_n: A_n $, corresponding to morphisms $ \overline s: A_1 \times \dots \times A_n \to B_2 $ and $ \overline t: A_1 \times \dots \times A_n \to B_1 $, the application $ st: B_2 $ corresponds to the composite of the product morphism with the evaluation morphism $ A_1 \times \dots \times A_n \to B_2^{B_1} \times B_1 \to B_2 $.
\begin{center}
  \begin{tikzcd}
    & A_1 \times \dots \times A_n \arrow[ld, "\overline s"'] \arrow[rd, "\overline t"] \arrow[d, "{\langle \overline s, \overline t \rangle}", dashed] &\\
    B_2^{B_1} & B_2^{B_1} \times B_1 \arrow[l, "\pi_1"] \arrow[r, "\pi_2"'] \arrow[d, "ev"] & B_1\\
    & B_2 &
  \end{tikzcd}
\end{center}

Given a term $ t: B $ with free variables $ x_1: A_1, \dots, x_n: A_n $, the abstraction $ (\lambda x_n, t): A_n \to B $ corresponds to using the adjunction corresponding to the exponential object of $ A_n $:
\[ C(A_1 \times \dots \times A_{n-1} \times A_n, B) \simeq C(A_1 \times \dots \times A_{n-1}, B^{A_n}). \]

\section{Scott's Representation Theorem}
The correspondence above for a simply-typed $ \lambda $-calculus raises a question whether such a correspondence also exists for an untyped $ \lambda $-calculus. As Definition \ref{def:endomorphism-theory} shows that in fairly general circumstances we can take one object $ c $ in a category $ C $ and consider the morphisms $ t: C(c^n, c) $ as terms in an untyped $ \lambda $-calculus. Hyland calls this the `endomorphism theory' of $ c $.

\TODO Remark upon the significance of having a reflexive object

We can then wonder whether this construction is part of a correspondence between untyped $ \lambda $-calculi and some class of categories. This question finds a partial answer in the following theorem, originally proven in a very syntactical way by Dana Scott (see \cite{curry}, page 418).
\begin{lemma}
  We can obtain every untyped $ \lambda $-calculus as the endomorphism theory of some object in some category.
\end{lemma}
\begin{proof}
  Let $ L $ be a $ \lambda $-theory. First of all, for $ a_1, a_2: L_0 $, we define
  \begin{align*}
    a_1 \circ a_2 &= \lambda x_1, a_1 (a_2 x_1);\\
    (a_1, a_2) &= \lambda x_1, x_1 a_1 a_2;\\
    \langle a_1, a_2 \rangle &= \lambda x_1, (a_1 x_1, a_2 x_1);\\
    \pi_i &= \lambda x_1, x_1 (\lambda x_2 x_3, x_{i + 1}).
  \end{align*}
  Although, actually, since every one of these starts with a $ \lambda $-abstraction, we would need to lift the constants $ a_i $ to $ \iota_{0, 1}(a_i): L_1 $ to make the definitions above typecheck.

  Note that $ \pi_i (a_1, a_2) = a_i $ and $ \pi_i \circ \langle a_1, a_2 \rangle = \lambda x_1, a_i x_1 $.

  Also, note that by replacing the $ x_i $ by $ x_{n + i} $ and the $ \iota_{0, 1}(a_i) $ by $ \iota_{n, 1}(a_i) $, we obtain definitions not only for $ n = 0 $, but for all $ n \geq 0 $.

  For our category, we take the category $ \mathbb R $ which Hyland calls the `category of retracts'. It is the Karoubi envelope of the monoid of the terms without free variables under composition. That is, we consider the category of idempotent functions:
  \[ \mathbb R_0 = \{ A : L_0 \mid A \circ A = A \} \quad \text{and} \quad \mathbb R(A, B) = \{ f: L_0 \mid B \circ f \circ A = f \}, \]
  with $ \id A = A $ and composition given by $ \circ $.

  This category has a terminal object $ I = \lambda x_1 x_2, x_2 : \mathbb R $.

  The category has binary products with projections and product morphisms
  \[ A_1 \times A_2 = \langle p_1, p_2 \rangle, \quad p_i = A_i \circ \pi_i \quad \text{and} \quad \langle f, g \rangle. \]

  The category also has exponential objects
  \[ C^B = \lambda x_1, B \circ x_1 \circ C \]
  with evaluation morphism $ \epsilon_{BC}: C^B \times B \to C $ given by
  \[ \epsilon_{BC} = \lambda x_1, C(\pi_1 x_1 (B (\pi_2 x_1))), \]
  which is universal because we can lift a morphism $ f: \mathbb R(A \times B, C) $ to a morphism $ \varphi(f): \mathbb R(A, C^B) $ given by
  \[ \varphi(f) = \lambda x_1 x_2, f (x_1, x_2). \]
  Note that for $ g: \mathbb R(A, C^B) $, the inverse $ \varphi^{-1}(g) $ is given by
  \[ \epsilon \circ \langle g \circ \pi_1, B \circ \pi_2 \rangle: \mathbb R(A \times B, C). \]

  Now, consider the object $ U = \lambda x_1, x_1 : \mathbb R $. Note that for all $ A: \mathbb R $, we have morphisms $ A: \mathbb R(U, A) $ and $ A: \mathbb R(A, U) $, which exhibit $ A $ as a retract of $ U $. In particular, $ U^U $ is a retract of $ U $, so $ U $ is a reflexive object.

  Therefore, $ E(U) $, the endomorphism theory of $ U $, has a $ \lambda $-theory structure. Note that the finite powers of $ U $ in $ \mathbb R $ are given by $ U^0 = I $ and $ U^{n + 1} = U^n \times U $.

  We have $ E(U)_n = \mathbb R(U^n, U) = \{ f: L_0 \mid U \circ f \circ U^n = f \} $. The variables of $ E(U) $ are the projections of $ U^n $:
  \[ q_{n, i} = \pi_2 \circ \underbrace{\pi_1 \circ \dots \circ \pi_1}_{n - i}. \]
  The substitution is given by composition with the product morphism:
  \[ f \bullet g = f \circ \langle \langle \langle I, g_1 \rangle, \dots \rangle, g_n \rangle. \]
  We have $ U^U = \lambda f, U \circ f \circ U = \lambda x_1 x_2, x_1 x_2 $. Using the equivalence $ \mathbb R(U^n \times U, U) \simeq \mathbb R(U^n, U^U) $ and the retraction $ U^U: U \to U^U $, the abstraction and application $ \lambda $ and $ \rho $ are given by
  \[ \lambda(f) = \lambda x_1 x_2, \iota_{0, 2}(f)(x_1, x_2), \quad \rho(g) = \lambda x_1, \iota_{0, 1}(g) (\pi_1 x_1) (\pi_2 x_1). \]
  for $ f: \mathbb R(U^{n + 1}, U) $ and $ g: \mathbb R(U^n, U) $.

  Now, we have bijections $ \psi_0: E(U)_0 \xrightarrow{\sim} L_0 $, given by
  \[ \psi_0(f) = f (\lambda x_1, x_1) \quad \text{and} \quad \psi_0^{-1}(g) = \lambda x_1, \iota_{0, 1}(g). \]
  We can extend this to any $ n $, by reducing any term to a constant by repeatedly using $ \lambda $, then applying the bijection, and then lifting it again using $ \rho $. Explicitly, we obtain
  \[ \psi_n(f) = \iota_{0, n}(f) ((((\lambda x_{n+1}, x_{n+1}), x_1), \dots), x_n), \quad \text{and} \quad \psi^{-1}_n(g) = \lambda x_1, g \bullet (q_i x_1)_i. \]
  It is not hard to verify that this is indeed a bijection, using at one point the fact that $ f: \mathbb R(U^n, U) $ is defined by $ f \circ U^n = f $, for
  \[ U^n = \lambda x_1, (((\lambda x_2, x_2), q_{n, 1} x_1 ), \dots, q_{n, n} x_1 ). \]
  It is also pretty straightforward to check that
  \begin{align*}
    \psi(q_{n, i}) &= x_i, & \psi(f) \bullet (\psi(g_i))_i &= \psi(f \bullet g),\\
    \psi(\lambda(h)) &= \lambda(\psi(h)), & \psi(\rho(h^\prime)) &= \rho(\psi(h^\prime))
  \end{align*}
  for $ f: E(U)_m $, $ g: E(U)_n^m $, $ h: E(U)_{n + 1} $ and $ h^\prime : E(U)_n $. Therefore, $ \psi $ is an isomorphism of $ \lambda $-theories.
\end{proof}

\section{The Taylor Fibration}

In his dissertation, Paul Taylor shows that $ \mathbb R $ is not only cartesian closed, but also \iindex{relatively cartesian closed}.

In Section \ref{sec:dependent-products}, we studied internal and external representations of families of objects in a category and how they behaved under substitutions (pullbacks). This was to arrive at a definition for dependent products and sums, as the right and left adjoints to the pullback (or substitution) functor $ \alpha^*: (C \downarrow A) \to (C \downarrow B) $ along some morphism $ \alpha : C(B, A) $.

Now, some categories are not locally cartesian closed. That is: not all substitution functors $ \alpha^* $ exist or have a right adjoint. One way to look at this, is that not all morphisms $ X \to A $ represent an family. In these categories, we can carefully choose a subset of the morphisms to represent our indexed families. We will call a morphism representing an indexed family a \iindex{display map}. In most cases, we have quite a bit of choice which maps we want to take as our display maps. However, to make sure that indexed families are well-behaved, a class of display maps needs to have some properties:
\begin{enumerate}
  \item The pullback of a display map along any morphism exists and is a display map.
  \item The composite of two display maps is a display map.
  \item $ C $ has a terminal object and any terminal projection is a display map.
\end{enumerate}
\begin{remark}
  A maximal example of a class of display maps is, for example, in the category $ \SET $, where we can take our class of display maps to equal the class of all morphisms of $ C $.
\end{remark}
\begin{remark}
  A `minimal' example of a class of display maps is the class of (maps isomorphic to) product projections in a category with finite products. In this case, all indexed families are constant, and then dependent sums and products become binary products and exponential objects.
\end{remark}

Now, let $ C $ be a category with a class of display maps. For any $ A : C $, we get a category $ C^A $ as a full subcategory of the slice category $ C \downarrow A $, with as objects the display maps $ f: X \to A $. A morphisms from in $ f: X \to A $ to $ g: Y \to A $ is a morphism $ \varphi: X \to Y $ such that the following diagram commutes.
\begin{center}
  \begin{tikzcd}
    X \arrow[rd, "f"'] \arrow[rr, "\varphi"] & & Y \arrow[ld, "g"]\\
    & A &
  \end{tikzcd}
\end{center}
Note that for the terminal object $ I : C $, $ C^I $ is still equivalent to $ C $.

Also note that since display maps are closed under pullbacks, we can construct the pullback functor $ \alpha^*: (C \downarrow A) \to (C \downarrow B) $ for all $ \alpha: C(B, A) $.

Note that since composing two display maps gives a display map again, and since the dependent sum is given by postcomposition, the pullback functor $ \alpha^* $ has a left adjoint for all display maps $ \alpha $. That is: the fiber categories $ C^A $ have dependent sums over display maps.

The question whether the fiber categories $ C^A $ also have dependent products over display maps, brings us to the definition of relative cartesian closedness.
\begin{definition}
  A category $ C $ is \index{cartesian closed!relatively}\textit{cartesian closed relative} to a class of display maps $ D $, if the substitution functors $ \alpha^* $ along display maps have right adjoints.
\end{definition}
