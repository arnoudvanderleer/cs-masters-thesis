\chapter{Algebraic Structures}

\section{Algebraic Theories}\label{sec:algebraic-theories}

\begin{definition}
  We define an \iindex{algebraic theory} $ T $ to be a sequence of sets $ T_n $ indexed over $ \mathbb N $ with for all $ 1 \leq i \leq n $ elements ("variables" or "projections") $ x_{n, i}: T_n $ (we usually leave $ n $ implicit), together with a substitution operation
  \[ \_ \bullet \_: T_m \times T_n^m \to T_n \]
  for all $ m, n $, such that
  \begin{align*}
    x_j \bullet g &= g_j\\
    f \bullet (x_{l, i})_i &= f\\
    (f \bullet g) \bullet h &= f \bullet (g_i \bullet h)_i
  \end{align*}
  for all $ 1 \leq j \leq l $, $ f: T_l $, $ g: T_m^l $ and $ h: T_n^m $.
\end{definition}

\begin{definition}
  A \textit{morphism}\index{algebraic theory!morphism} $ F $ between algebraic theories $ T $ and $ T^\prime $ is a sequence of functions $ F_n: T_n \to T^\prime_n $ (we usually leave the $ n $ implicit) such that
  \begin{align*}
    F_n(x_j) &= x_j\\
    F_n(f \bullet g) &= F_m(f) \bullet (F_n(g_i))_i
  \end{align*}
  for all $ 1 \leq j \leq n $, $ f: T_m $ and $ g: T_n^m $.
\end{definition}

Together, these form the category of algebraic theories \iindex{$ \AlgTh $}.

\begin{remark}
  We can construct binary products of algebraic theories, with sets $ (T \times T^\prime)_n = T_n \times T^\prime_n $, variables $ (x_i, x_i) $ and substitution
  \[ (f, f^\prime) \bullet (g, g^\prime) = (f \bullet g, f^\prime \bullet g^\prime). \]
  In the same way, the category of algebraic theories has all limits.
\end{remark}

Later on, we will see an example of a trivial algebraic theory (the terminal theory) $ T $, in which every $ T_n $ only contains one element. However, we can say a bit about nontrivial algebraic theories.
\begin{lemma}\label{lem:nontrivial-algebraic-theory}
  Let $ T $ be an algebraic theory, such that $ T_n $ has at least two distinct elements for some $ n $. Then for all $ 1 \leq i, j \leq m $ with $ i \not = j $, we have $ x_{m, i} \not = x_{m, j} $.
\end{lemma}
\begin{proof}
  We can also formulate the statement as: If there exist $ 1 \leq i, j \leq m $ with $ i \not = j $ such that $ x_{m, i} = x_{m, j} $, then all $ L_n $ contain at most one distinct element.

  So, suppose that $ x_{m, i} = x_{m, j} $ for some $ i \not = j $. For $ a, b : L_n $, we define $ v : L_n^m $ as
  \[ v_k = \left\{\begin{array}{ll} a & k = i\\ b & k \not = i \end{array}\right., \]
  so in particular, $ v_j = b $. Then we have
  \[ a = v_i = x_{m, i} \bullet v = x_{m, j} \bullet v = v_j = b, \]
  so $ L_n $ contains at most one distinct element.
\end{proof}

\section{Algebras}

\begin{definition}
  An \iindex{algebra} $ A $ for an algebraic theory $ T $ is a set $ A $, together with an action
  \[ \bullet: T_n \times A^n \to A \]
  for all $ n $, such that
  \begin{align*}
    x_j \bullet a &= a_j\\
    (f \bullet g) \bullet a &= f \bullet (g_i \bullet a)_i
  \end{align*}
  for all $ j $, $ f: T_m $, $ g: T_n^m $ and $ a: A^n $.
\end{definition}

\begin{definition}
  For an algebraic theory $ T $, a \textit{morphism}\index{algebra!morphism} $ F $ between $ T $-algebras $ A $ and $ A^\prime $ is a function $ F: A \to A $ such that
  \[ F(f \bullet a) = f \bullet (F(a_i))_i \]
  for all $ f: T_n $ and $ a: A^n $.
\end{definition}

Together, these form the category of $ T $-algebras \iindex{$ \Alg T $}.

\begin{remark}
  The category of algebras has all limits. The set of a limit of algebras is the limit of the underlying sets.
\end{remark}

\begin{remark}
  Note that for an algebraic theory $ T $, the $ T_n $ are all algebras for $ T $, with the action given by $ \bullet $.
\end{remark}

\begin{definition}[Pullback of algebras]\label{def:algebra-pullback}
  If we have a morphism of algebraic theories $ f: T^\prime \to T $, we have a functor $ \Alg T \to \Alg T^\prime $. It endows $ T^\prime $-algebras with an action from $ T $ given by $ g \bullet_{T^\prime} a = f(g) \bullet_T a $. Then $ T^\prime $-algebra morphisms commute with this $ T $-action, so we indeed have a functor.
\end{definition}

We can also consider the category of `all' algebraic theory algebras together. That is, the category $ C $ with $ C_0 = \sum_{T : \AlgTh}, \Alg T $ and $ C((T, A), (T^\prime, A^\prime)) $ consisting of pairs $ (f, f^\prime): \AlgTh(T, T^\prime) \times \SET(A, A^\prime) $ such that for all $ t: T_n $ and $ a: A^n $,
\[ f^\prime(t \bullet a) = f(t) \bullet (f^\prime(a_i))_i. \]
We then have a functor $ P: C \to \AlgTh $, projecting onto the first coordinate.

\begin{lemma}\label{lem:algebra-fibration}
  $ P $ is a fibration.
\end{lemma}
\begin{proof}
  Given an algebraic theory morphism $ f: \AlgTh(S, T) $ and a $ T $-algebra $ A_T $, Definition \ref{def:algebra-pullback} gives an $ S $-algebra $ A_S $ with underlying set $ A_T $. The cartesian morphism is $ (f, \id{A_T}): C((S, A_S), (T, A_T)) $.

  It is cartesian because for $ (R, A_R): C $ and $ (g, g^\prime): C((R, A_R), (T, A_T)) $ and $ h: \AlgTh(R, S) $ with $ h \cdot f = g $, the required morphism $ C((R, A_R), (S, A_S)) $ is given by $ g^\prime: \SET(A_R, A_S) $.
\end{proof}

\section{Presheaves}

\begin{definition}
  A \iindex{presheaf} $ P $ for an algebraic theory $ T $ is a sequence of sets $ P_n $ indexed over $ \mathbb N $, together with an action
  \[ \bullet: P_m \times T_n^m \to P_n \]
  for all $ m, n $, such that
  \begin{align*}
    t \bullet (x_{l, i})_i &= t\\
    (t \bullet f) \bullet g &= t \bullet (f_i \bullet g)_i
  \end{align*}
  for all $ t: P_l $, $ f: T_m^l $ and $ g: T_n^m $.
\end{definition}

\begin{definition}
  For an algebraic theory $ T $, a \textit{morphism}\index{presheaf!morphism} $ F $ between $ T $-presheaves $ P $ and $ P^\prime $ is a sequence of functions $ F_n: P_n \to P^\prime_n $ such that
  \[ F_n(t \bullet f) = F_m(t) \bullet f \]
  for all $ t: P_m $ and $ f: T_n^m $.
\end{definition}

Together, these form the category of $ T $-presheaves \iindex{$ \Pshf T $}.

\begin{remark}
  The category of presheaves has all limits. The $ n $th set $ \overline{P}_n $ of a limit $ \overline{P} $ of presheaves $ P_i $ is the limit of the $ n $th sets $ P_{i, n} $ of the presheaves in the limit diagram.
\end{remark}

An analogue to Lemma \ref{lem:algebra-fibration} shows that, like with algebras, the total category of presheaves is fibered over the category of algebraic theories.

\section{\texorpdfstring{$ \lambda $-}{lambda-}theories}

Let $ \iota_{m, n} : T_m \to T_{m + n} $ be the function that sends $ f $ to $ f \bullet (x_{m + n, 1}, \dots, x_{m + n, m}) $. Note that
\[ \iota_{m, n}(f) \bullet g = f \bullet (g_i)_{i \leq m} \quad \text{and} \quad \iota_{m, n}(f \bullet g) = f \bullet (\iota_{m, n}(g_i))_i. \]

For tuples $ x : X^m $ and $ y: X^n $, let $ x + y $ denote the tuple $ (x_1, \dots, x_m, y_1, \dots, y_n) : X^{m + n} $.

\begin{definition}[$ \lambda $-theory]
  A \iindex{$ \lambda $-theory} is an algebraic theory $ L $, together with sequences of functions $ \lambda_n: L_{n + 1} \to L_n $ and $ \rho_n: L_n \to L_{n + 1} $, such that
  \begin{align*}
    \lambda_m(f) \bullet h &= \lambda_n(f \bullet ((\iota_{n, 1}(h_i))_i + (x_{n + 1})))\\
    \rho_n(g \bullet h) &= \rho_m(g) \bullet ((\iota_{n, 1}(h_i))_i + (x_{n + 1}))
  \end{align*}
  for all $ f: L_{m + 1} $, $ g: L_m $ and $ h: L_n^m $.
\end{definition}

Together, these form the category of $ \lambda $-theories \iindex{$ \LamTh $}.

\begin{definition}[$ \beta $- and $ \eta $-equality]
  We say that a $ \lambda $-theory $ L $ satisfies $ \beta $-equality (or that it is a $ \lambda $-theory with $ \beta $) if $ \rho_n \circ \lambda_n = \id{L_n} $ for all $ n $. We say that is satisfies $ \eta $-equality if $ \lambda_n \circ \rho_n = \id{L_{n + 1}} $ for all $ n $.
\end{definition}

\begin{definition}[$ \lambda $-theory morphism]
  A \textit{morphism}\index{$ \lambda $-theory!morphism} $ F $ between $ \lambda $-theories $ L $ and $ L^\prime $ is an algebraic theory morphism $ F $ such that
  \begin{align*}
    F_n(\lambda_n(f)) &= \lambda_n(F_{n + 1}(f))\\
    \rho_n(F_n(g)) &= F_{n + 1}(\rho_n(g))
  \end{align*}
  for all $ f: L_{n + 1} $ and $ g: L_n $.
\end{definition}

\begin{remark}
  The category of lambda theories has all limits, with the underlying algebraic theory of a limit being the limit of the underlying algebraic theories.
\end{remark}

\begin{definition}
  A $ \lambda $-theory algebra or presheaf is an algebra or presheaf for the underlying algebraic theory.
\end{definition}

\section{Examples}

There is a lot of different examples of algebraic theories and their algebras. Some of these even turn out to be $ \lambda $-theories. In this section, we will discuss a couple of these.

\subsection{The free algebraic theory on a set}
\begin{example}
  Let $ S $ be a set. We can construct an algebraic theory $ F(S) $ by taking $ F(S)_n = S \sqcup \{ 1, \dots, n \} $ with projections $ x_i = i $ and substitution
  \begin{align*}
    i \bullet g &= g_i & s \bullet g &= s
  \end{align*}
  for $ i : \{ 1, \dots, n \} $ and $ s : S $.

  If we have a function $ f: S \to S^\prime $, we get a morphism $ F(f): F(S) \to F(S^\prime) $ given by
  \begin{align*}
    F(f)_n(i) &= i & F(f)_n(s) &= f(s)
  \end{align*}
  for $ i : \{ 1, \dots, n \} $ and $ s : S $.

  Also, $ F $ obviously respects the identity and substitution morphisms, so it is a functor.
\end{example}

Note that we have a forgetful functor $ (\cdot)_0 $ that sends a morphism of algebraic theories $ g: T \to T^\prime $ to the function $ f_0: T_0 \to T^\prime_0 $.

\begin{lemma}
  The algebraic theory $ F(S) $ defined above, is the free algebraic theory on the set $ S $.
\end{lemma}
\begin{proof}
  Let $ T $ be an algebraic theory. We have an equivalence
  \[ \AlgTh(F(S), T) \cong \SET(S, T_0), \]
  sending $ f: \AlgTh(F(S), T) $ to $ f_0: S = S \sqcup \emptyset \to T_0 $ (this is trivially natural in $ S $ and $ T $) and $ f: \SET(S, T_0) $ to the functions $ g_n: F(S)_n \to T_n $ given by
  \begin{align*}
    g_n(i) &= x_i & g_n(s) &= f(s) \bullet ().
  \end{align*}
\end{proof}

The proofs that $ F(S) $ is an algebraic theory and that $ F(f) $ and $ g $ are algebraic theory morphisms is an easy exercise in case distinction.

\begin{corollary}
  $ F(\emptyset) $ is the initial algebraic theory.
\end{corollary}
\begin{proof}
  For $ S = \emptyset $, the equivalence of hom-sets becomes
  \[ \AlgTh(F(\emptyset), T) \cong \SET(\emptyset, T_0) \]
  and the latter has exactly one element.
\end{proof}

\begin{lemma}
  There is an adjoint equivalence between the category $ \Alg{F(S)} $ and the coslice category $ S \downarrow \SET $.
\end{lemma}
\begin{proof}
  For the equivalence, we send a $ F(S) $-algebra $ A $ to the set $ A $ with morphism $ s \mapsto s \bullet () $. An algebra morphism $ f: A \to B $ is sent to the coslice morphism $ f: (S \to A) \to (S \to B) $. This constitutes a functor.

  Note that the category of $ F(S) $-algebras is univalent.

  Also, the functor is fully faithful, since one can show that for $ F(S) $-algebras, the coslice morphism $ \varphi: (f: S \to A) \to (f^\prime: S \to B) $ also has the structure of an algebra morphism $ \varphi: A \to B $.

  Lastly, the functor is essentially surjective, since we can lift a coslice $ f: S \to X $ to a $ F(S) $-algebra $ X $, with action
  \[ i \bullet x = x_i \quad \text{and} \quad s \bullet x = f(s). \]

  Therefore, the functor $ \Alg{F(S)} \to S \downarrow \SET $ is an adjoint equivalence.

  The proofs of these facts work by simple case distinction, and by using the properties of the coslice and algebra morphisms.
\end{proof}

$ F(\emptyset) $ is, in some sense, the smallest nontrivial algebraic theory. Then $ F(S) $ is the smallest nontrivial algebraic theory that has the elements of $ S $ as constants.

\subsection{The free \texorpdfstring{$ \lambda $}{lambda}-theory on a set}

Like with the free algebraic theory, we will construct the free $ \lambda $-theory as the smallest nontrivial $ \lambda $-theory (which is the $ \lambda $-calculus) with some additional constants.

Let $ S $ be a set. Consider the sequence of inductive types $ (\Lambda(S)_n)_n $ with the following constructors:
\begin{align*}
  \mathtt{Var}_n &: \{ 1, \dots, n \} \to \Lambda(S)_n;\\
  \mathtt{App}_n &: \Lambda(S)_n \to \Lambda(S)_n \to \Lambda(S)_n;\\
  \mathtt{Abs}_n &: \Lambda(S)_{n + 1} \to \Lambda(S)_n;\\
  \mathtt{Con}_n &: S \to \Lambda(S)_n.\\
\end{align*}

Define a substitution operator $ \bullet: \Lambda(S)_m \times \Lambda(S)_n^m \to \Lambda(S)_n $ by induction on the first argument:
\begin{align*}
  \mathtt{Var}_m(i) \bullet g &= g_i;\\
  \mathtt{App}_m(a, b) \bullet g &= \mathtt{App}_n(a \bullet g, b \bullet g);\\
  \mathtt{Abs}_m(a) \bullet g &= \mathtt{Abs}_n(a \bullet ((g_i \bullet (x_{n + 1, j})_j)_i + (x_{n + 1})));\\
  \mathtt{Con}_m(s) \bullet g &= \mathtt{Con}_n(s).
\end{align*}

And then quotient $ \Lambda(S) $ by the relation generated by
\[ \mathtt{App}_m(\mathtt{Abs}_m(f), g) \sim f \bullet ((x_{n, i})_i + (g)) \]
for all $ f: \Lambda(S)_{n + 1} $ and $ g: \Lambda(S)_n $.

\begin{example}
  We can give the sequence of sets $ \Lambda(S) $ an algebraic theory structure with variables $ x_{m, i} = \mathtt{Var}_m(i) $ and the substitution operator $ \bullet $ defined above. We can give $ \Lambda(S) $ a $ \lambda $-theory structure with $ \beta $-equality by taking
  \[ \lambda_n(f) = \mathtt{Abs}_n(f) \quad \text{and} \quad \rho_n(f) = \mathtt{App}_{n + 1}(f \bullet (\mathtt{Var}_{n + 1}(i))_i, \mathtt{Var}_{n + 1}(n + 1)). \]

  Now, given a function $ S \to S^\prime $, we define a morphism $ \LamTh(\Lambda(S), \Lambda(S^\prime)) $ by induction, sending $ \mathtt{Var}(i) $, $ \mathtt{App}(a, b) $ and $ \mathtt{Abs}(a) $ in $ \Lambda(S) $ to their corresponding elements in $ \Lambda(S^\prime) $ and sending $ \mathtt{Con}(s) $ to $ \mathtt{Con}(f(s)) $.
\end{example}

Note that, like with the previous example, we have a forgetful functor $ (\dot)_0: \LamTh \to \SET $.

\begin{lemma}
  $ \Lambda(S) $ is the free $ \lambda $-theory on $ S $.
\end{lemma}
\begin{proof}
  Let $ L $ be a $ \lambda $-theory. We have an equivalence
  \[ \LamTh(\Lambda(S), L) \cong \SET(S, L_0), \]
  sending $ f: \LamTh(\Lambda(S), L) $ to $ f_0 \vert_S: S \to L_0 $ (again, trivially natural in $ S $ and $ L $) and conversely, $ g: \SET(S, L_0) $ to the inductively defined $ f: \LamTh(\Lambda(S), L) $ given by
  \begin{align*}
    f(\mathtt{Var}(i)) &= x_i;\\
    f(\mathtt{App}(a, b)) &= \rho(f(a)) \bullet ((x_{n, i})_i + (f(b)));\\
    f(\mathtt{Abs}(a)) &= \lambda(f(a));\\
    f(\mathtt{Con}(s)) &= g(s) \bullet ().
  \end{align*}
\end{proof}

The proofs that $ \Lambda(S) $ is indeed a $ \lambda $-theory and that $ \Lambda(f) $ and $ g $ are $ \lambda $-theory morphisms, mainly work by definition of $ \bullet $, $ \lambda $ and $ \rho $, by induction on the terms of $ \Lambda(S) $ and by invoking the properties of the $ \lambda $-theory $ L $.

\begin{corollary}
  The `pure' lambda calculus is the initial $ \lambda $-theory.
\end{corollary}
\begin{proof}
  If we take $ S = \emptyset $, $ \Lambda(\emptyset) $ is the lambda calculus, which we will call $ \Lambda $. We have, like with the free algebraic theory, that $ \Lambda(\emptyset) $ is the initial $ \lambda $-theory.
\end{proof}

\subsubsection{About $ \Lambda $-algebra morphisms}

\begin{lemma}\label{lem:make-is-lambda-algebra-morphism}
  Let $ A $ and $ B $ be $ \Lambda $-algebras and let $ f: \SET(A, B) $ be a function that preserves the application and the $ \Lambda $-definable constants:
  \[ f((x_1 x_2) \bullet (a, b)) = (x_1 x_2) \bullet (f(a), f(b)) \quad \text{and} \quad f(s \bullet ()) = s \bullet () \]
  for all $ a, b: A $ and $ s: \Lambda_0 $. Then $ f $ is a $ \Lambda $-algebra morphism.
\end{lemma}
\begin{proof}
  Note that for $ s: \Lambda_{n + 1} $ and $ a: A^{n + 1} $,
  \[ (x_1 x_2) \bullet (\lambda(s) \bullet (a_i)_i, a_{n + 1}) = s \bullet a. \]
  By induction, we can express $ s \bullet a $ using a combination of $ (x_1 x_2) \bullet (\cdot, \cdot) $ and $ \lambda^n(s) $:
  \begin{align*}
    f(s \bullet a) &= f((x_1 x_2) \bullet (\dots ((x_1 x_2) \bullet (\lambda^n(s) \bullet (), a_1), \dots), a_n))\\
    &= (x_1 x_2) \bullet (\dots ((x_1 x_2) \bullet (f(\lambda^n(s) \bullet ()), f(a_1)), \dots), f(a_n))\\
    &= s \bullet (f(a_i))_i,
  \end{align*}
  so $ f $ is a $ \Lambda $-algebra morphism.
\end{proof}

\subsection{The free object algebraic theory}

\begin{example}
  Take a category $ C $, with a forgetful functor $ G: C \to \SET $ and a free functor $ F: \SET \to C $. Let $ \eta: \id \SET \Rightarrow F \bullet G $ be the unit of the adjunction and let $ \varphi: C(F(c), d) \cong \SET(c, G(d)) $ be the natural equivalence of homsets.

  We define an algebraic theory $ T $ with $ T_n = G(F(\{ 1, \dots, n \})) $, projections $ x_{n, i} = \eta_{\{1, \dots, n\}}(i) $. For the substitution, note that we take $ t_1, \dots, t_m: T_n $, so we have $ t: \{ 1, \dots, m \} \to G(F(\{ 1, \dots, n \})) $. We then take
  \[ s \bullet t = G(\varphi^{-1}(t))(s). \]

  Now, given an object $ c: C $, we can create a $ T $-algebra $ \alpha(c) $, with set $ G(c) $ and action
  \[ s \bullet t = G(\varphi^{-1}(t))(s). \]
  Also, given a morphism $ f: C(c, d) $. This gives a morphism $ G(f): \alpha(c) \to \alpha(d) $. Therefore, $ \alpha: C \to \Alg T $ is a functor.
\end{example}

The proofs that $ T $ is an algebraic theory, that $ G(c) $ is an algebra and that $ G(f) $ is an algebra morphism mainly rely on the fact that $ \varphi $ is natural.

So we have a functor from $ C $ to the category of $ T $-algebras. One can wonder whether there also is a functor the other way, or whether $ \alpha $ is even an equivalence. This is hard to characterize precisely, but in algebra, there is a broad class of examples where the functor is an equivalence, so where $ C $ is equivalent to $ \Alg T $. That is probably why $ T $ is called an \textit{algebraic theory}.

The idea is that if an object of $ C $ is a set, together with some operations between its elements, one can carefully choose some elements of $ T_0 $, $ T_1 $, $ T_2 $ etc., which act on an algebra like the specified operations.

\begin{example}\label{ex:free-monoid-theory}
  For $ C $ the category of monoids, $ \alpha: C \to \Alg T $ is an adjoint equivalence.

  Note that $ T_n $ is the free monoid on $ n $ elements. Its elements can be viewed as strings $ (x_1 x_5 x_3 x_{18} \dots x_7) $ with the characters $ x_1, \dots, x_n $, with the $ x_i $ the generators of the monoid, acting as the projections of the algebraic theory.

  Let $ A $ be a $ T $-algebra. We can give $ A $ a monoid structure by taking, for $ a, b: A $,
  \[ a b = (x_1 x_2) \bullet (a, b) \]
  and unit element
  \[ 1 = () \bullet (). \]
  Then the laws like associativity follow from those laws on the monoid and from the fact that the action on the algebra commutes with the substitution:
  \[ a (b c) = (x_1 (x_2 x_3)) \bullet (a, b, c) = ((x_1 x_2) x_3) \bullet (a, b, c) = (a b) c. \]
  Note that if we take a monoid, turn it into a $ T $-algebra and then into a monoid again, we still have the same underlying set, and it turns out that the monoid operation and unit element are equal to the original monoid operation and unit element. Therefore, $ \alpha $ is essentially surjective. It is also fully faithful, since any $ T $-algebra morphism respects the action of $ T $, which makes it into a monoid morphism. Therefore, $ \alpha $ is an adjoint equivalence.
\end{example}

\begin{remark}
  In the same way, one can characterize groups, rings and $ R $-algebras (for $ R $ a ring) as algebras of some algebraic theory. On the other hand, one can not use this method to describe fields as algebras for some theory $ T $, because one would need to describe the inverse $ z \mapsto z^{-1} $ operation as $ t \bullet (z) $ for some $ t: T_1 $, with $ z z^{-1} = 1 $, but since the elements of the algebraic theory act on all (combinations of) elements of the algebra, one would be able to take the inverse $ 0^{-1} = t \bullet (0) $ with $ 0 0^{-1} = 1 $, which would make no sense.
\end{remark}

\begin{remark}
  Another counterexample is the category $ \mathbf{Top} $ of topological spaces. We have a forgetful functor $ G: \mathbf{Top} \to \SET $ that just forgets the topology. On the other hand, we have a free functor $ F: \SET \to \mathbf{Top} $ which endows a set with the discrete topology. The construction above yields the inital algebraic theory $ T_n = \{ 1, \dots, n \} $, with an algebra action on every topological space $ i \bullet (a_1, \dots, a_n) = a_i $. Now, note that we can endow the set $ \{ \top, \bot \} $ with four different, nonisomorphic topologies, which all yield the same $ T $-algebra. In other words: the $ T $-algebra structure does not preserve the topological information. Therefore, the functor $ \alpha: \mathbf{Top} \to \Alg T $ is not an equivalence.
\end{remark}

\subsection{The terminal theory}
\begin{example}
  We can create a (somewhat trivial) algebraic theory $ T $ by taking $ T_n = \{ \star \} $, with projections $ x_i = \star $ and substitution $ \star \bullet \star = \star $. Taking $ \lambda(\star) = \star $ and $ \rho(\star) = \star $, we give it a $ \lambda $-theory structure (with $ \beta $ and $ \eta $-equality). Checking that this is indeed an algebraic theory and even a $ \lambda $-theory is trivial.

  Now, given any other algebraic theory $ T^\prime $, there exists a unique function $ T^\prime_n \to T_n $ for every $ n $, sending everything to $ \star $. These functions actually constitute an algebraic theory morphism $ T^\prime \to T $. If $ T^\prime $ is a $ \lambda $-theory, the algebraic theory morphism is actually a $ \lambda $-theory morphism. Again, checking this is trivial.

  Therefore, $ T $ is the terminal algebraic theory and $ \lambda $-theory.
\end{example}

\begin{lemma}
  $ \{ \star \} $ is the only algebra of the terminal theory.
\end{lemma}
\begin{proof}
  Let $ A $ be a $ T $-algebra. First of all, we have an element $ \star_A = \star_T \bullet_{0} () $. Secondly, for all elements $ \star, \star^\prime: A $, we have
  \[ \star = x_1 \bullet (\star, \star^\prime) = \star \bullet (\star, \star^\prime) = x_2 \bullet (\star, \star^\prime) = \star^\prime. \]
  Therefore, $ A = \{ \star \} $, which allows exactly one possible $ T $-action:
  \[ \star \bullet (\star, \dots, \star) = \star. \]
\end{proof}

\subsection{The endomorphism theory}

\begin{definition}\label{def:endomorphism-theory}
  Suppose that we have a category $ C $ and an object $ X: C $, such that all powers $ X^n $ of $ X $ are also in $ C $.
  The \iindex{endomorphism theory} $ E(X) $ of $ X $ is the algebraic theory given by $ E(X)_n = C(X^n, X) $ with projections as variables $ x_{n, i}: X^n \to X $ and a substitution that sends $ f: X^m \to X $ and $ g_1, \dots, g_m: X^n \to X $ to $ f \circ \langle g_i \rangle_i: X^n \to X^m \to X $.
\end{definition}

\begin{definition}
  Now, suppose that the exponential object $ X^X $ exists, and that we have morphisms back and forth $ abs: X^X \to X $ and $ app: X \to X^X $. Let, for $ Y: C $, $ \varphi_Y $ be the isomorphism $ C(X \times Y, X) \xrightarrow{\sim} C(Y, X^X) $.
  We can give $ E(X) $ a $ \lambda $-theory structure by setting, for $ f: E(X)_{n + 1} $ and $ g: E(X)_n $,
  \[ \lambda(f) = abs \circ \varphi_{X^n}(f) \qquad \rho(g) = \varphi_{X^n}^{-1}(app \circ g). \]
\end{definition}

The proofs that $ E(X) $ is an algebraic theory and a $ \lambda $-theory, use properties of the product, and naturality of the isomorphism $ \varphi_Y $.

\subsection{The theory algebra}
\begin{example}
  Let $ T $ be an algebraic theory and $ n $ a natural number. We can endow the $ T_n $ with a $ T $-algebra structure, by taking the substitution operator of $ T $ as the $ T $-action. Since this commutes with the substitution operator and the projections, $ T_n $ is a $ T $-algebra.
\end{example}

\subsection{The initial presheaf}
\begin{example}
  Let $ T $ be an algebraic theory. We can construct a $ T $-presheaf $ P $, with $ P_n = \emptyset $. Then $ \bullet : P_m \times T_n^m \to P_m $ is trivial, and the presheaf laws hold trivially.
\end{example}

\begin{lemma}
  This is indeed the initial presheaf.
\end{lemma}
\begin{proof}
  Let $ Q $ be a $ T $-presheaf. For all $ n $, since $ P_n $ is empty, there is only one possible function $ P_n \to Q_n $. These functions trivially satisfy the presheaf morphism laws, so they constitute the unique presheaf morphism $ P \to Q $.
\end{proof}

\subsection{The theory presheaf}
\begin{example}
  Let $ T $ be an algebraic theory. We can endow $ T $ with a $ T $-presheaf structure, by taking the substitution operator of $ T $ as the action on $ T $. Since this commutes with the substitution operator and the projections, $ T $ is a $ T $-presheaf.
\end{example}

\begin{lemma}\label{lem:presheaf-Yoneda}
  Given an algebraic theory $ T $ and a $ T $-presheaf $ Q $, we have for all $ n $ a bijection of sets
  \[ \varphi: \Pshf T(T^n, Q) \cong Q_n. \]
\end{lemma}
\begin{proof}
  For $ f: \Pshf T(T^n, Q) $, take $ \varphi(f) = f_n(x_1, \dots, x_n) $.

  Conversely, for all $ q: Q_n $ and all $ t_1, \dots, t_n: T_m^n $ take
  \[ \varphi^{-1}(q)_m(t_1, \dots, t_n) = q \bullet t. \]
\end{proof}

\subsection{The "+l" presheaf}

\begin{example}[The `+l' presheaf]
  Given a $ T $-presheaf $ Q $, we can construct a presheaf $ A(Q, l) $ with $ A(Q, l)_n = Q_{n + l} $ and, for $ q: A(Q, l)_m $ and $ f: T_n^m $, action
  \[ q \bullet_{A(Q, l)} f = q \bullet_Q ((\iota_{n, l} (f_i))_i + (x_{n + i})_i). \]
\end{example}

\begin{lemma}
  For all $ l $ and $ T $-presheaves $ Q $, $ A(Q, l) $ is the exponential object $ Q^{T^l} $.
\end{lemma}
\begin{proof}
  We will show that $ A(-, l) $ constitutes a right adjoint to the functor $ - \times T^l $. We will do this using universal arrows.

  For $ Q $ a $ T $-presheaf, take the arrow $ \varphi: A(Q, l) \times T^l \to Q $ given by $ \varphi(q, t) = q \bullet_Q ((x_{n, i})_i + t) $ for $ q: A(Q, l)_n = Q_{n + l} $ and $ t: T^l_n $.

  Now, given a $ T $-presheaf $ Q^\prime $ and a morphism $ \psi: Q^\prime \times T^l \to Q $. Define $ \tilde \psi: Q^\prime_n \to A(Q, l)_n $ by $ \tilde \psi(q) = \psi(\iota_{n, l}(q), (x_{n + i})_i) $.

  Then $ \psi $ factors as $ \varphi \circ (\tilde \psi \times \id{T^l}) $. Also, some equational reasoning shows that $ \tilde \psi $ is unique, which proves that $ \varphi $ indeed is a universal arrow.
\end{proof}
