\chapter{The formalization}

\section{Statistics}

\section{Components}

\section{Displayed categories}\label{sec:displayed-categories}

\subsection{Fibrations}

\subsection{Defininig objects by their category instead of the other way around}

\subsection{\_ax for categories and their objects}

\subsection{Cartesian vs cartesian' for products}

\subsection{Limits}

\subsubsection{Limits in a fibered category}

\section{Inductive types}

\section{The formalization of the \texorpdfstring{$ \lambda $}{lambda}-calculus}
Defining Lambda Calculus in a different way (not as an axiomatized HIT)
  - As set quotient instead of HIT
  - With a signature

\section{Tuples}
$ stn m \to A $ vs vec A

\section{Products}
$ T \times (T \times \dots \times T) $ vs $ T \times T^n vs T^(S n) $
Terminal as product over empty set
over any set with a function to empty.

\section{The \texorpdfstring{$ n + p $}{n + p}-presheaf}
L (S n) (for lambda) vs L (n + 1) (stemming from the naive implementation of the L (n + p) presheaf)

\section{Quotients}
Quotients (by hrel or eqrel) vs coproducts (generalizing to arbitrary category with coproduct) vs a category with some structure

\section{The Karoubi envelope}
KanExtension instead of specific construction at KaroubiEnvelope

\section{Univalence}
Univalence bewijzen via isweqhomot vs direct

\section{Equality, Iso's and Equivalence (of categories)}
It is important to choose the right kind of equality to prove.
