\chapter{Category Theoretic Preliminaries}

I will assume a familiarity with the category-theoretical concepts presented in \cite{CT4P}. These include categories, functors, isomorphisms, natural transformations, adjunctions, equivalences and limits.

\section{Notation}
For an object $ c $ in a category $ C $, I will write $ c: C $.

For a morphism $ f $ between objects $ c $ and $ c^\prime $ in a category $ C $, I will write $ f: C(c, c^\prime) $ or $ f: c \to c^\prime $.

For composition of morphisms $ f: C(c, d) $ and $ g: C(d, e) $, I will write $ f \cdot g $.

For composition of functors $ F: A \to B $ and $ G: B \to C $, I will write $ F \bullet G $.

\section{Universal Arrows}

One concept in category theory that can be used to describe a lot of limits and adjunctions is that of a universal arrow (see for example \cite{MacLane}, Part III)
\begin{definition}
  A \iindex{universal arrow} from an object $ c: C $ to a functor $ F: D \to C $ consists of an object $ d: D $ and a morphism $ f: D(c, F(d)) $ such that for every similar pair $ (d^\prime, f^\prime) $, $ f^\prime $ factors uniquely as $ f \cdot F(g) $ for some $ g: C(d, d^\prime) $:
  \begin{center}
    \begin{tikzcd}
      c \arrow[d, "f"'] \arrow[rd, "f^\prime"] &\\
      F(d) \arrow[r, "F(g)"', dashed] & F(d^\prime)
    \end{tikzcd}
  \end{center}
\end{definition}

There is also the dual concept of a universal arrow $ (d, f) $ from a functor $ F $ to an object $ c: C $. Its universal property yields the following diagram:
\begin{center}
  \begin{tikzcd}
    F(d^\prime)\arrow[rd, "f^\prime"']  \arrow[r, "F(g)", dashed] & F(d) \arrow[d, "f"]\\
    & c
  \end{tikzcd}
\end{center}

\section{Adjunctions}

Recall that an \iindex{adjunction} $ L \dashv R $ is a pair of functors
\begin{center}
  \begin{tikzcd}
    D \arrow[r, bend right, "R"'{name=R}] & C \arrow[l, bend right, "L"'{name=L}]
  \end{tikzcd}
\end{center}
with natural transformations (the unit and counit)
\[ \eta: \id C \Rightarrow L \bullet R \quad \text{and} \quad \epsilon: R \bullet L \Rightarrow \id D \]
such that the diagrams
\begin{center}
  \begin{tikzcd}[column sep = .5cm]
    L \arrow[rd, "\eta \bullet L"'] \arrow[rr, "\id L"] & & L\\
    & L \bullet R \bullet L \arrow[ur, "L \bullet \epsilon"']
  \end{tikzcd}
  \qquad
  \begin{tikzcd}[column sep = .5cm]
    R \arrow[rd, "R \bullet \eta"'] \arrow[rr, "\id R"] & & R\\
    & R \bullet L \bullet R \arrow[ur, "\epsilon \bullet R"']
  \end{tikzcd}
\end{center}
commute. Here the natural transformation $ \eta \bullet L: L \bullet R \bullet L $ is the natural transformation $ \eta $ whiskered on the right by $ L $, and the other whiskered transformations are similar.

An alternative characterization (see \cite{MacLane}, chapter IV.1, Theorem 2) of an adjunction $ L \dashv R $ is as a natural bijection
\[ \varphi: D(L(c), d) \xrightarrow{\sim} C(c, R(d)). \]
Naturality means that for all $ f: C(c^\prime, c) $, $ g: D(d, d^\prime) $ and $ h: D(L(c), d) $,
\[ \varphi(L(f) \cdot h \cdot g) = f \cdot \varphi(h) \cdot R(g). \]

Lastly, one can construct an adjunction using universal arrows. This lends itself particularly well for a formalization, where it is often preferable to have as little `demonstranda' as possible:
\begin{lemma}
  One can construct an adjunction $ (F, G, \eta, \epsilon) $ as above from only the functor $ L: C \to D $ and, for each $ c: C $, a universal arrow $ (R(c), \epsilon_c) $ from $ L $ to $ c $.
\end{lemma}
\begin{proof}
  See \cite{MacLane}, Chapter IV.1, Theorem 2 (iv).
\end{proof}

\subsection{Adjoint equivalences}
An (adjoint) equivalence of categories has multiple definitions. The one we will use here is the following:

\begin{definition}\label{def:equivalence-of-categories}
  An \iindex{adjoint equivalence} between categories $ C $ and $ D $ is a pair of adjoint functors $ L \dashv R $ like above such that the unit $ \eta: \id{C} \Rightarrow L \bullet R $ and counit $ \epsilon: R \bullet L \Rightarrow \id{D} $ are isomorphisms of functors.
\end{definition}

\subsection{Exponential objects}
Note that in the category of sets, for all $ X, Y: \SET $, we have a set $ (X \to Y) $. Also, for all $ X, Y, Z $, there is a (natural) bijection
\[ (X \times Y \to Z) \cong (X \to (Y \to Z)) \]
which we can also write as
\[ \SET(X \times Y, Z) \cong \SET(X, (Y \to Z)). \]
In other words, we have functors $ X \mapsto X \times Y $ and $ Z \mapsto (Y \to Z) $, and these two form an adjunction. The following generalizes this
\begin{definition}
  A category $ C $ has \textit{exponential objects}\index{exponential objects} (or \textit{exponentials}) if for all $ c: C $, the functor $ c^\prime \mapsto c^\prime \times c $ has a right adjoint, which we denote $ d \mapsto d^c $.
\end{definition}

\begin{remark}
  It is actually very well possible that a category does not have all exponentials, but it has some objects $ c, d, d^c: C $ with a natural bijection
  \[ C(d^\prime \times c, d) \cong C(d^\prime, d^c). \]
  Then $ d^c $ is still called an exponential object.
\end{remark}

\section{Fibrations}
Let $ P : E \to B $ be a functor. In this case, we will view this as the category $ E $ `lying over' the category $ B $, with for every point $ b: B $, a slice $ E_b = P^{-1}(B) $ lying `above' $ b $.

\begin{definition}
  A morphism $ f: E(y, z) $ is called \iindex{cartesian} if for all $ g: E(x, z) $ and $ h: B(P(x), P(y)) $ with $ h \cdot P(f) = P(g) $, there exists $ \bar h: E(x, y) $ such that $ P(\bar h) = h $ and $ \bar h \cdot f = g $, like illustrated in the following diagram from \cite{nlab:grothendieck_fibration}
  \begin{center}
    \begin{tikzcd}[row sep = .5in]
      E \arrow[d, "P"] &x \arrow[rr, "\forall g", bend left] \arrow[r, "\exists! \bar h"', dashed] & y \arrow[r, "f"'] & z\\
      B & P(x) \arrow[r, "\forall h"'] \arrow[rr, "P(g)", bend left] & P(y) \arrow[r, "P(f)"'] & P(z)
    \end{tikzcd}
  \end{center}
\end{definition}

\begin{definition}
  $ P $ is a \iindex{fibration} if for all $ y: E $ and morphisms $ f: B(x, P(y)) $, there exist an object $ \bar x: E $ and a cartesian morphism $ \bar f: E(\bar x, y) $ such that $ P(\bar x) = x $ and $ P(\bar f) = f $:
  \begin{center}
    \begin{tikzcd}
      E \arrow[d, "P"] & \bar x \arrow[r, "\exists \bar f", dashed] & y\\
      B & x \arrow[r, "\forall f"] & P(y)
    \end{tikzcd}
  \end{center}
\end{definition}

\section{(Co)slice categories}
Given an object in a category $ c: C $, the morphisms to and from $ c $ constitute the slice and coslice categories
\begin{definition}
  The \iindex{slice category} $ C \downarrow c $ is the category with as objects the morphisms to $ c $:
  \[ (C \downarrow c)_0 = \sum_{c^\prime: C} C(c^\prime, c). \]
  The morphisms from $ (c^\prime, f) $ to $ (c^{\prime\prime}, f^\prime) $ are the morphisms $ g: c^\prime \to c^{\prime\prime} $ making the following diagram commute.
  \begin{center}
    \begin{tikzcd}[column sep=.5cm]
      c^\prime \arrow[rr, "g"] \arrow[rd, "f"'] & & c^{\prime\prime} \arrow[ld, "f^\prime"]\\
      & c
    \end{tikzcd}
  \end{center}
\end{definition}
The \textit{coslice category}\index{slice category!co-} $ c \downarrow C $ is similar, but with the morphisms \textit{from} $ c $ instead of \textit{to} $ c $:
\[ (c \downarrow C)_0 = \sum_{c^\prime: C} C(c, c^\prime). \]

\section{Dependent products and sums}\label{sec:dependent-products}
The following is based loosely on Section 4.1 of \cite{taylor}.

Take a category $ C $. To talk about dependent sums $ \sum_{a: A} X_a $ and products $ \prod_{a: A} X_a $ in $ C $, we first need some way to construct the family of objects $ (X_a)_a $. Of course, we can do this \textit{externally} using a set $ A $, and picking an object $ X_a : C $ for every element $ a : A $. We then have a category of such families $ C^A $, with objects $ (X_a)_a $ and morphisms $ (f_a)_a: C^A((X_a)_a, (Y_a)_a) $, with $ f_a: X_a \to Y_a $. We write $ C^A $ because we can view this as just the $ A $-fold power of $ C $. Now, this assignment of categories $ A \mapsto C^A $ can be turned into a contravariant (pseudo)functor of $ 2 $-categories $ \op \SET \to \Cat $. It sends a morphism $ f: A \to B $ to the `relabeling' or `substitution' functor $ C^B \to C^A $, $ (X_b)_b \mapsto (X_{f(a)})_a $.

However, there is also an \textit{internal} representation, as a morphism $ X \to A $. We can turn the collection of these morphisms over all the $ A : C $ simultaneously into a category $ C^2 $ (abusing notation a bit, writing $ 2 $ for the two-point category $ \bullet \to \bullet $). Then taking codomains $ (X \to A) \mapsto A $ gives a functor $ C^2 \to C $. The fiber of this functor above $ A $ is the slice category $ (C \downarrow A) $.

In $ \SET $, the external and internal ways of indexing are actually equivalent, because given a family $ (X_a)_a $ we can construct a morphism $ f: \sum_{a : A} X_a \to A $ and conversely, we can recover the family $ (X_a)_a $ as $ (f^{-1}(a))_a $.

Also note that for $ \SET $, if we consider an indexed family $ (X_a)_a $ as some function $ X: A \to \SET $, then substitution over $ \alpha: B \to A $ is just given by postcomposition $ X \circ \alpha: B \to \SET $. It turns out that in the internal representation this arises as the pullback
\begin{center}
  \begin{tikzcd}
    \sum_{b : B} X_{\alpha(b)} \arrow[r] \arrow[d, "\alpha^* f"] \arrow[dr, phantom, "\lrcorner", very near start] & \sum_{a : A} X_a \arrow[d, "f"]\\
    B \arrow[r, "\alpha"] & A
  \end{tikzcd}
\end{center}
This can be turned into a pullback or `substitution' functor $ \alpha^* $, which Taylor calls $ \mathtt{P}\alpha $. This turns the functor $ \SET^2 \to \SET $ into a fibration. We can construct $ \alpha^* : (C \downarrow A) \to (C \downarrow B) $ for any category $ C $ with pullbacks and any morphism $ \alpha: C(B, A) $. The existence of this pullback makes the functor $ C^2 \to C $ into a fibration.

Now, in $ \SET $, for a family $ (X_a)_a $, consider the dependent product $ \prod_{a : A} X_a $. Its elements $ (x_a)_a $ can be identified with morphisms from the terminal set: $ \{ \star \} \to \prod_{a : A} X_a $, sending $ \star $ to $ (x_a)_a $. However, they can also be identified with the morphisms $ \varphi: A \to \sum_{a : A} X_a $ that make the following diagram commute, sending $ a $ to $ x_a $:
\begin{center}
  \begin{tikzcd}
    A \arrow[rd, "\id A"] \arrow[rr, " \varphi "] & & \sum_{a : A} X_a \arrow[ld, "f"]\\
    & A
  \end{tikzcd}
\end{center}
These are morphisms in $ (\SET \downarrow A) $ from $ (A, \id A) $ to $ (\sum_{a : A} X_a, f) $. Note that for the terminal morphism $ \alpha: A \to \{ \star \} $, we have $ \id A = \alpha^*(\id{\{\star\}}) $. To summarize, we have an equivalence
\[ (\SET \downarrow A)(\alpha^*(\id{\{\star\}}), f) \simeq (\SET \downarrow {\{\star\}})( \id{\{ \star \}}, \prod_{a : A} X_a). \]
Now, given a family of families $ ((X_b)_{b : B_a})_{a : A} $, we can wonder whether we can construct the family of dependent products $ (\prod_{b : B_a} X_b)_a $. In $ \SET $, this is definitely possible, and from this, we get an equivalence again
\[ (\SET \downarrow (\sum_{a : A} B_a))(\alpha^*(\id A), f) \simeq (\SET \downarrow A)(\id A, (\prod_{b : B_a} X_b)_a). \]
These equivalences suggest an adjunction $ \alpha^* \dashv \prod_{\dots} $. We can use this to define in general
\begin{definition}
  For a category $ C $ and a morphism $ \alpha: B \to A $, the \iindex{dependent product} along $ \alpha $ is, if it exists, the right adjoint to the pullback functor:
  \begin{center}
    \begin{tikzcd}
      (C \downarrow A) \arrow[r, shift left=2, "\alpha^*"{name=A}] &
      (C \downarrow B) \arrow[l, shift left=2, "\prod_\alpha"{name=B}]
      \ar[from=A, to=B, symbol=\dashv]
    \end{tikzcd}
  \end{center}
\end{definition}
\begin{remark}
  As argued above, we can recover the familiar dependent product $ \prod_{a : A} X_a $ of a family $ (X_a)_a $ as the dependent product $ \prod_\alpha f $ along the terminal morphism $ \alpha: A \to I $, with $ f: \sum_a X_a \to A $ the internal representation of the family. Here we use the equivalence between $ (C \downarrow I) $ and $ C $.
\end{remark}

Now we turn our attention to dependent sums. In $ \SET $, let $ (X_a)_a $ and $ (B_a)_a $ be two families over $ A $ and let $ ((Y_b)_{b : B_a})_{a : A} $ be a family of families. Let $ \alpha: \sum_{a : A} B_a \to A $ be the internal representation of $ (B_a)_a $. A family of maps $ f_a : (\sum_{b : B_a} Y_b)_a \to X_a $ consists of maps $ Y_b \to X_a $ for all $ b : B_a $, so these are maps $ f_b : Y_b \to X_{\alpha(b)} $. This gives an equivalence
\[ (\SET \downarrow A)((\sum_{b : B_a} Y_b)_a, (X_a)_a) \simeq (\SET \downarrow (\sum_{a : A} B_a))((Y_b)_b, \alpha^*((X_a)_a)). \]
This, again, suggests an adjunction which we will use as a definition.
\begin{definition}
  For a category $ C $ and a morphism $ \alpha: B \to A $, the \iindex{dependent sum} along $ \alpha $ is, if it exists, the left adjoint to the pullback functor:
  \begin{center}
    \begin{tikzcd}
      (C \downarrow A) \arrow[r, shift right=2, "\alpha^*"'{name=B}] &
      (C \downarrow B) \arrow[l, shift right=2, "\sum_\alpha"'{name=A}]
      \ar[from=A, to=B, symbol=\dashv]
    \end{tikzcd}
  \end{center}
\end{definition}

However, note that the conversion from an external to an internal representation in $ \SET $ already contained a dependent sum, which is no coincidence. It turns out that in practice, we will never have a hard time obtaining dependent sums:
\begin{lemma}
  Let $ \alpha : C(B, A) $ be a morphism in a category. If the pullback functor $ \alpha^*: (C \downarrow A) \to (C \downarrow B) $ exists, it has a left adjoint given by postcomposition with $ \alpha $.
\end{lemma}
\begin{proof}
  For morphisms $ f: X \to A $, $ g: Y \to B $, the universal property of the pullback, with the following diagram
  \begin{center}
    \begin{tikzcd}
      Y \arrow[r, dashed, "\varphi"'] \arrow[rr, "\psi", bend left] \arrow[dr, "g"'] & \alpha^* X \arrow[d, "\alpha^* f"'] \arrow[r] \arrow[rd, phantom, "\lrcorner", very near start] & X \arrow[d, "f"]\\
      & B \arrow[r, "\alpha"'] & A
    \end{tikzcd}
  \end{center}
  gives an equivalence between morphisms $ \varphi: Y \to \alpha^* X $ that commute with $ g $ and $ \alpha^* f $, and morphisms $ \psi: Y \to X $ that commute with $ g $, $ f $ and $ \alpha $. In other words:
  \[ (C \downarrow A)(g \cdot \alpha, f) \simeq (C \downarrow B)(g, \alpha^*(f)), \]
  which shows the adjunction.
\end{proof}

Now, let $ f : X \to A $ be the internal representation of an indexed family $ (X_a)_a $ and let $ \alpha : A \to I $ be the terminal projection. We have $ \sum_{a : A} X_a = f \circ \alpha : X \to I $. By the equivalence between $ (C \downarrow I) $ and $ C $, we see that the dependent sum of the family $ (X_a)_a $ is exactly $ X $. Therefore, our attention is mainly focused on the dependent product.

We will close this section with a name for a category that has all dependent products:

\begin{definition}
  A \index{cartesian closed!locally}\textit{locally cartesian closed} category is a category $ C $ with pullbacks such that each pullback functor $ \alpha^* $ has a right adjoint.
\end{definition}

Apart from having dependent sums and products, there also is the following theorem that shows the significance of locally cartesian closedness:
\begin{lemma}\label{lem:locally-cartesian-closed}
  A category $ C $ is locally cartesian closed iff $ (C \downarrow A) $ is cartesian closed for each $ A : C $.
\end{lemma}
\begin{proof}
  See the end of Section 1.3 of \cite{freyd}.
\end{proof}

\section{(Weakly) terminal objects}
\begin{definition}
  If a category has an object $ t $, such that there is a (not necessarily unique) morphism to it from every other object in the category, $ t $ is said to be a \iindex{weakly terminal object}.
\end{definition}

\begin{definition}
  Let $ C $ be a category with terminal object $ t $. For an object $ c: C $, a \iindex{global element} of $ c $ is a morphism $ f: C(t, c) $.
\end{definition}


\section{Kan Extensions}
One of the most general and abstract concepts in category theory is the concept of \textit{Kan extensions}. In \cite{MacLane}, Section X.7, MacLane notes that
\begin{quote}
  The notion of Kan extensions subsumes all the other fundamental concepts of category theory.
\end{quote}
In this thesis, we will use left Kan extension a handful of times. It comes in handy when we want to extend a functor along another functor in the following way:

Let $ A $, $ B $ and $ C $ be categories and let $ F : A \to B $ be a functor.
\begin{definition}
  Precomposition gives a functor between functor categories $ F_* : [B, C] \to [A, C] $. If $ F_* $ has a left adjoint, we will denote call this adjoint functor the \textit{left Kan extension}\index{Kan extension!left} along $ F $ and denote it $ \mathrm{Lan}_F : [A, C] \to [B, C] $.

  \begin{center}
    \begin{tikzcd}[column sep=.5cm]
      A \arrow[rr, "F"] \arrow[rd, dashed, "F_* G"'] & & B \arrow[ld, "G"]\\
      & C
    \end{tikzcd} \hspace{2cm} \begin{tikzcd}[column sep=.5cm]
      A \arrow[rr, "F"] \arrow[rd, "G"'] & & B \arrow[ld, dashed, "\Lan F G"]\\
      & C
    \end{tikzcd}
  \end{center}

  Analogously, when $ F_* $ has a right adjoint, one calls this the \textit{right Kan extension}\index{Kan extension!right} along $ F $ and denote it $ \mathrm{Ran}_F: [A, C] \to [B, C] $.
\end{definition}

If a category has limits (resp. colimits), we can construct the right (resp. left) Kan extension in a `pointwise' fashion (see Theorem X.3.1 in \cite{MacLane} or Theorem 2.3.3 in \cite{Kashiwara}). Below, I will outline the parts of the construction that we will need explicitly in this thesis.
\begin{lemma}
  If $ C $ has colimits, $ \Lan F {} $ exists.
\end{lemma}
\begin{proof}
  First of all, for objects $ b: B $, we take
  \[ \Lan F G(b) := \text{colim} \left( (F \downarrow b) \to A \xrightarrow G C \right). \]

  Here, $ (F \downarrow b) $ denotes the comma category with as objects the morphisms $ B(F(a), b) $ for all $ a: A $, and as morphisms from $ f: B(F(a), b) $ to $ f^\prime: B(F(a^\prime), b) $ the morphisms $ g: A(a, a^\prime) $ that make the diagram commute:
  \begin{center}
    \begin{tikzcd}[column sep=.5cm]
      F(a) \arrow[rr, "F(g)"] \arrow[rd, "f"'] & & F(a^\prime) \arrow[ld, "f^\prime"]\\
      & b
    \end{tikzcd}
  \end{center}
  and $ (F \downarrow b) \to A $ denotes the projection functor that sends $ f: B(F(a_1), b) $ to $ a_1 $.

  Now, a morphism $ h: B(b, b^\prime) $ gives a morphism of diagrams, sending the $ F(a) $ corresponding to $ f: B(G(a), b) $ to the $ F(a) $ corresponding to $ f \cdot h: B(G(a), b^\prime) $. From this, we get a morphism $ \Lan F G(h): C(\Lan F G(b), \Lan F G(b^\prime)) $.

  The unit of the adjunction is a natural transformation $ \eta: \id{[A, C]} \Rightarrow \Lan F {} \bullet F_* $. We will define this pointwise, for $ G: [A, C] $ and $ a: A $. Our diagram contains the $ G(a) $ corresponding to $ \id{F(a)}: (F \downarrow F(a)) $ and the colimit cocone gives a morphism
  \[ \eta_G(a) : C(G(a), \Lan F G (F(a))), \]
  the latter being equal to $ (\Lan F {} \bullet F_*)(G)(a) $.

  The counit of the adjunction is a natural transformation $ \epsilon: F_* \bullet \Lan F {} \Rightarrow \id{[B, C]} $. We will also define this pointwise, for $ G: [B, C] $ and $ b: B $. The diagram for $ \Lan F (F_* G)(b) $ consists of $ G(F(a)) $ for all $ f: B(F(a), b) $. Then, by the universal property of the colimit, the morphisms $ G(f): C(G(F(a)), G(b)) $ induce a morphism
  \[ \epsilon_G(b) : C(\Lan F (F_* G)(b), G(b)). \]
\end{proof}

\begin{lemma}
  If $ F : A \to B $ is a fully faithful functor, and $ C $ is a category with colimits, $ \eta: \id{[A, C]} \Rightarrow \Lan F {} \bullet F_* $ is a natural isomorphism.
\end{lemma}
\begin{proof}
  To show that $ \eta $ is a natural isomorphism, we have to show that $ \eta_G(a^\prime): G(a^\prime) \Rightarrow \Lan F G(F(a^\prime)) $ is an isomorphism for all $ G: [A, C] $ and $ a^\prime: A $. Since a left adjoint is unique up to natural isomorphism (\cite{CT4P}, Exercise 153), we can assume that $ \Lan F G(F(a^\prime)) $ is given by
  \[ \text{colim} ((F \downarrow F(a^\prime)) \to A \xrightarrow G C). \]
  Now, the diagram for this colimit consists of $ G(a) $ for each arrow $ f: B(F(a), F(a^\prime)) $. Since $ F $ is fully faithful, we have $ f = F(\overline f) $ for some $ \overline f: A(a, a^\prime) $. If we now take the arrows $ G(\overline f): C(G(a), G(a^\prime)) $, the universal property of the colimit gives an arrow
  \[ \varphi: C(\Lan F G(F(a^\prime)), G(a^\prime)) \]
  which constitutes an inverse to $ \eta_G(a^\prime) $. The proof of this revolves around properties of the colimit and its (induced) morphisms.
\end{proof}

\begin{remark}
  In the same way, if $ C $ has limits, $ \epsilon $ is a natural isomorphism.
\end{remark}

\begin{corollary}
  If $ C $ has limits or colimits, precomposition of functors $ [B, C] $ along a fully faithful functor is (split) essentially surjective.
\end{corollary}
\begin{proof}
  For each $ G: [A, C] $ we take $ \Lan F G: [B, C] $, and we have $ F_*(\Lan F G) \cong G $.
\end{proof}

\begin{corollary}
  If $ C $ has colimits (resp. limits), left (resp. right) Kan extension of functors $ [A, C] $ along a fully faithful functor is fully faithful.
\end{corollary}
\begin{proof}
  Since left Kan extension along $ F $ is the left adjoint to precomposition, we have
  \[ [A, C](\Lan F G, \Lan F G^\prime) \cong [B, C](G, F_*(\Lan F G^\prime)) \cong [B, C](G, G^\prime). \]
\end{proof}


\section{The Karoubi envelope}
Let $ C $ be a category. If we have a retraction-section pair
\begin{tikzcd}
  c \arrow[r, shift left, "r"] & d \arrow[l, shift left, "s"]
\end{tikzcd}
we have (by definition) $ s \cdot r = \id d $. On the other hand, $ r \cdot s: c \to c $ is an idempotent morphism, since $ r \cdot s \cdot r \cdot s = r \cdot s $. Conversely, we can wonder whether for some idempotent morphism $ a: c \to c $, we can find a retraction-section pair $ (r, s) $ such that $ a = r \cdot s $. If this is the case, we say that the idempotent $ a $ \textit{splits}\index{split idempotent}. If $ a $ does not split, we can wonder whether we can find an embedding $ \iota_C : C \hookrightarrow \overline C $ such that the idempotent $ \iota_C(a): \iota_C(c) \to \iota_C(c) $ does split. This is one way to arrive at the \textit{Karoubi envelope}.

\begin{definition}
  We define the category $ \overline C $. The objects of $ \overline C $ are tuples $ (c, a) $ with $ c: C $, $ a: C(c, c) $ such that $ a \cdot a = a $. The morphisms between $ (c, a) $ and $ (d, b) $ are morphisms $ f: C(a, b) $ such that $ a \cdot f \cdot b = f $. The identity morphism on $ (c, a) $ is given by $ a $ and $ \overline C $ inherits morphism composition from $ C $.
\end{definition}
This category is called the \iindex{Karoubi envelope}, the \textit{idempotent completion}\index{idempotent completion|see{Karoubi envelope}}, the \textit{category of retracts}\index{category of retracts|see{Karoubi envelope}}, or the \textit{Cauchy completion}\index{Cauchy completion|see{Karoubi envelope}} of $ C $.

\begin{remark}
  Note that for a morphism $ f: \overline C((c, a), (d, b)) $,
  \[ a \cdot f = a \cdot a \cdot f \cdot b = a \cdot f \cdot b = f \]
  and in the same way, $ f \cdot b = f $.
\end{remark}

\begin{definition}
  We have an embedding $ \iota_C: C \to \overline C $, sending $ c: C $ to $ (c, \id{c}) $ and $ f: C(c, d) $ to $ f $.
\end{definition}

\begin{lemma}
  Every object $ c: \overline C $ is a retract of $ \iota_C(c_0) $ for some $ c_0: C $.
\end{lemma}
\begin{proof}
  Note that $ c = (c_0, a) $ for some $ c_0: C $ and an idempotent $ a: c \to c $. We have morphisms
  \begin{tikzcd}
    \iota_C(c) \arrow[r, shift left, "a_\rightarrow"] & (c, a) \arrow[l, shift left, "a_\leftarrow"]
  \end{tikzcd}, both given by $ a $. We have $ a_\leftarrow \cdot a_\rightarrow = a = \id{(c, a)} $, so $ (c, a) $ is a retract of $ \iota_C(c) $.
\end{proof}

\begin{lemma}
  In $ \overline C $, every idempotent splits.
\end{lemma}
\begin{proof}
  Take an idempotent $ e: \overline C(c, c) $. Note that $ c $ is given by an object $ c_0: C $ and an idempotent $ a: C(c_0, c_0) $. Also, $ e $ is given by some idempotent $ e: C(c_0, c_0) $ with $ a \cdot e \cdot a = e $.

  Now, we have $ (c_0, e): \overline C $ and morphisms
  \begin{tikzcd}
    (c_0, a) \arrow[r, shift left, "e_\rightarrow"] & (c_0, e) \arrow[l, shift left, "e_\leftarrow"]
  \end{tikzcd}, both given by $ e $. We have $ e_\leftarrow \cdot e_\rightarrow = e = \id{(c_0, e)} $, so $ (c_0, e) $ is a retract of $ (c_0, a) $. Also, $ e = e_\rightarrow \cdot e_\leftarrow $, so $ e $ is split.
\end{proof}

\begin{remark}
  Note that the embedding is fully faithful, since
  \[ \overline C((c, \id c), (d, \id d)) = \{ f: C(c, d) \mid \id c \cdot f \cdot \id d = f \} = C(c, d). \]
\end{remark}

\begin{remark}
  Let $ D $ be a category. Suppose that we have a retraction-section pair in $ D $, given by
  \begin{tikzcd}
    d \arrow[r, shift left, "r"] & d^\prime \arrow[l, shift left, "s"]
  \end{tikzcd}.
  Now, suppose that we have an object $ c: D $ and a morphism $ f $ with $ (r \cdot s) \cdot f = f $. Then we get a morphism $ s \cdot f: d^\prime \to c $ such that $ f $ factors as $ r \cdot (s \cdot f) $. Also, for any $ g $ with $ r \cdot g = f $, we have
  \[ g = s \cdot r \cdot g = s \cdot f. \]
  \begin{center}
    \begin{tikzcd}
      d \arrow[r, left, "r"] \arrow[rd, "f"'] & d^\prime \arrow[r, "s"] \arrow[d, "s \cdot f" description] & d \arrow[ld, "f"]\\
      & c
    \end{tikzcd}.
  \end{center}
  Therefore, $ d^\prime $ is the equalizer of \begin{tikzcd}
    d \arrow[r, shift left, "\id d"] \arrow[r, shift right, "r \cdot s"'] & d
  \end{tikzcd}. In the same way, it is also the coequalizer of this diagram.

  Now, note that if we have a coequalizer $ c^\prime $ of $ \id c $ and $ a $, and an equalizer $ d^\prime $ of $ \id d $ and $ b $, the universal properties of these give an equivalence
  \[ D(c^\prime, d^\prime) \cong \{ f: D(c, d^\prime) \mid a \cdot f = f \} \cong \{ f: D(c, d) \mid a \cdot f = f = f \cdot b \}. \]
  \begin{center}
    \begin{tikzcd}
      c \arrow[r, shift left, "\id c"] \arrow[r, shift right, "a"'] & c \arrow[r] \arrow[d] \arrow[rd] & c^\prime \arrow[d]\\
      d & d \arrow[l, shift right, "\id d"'] \arrow[l, shift left, "b"] & d^\prime \arrow[l]
    \end{tikzcd}
  \end{center}
\end{remark}

Since a functor preserves retracts, and since every object of $ \overline C $ is a retract of an object in $ C $, one can lift a functor from $ C $ (to a category with (co)equalizers) to a functor on $ \overline C $.

For convenience, the lemma below works with pointwise left Kan extension using colimits, but one could also prove this using just (co)equalizers (or right Kan extension using limits).
\begin{lemma}
  Let $ D $ be a category with colimits. We have an adjoint equivalence between $ [C, D] $ and $ [\overline C, D] $.
\end{lemma}
\begin{proof}
  We already have an adjunction $ \Lan {\iota_C} {} \dashv \iota_{C*} $. Also, since $ \iota_C $ is fully faithful, we know that $ \eta $ is a natural isomorphism. Therefore, we only have to show that $ \epsilon $ is a natural isomorphism. That is, we need to show that $ \epsilon_G(c, a): D(\Lan {\iota_C} (\iota_{C*} G) (c, a), G(c, a)) $ is an isomorphism for all $ G: [\overline C, D] $ and $ (c, a): \overline C $.

  One of the components in the diagram of $ \Lan {\iota_C} (\iota_{C*} G) (c, a) $ is the $ \iota_{C*} G(c) = G(c, \id c) $ corresponding to $ a: \iota_C(c) \to (c, a) $. This component has a morphism into our colimit
  \[ \varphi: C(G(\iota_C(c)), \Lan {\iota_C} (\iota_{C*} G) (c, a)). \]
  Note that we can view $ a $ as a morphism $ a: \overline C((c, a), \iota_C(c)) $. This gives us our inverse morphism
  \[ G(a) \cdot \varphi: C(G(c, a), \Lan {\iota_C} (\iota_{C*} G) (c, a)). \]
\end{proof}

\begin{lemma}
  The formation of the opposite category commutes with the formation of the Karoubi envelope.
\end{lemma}
\begin{proof}
  An object in $ \overline{\op C} $ is an object $ c: \op C $ (which is just an object $ c: C $), together with an idempotent morphism $ a: \op C(c, c) = C(c, c) $. This is the same as an object in $ \op{\overline C} $.

  A morphism in $ \overline{\op C}((c, a), (d, b)) $ is a morphism $ f: \op C(c, d) = C(d, c) $ such that
  \[ b \cdot_C f \cdot_C a = a \cdot_{\op C} f \cdot_{\op C} b = f. \]
  A morphism in $ \op{\overline C}((c, a), (d, b)) = \overline C((d, b), (c, a)) $ is a morphism $ f: C(d, c) $ such that $ b \cdot f \cdot a = f $.

  Now, in both categories, the identity morphism on $ (c, a) $ is given by $ a $.

  Lastly, $ \overline {\op C} $ inherits morphism composition from $ \op C $, which is the opposite of composition in $ C $. On the other hand, composition in $ \op{\overline C} $ is the opposite of composition in $ \overline C $, which inherits composition from $ C $.
\end{proof}

\begin{corollary}
  As the category $ \SET $ is cocomplete, we have an equivalence between the category of presheaves on $ C $ and the category of presheaves on $ \overline C $:
  \[ [\op C, \SET] \cong [\overline{\op C}, \SET] \cong [\op{\overline C}, \SET]. \]
\end{corollary}

\section{Monoids as categories}\label{sec:monoid-category}
Take a monoid $ M $.
\begin{definition}
  We can construct a category $ C_M $ with $ C_{M0} = \{ \star \} $, $ C_M(\star, \star) = M $. The identity morphism on $ \star $ is the identity $ 1: M $. The composition is given by multiplication $ g \cdot_{C_M} f = f \cdot_M g $.
\end{definition}

\begin{remark}
  Actually, we have a functor from the category of monoids to the category of setcategories (categories whose object type is a set).

  A monoid morphism $ f: M \to M^\prime $ is equivalent to a functor $ F_f: C_M \to C_{M^\prime} $. Any functor between $ C_M $ and $ C_{M^\prime} $ sends $ \star_M $ to $ \star_{M^\prime} $. The monoid morphism manifests as $ F_f(m) = f(m) $ for $ m: C_M(\star, \star) = M $.
\end{remark}

\begin{lemma}
  An isomorphism of monoids gives an (adjoint) equivalence of categories.
\end{lemma}
\begin{proof}
  Given an isomorphism $ f: M \to M^\prime $. Then we have functors $ F_f: C_M \to C_{M^\prime} $ and $ F_{f^{-1}}: C_{M^\prime} \to C_M $. Take the identity natural transformations $ \eta: \id{C_M} \Rightarrow F_f \bullet F_{f^{-1}} $ and $ \epsilon: F_{f^{-1}} \bullet F_f \Rightarrow \id{C_{M^\prime}} $. Of course these are natural isomorphisms.
\end{proof}

\begin{definition}
  A \textit{right monoid action}\index{monoid action} of $ M $ on a set $ X $ is a function $ X \times M \to X $ such that for all $ x: X $, $ m, m^\prime: M $,
  \[ x 1 = x \qquad \text{and} \qquad (x m) m^\prime = x (m \cdot m^\prime). \]
\end{definition}

\begin{definition}
  A \textit{morphism}\index{monoid action!morphism} between sets $ X $ and $ Y $ with a right $ M $-action is an $ M $-equivariant function $ f: X \to Y $: a function such that $ f(xm) = f(x)m $ for all $ x: X $ and $ m: M $.
\end{definition}

These, together with the identity and composition from $ \SET $, constitute a category \iindex{$ \RAct M $} of right $ M $-actions.

\begin{lemma}
  Presheaves on $ C_M $ are equivalent to sets with a right $ M $-action.
\end{lemma}
\begin{proof}
  This correspondence sends a presheaf $ F $ to the set $ F(\star) $, and conversely, the set $ X $ to the presheaf $ F $ given by $ F(\star) := X $. The $ M $-action corresponds to the presheaf acting on morphisms as $ xm = F(m)(x) $. A morphism (natural transformation) between presheaves $ F \Rightarrow G $ corresponds to a function $ F(\star) \to G(\star) $ that is $ M $-equivariant, which is exactly a monoid action morphism.
\end{proof}

\begin{remark}
  Since the category of sets with an $ M $-action is equivalent to a presheaf category, it has all limits. However, we can make this concrete. The set of the product $ \prod_i X_i $ is the product of the underlying sets. The action is given pointwise by $ (x_i)_i m = (x_i m)_i $.
\end{remark}

Note that the initial set with $ M $-action is $ \{ \star \} $, with action $ \star m = \star $.

\begin{lemma}\label{lem:global-action-elements}
  The global elements of a set with right $ M $-action correspond to the elements that are invariant under the $ M $-action.
\end{lemma}
\begin{proof}
  A global element of $ X $ is a morphism $ \varphi: \{ \star \} \to X $ such that for all $ m: M $, $ \varphi(\star)m = \varphi(\star m) = \varphi(\star) $. Therefore, it is given precisely by the element $ \varphi(\star): X $, which must be invariant under the $ M $-action.
\end{proof}

\begin{lemma}
  The category $ C $ of sets with an $ M $-action has exponentials.
\end{lemma}
\begin{proof}
  Given sets with $ M $-action $ X $ and $ Y $. Consider the set $ C(M \times X, Y) $ with an $ M $-action given by $ \phi m^\prime(m, x) = \phi(m^\prime m, x) $. This is the exponential object $ X^Y $, with the (universal) evaluation morphism $ X \times X^Y \to Y $ given by $ (x, \phi) \mapsto \phi(1, x) $.
\end{proof}

\begin{definition}
  We can view $ M $ as a set $ U_M $ with right $ M $-action $ m n = m \cdot_M n $ for $ m: U_M $ and $ n: M $.
\end{definition}

\subsection{Extension and restriction of scalars}

Let $ \varphi: M \to M^\prime $ be a morphism of monoids.

Remember that sets with a right monoid action are equivalent to presheaves on the monoid category. Also, $ \varphi $ is equivalent to a functor between the monoid categories. The following is a specific case of the concepts in the section about Kan extension:

\begin{lemma}
  We get a \iindex{restriction of scalars} functor $ \varphi_* $ from sets with a right $ M^\prime $-action to sets with a right $ M $-action.
\end{lemma}
\begin{proof}
  Given a set $ X $ with right $ M^\prime $-action, take the set $ X $ again, and give it a right $ M $-action, sending $ (x, m) $ to $ x \varphi(m) $.

  On morphisms, send an $ M^\prime $-equivariant morphism $ f: X \to X^\prime $ to the $ M $-equivariant morphism $ f: X \to X^\prime $.
\end{proof}

Since $ \SET $ has colimits, and restriction of scalars corresponds to precomposition of presheaves (on $ C_{M^\prime} $), we can give it a left adjoint. This is the (pointwise) left Kan extension, which boils down to:

\begin{lemma}
  We get an \iindex{extension of scalars} functor $ \varphi^* $ from sets with a right $ M $-action to sets with a right $ M^\prime $-action.
\end{lemma}
\begin{proof}
  Given a set $ X $ with right $ M $-action. Take $ Y = X \times M^\prime / \sim $ with the relation $ (x m, m^\prime) \sim (x, f(m) \cdot m^\prime) $ for $ m: M $. This has a right $ M^\prime $-action given by $ (x, m^\prime)n^\prime = (x, m^\prime n^\prime) $.

  On morphisms, it sends the $ m $-equivariant $ f: X \to X^\prime $ to the morphism $ (x, m^\prime) \mapsto (f(x), m^\prime) $.
\end{proof}

\begin{lemma}\label{lem:scalar-extension-monoid-monoid-action}
  For $ U_M $ the set $ M $ with right $ M $-action, we have $ \varphi^*(U_M) \cong U_{M^\prime} $.
\end{lemma}
\begin{proof}
  The proof relies on the fact that for all $ m: U_M $ and $ m^\prime : M^\prime $, we have
  \[ (m, m^\prime) \sim (1, \varphi(m) m^\prime). \]
\end{proof}

Consider the category $ D $ with $ D_0 = M^\prime $ and
\[ D(m^\prime, \overline m^\prime) = \{ m: M \mid \varphi(m) \cdot m^\prime = \overline m^\prime \}. \]

\begin{lemma}\label{lem:scalar-extension-terminal}
  Suppose that $ D $ has a weakly terminal element. Then for $ I_M $ the terminal object in the category of sets with a right $ M $-action, we have $ \varphi^*(I_M) \cong I_{M^\prime} $.
\end{lemma}
\begin{proof}
  If $ D $ has a weakly terminal object, there exists $ m_0 : M^\prime $ such that for all $ m^\prime: M^\prime $, there exists $ m: M $ such that $ \varphi(m) \cdot m^\prime = m_0 $.

  The proof relies on the fact that every element of $ \varphi^*(I_M) $ is given by some $ (\star, m^\prime) $, but then
  \[ (\star, m^\prime) = (\star \cdot m, m^\prime) \sim (\star, \varphi(m) \cdot m^\prime) = (\star, \overline{m^\prime}), \]
  so $ \varphi^*(I_M) $ has exactly $ 1 $ element.
\end{proof}

\begin{remark}
  For $ \varphi^* $ to preserve terminal objects, we actually only need $ D $ to be connected. The fact that $ \varphi^*(I_M) $ is a quotient by a symmetric and transitive relation then allows us to `walk' from any $ (\star, m^\prime_1) $ to any other $ (\star, m^\prime_2) $ in small steps.
\end{remark}

For any $ m^\prime_1, m^\prime_2: M^\prime $, consider the category $ D_{m^\prime_1, m^\prime_2} $, given by
\[ D_{m^\prime_1, m^\prime_2, 0} = \{ (m^\prime, m_1, m_2): M^\prime \times M \times M \mid m_i^\prime = \varphi(m_i) \cdot m^\prime \} \]
and
\[ D_{m^\prime_1, m^\prime_2}((m^\prime, m_1, m_2), (\overline m^\prime, \overline m_1, \overline m_2)) = \{ m: M \mid \varphi(m) \cdot m^\prime = \overline m^\prime, m_i = \overline m_i \cdot m \}. \]

\begin{lemma}\label{lem:scalar-extension-product}
  Suppose that $ D_{m^\prime_1, m^\prime_2} $ has a weakly terminal object for all $ m^\prime_1, m^\prime_2: M^\prime $. Then for sets $ A $ and $ B $ with right $ M $-action, we have $ \varphi^*(A \times B) \cong \varphi^*(A) \times \varphi^*(B) $.
\end{lemma}
\begin{proof}
  Now, any element in $ \varphi^*(A) \times \varphi^*(B) = (A \times M^\prime / \sim) \times (B \times M^\prime / \sim) $ is given by some $ (a, m^\prime_1, b, m^\prime_2) $.

  The fact that $ D_{m^\prime_1, m^\prime_2} $ has a weakly terminal object means that we have some $ \overline m^\prime: M^\prime $ and $ \overline m_1, \overline m_2: M $ with $ m_i^\prime = \varphi(\overline m_i) \cdot \overline m^\prime $. Therefore,
  \[ (a, m^\prime_1, b, m^\prime_2) = (a, \varphi(\overline m_1) \cdot \overline m^\prime, b, \varphi(\overline m_2) \cdot \overline m^\prime) \sim (a \overline m_1, \overline m^\prime, b \overline m_2, \overline m^\prime), \]
  so this is equivalent to some element in $ \varphi^*(A \times B) = (A \times B \times M^\prime / \sim) $. Note that this trivially respects the right $ M^\prime $-action.

  The fact that $ (\overline m^\prime, \overline m_1, \overline m_2) $ is weakly terminal also means that for all $ m^\prime: M^\prime $ and $ m_1, m_2: M $ with $ m_i^\prime = \varphi(m_i) \cdot m^\prime $, there exists $ m: M $ such that $ \varphi(m) \cdot m^\prime = \overline m^\prime $ and $ m_i = \overline m_i \cdot m $. This means that the equivalence that we established is actually well-defined: equivalent elements in $ \varphi^*(A) \times \varphi^*(B) $ are sent to equivalent elements in $ \varphi^*(A \times B) $.

  Therefore, we have an isomorphism $ \psi: \varphi^*(A) \times \varphi^*(B) \xrightarrow{\sim} \varphi^*(A \times B) $. Now we only need to show that the projections are preserved by this isomorphism. To that end, take $ x = (a, m^\prime_1, b, m^\prime_2) \sim (a \overline m_1, \overline m^\prime, b \overline m_2, \overline m^\prime) : \varphi^*(A) \times \varphi^*(B) $. We have
  \[ \varphi^*(\pi_1)(\psi(x)) = (a \overline m_1, \overline m^\prime) = \pi^\prime_1(x). \]
  In the same way, $ \varphi^*(\pi_2) \circ \psi = \pi^\prime_2 $ and this concludes the proof.
\end{proof}
