\chapter{Univalent Foundations}

\section{Dependent type theory}
Univalent foundations takes place in a framework of type theory. In this section, we will quickly introduce some topics that we will need in subsequent sections.

First of all, \iindex{type theory} is the study of `type systems'. A \iindex{type system} is a collection of \textit{terms} or \textit{elements}, each of which has a corresponding type. A type is much like a set in set theory in that it has elements (for example, $ \mathrm{true}: \mathrm{Bool} $ or $ 1: \mathbb N $), but there are important differences. First of all, in type theory, terms are declared together with their type, so every term has exactly one type, whereas in set theory, something can be an element of multiple sets, like $ 5 \in \mathbb N $ and $ 5 \in \mathbb R $. Secondly, equality in type theory can work a bit differently than in set-based mathematics, but I will cover this in the next section.

Of course, in computer science, we are very familiar with type systems. In most programming languages, values explicitly (e.g. in Java) or implicitly (e.g. in Python) have a type associated to them. For example, \texttt{"Hello, World"} is of type \textit{string}, \texttt{true} and \texttt{false} are of type \textit{boolean}, \texttt{1} is of type \textit{integer} and \texttt{-5.8} is of type \textit{floating point number}. And here, already, we see some subtleties: due to their different internal representations, many programming languages distinguish between integers and floating point numbers, even though every integer can be considered to be a floating point number. Usually, programming languages resolve this by offering methods (sometimes in the form of \textit{coercions}) to convert between the two types (discarding what comes after the decimal point when converting a floating point number to an integer). Another coercion that occurs sometimes is the conversion of a \textit{character} like \texttt{a}, \texttt{1} or \texttt{\&} to an integer and back.

So if we have types and elements, are types also elements of some type? This is a tricky question, because if we say that there exists a type $ \TYPE $ which contains all types (and therefore, itself), we introduce inconsistency in our type system (see \autocite{girard}). This is just like having a `set of all sets', which results in problematic questions like ``does its subset, containing only the sets which do not contain itself, contain itself?'' This is usually solved by either stating that types do not themselves have a type, or by assuming the existence of \textit{type universes} $ U_1, U_2, \dots $, such that every type is an element of some $ U_i $. Note that universes are allowed to be inclusive or not: we may have $ U_n \subseteq U_{n + 1} $. In this thesis, we will not explicitly mention particular universes. We will just write $ \TYPE $ to denote some type universe (or its category of types and functions). This is called \textit{typical ambiguity}, meaning that every theorem holds for all (suitable) assignments of universes to these instances of $ \TYPE $.

One notable class of types is given by the \textit{function types}. For types $ A $ and $ B $, our type system might have the type $ A \to B $. An element $ f $ of this type can be combined with an element $ a $ of $ A $ to give an element $ f(b) $ of $ B $. Of course, the elements of this type are thought of (and usually are) functions from $ A $ to $ B $.

Now, a type system may or may not have all kinds of constructs. One of these constructs is \index{dependent type}\textit{dependent types}. A dependent type is a type which depends on values of other types. For example, $ \mathtt{array}(T, n) $, the type of arrays of length $ n $, with elements of type $ T $. The study of type systems that have such dependent types is called \iindex{dependent type theory}.

Suppose that we have a type $ A $, and a dependent type which we will write $ B: A \to \TYPE $.

One of the possible constructs in a type system is a type $ \sum_{a: A} B(a) $ called the \textit{dependent sum}, consisting of all pairs $ (a, b) $ with $ a : A $ and $ b: B(a) $. So every element of $ \sum_{a: A} B(a) $ gives an element of one $ B(a) $ (for some $ a : A $). Note that for the constant dependent type $ B(a) = B $, $ \sum_{a: A} B $ is the product type $ A \times B $.

Another construct which may exist, is a type $ \prod_{a: A} B(a) $, consisting of all `functions' $ f $ which map elements $ a: A $ to elements $ f(a): B(a) $. Every element of $ \prod_{a: A} B(a) $ gives therefore elements of all the $ B(a) $ simultaneously. Note that $ \prod_{a: A} B $ is the function type $ A \to B $.

Lastly, there is a very strong connection between logic and type theory. This is called the \iindex{Curry-Howard correspondence} or sometimes referred to as \iindex{products as types}. We can view a type $ T $ as the proposition ``$ T $ has an element''. An element of $ T $ is then a `proof' or `witness' of this proposition. If the type system is strong enough, it allows us to do mathematics in it, where:
\begin{itemize}
  \item A function $ f: A \to B $ that takes an argument of type $ A $ and yields a result of type $ B $ corresponds to the proof of $ B $ under the hypothesis that $ A $ holds: ``suppose that $ A $, then $ B $''. Note that the notation $ A \to B $ also makes sense if we think of $ A $ and $ B $ propositions.
  \item The empty type $ \mathbf 0 $ corresponds to `false'. Note that for all types $ A $, we can construct a function $ \mathbf 0 \to A $. In other words, we can prove everything from `false'.
  \item The unit type $ \mathbf 1 $ corresponds to `true'.
  \item The negation of $ T $ is the function type $ T \to \mathbf 0 $.
  \item The conjuction ``$ A $ and $ B $'' is given as the product type $ A \times B $.
  \item The disjunction ``$ A $ or $ B $'' is given as the propositional truncation of the coproduct, union or sum type $ A \sqcup B $.
  \item The statement ``For all $ a: A $, $ B(a) $ holds'', for some dependent type $ B $ (i.e. predicate) over $ A $, is given as the dependent product $ \prod_{a: A} B(a) $.
\end{itemize}
If we think of types as propositions, the question whether some axiom is assumed or not becomes the question whether some (family of) type(s) is inhabited. Note that most of the axioms that we list here require the notion of mere propositions, propositional truncation and mere existence, which we will cover from Section \ref{sec:homotopy-props-sets} onwards.
\begin{itemize}
  \item If for all mere propositions $ A $, the type $ \Vert A \lor (A \to \mathbf 0) \Vert $ is inhabited, we say that the type system assumes the \iindex{axiom of excluded middle}: ``Either $ A $ is true, or $ A $ is not true''. This is axiom allows us to prove $ A $ from `not not $ A $'.
  \item If for all dependent types $ C $ over $ A $ and $ B $, the type
    \[ \left(\prod_{a: A} \exists (b: B), C(a, b) \right) \to \exists (f: A \to B), \prod_{a: A} C(a, f(a)) \]
    is inhabited, we say that the type system assumes the \index{axiom of choice!propositional}\textit{propositional axiom of choice}: ``The product of a family of nonempty sets is nonempty''.
  \item In any type theory which has the required constructs, the following holds: For all dependent types $ C $ over $ A $ and $ B $, the type
    \[ \left(\prod_{a: A} \sum_{b: B} C(a, b) \right) \to \sum_{f: A \to B} \prod_{a: A} C(a, f(a)) \]
    is inhabited. This axiom (or actually more of a theorem) is called the \index{axiom of choice!type theoretic}\textit{type theoretic axiom of choice}.
\end{itemize}

\section{The univalence principle}\label{sec:univalence-principle}
\enquote{\textit{Isomorphic objects are equivalent.}}

This principle is visible in most of mathematics: Sets with a bijection have the same number of elements, isomorphic groups have the same properties, and since the universal property of limits makes them ``unique up to unique isomorphism'', we can talk about `the' limit of some diagram in a category.

Now, the \iindex{univalence principle} takes this a step further. It states that

\enquote{\textit{Isomorphic objects are equal.}}

Univalent foundations seeks to be a foundation for mathematics that is in line with this principle. This is often done within the framework of `Martin-Löf dependent type theory', a type theory developed by Per Martin-Löf \autocite{martin-lof-type-theory}. It is a family of dependent type systems with dependent products, dependent sums, an empty type, a unit type and union types. It is a constructive type theory, so it does not automatically assume the axiom of excluded middle or the propositional axiom of choice. It is however compatible with these axioms, so one can still assume these alongside its usual axioms.

It is important to note that Martin-Löf type theory has \textit{identity types}: given a type $ T $ and elements $ x, y: T $, we have a type $ \mathrm{Id}_T(x, y) $ (note that this is a dependent type), which we will usually denote with $ x = y $. An element $ p: x = y $ is a proof that $ x $ is `equal' to $ y $. This type comes with an interesting induction principle called \iindex{path induction}: we can show any statement about paths $ p: x = y $ for generic $ x $ and $ y $, if we can show it about the `trivial' path $ \mathrm{refl}: x = x $ for generic $ x $. For example, symmetry of the equality boils down to a function $ \prod_{x y : T} x = y \to y = x $. We construct this using path induction with the function that sends $ \mathrm{refl}: x = x $ to itself. For more information, see \autocite{hottbook}, Section 1.12.1.

In set-theoretic mathematics, there is the concept of a `bijection' $ S \simeq T $ of sets (or an isomorphism in the category $ \SET $), which is often treated as an equivalence. It consists of functions $ f: S \to T $ and $ g: T \to S $ with $ f \cdot g = \id S $ and $ g \cdot f = \id T $. In type theory, we have a similar concept, which is called `equivalence' (of types) $ S \simeq T $. Since bijections are not well-behaved for types that are not sets (see \ref{sec:homotopy-props-sets}), because in those cases, a function $ f: S \to T $ can have multiple distinct inverses. Therefore, we define $ S \simeq T := \sum_{f: S \to T} \mathrm{isequiv}(f) $, for some predicate $ \mathrm{isequiv}: (S \to T) \to \TYPE $ (see \autocite{hottbook}, Equation 2.4.10). However, intuitively we can still think of these as bijections.

Using the identity type, we can make our statement of the univalence principle more precise (and a bit stronger). For types, we can construct a function
\[ \mathrm{idtoequiv}: \prod_{S, T: \TYPE} (S = T) \to (S \simeq T). \]
We construct this function using path induction with the identity bijection $ \prod_S \id S: (S \simeq S) $. In fact, this is a specific case of the following: for a category $ C $, if we denote the type of isomorphisms between objects $ c $ and $ d $ with $ c \cong d $, we can construct a function
\[ \mathrm{idtoiso}: \prod_{c, d: C} (c = d) \to (c \cong d), \]
using path induction with the identity isomorphism $ \prod_{c: C} \id c: (c \cong c) $. We can then formulate the univalence principle for categories as

\enquote{For all $ c, d : C $, $ \mathrm{idtoiso}_{c, d}: (c = d) \to (c \cong d) $ is an equivalence.}

A category that adheres to the univalence principle is called a \iindex{univalent category}.

Note that if $ B $ is a univalent category and $ A $ is any cateory, the functor category $ [A, B] = B^A $ is univalent as well. In particular, the presheaf category $ P A = [\op A, \SET] $ is univalent.

Also, if $ A $ is univalent, any full subcategory $ B \subseteq A $ is also univalent, because isomorphisms and equalities of objects in $ B $ can be shown to be the `the same' as their isomorphisms and equalities in $ A $.

\begin{lemma}
  For every category $ C $, there exists a univalent category $ D $ with a weak equivalence
  \[ \iota: C \hookrightarrow D. \]
\end{lemma}
\begin{proof}
  One construction takes $ D $ to be the full subcategory of the presheaf category $ PC $ of objects that are isomorphic (there \textit{merely exists} an isomorphism) to the Yoneda embedding $ \yo(X) $ of some object $ X : C $. See \autocite{univalent-categories}, Theorem 8.5.
\end{proof}
We call such a univalent category a \iindex{Rezk completion} of $ C $.

\begin{remark}
  We can call it \textit{the} Rezk completion, since it is unique: Precomposition with the weak equivalence $ \iota: C \hookrightarrow D $ gives an adjoint equivalence on the functor categories $ [D, E] \simeq [C, E] $ for any univalent category $ E $. From this, it follows that $ \iota $ is the ``initial functor'' from $ C $: any functor from $ C $ to a univalent category factors uniquely through $ \iota $.

  Now, two initial functors $ F : C \to D $ and $ F^\prime : C \to D^\prime $ factor uniquely through each other:
  \[ F^\prime = F \bullet G \quad \text{and} \quad F = F^\prime \bullet G^\prime \]
  for some functors $ G $ and $ G^\prime $ and we can show that
  \[ G \bullet G^\prime = \id D \quad \text{and} \quad G^\prime \bullet G = \id{D^\prime}. \]
  From this, we can construct an equivalence $ D \cong D^\prime $. Since $ D $ and $ D^\prime $ are univalent, this equivalence gives an equality of categories, and because $ F \bullet G = F^\prime $, we have an equality of pairs
  \[ (D, F) = (D^\prime, F^\prime). \]
\end{remark}

\section{The univalence axiom}
Now, even for a basic category, like the category of types, it seems impossible to prove that the univalence principle holds. However, this is no surprise: it turns out that it is independent of the axioms of Martin-Löf type theory (\TODO \textbf{Is this true? I need a reference for this, but could not find one}). That is: it cannot be proven, but assuming it as an axiom does not yield contradictions.

As mentioned before, univalent foundations attempts to develop as much of mathematics as possible along the univalence principle. Therefore, we assume as our first axiom the \iindex{univalence axiom}:
\begin{axiom}
  For all $ S, T: \TYPE $, the function $ \mathrm{idtoequiv}_{S, T}: (S = T) \to (S \simeq T) $ is an equivalence.
\end{axiom}
In other words:
\begin{axiom}
  $ \TYPE $ is univalent.
\end{axiom}

\begin{remark}
  One consequence of the independence of the univalence axiom is that equivalent objects are `indiscernible'. That is: even if we do not yet assume the univalence axiom, we cannot formulate a property that is satisfied by some type, but not by another, equivalent type. This is because such a property would yield a contradiction when we would assume the univalence axiom.
\end{remark}

Now, the question arises: how about the univalence axiom for categories other than $ \TYPE $? Do we need to keep assuming an additional axiom for every category that we want to be univalent? It turns out that this is not necessary. In practice, most categories consist of `sets (or types) with additional structure'. For example: topological spaces, groups, $ \lambda $-theories and algebraic theory algebras. In such categories, we can leverage the univalence of $ \TYPE $ to show that for isomorphic objects, their underlying types are equal. Also morphisms are usually defined in such a way that they `preserve' the `additional structure', which is what we need to show that the category is univalent.

Also, Theorem 4.5 in \autocite{univalent-categories} shows that if a category $ B $ is univalent (in the paper, categories are called `precategories` and univalent categories are just called `categories'), then the functor category $ A \rightarrow B $ is also univalent. In particular, the category of (pre)sheaves $ A \to \SET $ is univalent.

Therefore, the univalence axiom is a very powerful axiom, and we usually do not need to assume additional axioms to show that more categories satisfy the univalence principle.

The last structure in this section for which we want to consider the univalence axiom, is the 2-category $ \Cat $ of categories. In general, we cannot show that this satisfies the univalence principle. However, we will restrict our attention to the sub-2-category of univalent categories, which are the categories that we want to study. Then Theorem 6.8 in \autocite{univalent-categories} shows that for univalent categories $ C $ and $ D $, there is an equivalence between $ C = D $ and $ C \simeq D $, where $ C \simeq D $ denotes the type of (adjoint) equivalences of categories (see Definition \ref{def:equivalence-of-categories}).

Lastly, a result about univalent categories that we will use a couple of times in this thesis:
\begin{lemma}\label{lem:univalen-category-equivalence}
  For a functor between univalent categories $ F: A \to B $, the types `$ F $ is an adjoint equivalence' and `$ F $ is a weak equivalence' are equivalent propositions (see \ref{sec:homotopy-props-sets}).
\end{lemma}
\begin{proof}
  See \autocite{univalent-categories}, Lemma 6.8.
\end{proof}

\section{hProps and hSets}\label{sec:homotopy-props-sets}
If we have a type $ T $ and objects $ x $ and $ y $, we can wonder how many elements $ x = y $ has. In set-based mathematics, this would be a nonsensical question: two elements of a set are either equal or not equal. Therefore, we can expect the answer to be that $ x = y $ has at most one element. And indeed, if we do not assume the univalence axiom, we can assume another axiom, called `uniqueness of identity proofs', which states that for $ p, q: x = y $, we have a proof of equality $ h: p = q $.

On the other hand, suppose that we do assume the univalence axiom. Consider the two-element type $ T = \{ -1, 1 \} $. We can construct two different equivalences $ \id T, \sigma: T \simeq T $:
\[ \id T(x) = x \quad \text{and} \quad \sigma(x) = -x. \]
By the univalence axiom, we must have that the identity type $ (T = T) $ has (at least) two distinct elements, corresponding to $ \id T $ and $ \sigma $. Therefore, the univalence axiom is not compatible with uniqueness of identity proofs, and we see that in a univalent setting, some identity types have more than one element.

A consequence of this is that types in general have too little structure to serve as a foundation for mathematics that was originally set-based. For example, suppose that we want to formalize the theory of groups. A group homomorphism $ f: \mathbf{Grp}(H, G) $ is defined as a function on the underlying types $ f_1: H \to G $, together with a proof that it commutes with the group operations: $ f_2: \prod_{x, y: H} f_1(x \circ y) = f_1(x) \circ f_1(y) $. Now, normally in group theory, to show that two homomorphisms $ f, g: \mathbf{Grp}(H, G) $ are equal, we show that $ \prod_{x: H} f_1(x) = g_1(x) $, the `data' is equal. However, if we are working with types instead of sets, we also need to show that the proofs of the `properties' are equal: $ f_2 = g_2 $ (note that we actually would need to transport $ f_2 $ here, for this equality to typecheck). This makes showing equality of morphisms much more complicated for concrete groups, and sometimes outright impossible for generic groups.

To deal with this, we need the concepts of mere propositions and (homotopy) sets:

\begin{definition}
  A \iindex{mere proposition} is a type $ T $ such that for all $ x, y: T $, $ x = y $.
\end{definition}

\begin{definition}
  A \iindex{homotopy set} is a type $ T $ such that for all $ x, y: T $, $ x = y $ is a mere proposition.
\end{definition}

Since the identity types for a homotopy set are mere propositions, a homotopy set mimicks a set in set theory, where equality between elements `is' or `is not'. If we restrict the underlying type of a group to be a homotopy set, it can be shown that $ \prod_{x, y: H} f_1(x \circ y) = f_1(x) \circ f_1(y) $ is a mere proposition, so $ f_2 = g_2 $ trivially. This is often true when translating definitions from set theory to univalent foundations: if we base our objects on homotopy sets instead of types, we only have to worry about equality of `data', the equality of `properties' follows automatically.

For similar reasons, we restrict the hom-types $ C(c, d) $ of categories to be sets. Note that $ \TYPE $ is not a category under this definition (it is a `precategory', for some definition of precategory), because in general, the type of functions between sets $ S \to T $ is not a set.

\section{Truncation and (mere) existence}
As mentioned before, if we want to do mathematics in a dependent type theory, we can `encode' propositions as types. The elements of the type correspond to the proofs that the proposition is true, see for example the identity types. However, we need to be careful here about the distinction between types in general and mere propositions:

A proof that a type $ T $ is nonempty usually consists of giving an element $ t : T $. If we encode the statement ``$ T $ is nonempty'' as $ T $, and if $ T $ is not a mere proposition, then ``$ T $ is nonempty'' is not a mere proposition, so it has multiple distinct proofs. In some cases, this is exactly what we want, because we want to retrieve the chosen element of $ T $. However, in some cases, we want the fact that a set is nonempty (or any statement in general) to be a mere proposition, to avoid having to carry around a specific element and having to prove equality of two specific elements. For such cases, we have the `propositional truncation':

\begin{definition}
  For a type $ T $, a type $ \Vert T \Vert $ exists (\autocite{hottbook}, Section 3.7) with the properties that for all $ t: T $, we have an element $ \vert t \vert : \Vert T \Vert $ and that $ \Vert T \Vert $ is a mere proposition. It has a recursion principle stating that for a mere proposition $ B $, a function $ f: T \to B $ induces a function $ \vert f \vert: \Vert T \Vert \to B $ that commutes with $ \vert \cdot \vert $. This object is called the \index{truncation!propositional}\textit{propositional truncation}.
\end{definition}
The propositional truncation forgets the details about a type, and only keeps the information about whether it is inhabited or not. The recursion principle means that if we are trying to prove a mere proposition based on some element $ \vert t \vert: \Vert T \Vert $, we can pretend that we do have a concrete element $ t: T $.

There also is the concept of higher order truncations. For example, the \index{truncation!set}\textit{set truncation} $ \Vert A \Vert_0 $, which is a homotopy set and has an equivalence $ (\Vert A \Vert_0 \to B) \simeq (A \to B) $ for any set $ B $. However, these higher truncation types become increasingly harder to construct, and in this thesis, we will only need to consider the propositional truncation.

Often, when we prove a theorem or lemma that ``there exists some $ x: X $ such that $ Y(x) $'', what we actually mean is that we can construct an element $ x: X $ and then an element $ y : Y(x) $ of the dependent type $ Y $ over $ X $. This is equivalent to constructing an element of $ \sum_{x: X} Y(x) $. However, this is in general not a mere proposition. If we want to express that the set of such $ x $ is nonempty as a mere proposition, we talk about \iindex{mere existence}:
\begin{definition}\label{def:mere-existence}
  Given a dependent type $ Y $ over $ X $, if we say that there \textit{merely exists} an element $ x: X $ such that $ Y(x) $, we mean that we have an element of the propositional truncation
  \[ h: \left(\exists x: Y(x)\right) := \left\Vert \sum_{x: X} Y(x) \right\Vert. \]
\end{definition}

For example, if we have objects in a category $ c, d: C $ and we want to talk about a retraction $ f $ of $ c $ onto $ d $, we commonly define this as ``a morphism $ f_1: C(c, d) $, such that there exists a `section': a morphism $ f_2: C(d, c) $ with $ f_2 \cdot f_1 = \id d $''. Now, we commonly consider $ f_1 $ to be the `data' of the retraction; we consider retractions $ f $ and $ f^\prime $ to be the same retraction if $ f_1 = f^\prime_1 $. This means that being a retraction is about the `mere existence' of a section. Note, however, that in this case, we cannot use the section in constructions, except when we are trying to prove mere propositions.

Note that for the Curry-Howard correspondence, the product or conjunction $ A \land B = A \times B $ of two mere propositions is again a mere proposition. However, the union $ A \sqcup B $ is not necessarily a mere proposition. To make it into a mere proposition, we need to take the propositional truncation $ A \lor B = \Vert A \sqcup B \Vert $.

\section{Equality and homotopy}
Univalent Foundations is often mentioned together with Homotopy Type Theory, because they are related but distinct concepts. Therefore, we will mention it here.

As we saw before, if we do not assume uniqueness of identity proofs, given two elements $ x, y: T $, we can have multiple distinct proofs that $ x = y $, which is counterintuitive. One way to think about this, is the perspective of homotopy type theory. In homotopy type theory, one considers a type $ T $ to be a `space', intuitively similar to a topological space, but without an explicit topology. The elements $ t : T $ are then the points of the space. Elements of the identity type $ (s = t) $ are interpreted as `paths' from $ s $ to $ t $. That is why the induction principle of $ (s = t) $ is called `path induction'. Of course, we can go higher: for $ p, q : x = y $, the elements $ (p = q) $ are `paths between paths', (path) `homotopies', `sheets' or `1-cells', for homotopies $ h, h^\prime: p = q $, the elements of $ h = h^\prime $ are paths between paths between paths, `volumes' or `2-cells' etc.

If we have a `geometric' interpretation of our type theory, we can investigate the `shape' of a (nonempty) type $ T $, given by the structure of the (higher) identity types.

\begin{figure}
  \begin{subfigure}{.3\textwidth}
    \includesvg[pretex=\tiny, width=\columnwidth]{img/homotopy-plane}
    \caption{}
    \label{fig:homotopy-plane}
  \end{subfigure}
  \begin{subfigure}{.3\textwidth}
    \includesvg[pretex=\tiny, width=\columnwidth]{img/homotopy-cylinder}
    \caption{}
    \label{fig:homotopy-cylinder}
  \end{subfigure}
  \begin{subfigure}{.3\textwidth}
    \includesvg[pretex=\tiny, width=\columnwidth]{img/homotopy-sphere}
    \caption{}
    \label{fig:homotopy-sphere}
  \end{subfigure}
  \caption{Different possible homotopy structures}
\end{figure}

First of all, if we have $ x, y : T $ for which $ x = y $ does not have an inhabitant, $ x $ and $ y $ lie in different `connected components'. Now, we focus on a connected type:

For example, are all elements $ x, y: T $ of the type equal to each other, and are all elements $ p, q: x = y $ of all the (higher) identity types also equal in a unique way? Then we have a \iindex{contractible type}, which intuitively looks somewhat like a plane (Figure \ref{fig:homotopy-plane}).

It is also possible that any two elements $ x, y : T $ are equal, but that there are distinct paths $ p, q: x = y $, with no homotopy between them. Then intuitively $ T $ looks like a circle or a tube or a projective space, or something more complicated (Figure \ref{fig:homotopy-cylinder}).

\begin{remark}
  Note that we can give the type of paths $ x = x $ a group structure and $ x = y $ is a `torsor' for this group. If $ x = y $ has exactly two distinct elements, the group $ x = x $ looks like $ \mathbb Z / 2 \mathbb Z $, which suggests some projective plane-like structure.

  For something to look like a circle or tube, we need this group $ x = x $ to be isomorphic to $ \mathbb Z $. For an example, see \autocite{circle-fundamental-group}.
\end{remark}

As a third example, consider a type $ T $ in which any two elements $ x, y : T $ are equal, and any two paths $ p, q: x = y $ have two distinct homotopies $ h, h^\prime: p = q $ between them. Then we can imagine the type to look somewhat like a sphere (Figure \ref{fig:homotopy-sphere}) or something more complicated.

This is the lens through which homotopy type theory studies types.

\section{Transport}

\begin{figure}
  \begin{subfigure}{.45\textwidth}
    \includesvg[pretex=\tiny, width=.7\columnwidth]{img/fibration}
    \caption{}
    \label{fig:fibration}
  \end{subfigure}
  \begin{subfigure}{.45\textwidth}
    \includesvg[pretex=\tiny, width=.7\columnwidth]{img/transport}
    \caption{}
    \label{fig:transport}
  \end{subfigure}
  \caption{A fibration with a path in the base space, giving rise to transport in the fibration}
\end{figure}

Suppose that we have a dependent type $ B: A \to UU $. Intuitively, $ B $ is a collection of spaces, lying over the points of $ A $ (Figure \ref{fig:fibration}). Now, if we have a path $ p: a = a^\prime $ in $ A $, we can use path induction to get a function $ \mathrm{transport}_p: B(a) \to B(a^\prime) $ (Figure \ref{fig:transport}).

For example, this allows us to transfer an element from $ \mathtt{array}(T, n + n) $ to $ \mathtt{array}(T, 2 \cdot n) $ with transport over the path $ n + n = 2 \cdot n $. Also, if we assume the univalence axiom, and we have, for example, an isomorphism of groups $ f: G \cong G^\prime $ and a proof $ h $ that $ G $ is semisimple, we can transport $ h $ along the equality given by $ f $ to a proof that $ G^\prime $ is semisimple.

\begin{example}\label{ex:transport-array}
  In addition, consider the following example. We take $ A $ to be a type universe, and let $ B: T \mapsto \mathtt{array}(T, 3) $. Consider the types $ a = \{ \top, \bot \} $ and $ a^\prime = \{ 0, 1 \} $. We have equivalences $ \overline p $ and $ \overline q $ between $ a $ and $ a^\prime $:
  \[ \overline p(\top) = 1, \overline p(\bot) = 0 \quad \text{and} \quad \overline q(\top) = 0, \overline q(\bot) = 1. \]
  That means that we have distinct equalities $ p, q: a = a^\prime $. Now, suppose that we have an array $ x = [\top, \top, \bot]: B(a) $. Since $ a = a^\prime $, we would want to treat $ x $ as an element of $ B(a^\prime) $, but then we need to make a choice whether we treat it as $ [1, 1, 0] $ or $ [0, 0, 1] $ (or something else altogether). This is exactly the question whether we transport along $ p $ or $ q $, and therefore, when our base type is not a set, it is important to be aware that we are transporting a property along an equality, and we need to be careful which equality it is that we transport along.
\end{example}

Another place where transports occur frequently is when proving equality $ (x, y) = (x^\prime, y^\prime) $ of elements of a dependent sum $ \sum_{a: A} B(a) $. We can start componentwise by proving $ p: x = x^\prime $, but after this, we cannot directly prove $ q: y = y^\prime $, because these two live in different types: $ B(x) $ and $ B(x^\prime) $ respectively. Therefore, we need to transport, and then prove
\[ \mathrm{transport}_p(y) = y^\prime: B(x^\prime). \]

\subsection{Caution: `transport hell'}
Now, it seems that we can transport all properties and data along equalities. And of course, that is true, but some caution needs to be taken with this when working with a proof assistant.

For example, consider the situation in Example \ref{ex:transport-array}. As mentioned, we can transport $ x $ to get an element $ y := \mathrm{transport}_p(x) $, which is an element of $ B(a^\prime) $, but now suppose that we want to compute something using its first coefficient $ y_1 $. Of course, on paper it quickly becomes clear that $ y_1 = 1 $, but in practice, it takes quite some work to convince a proof assistant of that fact. Of course, we could write a lemma which states that for any array $ x $, any equivalence $ \overline p $ with corresponding equality $ p $, and any index $ i $,
\[ \mathrm{transport}_p(x)_i = \overline p(x_i). \]
However, at that point, we have more or less constructed our own function between $ B(a) $ and $ B(a^\prime) $, which is much easier to work with than $ \mathrm{transport}_p $.

Experience teaches that in general, it is fine to transport properties $ b: B(a) $ of which we will never need the value, only the fact that it (merely) exists. On the other hand, for properties and constructions of which we might later want to use the actual value, it is much better to transfer them `by hand'. Often, but not always, this criterion coincides with $ B(a) $ being a mere proposition.

Another case where unwanted transports often occur is when showing the equality of two complicated elements of a type. For example, $ ((x_1, x_2), (x_3, x_4)) = ((x^\prime_1, x^\prime_2), (x^\prime_3, x^\prime_4)) $, both elements of
\[ \sum_{(x_1, x_2): \sum_{a: A} B(a)} \sum_{x_3 : C(x_1)} D(x_1, x_2, x_3). \]
This example involves proofs:
\begin{itemize}
  \item $ p : x_1 = x^\prime_1 $;
  \item $ q : \mathrm{transport}_p(x_2) = x^\prime_2 $;
  \item $ r : \mathrm{transport}_{(p, q)}(x_3) = x^\prime_3 $, which can be simplified to $ \mathrm{transportf}_p(x_3) = x^\prime_3 $, since $ x_3 $ does not depend on $ x_2 $
  \item and finally $ s: \mathrm{transport}_{((p, q), r)}(x_4) = x^\prime_4 $.
\end{itemize}
In general, these latter proofs $ r $ and $ s $ are very challenging and best avoided whenever possible. The situation in which one has to work with such complicated transports, which make a proof (seemingly unnecesarily) complicated is called \iindex{transport hell}. Some mathematical tools have been developed (see for example \ref{sec:displayed-categories} about displayed categories) that help avoid transport hell in some cases. Additionally, the structure or `strategy' of a proof can sometimes be changed in order to avoid the worst of the transports. However, it is often not possible to avoid transports altogether, and this is one of the reasons why proving something in a proof assistant takes a lot more work than proving it on paper.
